\section{The StreamIt Language}

StreamIt is an architecture-independent programming language for
high-performance streaming
applications~\cite{streamitcc,streamit-lang-spec}.  As described
previously, it represents programs as a set of independent actors
(referred to as {\it filters} in StreamIt) that use explicit data
channels for all communication.  Each filter contains a {\tt work}
function that executes a single step of the filter.  From within {\tt
work}, filters can {\it push} items onto the output channel, {\it pop}
items from the input channel, or {\it peek} at an input item without
removing it from the channel.  
%Peeking enables a natural expression of
%sliding-window computations (e.g., FIR filters).  
While peeking requires special care in parts of our analysis, it is
critical for exposing data parallelism in sliding-window filters
(e.g., FIR filters), as they would otherwise need internal state.

StreamIt provides three hierarchical primitives for composing filters
into larger stream graphs.  A {\it pipeline} connects streams
sequentially; for example, there is a four-element pipeline beginning
with {\tt UnwrapPhase} in the Vocoder example
(Figure~\ref{fig:vocoder}).  A {\it splitjoin} specifies independent,
task-parallel streams that diverge from a common {\it splitter} and
merge into a common {\it joiner}.  For example, the {\tt AdapativeDFT}
filters in Figure~\ref{fig:vocoder} form a two-element splitjoin (each
filter is configured with different parameters).  The final
hierarchical primitive in StreamIt is the {\it feedbackloop}, which
provides a way to create cycles in the graph.  In practice,
feedbackloops are rare and we do not consider them in this paper.

In this paper, we require that the push, pop, and peek rates of each
filter are known at compile time.  This enables the compiler to
calculate a steady-state for the stream graph: a repetition of each
filter that does not change the number of items buffered on any data
channel~\cite{lee87,karczmarek:lctes:2003}. In
combination with a simple program analysis that estimates the number
of operations performed on each invocation of a given work function,
the steady-state multiplicities offer an estimate of the work
performed by a given filter as a fraction of the overall program
execution.  This estimate is important for our software pipelining
technique.
