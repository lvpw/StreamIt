Video playback devices rely on compression 
algorithms to minimize storage, transmission bandwidth, and overall cost. 
Compression techniques have high realtime and sustained
throughput requirements, and the end of CPU clock scaling 
means that parallel implementations
for novel system architectures are needed. Parallel implementations
increase the complexity of application design.
Current languages force the programmer to trade off productivity for
performance;
the performance demands dictate that the parallel programmer
choose a low-level language in which he can explicitly 
control the degree of parallelism and tune his code for performance. 
This methodology is not cost effective because this architecture-specific code is 
neither malleable nor portable. 
Reimplementations must be written from scratch for each of the 
existing 
parallel and reconfigurable architectures.

This thesis shows that multimedia compression algorithms, composed of many 
independent processing stages, are a good match for the streaming model of computation.
Stream programming models afford certain advantages in terms of programmability, robustness, 
and achieving high performance. 

This thesis intends to influence language design towards the inclusion of
features that lend to the efficient implementation and parallel 
execution of streaming applications like image and video compression algorithms. 
Towards this I contribute $i)$ a clean, malleable, and portable implementation of an 
MPEG-2 encoder and decoder expressed in a streaming fashion, $ii)$ an 
analysis of how a streaming language improves programmer productivity, $iii)$ an 
analysis of how a streaming language enables scalable parallel execution, $iv)$
an enumeration of the language features that are needed to cleanly express
compression algorithms, $v)$ an enumeration of the language features that 
support large scale application development and promote software engineering
principles such as portability and reusability.
This thesis presents a case study of MPEG-2 encoding and decoding
to explicate points about language expressiveness. 
The work is in the context of the StreamIt programming language.


