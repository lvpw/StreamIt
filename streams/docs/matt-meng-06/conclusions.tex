\chapter{Conclusions}
\label{chapter:conclusions}

Compression schemes play a key role in the proliferation of 
multimedia applications and the digital media explosion. 
At the same time, applications must be written 
for a plethora of unique parallel architectures. 
This thesis has shown that stream programming is an 
ideal model of computation for realizing image and 
video compression schemes. 
For this domain, stream programming boosts programmer 
productivity and enables scalable parallel execution 
of an application on a variety of architecture targets. 
Streaming language features allow a programmer to 
efficiently express a computation and expose 
parallelism, enabling a compiler to provide scalable performance.

This thesis shows these points through the following contributions: 
$(i)$ clean, malleable, and portable MPEG-2 encoder and 
decoder implementations expressed in a streaming fashion, 
$(ii)$ an analysis showing that a streaming language 
improves programmer productivity, 
$(iii)$ an analysis showing that a streaming language 
enables parallel execution, 
$(iv)$ an enumeration of the language features that are 
needed to cleanly express compression algorithms, 
$(v)$ an enumeration of the language features that 
support large scale application development and promote 
software engineering principles such as portability and reusability. 

This work was performed in the context of the StreamIt 
programming language, for its ability to express streaming 
computations, and the MPEG-2 video compression scheme. 
However the work is relevant to the domain of multimedia
codecs, including JPEG and MPEG-4.
Currently the H.264 
compression scheme is poised to supersede MPEG-2 video 
compression; assuming the language suggestions in this 
paper make their way into streaming languages, the most 
interesting research direction would be scalable, portable, 
and malleable implementations of H.264 codecs, expressed in a 
streaming fashion.

\section{Future Work}
\label{sec:future}

%% First, as the current transformation has the potential to increase the
%% size of the file, we plan to explore lightweight techniques for
%% re-compressing a data stream that is already partially compressed.
%% This should be straightforward in the case of Apple Animation; for
%% example, a run-length encoded unit can be extended without needing to
%% be rediscovered.

There remain rich areas for future work in computing on compressed
data.  First, the compressed processing technique can be applied far
beyond the current focus.  In its current form, the technique could be
evaluated on video operations such as thresholding, color depth
reduction, sepia toning, saturation adjustment, and color replacement.
With minor extensions (see Section~\ref{sec:extensions}), the
technique can support video operations such as cropping, padding,
histograms, image flipping, sharpening, and blurring.  The technique
may also have applications in an embedded setting, where it could
offer power savings---for example, in processing the RAW data format
within digital cameras.  It may even be possible to do sparse matrix
operations using the technique; in addition to compressing the
locations of the zero elements, LZ77 would also compress repetetive
patterns in the non-zero elements.

Research is also underway to apply a similar technique to lossy,
DCT-based compression formats.  The streaming model cf computation
also offers key advantages in this domain, as neighboring actors that
compute linear functions can be algebraically simplified at compile
time~\cite{aalamb}.  For example, a JPEG transcoder typically performs
an iDCT (during decompression), followed by the user's transformation,
followed by a DCT (during compression).  If the user's transformation
is also linear (e.g., color inversion) then all three stages can be
automatically collapsed, thereby eliminating the decompression and
re-compression steps.  Preliminary experiments in this direction
indicate speedups upwards 10x.  By extending the framework to multiple
compression formats, users will be able to write their transformations
once, in a high-level language, and rely on the compiler to map the
computations to each of the compresed domains.

