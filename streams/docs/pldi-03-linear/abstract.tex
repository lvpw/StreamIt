As DSP programming is becoming more complex, there is an increasing
need for high-level abstractions that can be efficiently compiled.
Toward this end, we present a set of aggressive optimizations that
target linear sections of a stream program.  Our input language is
StreamIt, which represents programs as a hierarchical graph of
autonomous filters.  A filter is linear if each of its outputs can be
represented as an affine combination of its inputs.  Linear filters
are common in DSP applications; examples include FIR filters,
expanders, compressors, FFTs and DCTs.

We demonstrate that several traditional optimizations on linear
filters can be completely automated by the compiler.  First, we
present a linear extraction analysis that automatically detecs linear
filters based on the C-like code in their work function.  Then, we
give a procedure for combining adjacent linear filters into a single
filter, as well as for translating a linear filter to operate in the
frequency domain.  We also present an optimization selection
algorithm, which finds the sequence of combination and frequency
transformations that will give the maximal benefit.

We have completed a fully-automatic implementation of the above
techniques as part of the StreamIt compiler, and we demonstrate
performance improvements that average 450\% over our benchmark suite.
As our benchmarks are written in StreamIt at a high level of
abstraction, we see the optimizations as an enabling technology for
non-expert DSP programmers to leverage complex, state-of-the-art
techniques.
