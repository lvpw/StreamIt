\begin{figure}[t]
  \psfig{figure=images/part-algorithm.eps,width=3.5in}
  \caption{Algorithm for optimization selection.
  \protect\label{fig:part-alg}}
\end{figure}

\begin{figure}[t]
  \psfig{figure=images/part-algorithm2.eps,width=3.5in}
  \caption{Type declarations for code in Figure~\ref{fig:part-alg}.}
\end{figure}

\begin{figure}[t]
  \psfig{figure=images/part-algorithm3.eps,width=3.5in}
  \caption{Cost functions for optimization selection.}
\end{figure}

\section{Optimization Selection}

MOTIVATION

- only beneficial to combine certain filters

  - further, there could be two filters that are less efficient when
    combined, but combining them allows for larger combinations that
    enable more savings

- can only combine some parts of certain structures
- can only go to frequency if the pop rate is 1

  - thus, have to weight the tradeoff between greater linear savings
    if combined (and raised pop rate), vs. frequency savings if
    transformed at a given node

- further, some matrix multiplies are more or less expensive than
  others.  for example, if there are many zeros, they can be factored
  out.  If there is symmetry, something might be more efficient;
  wouldn't want to lose that symmetry by combining with a higher node.

  - differing sizes could be subjected to different cost functions on
  a given architecture, and with a given matrix multiply routine

- also, there is an overhead to different stream configurations.  For
  examples, on some parallel architectures, there could be load
  balancing issues involved with combining more streams

- also note that in frequency, can't take advantage of zero entries of
  matrix.

ALGORITHM

Ours takes into account the following aspects:

cost function

  - for linear nodes, the number of multiplies and adds (disregards
    zero entries of matrix)

  - for frequency nodes, a speedup over the linear node that was
    obtained from profiling.  We show a factor of 50 improvement over
    a linear node with no zero's in it.

      - also incorporates knowledge on where a frequency
        transformation is possible

the way it works
  - intuitively
  - reference to pseudocode

optimizations
  - symmetery, etc.

EXAMPLE

- target detect or filterbank

---

