%% Upward of fifty percent of the code that runs the DSP(s) in a modern
%% cell phone is coded in assembly with the rest written in C. Hand
%% optimized assembly code typically makes the best use of the available
%% resources such as power, specialized coprocessors, and specialized
%% instructions.  The problem with assembly code is that the same
%% algorithm must be mapped time and time again whenever a new chip comes
%% out. The life cycle of a typical DSP is much shorter than the life
%% cycle of a general purpose microprocessor -- each new generation is
%% separated by months rather than years.
 
%% Therefore frequently reimplementing algorithms by hand is a costly,
%% arduous process that increases cost and slows the pace of
%% advances. Engineers must spend time working out details rather than
%% focusing on solving harder problems. Compilers were invented forty
%% years ago exactly to let engineers focus on the problem at hand rather
%% than spend time with machine specific details. Compilers for DSP
%% architectures have a difficult job, and are not very good at mapping a
%% program written in a general purpose language like C into the
%% specialized instructions provided by DSPs. Many of the instructions
%% provided by a DSP are targeted for a very specific application (like
%% FIR filtering), but most general purpose languages have no way to
%% describe higher level behavior other than functionally. If you don't
%% express your algorithm in the same way that the compiler expects to
%% encounter it, the resulting program will not take best advantage of
%% the available DSP resources.

\section{Introduction}
Digital computation is becoming a ubiquitous element of modern life.
Everything from cell phones to HDTV systems to satellite radios
require increasingly sophisticated algorithms for digital signal
processing.  Optimization is especially important in this domain, as
embedded devices commonly have high performance requirements and tight
resource constraints.  Consequently, there are often two stages to the
development process: first, the algorithm is designed and simulated at
a high level of abstraction, and second, it is optimized and
re-implemented at a low level by an expert DSP programmer.  In order
to achieve high performance, the DSP programmer needs to take
advantage of architecture-specific features and constraints (usually
via extensive use of assembly code) as well as global properties of
the application that could be exploited to obtain algorithmic
speedups.  Apart from requiring expert knowledge, this effort is
time-consuming, error-prone, and costly, and must be repeated for
every change in the target architecture and every adjustment to the
high-level system design.  As embedded applications continue to grow
in complexity, these factors will become unmanageable.  There is a
pressing need for high-level DSP abstractions that can be compiled
without any performance penalty.

In this paper, we develop a set of optimizations that lower the entry
barrier for high-performance stream programming.  Our work is done in
the context of StreamIt~\cite{streamit-asplos,streamitcc}, which is a
high-level language for signal processing applications.  A program in
StreamIt is comprised of a set of concurrently executing filters, each
of which contains its own address space and communicates with its
neighbors using FIFO queues.  Our analysis focuses on filters which
are {\it linear}: that is, each of their outputs can be expressed as
an affine combination of their inputs.  Linear filters are common in
DSP applications; examples include FIR filters, expanders,
compressors, FFTs and DCTs.

In practice, there are a host of optimizations that are applied to
linear portions of a stream graph.  In particular, neighboring linear
nodes can be combined into one, and large linear nodes can benefit
from translation into the frequency domain.  However, these
optimizations require detailed mathematical analysis and are tedious
and complex to implement.  They are only beneficial under certain
conditions---conditions that might change with the next version of the
system, or that might depend on neighboring components that are being
written by other developers.  To improve the modularity, portability,
and extensibility of stream programs, the compiler should be
responsible for identifying linear nodes and performing the
appropriate optimizations.  Towards this end, we make the following
contributions:
\begin{itemize}
\vspace{-6pt}

\item A linear dataflow analysis that extracts an abstract linear
representation from imperative C-like code.
\vspace{-6pt}

\item An automated transformation of neighboring linear nodes into a
single collapsed representation.
\vspace{-6pt}

\item An automated translation of linear nodes into the frequency
domain.
\vspace{-6pt}

\item An optimization selection algorithm that determines which
transformations are most beneficial to apply.
\vspace{-6pt}

\item A fully-automatic implementation of these techniques in the
StreamIt compiler, demonstrating an average speedup of 450\% and a
best-case speedup of 800\%.
\vspace{-6pt}

\end{itemize}
In the rest of this section, we give a motivating example and
background information on StreamIt.  Then we present our linear node
representation (Section~\ref{sec:linearrep}) and our supporting
dataflow analysis (Section~\ref{sec:dataflow}).  Next we describe
structural transformations on linear nodes
(Section~\ref{sec:combine}), a frequency domain optimization
(Section~\ref{sec:freq}) and an optimization selection algorithm
(Section~\ref{sec:partitioning}). Finally, we present results
(Section~\ref{sec:results}), related work (Section~\ref{sec:related})
and conclusions (Section~\ref{sec:conclusion}).

\subsection{Motivating Example}
\begin{figure}[t]
\vspace{-6pt}
\center
\epsfxsize=3.0in
\epsfbox{images/motivating-example.eps}
\vspace{-5pt}
\caption{Block diagram of two FIR filters.}
\vspace{-5pt}
\makeline
\vspace{-3pt}
\label{fig:motivating-fig}
\scriptsize
\begin{verbatim}
/* perform two consecutive FIR filters with weights w1, w2 */
void two_filters(float* w1, float* w2, int N) {
  int i;
  float data[N];         /* input data buffer */
  float buffer[N];       /* inter-filter buffer */
  
  for (i=0; i<N; i++) {  /* initialize the input data buffer */
    data[i] = get_next_input();
  }
  
  for (i=0; i<N; i++) {  /* initialize inter-filter buffer */
    buffer[i] = filter(w1, data, i, N);
    data[i] = get_next_input();
  }
  
  i = 0;
  while(true) {
    /* generate next output item */
    push_output(filter(w2, buffer, i, N));
    /* generate the next element in the inter-filter buffer */
    buffer[i] = filter(w1, data, i, N);
    /* get next data item */
    data[i] = get_next_input();
    /* update current start of buffer */
    i = (i+1)%N;
  }
}

/* perform N-element FIR filter with weights and data */
float filter(float* weights, float* data, int pos, int N) {
  int i;
  float sum = 0;

  /* perform weighted sum, starting at index pos */
  for (i=0; i<N; i++, pos++) {
    sum += weights[i] * data[pos];
    pos = (pos+1)%N;
  }
  return sum;
}
\end{verbatim}
\vspace{-18pt}
\caption{Two consecutive FIR filters in C.  Channels are represented
as circular buffers, and the scheduling is done by hand.
\protect\label{fig:motivating-example}}
\makeline
\vspace{-12pt}
\end{figure}

\begin{figure}[t]
\vspace{-6pt}
\scriptsize
\begin{verbatim}
float->float pipeline TwoFilters(float[N] w1, float[N] w2) {
  add FIRFilter(w1);
  add FIRFilter(w2);
}

float->float filter FIRFilter(float[N] weights) {
  work push 1 pop 1 peek N {
    float sum = 0;
    for (int i=0; i<N; i++) {
      sum += weights[i] * peek(i);
    }
    push(sum);
    pop();
  }
}
\end{verbatim}
\vspace{-18pt}
\caption{Two consecutive FIR filters in StreamIt.  Buffer management
and scheduling are handled by the compiler.\protect\label{fig:example-streamit}}
\vspace{-8pt}
\makeline
\vspace{-3pt}
\begin{verbatim}
float->float filter CollapsedTwoFilters(float[N] w1, float[N] w2) {
  float[N] combined_weights;

  init {  /* calculate combined_weights from w1 and w2 */  }

  work push 1 pop 1 peek N {
    float sum = 0;
    for (int i=0; i<N; i++) {
      sum += combined_weights[i]*peek(i);
      }
    push(sum);
    pop();
  }
}
\end{verbatim}
\vspace{-18pt}
\caption{Combined version of the two FIR filters.  Since each FIR
filter is linear, the weights can be combined into a single {\tt
combined\_weights} array.\protect\label{fig:example-combine}}
\vspace{-8pt}
\makeline
\vspace{-3pt}
%% float->float filter FreqTwoFilters() {
%%   complex[N] H;
%%   init {
%%     H = FFT(combined_weights);
%%   }
%%   work push L pop L peek N+L {
%%     float[N] X = FFT(peek(0..N+L-1)); /* input FFT */
%%     float[N] Y =  X .* H; /* element wise mult */
%%     float[N] y = IFFT(Y); /* inverse FFT */
%%     push(y[0..L-1]); /* push first L elts of y */
%%   }
%% }
\begin{verbatim}
float->float pipeline FreqTwoFilters(float[N] w1, float[N] w2) {
  float[N] combined_weights = ... ;     // calc. combined weights
  complex[N] H = fft(combined_weights); // take FFT of weights
  add FFT();                            // add FFT stage to stream
  add ElementMultiply(H);               // add multiplication by H
  add IFFT();                           // add inverse FFT
}
\end{verbatim}
\vspace{-18pt}
\caption{Combined version of two FIR filters in the frequency domain.
\protect\label{fig:example-frequency}}
\vspace{-10pt}
\makeline
\vspace{-10pt}
\end{figure}

To illustrate the program transformations that our technique is
designed to automate, consider a sequence of finite impulse response
(FIR) filters as shown in Figure~\ref{fig:motivating-fig}. The
imperative C style code that implements this simple DSP application is
shown in Figure~\ref{fig:motivating-example}. 
The program largely defies many standard compiler analysis
and optimization techniques because of its use of circular buffers and
the muddled relationship between {\tt data}, {\tt buffer} and the
output.

Figure~\ref{fig:example-streamit} shows the same filtering process
implemented in StreamIt. The StreamIt version is more abstract than
the C version.  It indicates the communication pattern between
filters, shows the structure of the original block diagram and leaves
the complexities of buffer management and scheduling to the compiler.

Two optimized versions of the FIR program are shown in
Figures~\ref{fig:example-combine} and~\ref{fig:example-frequency}.  In
Figure~\ref{fig:example-combine}, the programmer has combined the {\tt
weights} arrays from the two filters into a single, equivalent array.
This reduces the number of multiply operations by a factor of two.  In
Figure~\ref{fig:example-frequency}, the programmer has done the
filtering in the frequency domain, using the FFT and IFFT to translate
between time and frequency.  Computationally intensive streams are
more efficient in frequency than in time.

Our linear analysis can automatically derive both of the
implementations in Figures~\ref{fig:example-combine}
and~\ref{fig:example-frequency}, starting with the code in
Figure~\ref{fig:example-streamit}.  These optimizations free the
programmer from the burden of combining and optimizing linear filters
by hand.  Instead, the programmer can design modular filters at the
natural granularity for the algorithm in question and rely on the
compiler for combination and transformation.

\subsection{StreamIt}

StreamIt is a language and compiler for high-performance signal
processing~\cite{gordon-thesis,streamit-asplos,streamitcc}.  In a
streaming application, each data item is in the system for only a
small amount of time, as opposed to scientific applications where the
data set is used extensively over the entire execution.  Also, stream
programs have abundant parallelism and regular communication patterns.
The StreamIt language aims to expose these properties to the compiler
while maintaining a high level of abstraction for the programmer.

StreamIt programs are composed of processing blocks called {\it
filters}.  Each filter has an input tape from which it can read values
and an output tape to which it can write values.  Each filter also
contains a {\it work} function which describes the filter's atomic
execution step in the steady state.  If the first invocation of the
work function has different behavior than other executions, a
special {\it prework} function is defined.

The work function contains C-like imperative code, which can
access filter state, call external routines and produce and consume
data.  The input and output channels are treated as FIFO queues, which
can be accessed with three primitive operations: 
1) {\it pop()}, which returns the first item on the input tape and 
advances the tape by one item, 
2) {\it peek(i)}, which returns the value at the $i$th position
on the input tape, 
and 3) {\it push$(v)$}, which pushes value $v$ onto the output tape.  
Each filter must declare the maximum element it
will peek at, the number of elements it will pop, and the
number of elements that it will push during an execution of 
work.  These rates must be resolvable at compile time and constant
from one invocation of work to the next.

\begin{figure}[t]
\vspace{-6pt}
~~
\begin{minipage}{0.46in}
\centering
\psfig{figure=images/pipeline.eps,width=0.46in} \\
\end{minipage} 
~
\begin{minipage}{1.3in}
\centering
\psfig{figure=images/splitjoin.eps,width=1.3in} \\
\end{minipage}
~
\begin{minipage}{1.02in}
\centering
\psfig{figure=images/feedback.eps,width=1.02in} \\
\end{minipage} 
\\ ~ \\ {\bf \protect\small (a) A pipeline. ~~(b) A splitjoin. ~~(c) A feedbackloop.}
\caption{\protect\small Stream structures supported by StreamIt.
\protect\label{fig:structures}}
\vspace{-14pt}
\makeline
\vspace{-14pt}
\end{figure}

A program in StreamIt consists of a hierarchical graph of filters.
Filters can be connected using one of the three predefined structures
shown in Figure~\ref{fig:structures}: 1) {\it pipelines} represent the
serial computation of one filter after another, 2) {\it splitjoins}
represent explicitly parallel computation, and 3) {\it feedbackloops}
allow cycles to be introduced into the stream graph.  A {\it stream}
is defined to be either a filter, pipeline, splitjoin or
feedbackloop. Every subcomponent of a structure is a stream, and all
streams have exactly one input tape and exactly one output tape.

It has been our experience that most practical applications can be
represented using StreamIt's hierarchical structures.  Though
sometimes a program needs to be reorganized to fit into the structured
paradigm, there are benefits for both the programmer and the compiler
in having a structured language~\cite{streamitcc}.  In particular, the
linear analyses described in this paper rely heavily on the structure
of StreamIt, as they can focus on each hierarchical primitive rather
than dealing with the complexity of arbitrary graphs.
