\section{Detailed Example}
\label{sec:example}

We now discuss the StreamIt implementation of the Trunked Radio
illustrated in Figure~\ref{fig:radiodiagram}.  The Trunked Radio is a
frequency-hopping system in which the receiver switches between a set of
known frequencies whenever it hears certain tones from the transmitter.

The toplevel class, {\tt TrunkedRadio}, is implemented as a seven-stage
Pipeline (see Figure~\ref{fig:radiocode}).  The {\tt RFtoIF} stage
modulates the input signal from RF to a frequency band around the
current IF frequency.  To support a change in the IF frequency when
frequency hopping occurs, the {\tt RftoIF} filter contains a {\tt
setFreq} method that is invoked via a message from the {\tt
CheckFreqHop} stage.  The message is sent from {\tt CheckFreqHop} with a
latency range of $4N$ to $6N$, which means that {\tt RFtoIF} must
deliver between $4N$ and $6N$ items using the old modulation scheme
before changing to the new frequency.

The optional {\tt Booster} stage provides amplification for weak
signals, but is usually turned off to conserve power.  The {\tt Booster}
is toggled by a re-initialization message from the {\tt CheckQuality}
stage, which estimates the signal quality by the shape of the frequency
spectrum.  If all the frequencies have similar amplitudes, {\tt
CheckQuality} assumes that the signal-to-noise ratio is low and sends a
message to activate the {\tt Booster}.  This message is sent using
best-effort delivery.

The {\tt FFT} stage converts the signal from the time domain to the
frequency domain; please refer to p. 796 of \cite{clr} for a diagram
of the parallel FFT algorithm.  The StreamIt implementation consists
of a bit-reversal permutation followed by a series of {\tt Butterfly}
stages.  The bit-reversal phase illustrates how data can be reshuffled
with just a few SplitJoin constructs (see
Figure~\ref{fig:bitreverseorder}).  The {\tt Butterfly} stage--which
is parameterized to allow for a compact representation of the
FFT--also employs SplitJoins to select groups of items for its
computation.  We believe that the StreamIt version of the FFT is clean
and intuitive, as the SplitJoin constructs expose the natural
parallelism of the algorithm.




