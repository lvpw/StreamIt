\subsection{Layout}
\label{sec:layout}

The goal of the layout phase is to assign nodes in the stream graph to
computation nodes in the target architecture while minimizing the
communication and synchronization present in the final layout.  The
layout assigns exactly one node in the stream graph to one computation
node in the target.  The layout phase assumes that the given stream
graph will fit onto the computation fabric of the target and that the
filters are load balanced.  These requirements are satisfied by the
partitioning phase described above.

The layout phase of the StreamIt compiler is implemented using
simulated annealing \cite{simanneal}.  We choose simulated annealing
for its combination of performance and flexibility.  To adapt the
layout phase for a given architecture, we supply the simulated
annealing algorithm with three architecture-specific parameters: a
cost function, a perturbation function, and the set of legal layouts.
To change the compiler to target one tiled architecture instead of
another, these parameters should require only minor modifications.

The cost function should accurately measure the added communication
and synchronization generated by mapping the stream graph to the
communication model of the target.  Due to the static qualities of
StreamIt, the compiler can provide the layout phase with exact
knowledge of the communication properties of the stream graph.  The
terms of the cost function can include the counts of how many items
travel over each channel during an execution of the steady state.
Furthermore, with knowledge of the routing algorithm, the cost
function can infer the intermediate hops for each channel.  For
architectures with non-uniform communication, the cost of certain hops
might be weighted more than others.  In general, the cost function can
be tailored to suit a given architecture.

Phase ordering between stream graph parttioning and layout can lead to
suboptimal results. We plan to develop a unified approach for
partitioning and layout in the future.
