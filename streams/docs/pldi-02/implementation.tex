\begin{table}[t]
\begin{center}
\scriptsize
\begin{tabular}{|l|r|r|r|} \hline
Benchmark & Lines & Filters & Graph Size\\
\hline \hline
PCA Demo & 484 & 5 & 7\\
\hline
FM Radio & 411 & 5 & 27\\
\hline
perftest4 & 347 & 5 & 20\\
\hline
GSM Decoder & 3050 & 11 & 21 \\
\hline
\end{tabular}
\vspace{-6pt}
\caption{\protect\small Application Characteristics}
\label{tab:benchmarks}
\vspace{-12pt}
\end{center}
\end{table}

\section{Implementation and Evaluation}
\label{sec:results}

We have implemented a fully-functional prototype of the StreamIt
optimizing compiler as an extension to the Kopi Java Compiler, a
component of the open-source Kopi Project \cite{kopi}.  Our compiler
generates C code that is compiled with a StreamIt runtime library to
produce the final executable.  We have also developed a library in
Java that allows StreamIt code to be executed as pure Java, thereby
providing a verification mechanism for the output of the compiler.

The compilation process for streaming programs contains many novel
aspects because the basic unit of computation is a stream rather than
a procedure.  In order to compile stream modules separately, we have
developed a runtime interface--analogous to that of a procedure call
for traditional codes--that specifies how one can interact with a
black box of streaming computation.  The stream interface contains
separate phases for initialization and steady-state execution; in the
execution phase, the interface includes a contract for input items,
output items, and possible message production and consumption.  The
interface relies on the \sdep~function to specify message timing in
terms of a stream's input tape.

We have evaluated our compiler with StreamIt versions of the following
applications: 1) A GSM Decoder, which takes GSM-encoded parameters as
inputs, and uses these to synthesize audible speech, 2) A system from
the Polymorphic Computing Architecture (PCA) \cite{pca} which
encapsulates the core functionality of modern radar, sonar, and
communications signal processors, 3) A software-based FM Radio with
equalizer, and 4) A performance test from the SpectrumWare system that
implements an Orthogonal Frequency Division Multiplexor (OFDM)
\cite{spectrumware}.  Table \ref{tab:benchmarks} gives characteristics
of the above applications including the number of filters implemented
and the size of the stream graph as coded.

\begin{table}[t]
\begin{center}
\scriptsize
\begin{tabular}{|l|r|r|r|r|} \hline
& \multicolumn{2}{|c|}{StreamIt} &  \multicolumn{2}{|c|}{Hand Coded}\\
\hline 
Benchmark & Baseline & Fusion & Spectra & C \\
\hline \hline
PCA Demo & 1.3 & - & 3.4 & N/A\\
\hline
FM Radio & 5.8 & 4.9 & 9.9 & N/A\\
\hline
perftest4 & 330 & - & 330 & N/A\\
\hline
GSM Decoder & 4.88 & - & N/A & .47\\
\hline
\end{tabular}
\vspace{-6pt}
\caption{\protect\small Performance Results (in $\mu$sec/output)}
\label{tab:performance}
\vspace{-21pt}
\end{center}
\end{table}

%% \subsection{GSM Decoder}

%%   The decoder portion of the StreamIt GSM Vocoder takes GSM encoded
%% parameters as inputs, and uses these to synthesize audible speech.  This
%% is accomplished by processing the parameters through four main filters.
%% The RPE decoder filter produces some "pink noise" that very loosely
%% estimates the speech waveform, using quantized bit sequences and a
%% maximum value parameter from the encoded input.  This "pink noise" is
%% fed to the Long Term Prediction portion, which applies long-term
%% characteristics to the sequence through a delay filter within a feedback
%% loop.  The resulting signal is then sent to the Short Term Synthesis
%% filter, which decodes high frequency voice characteristics from the
%% encoded parameters and applies these to the signal.  Finally, the
%% Post-processing filter identifies peaks in the signal to make it audible.

%% \subsection{PCA Demo}

%% This application is representative of the core functionality needed by
%% modern radar, sonar, and communications signal processors.

In the Table~\ref{tab:performance}, we evaluate the performance of our
compiler by comparing the StreamIt implementation against either the
SpectrumWare implementation or (in the case of GSM) a hand-optimized C
version.  SpectrumWare \cite{spectrumware}~is a high-performance
runtime library for streaming programs, implemented in C++.  The
StreamIt language offers a higher level of abstraction than
SpectrumWare (see Section \ref{sec:oo-rat}), and yet the StreamIt
compiler is able to beat the SpectrumWare performance by a factor of
two for the PCA Demo and FM Radio.

For the GSM application, the extensively hand-optimized C version
incorporates many transformations that rely on the high-level
knowledge of the algorithm, and the StreamIt performs an order of
magnitude slower.

The StreamIt compiler infrastructure is far from complete.  We are in
the process of discovering all the optimization possibilities in this
new domain.  Our code generation strategy currently has many
inefficiencies, and in the future we plan to generate optimized
assembly code by interfacing with a code generator.  We strongly
believe that we can improve the current performance by at least an
order of magnitude on uniprocessors, and we have yet to take advantage
of the inherent data and pipeline parallelism in StreamIt programs for
parallel execution.
