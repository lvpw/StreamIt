\newpage
\mbox{~} \\ ~ \vspace{-70pt} ~ \\
\section{Conclusions} \vspace{-1pt}
\label{sec:conclusions}

%% Many of the applications that will drive the next generation of
%% computing systems---digital video editing, computer vision, computer
%% graphics and animation---operate on image and video formats that are
%% universally stored in compressed data formats.  

In order to accelerate operations on compressible data, this paper
presents a general technique for translating stream programs into the
compressed domain.  Given a natural program that operates on
uncompressed data, our transformation outputs a program that directly
operates on the compressed data format.  We support lossless
compression formats based on LZ77; while this paper describes the
application to Apple Animation, elsewhere we describe the application
to Flic Video, Microsoft RLE, and Targa~\cite{techreport}.  In the
general case, the transformed program may need to partially decompress
the data to perform the computation, though this decompression is
minimized throughout the process and significant compression ratios
are preserved without resorting to an explicit re-compression step.

%% While we formulated our transformation in terms of the cyclo-static
%% dataflow model, the techniques can be applied within other functional
%% and general-purpose languages so long as the right information is
%% available and certain constraints are satisfied.  The transformation
%% relies on a regular pattern of data access; we use a streaming
%% abstraction, but structured iteration over arrays could also suffice.
%% We rely on static data rates in actors, which could also be expressed
%% as functions with a fixed number of arguments and return values.
%% Actors (functions) must be pure, without side effects or unresolvable
%% dependences on potentially mutable data.  While these properties are
%% intrinsic to a language such as StreamIt, they also come naturally in
%% most functional languages and may be adaptable to general-purpose
%% languages in the form of a runtime library with a restricted API.

We implemented some of our transformations in the StreamIt compiler
and demonstrated excellent speedups.  Across a suite of 12 videos in
Apple Animation format, computing directly on compressed data offers a
speedup roughly proportional to the compression ratio.  For pixel
transformations (brightness, contrast, inverse) speedups range from
2.5x to 471x, with a median of 17x; for video compositing operations
(overlays and mattes) speedups range from 1.1x to 32x, with a median
of 6.6x.  While previous researchers have used special-purpose
compressed processing techniques to obtain speedups on lossy,
DCT-based codecs, we are unaware of a comparable demonstration for
lossless video compression.  As digital films and animated features
have embraced lossless formats for the editing process, the speedups
obtained may have practical value.
\vspace{-1pt}
