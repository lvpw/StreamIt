\chapter{Optimization}

     There are multiple metrics used to analyze performance
of a computer program - speed (throughput, or outputs per second),
space, power consumption, etc. We focus on speed and attempt to
minimize the computation performed to produce each output.
Obviously, this type of optimization has positive effects on the
other parameters. However, we are mainly concerned with speed
because it is simple to track, and due to falling hardware costs,
is frequently a program's bottleneck.

    There are two types of optimizations we consider.  The first is to
remove extraneous state variables from the linear state-space
representation. This reduces the memory allocation for a program
and reduces the number of loads and stores executed, which are
typically time intensive operations. It also eliminates
computations that involve the removed states.  The second
optimization is to reduce the parametrization of a state-space
representation, by changing the representation to one with more
zero and one entries in its matrices. This directly eliminates
computations, since all multiplications by zero or one are not
processed by the replacement algorithm.

\section{State-Space Transformations}

    For any state-space equation pair, there are an infinite
number of transformations to an equivalent state-space system.
These transformations involve a change of basis of the state
vector $\vec{\mathbf{x}}$ to $\mathbf{T} \vec{\mathbf{x}}$, where
$\mathbf{T}$ is an invertible matrix. Consider the state-update
equation $\vec{\dot{\mathbf{x}}} = \mathbf{A} \vec{\mathbf{x}} +
\mathbf{B} \vec{\mathbf{u}}$. Multiplying the entire equation by
$\mathbf{T}$ yields:
\begin{eqnarray*}
\mathbf{T} \vec{\dot{\mathbf{x}}} = \mathbf{TA} \vec{\mathbf{x}} +
\mathbf{TB} \vec{\mathbf{u}}
\end{eqnarray*}

    Since $\mathbf{T}^{-1} \mathbf{T} = \mathbf{I}$, we can write:
\begin{eqnarray*}
\mathbf{T} \vec{\dot{\mathbf{x}}} & = & \mathbf{TA}
(\mathbf{T}^{-1} \mathbf{T}) \vec{\mathbf{x}} + \mathbf{TB}
\vec{\mathbf{u}} = \mathbf{TA}
\mathbf{T}^{-1} (\mathbf{T} \vec{\mathbf{x}}) + \mathbf{TB} \vec{\mathbf{u}} \\
\vec{\mathbf{y}} & = & \mathbf{C} (\mathbf{T}^{-1} \mathbf{T})
\vec{\mathbf{x}} + \mathbf{D} \vec{\mathbf{u}} = \mathbf{C}
\mathbf{T}^{-1} (\mathbf{T} \vec{\mathbf{x}}) + \mathbf{D}
\vec{\mathbf{u}}
\end{eqnarray*}

    Where we have introduced the output equation as well. Let
$\vec{\mathbf{z}} = \mathbf{T} \vec{\mathbf{x}}$.
$\vec{\mathbf{z}}$ is a new state vector related to the old state
vector $\vec{\mathbf{x}}$ by the change of basis $\mathbf{T}$.
Substituting into the equations above we get:
\begin{eqnarray*}
\vec{\dot{\mathbf{z}}} & = & \mathbf{TA} \mathbf{T}^{-1} \vec{\mathbf{z}} + \mathbf{TB} \vec{\mathbf{u}} \\
\vec{\mathbf{y}} & = & \mathbf{C} \mathbf{T}^{-1}\vec{\mathbf{z}}
+ \mathbf{D}\vec{\mathbf{u}}
\end{eqnarray*}

    These is precisely the original state-space equation pair,
with $\mathbf{A}$, $\mathbf{B}$, and $\mathbf{C}$ transformed to
$\mathbf{T} \mathbf{A} \mathbf{T}^{-1}$, $\mathbf{T} \mathbf{B}$,
and $\mathbf{C} \mathbf{T}^{-1}$, respectively.

    For a StreamIt state-space representation $\mathrm{R}$, we must
determine how the other values change. The initialization state
update equation is essentially the same as the regular state
update equation, so $\mathbf{A_{pre}}$ and $\mathbf{B_{pre}}$ are
transformed to $\mathbf{T} \mathbf{A_{pre}} \mathbf{T}^{-1}$ and
$\mathbf{T} \mathbf{B}$ respectively. Since the old state vector
$\vec{\mathbf{x}}$ is multiplied by $\mathbf{T}$, the old initial
state vector is multiplied by $\mathbf{T}$. The number of states,
inputs, and outputs is the same, so $s$, $o$, and $u$ are
unchanged.

\section{State Removal}

    There are two types of states that can be removed from a state-space system
without changing its behavior - unreachable and unobservable
states. Informally, unreachable states are unaffected by inputs
and unobservable states have no effect on outputs. More formally,
the set of states in a system can be divided into reachable and
unreachable states where:
\begin{enumerate}
\item The unreachable states are not updated by any of the
reachable states.

\item The unreachable states are not updated by any inputs.
\end{enumerate}

    In terms of the state-space equation pair, this means $\mathbf{A}[i,j] =
0, \mathbf{B}[i,k] = 0$ where $i$ is the row of an unreachable
state, $j$ is the column of a reachable state, and $k$ is any of
the inputs.
    If all the unreachable states are initially zero, they
remain zero because they are not updated by a non-zero value
(either a reachable state or an input). Therefore, all unreachable
states that are not initialized can be removed from a
representation, since they do not effect the reachable states or
the outputs.

    The set of states in a system can also be divided into
observable and unobservable states where:
\begin{enumerate}
\item The observable states are not updated by any of the
unobservable states.

\item The outputs do not depend on the unobservable states.
\end{enumerate}

    In terms of the state-space equation pair, this means $\mathbf{C}[i,j] =
0, \mathbf{D}[k,j] = 0$ where $j$ is the column of an observable
state, $i$ is the row of an unobservable state, and $k$ is any of
the outputs.
    The unobservable states are not used to update the observable
states and are not used to determine the outputs. Therefore, all
unobservable states can be removed from a representation
(regardless of their initial values).

    A simple algorithm to isolate the unreachable and unobservable
states in a system by use of transformations is explained in
\cite{Mayne}. The algorithm works as follows: Perform row
operations on the augmented matrix $\left [ \begin{array} {cc}
\mathbf{A} & \mathbf{B} \end{array} \right ]$ to put it into a
type of row-echelon form\footnote{A matrix is in standard
row-echelon form if the first non-zero entry in each row is a 1
(called the leading 1) and the leading 1 in a higher row is to the
left of the leading 1 in a lower row. For our type of row-echelon
form, the \emph{last} non-zero entry in each row is a 1 (call it
the ending 1) and the ending 1 in a higher row is to the left of
the ending 1 in a lower row.}, and perform the corresponding
inverse column operations on $\mathbf{A}$ and $\mathbf{C}$ to keep
the system equivalent to the original (Performing a row operation
on a matrix is equivalent to left multiplying it by some
invertible matrix, and performing a column operation on a matrix
is equivalent to right multiplying it by some invertible matrix).
Once the augmented matrix is in the desired form, row $i$ of the
combined matrix represents an unreachable state if there are no
non-zero entries past the $i^{th}$ column. For unobservable
states, the combined matrix $\left [ \begin{array} {cc}
\mathbf{A}^T & \mathbf{C}^T
\end{array} \right ]$ is operated on instead.

    Using this algorithm, we can find the entire set of unobservable
states and remove them all. The only exceptions are those
unobservable states that affect observable states in the
initialization matrix $\matrix{A_{pre}}$. If $j$ is the column of
an observable state then we must have $\mathbf{A_{pre}}[i,j] = 0$
for all values of $i$, where $i$ is the row of an observable
state. Otherwise, the unobservable state $j$ cannot be removed,
because it affects at least one observable state, and therefore
may affect the outputs.

    More care must be taken when removing unreachable states. If an
unreachable state has a non-zero starting value, or is affected by
the initialization matrices, it cannot be removed. In either of
these cases, the unreachable state may attain a non-zero value,
and therefore may have an affect on the reachable states and/or
outputs. Additionally, an unreachable state $x_1$ that is updated
by a different unreachable state $x_2$ that cannot be removed may
eventually have a non-zero value, even if it ($x_1$) is initially
zero. Therefore, the unreachable state $x_1$ cannot be removed as
well.

    The last case may cause problems when trying to remove
unreachable states. If an unreachable state $x_1$ is updated by
unreachable states $x_2$ and $x_3$, we must check if those states
can be removed before determining if state $x_1$ can be removed.
If one of those states, say $x_2$, depends on $x_1$, we must
determine if $x_1$ can be removed before determining whether $x_2$
can be removed - resulting in an impossible `loop-like'
determination. Clearly, a more robust approach is necessary.

    Suppose we have found the set of unreachable states and they
form the first $k$ states of the state vector (we can do both of
these steps by isolating the unreachable states, then moving them
to the top of the state vector if necessary). Consider the
sub-matrix $\mathbf{A}[1:k;1:k]$ consisting of the first k rows
and first k columns of $\mathbf{A}$. This sub-matrix represents
how the unreachable states are updated based on each other.
Suppose this sub-matrix is in upper-triangular form, which means
that all entries below the main diagonal are zero. We can remove
states in the following manner:
\begin{enumerate}
\item Check the states in reverse order, from state $k$ to state
$1$.

\item For the $i^{th}$ state, check whether the state has an
initial value, is updated by the initialization matrices, or
depends on a state with a higher index. If any of these are true,
we cannot remove the state. Otherwise, we can remove the state.
\end{enumerate}

    Since the unreachable state sub-matrix is in upper-triangular
form, all unreachable states can only have dependencies on states
with a higher index. Furthermore, since we are working from the
state with highest index first, at each step in the algorithm we
can immediately determine whether or not a given state is
removable. Therefore we have found our robust approach to remove
unreachable states. What remains to be done is transforming the
sub-matrix to upper-triangular form.

    The QR algorithm, described in \cite{Trefethen}, is an iterative method of
converting any square matrix $\mathbf{P}$ to upper-triangular
form. The algorithm is essentially the following two step
procedure, applied as many times as necessary.
\begin{enumerate}
\item $\mathbf{Q} \mathbf{R} = \mathbf{P}$   (QR factorization of
P)

\item $\mathbf{P} = \mathbf{R} \mathbf{Q}$
\end{enumerate}

    The QR factorization of a matrix $\mathbf{P}$ factors
$\mathbf{P}$ into the product of an orthogonal matrix
$\mathbf{Q}$\footnote{An orthogonal matrix has the property that
its transpose is equal to its inverse} and an upper-triangular
matrix $\mathbf{R}$. Since $\mathbf{R} = \mathbf{Q}^{-1}
\mathbf{P}$, the QR algorithm is repeatedly transforming
$\mathbf{P}$ to $\mathbf{Q}^{-1} \mathbf{P} \mathbf{Q}$.

    Since $\mathbf{Q}$ is invertible, we can apply this
transformation to the unreachable state sub-matrix, where the
transformation matrix $\mathbf{T}$ is $\mathbf{Q}^{-1}$. Since we
want to keep the other states unchanged, the full transformation
matrix applied to $\mathbf{A}$, $\mathbf{B}$, $\mathbf{C}$ is
$\mathbf{T} = \left [ \begin{array} {cc} \mathbf{Q}^{-1} &
\mathbf{0} \\ \mathbf{0} & \mathbf{I} \end{array} \right ]$

\section{Putting Inputs into States}

    So far we have considered optimizations that affect $\mathbf{A}$,
$\mathbf{B}$, and $\mathbf{C}$. Since the optimizations are
entirely the result of state transformations, they do not affect
$\mathbf{D}$, which is independent of the choice of state-space
basis. By storing every input as a state, however, all the entries
of $\mathbf{D}$ are moved into $\mathbf{A}$ and can then be
changed by state optimizations.

    We have already discussed how to store inputs as states. When
every input is stored as a state, we find the new state-equation
pair is:
\begin{eqnarray*}
\left [ \begin{array} {c} \vec{\dot{\mathbf{x}}} \\
\vec{\dot{\mathbf{x_{inputs}}}} \end{array} \right ] & = & \left [
\begin{array} {cc} \mathbf{A} & \mathbf{B} \\ \mathbf{0} &
\mathbf{0} \end{array} \right ] \left [ \begin{array} {c}
\vec{\mathbf{x}} \\ \vec{\mathbf{x_{inputs}}} \end{array} \right ]
+ \left [ \begin{array} {c} \mathbf{0} \\ \mathbf{I} \end{array}
\right ] \vec{\mathbf{u}} \\
\vec{\mathbf{y}} & = & \left [ \begin{array} {cc} \mathbf{C} &
\mathbf{D} \end{array} \right ] \left [ \begin{array} {c}
\vec{\mathbf{x}} \\ \vec{\mathbf{x_{inputs}}} \end{array} \right ]
+ \mathbf{0} \vec{\mathbf{u}}
\end{eqnarray*}

    These states should be added before state-removal is
performed. It may seem counter-intuitive that we first add states,
then seek to remove them. However, the added states represent
computations involving $\mathbf{D}$, which were not considered
before. Removing some of these states results in reducing
computations involving $\mathbf{D}$.

\section{Parameter Reduction}

    After removing as many states as possible, including input
states, we want to change the state-space system to one with the
fewest number of non-zero, non-one entries (termed parameters). If
$\mathbf{A}$, $\mathbf{B}$, and $\mathbf{C}$ are completely
filled, there are $s*(s+o+u)$ parameters. Ackermann and Bucy
\cite{Ackermann/Bucy} show a general form for $\mathbf{A}$ and
$\mathbf{C}$ ($\mathbf{B}$ can be filled with parameters) to have
at most $s*(o+u)$ parameters, assuming there are no unobservable
or unreachable states. They derive this form using system impulse
responses. We will achieve this same form using row operations on
the augmented matrix $\left [
\begin{array} {cc} \mathbf{A}^T & \mathbf{C}^T \end{array} \right
]$. The form we want is:
\begin{eqnarray*}
\mathbf{A}^T & = & \left [ \begin{array} {ccccc} \mathbf{L_1} &
\mathbf{A_{12}} & \mathbf{A_{13}} & ... & \mathbf{A_{1u}} \\
\mathbf{0} & \mathbf{L_2} & \mathbf{A_{23}} & ... &
\mathbf{A_{2u}} \\ \mathbf{0} & \mathbf{0} & \mathbf{L_3} & ... &
\mathbf{A_{3u}} \\ ... & ... & ... & ... & ... \\ \mathbf{0} &
\mathbf{0} & \mathbf{0} & ... & \mathbf{L_u} \end{array} \right ] \\
\mathbf{C}^T & = & \left [ \begin{array} {ccccc} 1 & 0 & 0 & ... &
0 \\ 0 & 0 & 0 & ... & 0 \\ ... & ... & ... & ... & ... \\ 0 & 1 &
0 & ... & 0 \\ 0 & 0 & 0 & 0 & 0 \\ ... & ... & ... & ... & ... \\
0 & 0 & 0 & ... & 1 \end{array} \right ]
\end{eqnarray*}

    The matrices $\mathbf{L_i}$ are rectangular, and the matrices
$\mathbf{A_{ij}}$ are square, but do not necessarily have the same
dimensions as each other. These matrices have the form:
\begin{eqnarray*}
\mathbf{L_i} & = & \left [ \begin{array} {ccccc} 0 & 0 & ...
& 0 & * \\ 1 & 0 & ... & 0 & * \\ 0 & 1 & ... & 0 & * \\
... & ... & ... & ... & ... \\ 0 & 0 & ... & 1 & * \end{array}
\right ] \\
\mathbf{A_{ij}} & = & \left [ \begin{array} {cccc} 0 & 0 & ... & *
\\ ... & ... & ... & ... \\ 0 & 0 & ... & * \end{array} \right ]
\end{eqnarray*}

    The entries marked with a * are the parameters of the system.
This is known as the observable canonical form of the system. In
contrast, the reachable canonical form defines $\mathbf{A}$ and
$\mathbf{B}$ instead of $\mathbf{A}^T$ and $\mathbf{C}$, and
$\mathbf{C}$ may be filled with parameters instead of
$\mathbf{B}$.

    We present a simple algorithm, in pseudocode to attain the form
above. We do not include the necessary inverse column operations
that must go with all row operations.

\begin{scriptsize}
\begin{verbatim}
Reduce Parameters {
  currRow = 0; colA = 0; colC = 0;

  while(currRow < totalRows) {

   -find a non-zero entry in column colC at or below row currRow of C{transpose}, and swap it with the
    entry in row currRow;
   -set C{transpose}[currRow,colC] = 1 by scaling the row appropriately;
    make all entries above and below it zero by adding appropriate multiple of row currRow to other rows;

    currRow = currRow + 1;
    colC = colC + 1;

    do {
     -find a non-zero entry in column colA at or below row currRow of A{transpose}, and swap it with the
      entry in row currRow;
     -set A{transpose}[currRow,colA] = 1 by scaling the row appropriately;
      make all entries below it zero by adding appropriate multiple of row currRow to other rows;

      currRow = currRow + 1;
      colA = colA + 1;
    } while a non-zero entry in column colA is found

    colA = colA + 1;
  }
}
\end{verbatim}
\end{scriptsize}

    It is possible that one type of form has fewer parameters than the
other. Therefore, we perform the above algorithm on $\left [
\begin{array} {cc} \mathbf{A}^T & \mathbf{C}^T
\end{array} \right ]$ as noted to produce the observable form, and on $\left [
\begin{array} {cc} \mathbf{A} & \mathbf{B} \end{array} \right
]$ to produce the reachable form, and check which one has fewer
parameters.

\section{Staged Execution}

    Using input state variables corresponds to executing a state-space
block in two stages:
\begin{enumerate}
\item Put inputs into input state variables.

\item Execute the original block, using input states instead of
actual inputs.
\end{enumerate}

    We can add additional stages by having multiple sets of input
states - $\vec{\mathbf{x_{inputs1}}}$,
$\vec{\mathbf{x_{inputs2}}}$, etc. The first set gets saved in the
second set, the second set gets saved in the third set, etc.
Suppose there are $k$ input sets. We can write our state-space
equation pair as follows:
\begin{eqnarray*}
\left [ \begin{array} {c} \vec{\dot{\mathbf{x}}} \\ \vec{\dot{\mathbf{x_{inputsk}}}} \\
... \\ \vec{\dot{\mathbf{x_{inputs2}}}} \\
\vec{\dot{\mathbf{x_{inputs1}}}}
\end{array} \right ] & = & \left [ \begin{array} {ccccc}
\mathbf{A} & \mathbf{B} & \mathbf{0} & ... &
\mathbf{0} \\ \mathbf{0} & \mathbf{0} & \mathbf{I} & ... & \mathbf{0} \\
... & ... & ... & ... & ... \\ \mathbf{0} & \mathbf{0} &
\mathbf{0} & ... & \mathbf{I} \\ \mathbf{0} & \mathbf{0} &
\mathbf{0} & ... & \mathbf{0} \end{array} \right ] \left [
\begin{array} {c} \vec{\mathbf{x}} \\ \vec{\mathbf{x_{inputsk}}} \\ ...
\\ \vec{\mathbf{x_{inputs2}}} \\ \vec{\mathbf{x_{inputs1}}} \end{array} \right ]
+ \left [ \begin{array} {c} \mathbf{0} \\ \mathbf{0} \\ ... \\
\mathbf{0} \\ \mathbf{I} \end{array} \right ]
\vec{\mathbf{u}} \\
\vec{\mathbf{y}} & = & \left [ \begin{array} {ccccc} \mathbf{C} &
\mathbf{D} & ... & \mathbf{0} & \mathbf{0} \end{array} \right ]
\left [ \begin{array} {c} \vec{\mathbf{x}}
\\ \vec{\mathbf{x_{inputsk}}} \\ ... \\ \vec{\mathbf{x_{inputs2}}}
\\ \vec{\mathbf{x_{inputs1}}} \end{array} \right ] + \mathbf{0} \vec{\mathbf{u}}
\end{eqnarray*}

    By itself, executing the work of a filter in stages does not
result in any gain in performance. However, minimally
parameterizing the resulting system may be more productive than
minimally parameterizing the one or two execution stage system.
The canonical forms of the previous section do not in general
minimally parameterize the system; hence evaluating staged
execution remains an area of future research.
