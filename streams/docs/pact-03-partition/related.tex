\section{Related Work}

%\begin{verbatim}

%Compile-Time Partitioning and Scheduling of Parallel Programs
%http://delivery.acm.org/10.1145/20000/13313/p17-sarkar.pdf?key1=13313&key2=0398809401&coll=Portal&dl=Portal&CFID=9316306&CFTOKEN=51649022

%Partitioning Parallel Programs for Macro Dataflow (somewhat subsumed by above)
%http://delivery.acm.org/10.1145/320000/319863/p202-sarkar.pdf?key1=319863&key2=5697809401&coll=portal&dl=ACM&CFID=9315645&CFTOKEN=28761144

%Efficient load balancing for wide-area divide-and-conquer
%applications, PPoPP 1999.

%Look at topic 3 in load balancing from previous years of europar:
%http://link.springer.de/link/service/series/0558/tocs/t2400.htm

%\end{verbatim}

The idea of structuring a language hierarchically to aid analysis is
not a new concept. IF1 is an example of a language that is based
heavily on structure~\cite{Sked}. At the lowest level, IF1 has around
50 different types of primitive nodes that can be combined together to
form a hierachy of acyclic dataflow graphs. The compound nodes Select,
TagCase, ForAll, While, and Until link together subgraphs. Taking
advantage of this hierarchal structure, algorithms have been developed
that effectively partition IF1 programs with low order polynomial
running times~\cite{Sarker}. The StreamIt language is designed to be a
language that is both easy to analyze and lends itself well to
programming within the streaming domain. The parameterizable nature of
stream graphs allow the compiler to immediately detect most symmetries
and cut down the amount of redundant analysis.

Precise communication estimation is becoming increasingly important to
effectively compile for next generation architectures. Sophisticated
algorithms have been developed that can handle simple topologies such
clique, uniform, and ring structures~\cite{Tao}. However, more complex
topologies that mimize wire delays are becoming more prevalent.
Simple communication estimation functions such as wormhole routing
breaks down in these topologies. Dynamic programming was designed to
handle two dimension tile based architectures well. The techniques
used in dynamic programming can potentially generalize to more complex
topologies as well.

Code duplication can be an effective way of reducing the bottleneck
of an application. Most existing partitioning algorithms either ignore
the possibility of code duplication or only duplicate code when it is
absolutely clear it would help. The dynamic partitioner is intelligent
enough to duplicate stateless filters if it reduces bottleneck
of the program.