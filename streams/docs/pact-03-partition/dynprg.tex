\section{Partitioning Algorithm}

In this section we describe our partitioning algorithm, which operates
on an abstract representation of the stream graph and is well-suited
for many interesting problems.

\subsection{Problem Definition}

In this paper we focus on a static load-balancing problem, which we
define as follows.  We are given the following:
\begin{itemize}

\item A StreamIt stream graph $\mt{str}$ in which each filter $filter$
has a certain amount of steady-state $\mt{work}$ and can be run in
data parallel (to an arbitrary degree) if $\mt{isFissable(f)}$ is
true.

\item An integer $N$ denoting the number of available partitions.

\end{itemize}

The goal is to produce a mapping $\mt{map}$ from nodes to partitions
that minimizes the cost of the bottleneck node {\it while maintaining
a structured stream graph}.  Formally, this could be posed as:

\begin{center}
  \framebox{
  \begin{minipage}{3in}
    {\bf Minimize} $MAX_{i}  (\sum_{d~s.t.~map(d)=i\}} d.work)$

    ~~~Subject to:
    \begin{enumerate}
      
    \item Every node is assigned to a legal partition:
      \[
      \forall d \in s:~~ 0 < \mt{map}(d) < n
      \]
      
    \item There exists a structured stream graph with the same
    semantics as $\mt{str}$ in which there exists a container that
    exactly corresponds to each partition.  That is:
      \[
      \begin{array}{c}
      \forall d \in s,~\exists s' \in s,~\forall d' \in s: \\ ~ \\
      \left( \mt{map}(d')==map(d) \implies s' \in s' \right)
      \end{array}
      \]
    \end{enumerate}
  \end{minipage}}
\end{center}

Of course, it is difficult to infer whether condition (2) holds above,
as there are a large (if not infinite) number of possible
configurations of a given stream graph.  Thus, practically speaking,
the algorithm must also output a sequence of transformations on the
original graph that will yield a structured stream graph with the
desired partitioning.

\subsection{Dynamic Programming Solution}

The main insight behind our solution is this: most of the useful and
efficient partitionings of a {\it structured} stream graph {\tt s}
follow from either a vertical or a horizontal division of {\tt s}.  In
other words, a large percentage of the structured configurations for a
given stream program can be represented as a series of straight,
rectangular divisions the toplevel node.  And because the rectangular
sub-sections are overlapping, we can search this space very
efficiently by memoizing values that we have already seen, thereby
utilizing dynamic programming.  The details of the algorithm follow
below.

The partitioning algorithm is given in Figure~\ref{code:partition}.  The
stream representation is simplified over StreamIt: there are only two
types of structures, a container and a leaf node.  Each leaf node {\tt
d} represents a filter, with a {\tt d.work} field for the steady-state
work and an {\tt isFissable} function to detect data parallelism.
Each stream container {\tt s} is represented as a ragged array, with a
fixed height {\tt s.height} and a varying width; the width at row {\tt
y} is given by {\tt s.width}[y].  Intuitively, a stream container
represents a pipeline that contains splitjoins of varying widths; in
this representation, the children of the splitjoins are viewed as
children of a two-dimensional ragged pipeline, so that the partitioner
can cut across the initial splitjoin boundaries.

Like a standard dynamic programming algorithm, the partitioner has two
phases.  The first phase searches the space of configurations and
determines the best available cost, while the second phase descends
again through the space and traces back along the optimal cost to
construct the optimal partitions.  Shown in Figure~\ref{code:partition} is
the first phase of the algorithm, which is the most important.

For each stream {\tt s} in the graph, the search finds the minimum
cost if {\tt s} were assigned to {\tt n} tiles, where {\tt n} ranges
from {\tt 1} to {\tt N}, the maximum number of tiles available.  In
the base case, {\tt s} is a node, and its cost is given by {\tt
getNodeCost}, which simply parallelizes {\tt s} as much as
possible\footnote{In the actual implementation, we maintain an upper
limit on the parallelization factor, as communication costs quickly
outweigh the benefits of distributing the computation.}.  Otherwise,
{\tt s} is a container, and optimal cost is exhibited under some
partitioning of the children (see {\tt getContCost}.)

This is where the algorithm's assumption of rectangular partitions
makes the problem tractable.  If it is looking for the optimal cost of
{\tt s} when allocated to {\tt n} tiles, then it considers dividing
the problem in half, in all possible ways: splitting vertically at
every possible {\tt x} coordinate, splitting horizontally at every
every possible {\tt y} coordinate, and, further, allocating any
division of the {\tt n} tiles to the two children in every case.
These constitute the main triply-nested loops in the {\tt getContCost}
function.

There are two subtleties to the algorithm.  First is a fundamental
complexity that is added by the fact that each container is a ragged
array rather than just a rectangle.  Note that even though the array
is ragged, the algorithm is only undertaking to make {\it rectangular}
partitions (because the implementation complexity of ragged partitions
is too high.)  Thus, each time it recurses, it must check that there
is still a contiguous rectangular region of child streams.  This is
what the {\tt canCutVertically} function is for (see
Figure~\ref{code:partition}).

The second subtlety is specific to our implementation of the StreamIt
compiler.  It so happens that we can collapse all neighboring joiner
nodes into a single processing tile, but disconnected joiners require
their own tile.  Thus, the {\tt needsJoiner} function calculates
whether or not a joiner is needed for a stream (based on the search
parameters, and what it can infer about its parents).  If the stream
needs a joiner, then it subtracts one from the tile count {\tt n} that
it is searching for.  While this is an implementation-specific detail,
it illustrates that diverse (and sometimes seemingly non-hierarchical)
constraints can be straightforward to integrate into the algorithm.

\subsection{Taking Advantage of Symmetry}

The rectangular approach of the dynamic programming partitioner
reveals many overlapping sub-problems beyond the memoized cost of the
configurations.  In particular, it is a common pattern in stream
programs to have a single filter repeated several times, either in a
splitjoin (e.g., filterbanks, sensor arrays, cell phone servers) or in
a pipeline (e.g., fine-grained FIR filters, network routers, graphics
shaders).  

We detect structurally identical filters based on a hash value related
to the similarity we want to exploit.  For instance, we have
implemented the following optimizations:
\begin{enumerate}

\item {\bf Amortized work estimation.}  Typically, we infer the work
of each filter as a compile-time estimate of the average-case
performance of its work work function.  However, we also support
profile-guided feedback where the work function is timed on native
hardware.  Though too expensive to support across a graph with
hundreds of elements, a hash of the filter's control flow and
arithmetic operations reveals uniform structures that make this
operation vastly cheaper.  For instance, the expanded version of our
Radar application contains 52 filters, and though each has its own
parameters, all shared one of 7 structures, allowing us to get the
full benefit of profiling at only a fraction of the cost.

\item {\bf Aliasing of children and memo tables.}  Similarly, for
adjacent regions of stream containers that all share the same pattern
of work estimate, the entries of the memo table can be aliased to save
on computations.  Further, children themselves can be aliased when
they are structurally equivalent.  Almost all of our benchmarks
benefit from this optimization.

\item {\bf Automatically computing the best partitions.}  In certain
cases, such as a wide splitjoin filled with equally costly filters, it
is possible to calculate the best partition in a closed form without
recursing through the hierarchy.
\end{enumerate}

We believe that these are strong benefits of having a structured
stream programming model, as well as using a partitioner that
understands the symmetry encapsulated by the programming model, as
well as the application class itself.

\subsection{Interfacing with the Partitioner}

Our description of the partitioner thus far has left two important
questions unanswered.  First, how does one convert from a StreamIt
graph to the ragged array structure described above?  Generally
speaking, the translation should maximize the size of each array, as
this gives the partitioner more freedom (as less choices are
sequentialized by artificial boundaries) and therefore it chooses from
a broader array of configurations.  Second, how does the algorithm
express the series of transformations that need to be made on the
original stream graph before it will embody a given partitioning?

Both of these questions require a general and flexible means of
refactoring structured stream programs.  These are the techniques we
present in the next section; they are very important to the efficacy
of the partitioning algorithm.
