\Section{Concluding Remarks}
\vspace{-11pt}

As computer architectures change from the traditional monolithic
processors, to scalable wire-exposed and multicore processors, there
is a greater need for portable applications that expose parallelism
and communication to enable efficient and high performance
executions---while also boosting programmer productivity. StreamIt is
a programming language and a compilation infrastructure specifically
engineered to naturally expose and leverage stream abstractions that
are embodied in modern streaming applications. We have used StreamIt
to implement DSP codes (e.g., software radio, beamforming), image and
video codecs (e.g., MPEG-2 encoding and decoding), encryption
algorithms (e.g., DES and Serpent), and many other applications. The
language, compiler, and applications are available for download from
the project web page at http://cag.csail.mit.edu/streamit.

The goal of the StreamIt project is to boost productivity for
streaming application developers such that they focus on algorithmic
innovation rather than on performance tuning. The ability to leverage
domain specific language constructs affords optimization opportunities
that can deliver high performance from high levels of abstraction. 

We believe emerging languages such as X10 can provide a framework to
implement domain specific abstractions in a general purpose
programming model. We have designed and implemented a prototype bridge
to run StreamIt code as part of the X10 virtual machine. As a result,
application developers can use the streaming abstractions for the
computation that fit that model of computation, while different
abstractions can be used to describe other aspects of the computation.
A part of our ongoing work is concerned with evaluating the productivity
and performance merit of native interfaces to and from StreamIt codes.