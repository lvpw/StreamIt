\documentclass[times,10pt]{tr}
\usepackage{fullpage}
\usepackage{epsfig}
\usepackage{pslatex} 
\usepackage{subfigure} 
\usepackage{epsfig}
\usepackage{nopageno}

%%%%%%%%%%%%%%%%%%%%%%%%%%%%%%%%%%%%%%%%%%%%%%%%%%%%%%%%%%%%%%%%%%%%%%%%%%%
%%%%%%%%%%%%%%%%%%%%%%%%%%%%%%%%%%%%%%%%%%%%%%%%%%%%%%%%%%%%%%%%%%%%%%%%%%%
%%%%%%%%%%%%%%%%%%%%%%%%%%%%%%%%%%%%%%%%%%%%%%%%%%%%%%%%%%%%%%%%%%%%%%%%%%%

\title {\bf High-Productivity Stream Programming For High-Performance Systems}

\author{
  Rodric Rabbah, Bill Thies, Michael Gordon, Janis Sermulins, and Saman Amarasinghe\\
  Computer Science and Artificial Intelligence Laboratory\\
  Massachusetts Institute of Technology\\
  \{rabbah, thies, mgordon, janiss, saman\}@csail.mit.edu
}

\begin{document}
\maketitle

%%%%%%%%%%%%%%%%%%%%%%%%%%%%%%%%%%%%%%%%%%%%%%%%%%%%%%%%%%%%%%%%%%%%%%%%%%%
%%%%%%%%%%%%%%%%%%%%%%%%%%%%%%%%%%%%%%%%%%%%%%%%%%%%%%%%%%%%%%%%%%%%%%%%%%%
%%%%%%%%%%%%%%%%%%%%%%%%%%%%%%%%%%%%%%%%%%%%%%%%%%%%%%%%%%%%%%%%%%%%%%%%%%%

Applications that are structured around some notion of a ``stream''
are increasingly prevalent to common computing practices, and there is
evidence that streaming media applications already consume a
substantial fraction of the computation cycles on consumer
machines~\cite{rix98}. Furthermore, stream processing---of 
voice and video data---is central to a plethora of embedded
systems, including hand-held computers, cell phones, and DSPs. The
stream abstraction is also fundamental to high-performance systems
such as intelligent software routers, cell phone base stations, and
HDTV editing consoles.

Despite the prevalence of these applications, there is surprisingly
little language and compiler support for practical, large-scale stream
programming. Of course, the notion of a stream as a programming
abstraction was established decades ago~\cite{SICP}, and a number of
special-purpose stream languages exist today (see \cite{survey97} for
a review). Many of these languages and representations are elegant
and theoretically sound, but they are often too inflexible to support
straightforward development of modern stream applications, or their
implementations are too inefficient to use in practice. Consequently,
most programmers resort to general-purpose languages such as C or C++
to implement stream programs. Yet there are several reasons why
general-purpose languages are inadequate for stream programming. Most
notably, they do not provide a natural or intuitive representation of
streams, thereby reducing readability, robustness, and programmer
productivity. Moreover, because the widespread parallelism and
regular communication patterns of data streams are left implicit in
general-purpose languages, compilers are not stream-conscious and do
not perform stream-specific optimizations. As a result,
performance-critical loops are often hand-coded in a low-level
assembly language and must be re-implemented for each target
architecture. This practice is labor-intensive, error-prone, and very
costly.

The StreamIt language and compiler effort at MIT is geared toward
boosting programmer-productivity while concomitantly delivering
high-performance for a wide array of computing targets. The StreamIt
language features several novelties that  are essential for large
scale program development: the language is modular, parameterizable,
malleable and architecture independent. In addition, the language exposes the
widespread parallelism and communication patterns that are inherent in
many streaming programs, and as a result, the compiler can apply
aggressive optimizations that are infeasible to perform when using
conventional languages. The StreamIt compiler can deliver high
performance codes---with speedups ranging up to $4.5\times$---for a
wide array of computing targets, including embedded and desktop
uniprocessors (e.g., StrongARM, IA32, and IA64),  tiled and multicore
architectures (e.g., Raw~\cite{raw}), or grid computing 
systems (e.g., a cluster of workstations interconnected by a
high-speed local area network).

Our compilation infrastructure automates several domain-specific
optimizations that are well known in the DSP domain. In addition, the
compiler includes optimizations  for computation reordering, load
balancing, layout, and routing; all of which are significantly
important in high-performance parallel and distributed computing
systems. The compiler can also apply a series of cache-ware
optimizations to improve the performance of the memory system and
ameliorate the effects of the ``memory wall''. Furthermore, we have
developed a graphical integrated development environment~\cite{kkuo-pphec} that
provides an elaborate debugging framework to interpret and visually
represent streaming computation, including the flow of information in
parallel and distributed streaming programs.

We believe StreamIt makes a great stride toward  delivering the
promise of Moore's Law to the end-user. The StreamIt language,
compiler, and development environment empower software engineers and
application developers to productively program streaming system in
what is today  the near-exclusive domain of experts.

%%%%%%%%%%%%%%%%%%%%%%%%%%%%%%%%%%%%%%%%%%%%%%%%%%%%%%%%%%%%%%%%%%%%%%%%%%%
%%%%%%%%%%%%%%%%%%%%%%%%%%%%%%%%%%%%%%%%%%%%%%%%%%%%%%%%%%%%%%%%%%%%%%%%%%%
%%%%%%%%%%%%%%%%%%%%%%%%%%%%%%%%%%%%%%%%%%%%%%%%%%%%%%%%%%%%%%%%%%%%%%%%%%%
\vspace{+20pt}
\noindent {\bf Presentation Outline:} The   presentation   will
describe   the  StreamIt language and its salient
features~\cite{streamitcc,thies-ppopp-2005}. We will focus on the
hierarchical nature of the language, highlighting modularity,
malleability, and portability. In addition, the talk will provide an
overview of the StreamIt compiler infrastructure, which was released
publicly on our website (http://cag.csail.mit.edu/streamit).  We will
describe the compiler in the context of three research thrusts:
automating domain-specific DSP optimizations,  targeting distributed
communication-exposed architectures, and performing cache-ware
optimizations.

First, we will present a set of domain-specific optimizations for linear
sections of the stream graph~\cite{lamb-pldi-2003,sitij-thesis}.  A
computation is linear if each of its outputs can be represented as an
affine combination of its inputs (e.g., FIR filters, expanders,
compressors, FFTs).  The StreamIt compiler recognizes linear computation
using a simple dataflow analysis.  It then exploits the linear
properties to perform algebraic simplification and
to translate linear computations into the frequency domain (when
profitable).  These transformations yield an average speedup of $4.5\times$
on a Pentium~3.

Second, we will describe our backend support for tiled and multicore
processors, and distributed computing
platforms~\cite{streamit-asplos}. We use the MIT Raw architecture as
an evaluation vehicle for the former, and a cluster of Pentium~3
processors interconnected with a high-speed network as an evaluation
vehicle for the latter. To achieve good performance on these parallel
targets, the compiler includes phases for work estimation, load
balancing, layout, and communication scheduling.  The load balancing
stage utilizes a novel dynamic-programming algorithm that can be
extended to consider a range of hierarchical cost functions.  When
targeting a 16-tile Raw machine, the compiler achieves an average
speedup of $16\times$ compared to a 1-tile Raw machine, and an average
speedup of $9\times$ compared to a Pentium~3. The compiler yields similar
performance gains on the Pentium~3 cluster.

Third, we will present several cache aware optimizations to improve
instruction and data locality, and improve register allocation and
scheduling freedom~\cite{sermulins-lctes-2005}. The optimizations are
founded upon a simple and intuitive model that quantifies the temporal
locality of a streaming program. The cache aware optimizations in the
StreamIt compiler yield a 249\% average speedup (over unoptimized
code) for our streaming benchmark suite on a StrongARM~1110
processor. The optimizations also yield a 154\% speedup on a Pentium~3
and a 152\% speedup on an Itanium~2.

%%%%%%%%%%%%%%%%%%%%%%%%%%%%%%%%%%%%%%%%%%%%%%%%%%%%%%%%%%%%%%%%%%%%%%%%%%%
%%%%%%%%%%%%%%%%%%%%%%%%%%%%%%%%%%%%%%%%%%%%%%%%%%%%%%%%%%%%%%%%%%%%%%%%%%%
%%%%%%%%%%%%%%%%%%%%%%%%%%%%%%%%%%%%%%%%%%%%%%%%%%%%%%%%%%%%%%%%%%%%%%%%%%%

%%%%%%%%%%%%%%%%%%%%%%%%%%%%%%%%%%%%%%%%%%%%%%%%%%%%%%%%%%%%%%%%%%%%%%%%%%%
%%%%%%%%%%%%%%%%%%%%%%%%%%%%%%%%%%%%%%%%%%%%%%%%%%%%%%%%%%%%%%%%%%%%%%%%%%%
%%%%%%%%%%%%%%%%%%%%%%%%%%%%%%%%%%%%%%%%%%%%%%%%%%%%%%%%%%%%%%%%%%%%%%%%%%%


%%%%%%%%%%%%%%%%%%%%%%%%%%%%%%%%%%%%%%%%%%%%%%%%%%%%%%%%%%%%%%%%%%%%%%%%%%%
%%%%%%%%%%%%%%%%%%%%%%%%%%%%%%%%%%%%%%%%%%%%%%%%%%%%%%%%%%%%%%%%%%%%%%%%%%%
%%%%%%%%%%%%%%%%%%%%%%%%%%%%%%%%%%%%%%%%%%%%%%%%%%%%%%%%%%%%%%%%%%%%%%%%%%%

%\begin{small}
\bibliographystyle{abbrv}
\bibliography{references}
%\end{small}

%%%%%%%%%%%%%%%%%%%%%%%%%%%%%%%%%%%%%%%%%%%%%%%%%%%%%%%%%%%%%%%%%%%%%%%%%%%
%%%%%%%%%%%%%%%%%%%%%%%%%%%%%%%%%%%%%%%%%%%%%%%%%%%%%%%%%%%%%%%%%%%%%%%%%%%
%%%%%%%%%%%%%%%%%%%%%%%%%%%%%%%%%%%%%%%%%%%%%%%%%%%%%%%%%%%%%%%%%%%%%%%%%%%

\end{document}
