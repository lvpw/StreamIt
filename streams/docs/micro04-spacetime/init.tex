\section{Initialization Schedules}
\subsection{Peek Initialization Schedule}
The StreamIt compiler is responsible for the automatic scheduling of
the filters of the stream graph. In this context, we use the term
scheduling to refer to the calculation of a steady-state schedule for
a stream graph composed of StreamIt filters.  Calculating a
steady-state schedule is orthogonal to trace scheduling. The
steady-state schedule is calculated before entry to the space-time
backend.  Calculating a steady-state schedule for a StreamIt stream
graph is complicated by the presence of the {\tt peek} operation which
implies that some programs require a separate schedule for
initialization that is distinct from the steady-state schedule.  We
call this stage the {\it peek initialization} stage and its schedule
of execution, the {peek initialization schedule}.

A more detailed discussion of the peek initialization schedule is
beyond the scope of this paper, see \ref{streamitcc} for a more
detailed discussion.  But it suffices to know that there is
need for a peek initialization schedule because the steady-state schedule
must be periodic, that is it must preserve the number of live items on
each channel in the graph.  The peek initialization is needed if there
is a filter with $peek > pop$ because such a filter leaves $peek -
pop$ items on its input channel after every firing, and thus the
channel could never be emptied.  A separate schedule is needed to
place at least $peek - pop$ items in the input channel to such a filter.

The peek initialization schedule and the steady-state schedule are
both Single Appearance Schedules (SAS) meaning that each filter is
placed in a loop and appears exactly once in the schedule.  The number
of executions of a filter in a schedule is called its {\it
multiplicity} in later sections.  The natural execution order for both
the peek initialization schedule and the steady-state schedule is
given by a breadth-first traversal of the stream graph. Of course, the
trace scheduling phase has some freedom in ignoring the
information-flow dependencies of the trace graph when constructing a
trace schedule.  It is the responsibility of the {\it Prologue
Schedule} to ensure that the...

\subsection{Prologue Schedule}
