\section{The Raw Architecture}
\label{sec:raw}

The Raw Microprocessor \cite{raw10,raw} addresses the wire delay
problem \cite{raw13} by providing direct instruction set architecture
(ISA) analogs to three underlying physical resources of the processor:
gates, wires and pins. Because ISA primitives exist for these
resources, a compiler such as StreamIt has direct control over both
the computation and the communication of values between the functional
units of the microprocessor, as well as across the pins of the
processor.

The architecture exposes the gate resources as a scalable 2-D array of
identical, programmable tiles, that are connected to their immediate
neighbors by four on-chip networks.  Each network is 32-bit,
full-duplex, flow-controlled and point-to-point. On the edges of the
array, these networks are connected via logical channels \cite{raw11}
to the pins.  Thus, values routed through the networks off of the side
of the array appear on the pins, and values placed on the pins by
external devices (for example, wide-word A/Ds, DRAMS, video streams
and PCI-X buses) will appear on the networks.

\begin{figure}[t]
\centering
\psfig{figure=raw-diagram.eps,width=3.2in}
\caption{\protect\small Block diagram of the Raw architecture.
\protect\label{fig:raw-diagram}}
\vspace{-12pt}
\end{figure}

Each of the tiles contains a compute processor, some memory and two
types of routers--one static, one dynamic--that control the flow of
data over the networks as well as into the compute processor (see
Figure \ref{fig:raw-diagram}). The compute processor interfaces to the
network through a bypassed, register-mapped interface \cite{raw10}
that allows instructions to use the networks and the register files
interchangeably. In other words, a single instruction can read up to
two values from the networks, compute on them, and send the result out
onto the networks, with no penalty.  Reads and writes in this fashion
are blocking and flow-controlled, which allows for the computation to
remain unperturbed by unpredictable timing variations such as cache
misses and interrupts.

Each tile's static router has a virtualized instruction memory to
control the crossbars of the two static networks. Collectively, the
static routers can reconfigure the communication pattern across these
networks every cycle.  The instruction set of the static router is
encoded as a 64-bit VLIW word that includes basic instructions
(conditional branch with/without decrement, move, and nop) that
operate on values from the network or from the local 4-element
register file. Each instruction also has 13 fields that specify the
connections between each output of the two crossbars and the network
input FIFOs, which store values that have arrived from neighboring
tiles or the local compute processor.  The input and output
possibilities for each crossbar are: North, East, South, West,
Processor, to the other crossbar, and into the static router. The
FIFOs are typically four or eight elements large.

To route a word from one tile to another, the compiler inserts a route
instruction on every intermediate static router \cite{lee98spacetime}.
Because the routers are pipelined and compile-time scheduled, they can
deliver a value from the ALU of one tile to the ALU of a neighboring
tile in 3 cycles, or more generally, 2+N cycles for an inter-tile
distance of N hops.

{\bf Add a description of Raw off-chip I/O.}

The results of this paper were generated using btl, a cycle-accurate
simulator that models arrays of Raw tiles identical to those in the
.15 micron 16-tile Raw prototype ASIC chip.  With a target clock rate
of 450 MHz, the tile employs as compute processor an 8-stage, single
issue, in-order MIPS-style pipeline that has a 32 KB data cache, 32 KB
of instruction memory, and 64 KB of static router memory. All
functional units except the floating point and integer dividers are
fully pipelined. The mispredict penalty of the static branch predictor
is three cycles, as is the load latency. The compute processor's
pipelined single-precision FPU operations have a latency of 4 cycles,
and the integer multiplier has a latency of 2 cycles.

