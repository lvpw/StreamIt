\section{Code Generation}
\label{sec:codegen}
In this section we consider low-level code generation for the Raw
microprocessor. Each Raw tile is an amalgam of the filters that
execute on it. To 

As we step through the schedule for each stage of
execution, we append the communication and computation code to the
compute and switch processors considering the multiplicity of the
filter in each stage.

Talk about the need for an initialization stage...

* For the initialization and priming stage iteration over the stream
graph preserving the data-dependencies of the graph.

* For the steady-state: iterate over the schedule as calculated by the trace scheduling
algorithm of Section \ref{sec:scheduling}.

* static network!

\subsection{Intra-Trace Code Generation}
When we visit a trace in the schedule for each execution phase, we
iterate over the filters that compose the trace generating 
communication and computation code necessary to execute the trace.

\subsubsection{Compute Processor}
For linear traces, we generate parameterized, template assembly code
for both computation and communication as described in Section
\ref{sec:linear}.  As for non-linear traces we generate C code that is
compiled using Raw's GCC 3.3 port.  Each compute processor begins by
calling the {\tt init} function of all the filters mapped to it.
Next, it executes the {\tt work} of each filter in the sequence
determined by the traversal of the schedule of each execution phase.
The translation of the {\tt init} function, {\tt work} functions, and
any helper function is, for the large part, straightforward.  Most
StreamIt expressions have direct analogs in C except for the channel
expressions {\tt push()}, {\tt peek(index)} and {\tt pop()}.

One major benefit of StreamIt over other streaming languages is that
the StreamIt compiler is responsible for input buffer management.  The
programmer is free to call {\tt pop()} or {\tt peek(index)} without
having to worry about the structure of the buffer or updating the
buffer's index variable after each execution of the work function.
Currently, we classify filters as one of three types with respect to
buffer management: 
\begin{enumerate}
\item If a filter does not peek and there does not exist
control-flow dependent channel expressions, then a buffer is not needed
and {\tt pop()} statements are translated directly into static network
receives.  There exists other initialization considerations which are
beyond the scope of this paper.
\item If there exists peek statements, but $peek = pop$ then we can
use a simple linear buffer that is reset after each work function execution.
\item Otherwise, when $peek > pop$, we use a circular buffer to
account for the data remaining on the channel after the work function
invocation.
\end{enumerate}
Each {\tt push()} statement is translated directly into a static
network send.  

\subsubsection{Switch Processor}
The intra-trace computation code is rather simple.  As mentioned in
Section \ref{sec:traces}, individual traces are limited to pipelines,
thus we do not have to worry about data-reorganization within a trace.
While executing a trace a tile receives its data from the
neighboring tile mapped to its upstream filter and sends data to the
neighboring tile mapped to its downstream filter.  The end-points of a
trace send or receive their data to or from off-chip.  We route the
items off the side of the chip and generate the necessary port
multiplexing and/or DRAM instructions as described above.

