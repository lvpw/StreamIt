\vspace{-5pt}
\section{Related Work}
\label{sec:related}
\vspace{-5pt}

A large number  of programming languages have included  a concept of a
stream; see \cite{survey97} for a survey.  Those that are perhaps most
related  to StreamIt 1.0  are synchronous  dataflow languages  such as
LUSTRE~\cite{lustre}  and  ESTEREL~\cite{esterel92},  which require  a
fixed number of inputs to arrive simultaneously before firing a stream
node.  However,  most special-purpose stream languages  do not contain
features such as messaging and support for modular program development
that  are essential  for modern  stream applications.   Also,  most of
these  languages are so  abstract and  unstructured that  the compiler
cannot  perform  enough analysis  and  optimization  to  result in  an
efficient implementation.

At an abstract level, the stream  graphs of StreamIt share a number of
properties with the synchronous dataflow (SDF) domain as considered by
the Ptolemy project~\cite{ptolemyoverview}.  Each node in an SDF graph
produces and consumes a given number of items, and there can be delays
along the arcs between nodes  (corresponding loosely to items that are
peeked in  StreamIt).  As  in StreamIt, SDF  graphs are  guaranteed to
have  a static  schedule and  there are  a number  of  nice scheduling
results  incorporating  code  size and  execution  time~\cite{leesdf}.
However,  previous   results  on   SDF  scheduling  do   not  consider
constraints  imposed by  point-to-point messages,  and do  not include
StreamIt's level of programming language support.

The Imagine  architecture is  specifically designed for  the streaming
application domain~\cite{rixner98bandwidthefficient}.   It operates on
streams by  applying a computation  kernel to multiple data  items off
the stream register file.  The compute kernels are written in Kernel-C
while the applications stitching  the kernels are written in Stream-C.
Unlike StreamIt,  with Imagine  the user has  to manually  extract the
computation kernels  that fit  the machine resources  in order  to get
good   steady   state   performance   for   the   execution   of   the
kernel~\cite{kapasi:2001:ss}.
