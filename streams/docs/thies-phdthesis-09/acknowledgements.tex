\newpage
~ \vspace{-3.7\baselineskip}\\
\enlargethispage{0.3\baselineskip}
\section*{Acknowledgments}

I would like to start by expressing my deepest gratitude to my
advisor, colleague and friend, Saman Amarasinghe.
%Saman's committment to me has been downright scary.
%Simply put, Saman has changed my life.
%Simply put, Saman has been the mentor of a lifetime.
From 4am phone calls in Boston to weeks of one-on-one time in Sri
Lanka and India, Saman invested {\it unfathomable} time and energy
into my development as a researcher and as a person.  His extreme
creativity, energy, and optimism (not to mention mad PowerPoint
skills!) have been a constant source of inspiration, and whenever I am
at my best, it is usually because I am asking myself: {\it What would
Saman do}?  Saman offered unprecedented freedom for me to pursue
diverse interests in graduate school -- including weeks at a time
working with other groups -- and served as a fierce champion on my
behalf in every possible way.  I will forever treasure our deep sense
of shared purpose and can only aspire to impact others as much as he
has impacted me.

%Like a Papa Bear, any arguments between the two of us would be 
%completely dwarfed by the fierce battles he would wage on my behalf.

\vspace{-8pt}\paragraph*{Contributors to this dissertation} Many
people made direct contributions to the content of this dissertation.
The StreamIt project was a fundamentally collaborative undertaking,
involving the extended efforts of over 27 people.  I feel very lucky
to have been part of such an insightful, dedicated, and fun team.
Section~\ref{sec:streamit-project} provides a technical overview of
the entire project, including the division of labor.  In what follows
I am listing only a subset of each person's actual contributions.
Michael Gordon, my kindred Ph.D. student throughout the entire
StreamIt project, led the development of the parallelization
algorithms (summarized in Chapter~4), the Raw backend and countless
other aspects of the compiler.  Rodric Rabbah championed the project
in many capacities, including contributions to cache optimizations
(summarized in Chapter~4), teleport messaging (Chapter~3), the MPEG2
benchmarks, an Eclipse interface, and the Cell backend.  Michal
Karczmarek was instrumental in the original language design (Chapter
2) and teleport messaging, and also implemented the StreamIt scheduler
and runtime library.  David Maze, Jasper Lin, and Allyn Dimock made
sweeping contributions to the compiler infrastructure; I will forever
admire their skills and tenacity in making everything work.

Central to the StreamIt project is an exceptional array of
M.Eng. students, who I feel very privileged to have interacted with
over the years.  Andrew Lamb, Sitij Agrawal, and Janis Sermulins led
the respective development of linear optimizations, linear statespace
optimizations, and cache optimizations (all summarized in Chapter~4).
Janis also implemented the cluster backend, with support for teleport
messaging (providing results for Chapter~3).  Matthew Drake
implemented the MPEG2 codec in StreamIt, while Jiawen Chen implemented
a flexible graphics pipeline and Basier Aziz implemented mosaic
imaging.  Daviz Zhang developed a lightweight streaming layer for the
Cell processor; Kimberly Kuo developed an Eclipse user interface for
StreamIt; Juan Reyes developed a graphical editor for stream graphs;
and Jeremy Wong modeled the scalability of stream programs.  Kunal
Agrawal investigated bit-level optimizations in StreamIt.  Ceryen Tan
is improving StreamIt's multicore backend.

The StreamIt project also benefited from an outstanding set of
undergraduate researchers, who taught me many things.  Ali Meli, Chris
Leger, Satish Ramaswamy, Matt Brown, and Shirley Fung made important
contributions to the StreamIt benchmark suite (detailed in Chapter~2).
Steve Hall integrated compressed-domain transformations into the
StreamIt compiler (providing results for Chapter~5).  Qiuyuan Li
worked on a StreamIt backends for Cell, while Phil Sung targeted a
GPU.

%Qiuyuan Li worked on a StreamIt backend for the Cell processor, while
%Phil Sung worked on a backend for graphics processors.

Individuals from other research groups also impacted the StreamIt
project.  Members of the Raw group offered incredible support for our
experiments, including Anant Agarwal, Michael Taylor, Walter Lee,
Jason Miller, Ian Bratt, Jonathan Eastep, David Wentzlaff, Ben
Greenwald, Hank Hoffmann, Paul Johnson, Jason Kim, Jim Psota, Nathan
Schnidman, and Matthew Frank.
%
\newpage
\enlargethispage{0.3\baselineskip}
%
~ \vspace{-1.3\baselineskip}\\
\noindent Hank Hoffmann, Nathan Schnidman, and Stephanie Seneff also
provided valuable expertise on designing and parallelizing signal
processing applications.  External contributors to the StreamIt
benchmark suite include Ola Johnsson, Mani Narayanan, Magnus
Stenemo, Jinwoo Suh, Zain ul-Abdin, and Amy Williams.  Fabrice
Rastello offered key insights for improving our cache optimizations.
Weng-Fai Wong offered guidance on several projects during his visit
to the group.  StreamIt also benefited immensely from regular and
insightful conversations with stakeholders from industry, including
Peter Mattson, Richard Lethin, John Chapin, Vanu Bose, and Andy Ong.

Outside of the StreamIt project, additional individuals made direct
contributions to this dissertation.  In developing our tool for
extracting stream parallelism (Chapter~6), I am indebted to Vikram
Chandrasekhar for months of tenacious hacking and to Stephen McCamant
for help with Valgrind.  I thank Jason Ansel, Chen Ding, Ronny
Krashinsky, Viktor Kuncak, and Alex Salcianu, who provided valuable
feedback on manuscripts that were incorporated into this dissertation.
I am also grateful to Arvind and Srini Devadas for serving on my
committee on very short notice, and to Marek Olszewski for serving as
my remote agent of thesis submission!

\vspace{-8pt}\paragraph*{The rest of the story} Throughout my life,
I have been extremely fortunate to have had an amazing set of
mentors who invested a lot of themselves in my personal growth.  I
thank Thomas ``Doc'' Arnold for taking an interest in a nerdy high
school kid, and for setting him loose with chemistry equipment in a
Norwegian glacier valley -- a tactic which cemented my interest in
science, especially the kind you can do while remaining dry.
%a tactic which ignited not only my interest in science, but also in
%girls.
I thank Scott Camazine for taking a chance on a high school programmer
in my first taste of academic research, an enriching experience which
%was not only enriching and fun, but also 
opened up many doors for me in the future.  I thank Vanessa Colella
and Mitchel Resnick for making my first UROP experience a very
special one, as evidenced by my subsequent addiction to the UROP
program.  I thank Andrew Begel for teaching me many things, not
least of which is by demonstration of his staggering commitment,
capacity, and all-around coolness in mentoring undergraduates.  I'm
especially grateful to Brian Silverman, a mentor and valued friend
whose unique perspectives on everything from Life in StarLogo to
life on Mars have impacted me more than he might know.  I thank
Markus Zahn for excellent advice and guidance, both as my
undergraduate advisor and UROP supervisor.  Finally, I'm very
grateful to Kath Knobe, who provided unparalleled mentorship during
my summers at Compaq and stimulated my first interest in compilers
research.

Graduate school brought a new set of mentors.  I learned a great
deal from authoring papers or proposals with Anant Agarwal, Srini
Devadas, Fredo Durand, Michael Ernst, Todd Thorsen, and
Fr\'{e}d\'{e}ric Vivien, each of whom exemplifies the role of a
faculty member in nurturing student talent.  I am also very grateful
to Srini Devadas, Martin Rinard, Michael Ernst, and Arvind for being
especially accessible as counselors, showing interest in my work and
well-being even in spite of very busy schedules.  I could not have
imagined a more supportive environment for graduate study.

I thank Charles Leiserson and Piotr Indyk for teaching me about
teaching itself.  I will always remember riding the T with Charles
when a car full of Red Sox fans asked him what he does for a living.
Imagining the impressive spectrum of possible replies, I should not
have been surprised when Charles said simply, ``I teach''.  Nothing
could be more true, and I feel very privileged to have been a TA in
his class.

I'd like to thank my collaborators on projects other than StreamIt,
for enabling fulfilling and fun pursuits outside 
% the work described in 
of this dissertation.  In the microfluidics lab, I thank
J.P. Urbanski for many late nights ``chilling at the lab'', his
euphemism for a recurring process whereby he manufactures chips and
I destroy them.  His knowledge, determination, and overall good
nature are truly inspiring.  I also learned a great deal from David
Craig, Mats Cooper, Todd Thorsen, and Jeremy Gunawardena, 
%
\newpage
\enlargethispage{0.5\baselineskip}
%
~ \vspace{-1.5\baselineskip}\\
\noindent who were extremely supportive of our foray into
microfluidics.  I thank Nada Amin for her insights, skills, and
drive in developing our CAD tool, and for being an absolute pleasure
to work with.

%In the microfluidics lab, I learned an immense amount from
%J.P. Urbanski, microfluidic wizard extraordinaire whose work ethic
%is as intense as it is understated -- never again will I think
%lightly of a late night ``chilling at the lab''.

I'm very thankful to my collaborators in applying technology towards
problems in socio-economic development, from whom I've drawn much
support.  First and foremost is Manish Bhardwaj, whose rare
combination of brilliance, determination, and selflessness has been
a deep inspiration to me.  I also thank Emma Brunskill, who has been
a tremendous collaborator on many fronts, as well as
%whose independence and resourcefulness are always humbling.  
%I'm very grateful to Emma Brunskill, Somani Patnaik, 
Sara Cinnamon, Goutam Reddy, Somani Patnaik and Pallavi Kaushik for
being incredibly talented, dedicated, and fun teammates.
%I thank the Venerable Tenzin Priyadarshi and Scott Kennedy for
%valuable support and guidance.
%
%Further in the past, 
I am very grateful to Libby Levison for involving me in my first
project at the intersection of technology and development, without
which I might have gone down a very different path.  I also thank
Samidh Chakrabarti for being a great officemate and friend, and my
first peer with whom I could investigate this space together.

I am indebted to the many students and staff who worked with me on
the TEK project, including Marjorie Cheng, Tazeen Mahtab, Genevieve
Cuevas, Damon Berry, Saad Shakhshir, Janelle Prevost, Hongfei Tian,
Mark Halsey, and Libby Levison.  I also thank Pratik Kotkar,
Jonathan Birnbaum, and Matt Aasted for their work on the Audio Wiki.
I would not have been able to accomplish nearly as much without the
%continuous 
insights, dedication, and hard work of all these individuals.

% leaving out sri lanka guys, since it's not released:
% - Thayaparan Kailainathan
% - Mahendrakumar Senthivel
% - Thayarupan Rajendram
%
% leaving out some TEK authors who I don't even know:
% - Alexandro Artola
% - Binh D. Vo
% - Yuliya Litvak
% - Sheldon Chan
% - Sid Henderson

Graduate school would be nothing if not for paper deadlines, and I
feel very lucky to have been down in the trenches with such bright,
dependable, and entertaining co-authors.  Of people not already cited
as such, I thank Marten van Dijk, Blaise Gassend, Andrew Lee, Charles
W. O'Donnell, Kari Pulli, Christopher Rhodes, Jeffrey Sheldon, David
Wentzlaff, Amy Williams, and Matthias Zwicker for some of the best
end-to-end research experiences I could imagine.

Many people made the office a very special place to be.  Mary McDavitt
is an amazing force for good, serving as my rock and foundation
throughout many administrative hurricanes; I can't thank her enough
for all of her help, advice, and good cheer over the years.  I'm also
very grateful to Shireen Yadollahpour, Cornelia Colyer, and Jennifer
Tucker, whose helpfulness I will never forget.  Special thanks to
Michael Vezza, system administrator extraordinaire, for his extreme
patience and helpfulness in tending to my every question, and fixing
everything that I broke.

I thank all the talented members of the Commit group, and especially
the Ph.D. students and staff -- Jason Ansel, Derek Bruening, Vikram
Chandrasekhar, Gleb Chuvpilo, Allyn Dimock, Michael Gordon, David
Maze, Michal Karczmarek, Sam Larsen, Marek Olszewski, Diego Puppin,
Rodric Rabbah, Mark Stephenson, Jean Yang, and Qin Zhao.  On top of
tolerating {\it way} more than their fair share of StreamIt talks,
they offered the best meeting, eating, and traveling company ever.  I
especially thank Michael Gordon, my officemate and trusted friend, for
making 32-G890 one of my favorite places -- I'm really going to miss
our conversations (and productive silences!)

I'd like to extend special thanks to those who supported me in my job
search last spring.  I feel very grateful for the thoughtful counsel
of dozens of people on the interview trail, and especially to a few
individuals (you know who you are) who spent many hours talking to me
and advocating on my behalf.  This meant a great deal to me.  I also
thank Kentaro Toyama and others at MSR India for being very flexible
with my start date, as the submission of this thesis was gradually
postponed!

I am extremely fortunate to have had a wonderful support network to
sustain me throughout graduate school.  To the handful of close
friends who joined me for food, walks around town, or kept in touch
from a distance: thank you for seeing me through the thick and thin.
I'd like to especially call out to David Wentzlaff, Kunal Agrawal,
Michael Gordon and Cat Biddle, who held front-row season tickets to my
little world and made it so much better by virtue of being there.

Finally, I would like to thank my amazingly loving and supportive
parents, who have always been 100\% behind me no matter where I am
in life.  I dedicate this thesis to them.

%% \clearpage
%% ~ \\ \vspace{1.1in} ~ \\
%% \begin{center}
%% {\bf \large Relation to Prior Publications}
%% \end{center}
\section*{Relation to Prior Publications}
This dissertation alternately extends and summarizes prior
publications by the author.  Chapters 1 and 2 are significantly more
detailed than prior descriptions of the StreamIt
language~\cite{thies-cc02,thies-can02,amarasinghe-ijpp05} and include
an in-depth study of the StreamIt benchmark suite that has yet to be
published elsewhere.  Chapter~3 subsumes the prior description of
teleport messaging~\cite{thies-ppopp05}, including key changes to the
semantics and the first uniprocessor scheduling algorithm.  Chapter~4
is a condensed summary of prior
publications~\cite{gordon-asplos02,lamb-pldi03,agrawal-cases05,sermulins-lctes05,gordon-asplos06},
though with new text that often improves the exposition.  Chapter~5
subsumes the prior report on compressed-domain
processing~\cite{thies07compression}, offering enhanced functionality,
performance, and readability.  Chapter~6 is very similar to a recent
publication~\cite{thies-micro07}.  Some aspects of the author's work
on StreamIt are not discussed in this
dissertation~\cite{karczmarek-lctes03,chen-graphics05}.

Independent publications by other members of the StreamIt group are
not covered in this
dissertation~\cite{kuo05,drake-ipdps06,zhang_lightweight_2007}.  In
particular, the case study of implementing MPEG2 in StreamIt provides
a nice example-driven exposition of the entire
language~\cite{drake-ipdps06}.

%\vspace{1in}
%\begin{center}
%{\bf \large Funding Acknowledgment}
%\end{center}
\section*{Funding Acknowledgment}
This work was funded in part by the National Science Foundation
(grants EIA-0071841, CNS-0305453, ACI-0325297), DARPA (grants
F29601-01-2-0166, F29601-03-2-0065), the DARPA HPCS program, the MIT
Oxygen Alliance, the Gigascale Systems Research Center, Nokia, and a
Siebel Scholarship.
\clearpage

% NOT ACKNOWLEDGING:
%
% - Meha Senthil?
%
% StreamIt:
% - Vijay Saraswat
% - Ryan Newton
% - Ken Steele provided an IPAQ for some of the experiments in Chapter~4.
%
% Teaching:
% - David Liben Nowell
% 
% Microfluidics:
% - natalie andrew
%
% IIH:
% - Seema Kacker
% - Sourav Dey
% - Ajit Dash
% - Alex Krull
% - Oliver Venn
% - Jessica Leon
% - Nikhil Nadkarni
% - Catherine Dunn
%
% All these details:
%
% I am indebted to many additional students and collaborators who
% helped to pursue goals at the intersection of technology and
% development in my graduate school career.
%
% For external contributions to TEK, from Elsevier I thank Ammy
% Votglander, Craig Scott, Jeremy Alder, Spencer de Groot, and
% Christian Pruvost; from First Mile Solutions, I thank Rich
% Fletcher, Amir Alexander Hasson, and Olufemi Omojola; from the
% People's First Network, I thank David Leeming.  
%
% At Innovators In Health, I thank Manish Bhardwaj, Sara Cinnamon,
% Goutam Reddy, Emma Brunskill, Somani Patnaik, Pallavi Kaushik,
% Seema Kacker, Sourav Dey, Ajit Dash, Alex Krull, Oliver Venn,
% Jessica Leon, Nikhil Nadkarni, and Catherine Dunn.  
%
% I also thank Rich Fletcher, Michael Gordon, Jonathan Jackson,
% Jhonatan Rotenberg, Luis Sarmenta, Ammy Votglander for many
% conversations benefiting this research.  I would not have been
% able to accomplish nearly as much without the steadfast
% dedication and help of all of these individuals.
