\startchapter{Conclusions}
\label{chap:conclusions}

My thesis is that incorporating streaming abstractions into the
programming language can simultaneously improve both programmability
and performance.  Programmers are unburdened from providing low-level
implementation details, while compilers can perform parallelization
and optimization tasks that were previously beyond the reach of
automation.  This dissertation supports this thesis with the following
contributions:

\mybegin

\item We define the StreamIt language, one of the first programming 
languages to embrace synchronous dataflow as a model of computation.
StreamIt contains novel language constructs, including structured
streams, parameterized data reordering, and teleport messaging, that
improve both the programmability and the analyzability of stream
programs.  Teleport messaging addresses a long-standing problem in
synchronizing events across decoupled modules, and represents the
first general framework for delivering messages with respect to the
regular dataflow in the stream.  By providing a modular and composable
syntax, the StreamIt language becomes accessible to non-expert
programmers.  Simultaneously, the language preserves the rich static
properties of the streaming domain, exposing them to the compiler for
the sake of optimization.

\item We demonstrate that it is tractable to develop large-scale 
applications in a stream programming model.  Our experience comes on
two fronts.  First, we describe the development of the 34,000-line
StreamIt benchmark suite, consisting of large applications (such as
MPEG-2 encoding/decoding and GMTI radar processing) that were written
by programmers who were previously unfamiliar with StreamIt.  Our
survey of these benchmarks provides the first rigorous
characterization of the streaming domain, as well as new insights into
the utility and usability of various language features.  Second, we
develop a tool for migrating legacy C programs into a streaming
representation.  It is the first tool to use a dynamic analysis to
expose coarse-grained parallelism in C programs.  We show that this
tool is effective at extracting a synchronous dataflow graph from
large C applications, spanning MPEG-2, GMTI, MP3, and others.

\item We develop a new optimization for the streaming domain, allowing 
programmers to accelerate common video editing operations by a median
of 15x and a maximum of 471x.  This transformation maps stream
programs into the compressed domain, allowing them to operate directly
on compressed data formats rather than requiring a costly
decompression and re-compression on either side of processing.  Our
technique is the first to support compressed-domain processing of
LZ77-compressed data.  We apply our technique to accelerate
transformations such as color adjustment and video compositing on the
Apple Animation format.  Performance gains are proportional to the
compression factor.

\item We review the key optimization results in the StreamIt project, 
enabling programmers to obtain large speedups on many tasks.
Targeting a 16-core architecture, our compiler leverages a novel
combination of task, data, and pipeline parallelism to obtain a robust
speedup of over 11x (relative to a single core).  In optimizing linear
computations, our compiler mirrors the behavior of a DSP expert,
automatically combining linear nodes, translating them to the
frequency domain, and selecting the most profitable series of
transformations.  Linear optimizations yield an average performance
improvement of 5.5x, and a maximum improvement of 8.0x.  Finally, we
offer a set of cache optimizations that adjusts the schedule of filter
executions so as to improve the data and instruction locality.  It
offers an average benefit of 3.5x when targeting an embedded
processor.

\myend

Several of the transformations that are automated in the StreamIt
compiler are already accessible to expert programmers.  For example,
the optimization of linear nodes is a standard part of the DSP design
flow.  Cache optimizations similar to ours are routinely performed
during the manual tuning of an embedded system.  Streaming
applications can be parallelized in other languages with significant
help from the programmer.  However, the key benefit of StreamIt is
that all of these transformations become accessible to non-experts.
Using a stream programming model, the compiler can leverage new
information to automate transformations that were previously reserved
for technology wizards.

%While stream programming is a powerful tool, it is not a silver
%bullet to all parallel programming problems.
Perhaps the biggest limitation of the techniques described in this
dissertation is that they apply primarily to static-rate programs, in
which the input and output rates of actors are known at compile time.
The compiler depends on static rates for load-balancing task- and
pipeline-parallel actors; for optimizing linear filters (which are
implicitly static-rate); for cache optimizations; and for teleport
messaging (though Section~\ref{sec:messaging-future} describes how to
extend messaging to handle dynamism).  Some of the techniques
described do extend to dynamic rates, including the language's support
for hierarchical streams and parameterized data reordering, as well as
the compiler's support for coarse-grained data parallelism,
translation to the compressed domain (with minor modifications), and
our dynamic analysis for extracting parallelism from C programs.

In the long term, we envision that our optimizations of static-rate
graphs would have maximum impact when those graphs are embedded in a
more flexible and hybrid programming model.  By analogy to instruction
scheduling, one could consider our optimizations as focusing on the
basic block: a simple yet pervasive construct that can be stitched
together to perform a broad array of tasks.  While support for complex
control flow is important for functionality, aggressive optimization
within the basic block is essential for high performance.  Our
benchmark suite supports the premise that dynamic rates often occur at
only a few points in an application; out of our 29 most realistic
applications, 24 are completely static-rate, and out of those with any
% dynamic rates in user-defined filters: 17 / 530
dynamic rates, only 3\% of the user-defined filters have a dynamic
rate.  Thus, the compiler can focus on optimizing each static-rate
subgraph, while relying on the runtime system to orchestrate the
dynamic-rate boundaries.  In addition to supporting dynamism, we
envision that a hybrid programming model would provide support for
many models of computation (transactions, event-driven programs,
scientific codes, etc.) with high-performance streaming playing only
one part.  These models could likely be embedded in a general-purpose
programming language, using a separate library and runtime system for
each one.  Integrating such models of computation into a unified
authoring environment is an interesting direction for further
research.

What is the future of stream programming?  Well, the goal of academic
programming language research is frequently misunderstood.  While it
would not be objectionable to have a language like StreamIt take over
the world, this goal is rarely realistic or sustainable.  Rather, the
primary goal of our research is to influence the direction of future
languages.  There is broad precedent for such influence; for example,
Bjarne Stroustrup traces the detailed evolution of the C++ language,
tracing the impact of many previous languages on its structure and
feature set~\cite{stroustrup_design_1994}.  While many of these
languages are also well known (Fortran, C, Ada, Simula), there are
also important influences from lesser-known languages, many of which
are of academic origin (CPL, BCPL, ML, Clu).  Given the trend towards
multicore processors, and the increasing prevalence of streaming
applications, we anticipate the emergence of languages and libraries
with integrated support for efficient stream processing.  Already the
StreamIt system (which is open-source and available
online~\cite{streamitweb}) has been reused and extended by multiple
research groups, including UC
Berkeley~\cite{mani-permutations,bit-streaming}, IBM
Research~\cite{huang_liquid_2008,hormati_optimus:_2008}, University of
Michigan~\cite{kudlur_orchestratingexecution_2008,hormati_optimus:_2008},
Halmstad University~\cite{ola-techrep,andersson_model_07}, Johns
Hopkins University~\cite{duca-thesis}, and North Carolina State
University~\cite{won-thesis}.  It is our hope that the abstractions,
optimizations, and lessons learned as part of this dissertation will
serve to inform and inspire this research as well as the upcoming
generation of mainstream programming langauges.

%% don't mention:
%% - zeroing out arrays is expensive, should optimize away (move to
%%   lessons learned?)
%% - buffer reuse
%% - fusion strategies (contradicts other lctes research?)
