\chapter{Example StreamIt Program}
\label{chap:example-program}

~ \\ \vspace{-0.7in}

\noindent This appendix provides a complete program listing for a small StreamIt
benchmark, ChannelVocoder.  The stream graph for this benchmark can be found at 
the end of the listing.

~ \\ \vspace{-0.8in}

%[
/** 
 * Author:  Andrew Lamb
 *
 * This is a channel vocoder as described in MIT 6.555 Lab 2.  Its salient features
 * are a filter bank, each of which contains a decimator after a bandpass filter.
 * 
 * First the signal is conditioned using a lowpass filter with cutoff at 5000
 * Hz. Then the signal is "center clipped" which basically means that very high and
 * very low values are removed.  The sampling rate is 8000 Hz.
 *
 * Then, the signal is sent both to a pitch detector and to a filter bank with 200 Hz
 * wide windows (16 overall).
 * 
 * Thus, each output is the combination of 16 band envelope values from the filter
 * bank and a single pitch detector value. This value is either the pitch if the
 * sound was voiced or 0 if the sound was unvoiced.
 **/
void->void pipeline ChannelVocoder {
    int PITCH_WINDOW = 100; // the number of samples to base the pitch detection on
    int DECIMATION   = 50;  // decimation factor
    int NUM_FILTERS  = 16; 
    
    add FileReader<float>("input.dat");
    // low pass filter to filter out high freq noise
    add LowPassFilter(1, (2*pi*5000)/8000, 64); 
    add float->float splitjoin {
        split duplicate;
        add PitchDetector(PITCH_WINDOW, DECIMATION);
        add VocoderFilterBank(NUM_FILTERS, DECIMATION);
        join roundrobin(1, NUM_FILTERS);
    }
    add FileWriter<float>("output.dat");
}
%]
%[
/** 
 * Pitch detector.
 **/
float->float pipeline PitchDetector(int winsize, int decimation) {
    add CenterClip();
    add CorrPeak(winsize, decimation);
}

/**
 * The channel vocoder filterbank. 
 **/
float->float splitjoin VocoderFilterBank(int N, int decimation) {
    split duplicate;
    for (int i=0; i<N; i++) {
        add FilterDecimate(i, decimation);
    }
    join roundrobin;
}

/** 
 * A channel of the vocoder filter bank -- has a band pass filter centered at i*200
 * Hz followed by a decimator with decimation rate of decimation.
 **/
float->float pipeline FilterDecimate(int i, int decimation) {
    add BandPassFilter(2, 400*i, 400*(i+1), 64); 
    add Compressor(decimation);
}

/** 
 * This filter "center clips" the input value so that it is always within the range
 * of -.75 to .75
 **/
float->float filter CenterClip {
    float MIN = -0.75;
    float MAX =  0.75;
    work pop 1 push 1 {
        float t = pop();
        if (t<MIN) {
            push(MIN); 
        } else if (t>MAX) {
            push(MAX);
        } else {
            push(t);
        }
    }
}
%]
%[
/** 
 * This filter calculates the autocorrelation of the next winsize elements and then
 * chooses the max peak. If the max peak is under a threshold we output a zero. If
 * the max peak is above the threshold, we simply output its value.
 **/
float->float filter CorrPeak(int winsize, int decimation) {
    float THRESHOLD = 0.07;
    work peek winsize push 1 pop decimation {
        float[winsize] autocorr; // auto correlation
        for (int i=0; i<winsize; i++) {
            float sum = 0;
            for (int j=i; j<winsize; j++) {
                sum += peek(i)*peek(j);
            }
            autocorr[i] = sum/winsize;
        }

        // armed with the auto correlation, find the max peak in a real vocoder, we
        // would restrict our attention to the first few values of the auto corr to
        // catch the initial peak due to the fundamental frequency.
        float maxpeak = 0;
        for (int i=0; i<winsize; i++) {
            if (autocorr[i]>maxpeak) {
                maxpeak = autocorr[i];
            }
        }
    
        // output the max peak if it is above the threshold.
        // otherwise output zero.
        if (maxpeak > THRESHOLD) {
            push(maxpeak);
        } else {
            push(0);
        }
        for (int i=0; i<decimation; i++) {
            pop();
        }
    }
}          
%]
%[
/**
 * A simple adder which takes in N items and pushes out the sum of them.
 **/
float->float filter Adder(int N) {
    work pop N push 1 {
        float sum = 0;
        for (int i=0; i<N; i++) {
            sum += pop();
        }
        push(sum);
    }
}

/**
 * This is a bandpass filter with the rather simple implementation of a low pass
 * filter cascaded with a high pass filter. The relevant parameters are: end of
 * stopband=ws and end of passband=wp, such that 0<=ws<=wp<=pi gain of passband and
 * size of window for both filters. Note that the high pass and low pass filters
 * currently use a rectangular window.
 **/
float->float pipeline BandPassFilter(float gain, float ws, float wp, int numSamples) {
    add LowPassFilter(1, wp, numSamples);
    add HighPassFilter(gain, ws, numSamples);
}

/**
 * This filter compresses the signal at its input by a factor M.
 * Eg it inputs M samples, and only outputs the first sample.
 **/
float->float filter Compressor(int M) {
    work peek M pop M push 1 {
        push(pop());
        for (int i=0; i<(M-1); i++) {
            pop();
        }
    }
}
%]
\newpage~\\~\vspace{-5\baselineskip}\enlargethispage{\baselineskip}
%[
/** 
 * Simple FIR high pass filter with gain=g, stopband ws(in radians) and N samples.
 *
 * Eg
 *                 ^ H(e^jw)
 *                 |
 *      ---------  |     ---------
 *     |      |    |    |     |
 *     |      |    |    |     |
 *    <----------------------------------> w
 *                  pi-wc pi pi+wc
 *
 * This implementation is a FIR filter is a rectangularly windowed sinc function (eg
 * sin(x)/x) multiplied by e^(j*pi*n)=(-1)^n, which is the optimal FIR high pass
 * filter in mean square error terms.
 *
 * Specifically, h[n] has N samples from n=0 to (N-1)
 * such that h[n] = (-1)^(n-N/2) * sin(cutoffFreq*pi*(n-N/2))/(pi*(n-N/2)).
 * where cutoffFreq is pi-ws
 * and the field h holds h[-n].
 */
float->float filter HighPassFilter(float g, float ws, int N) {
    float[N] h;

    /* since the impulse response is symmetric, I don't worry about reversing h[n]. */
    init {
        int OFFSET = N/2;
        float cutoffFreq = pi - ws;
        for (int i=0; i<N; i++) {
            int idx = i + 1;
            /* flip signs every other sample (done this way so that it gets array destroyed) */
            int sign = ((i%2) == 0) ? 1 : -1;
            // generate real part
            if (idx == OFFSET) 
                /* take care of div by 0 error (lim x->oo of sin(x)/x actually equals 1)*/
                h[i] = sign * g * cutoffFreq / pi; 
            else 
                h[i] = sign * g * sin(cutoffFreq * (idx-OFFSET)) / (pi*(idx-OFFSET));
        }
    }

    /* implement the FIR filtering operation as the convolution sum. */
    work peek N pop 1 push 1 {
        float sum = 0;
        for (int i=0; i<N; i++) { 
            sum += h[i]*peek(i);
        }
        push(sum);
        pop();
    }
}
%]
%[
/** 
 * Simple FIR low pass filter with gain=g, wc=cutoffFreq(in radians) and N samples.
 * Eg:
 *                 ^ H(e^jw)
 *                 |
 *          ---------------------
 *          |      |      |
 *          |      |      |
 *    <----------------------------------> w
 *         -wc            wc
 *
 * This implementation is a FIR filter is a rectangularly windowed sinc function (eg
 * sin(x)/x), which is the optimal FIR low pass filter in mean square error terms.
 *
 * Specifically, h[n] has N samples from n=0 to (N-1)
 * such that h[n] = sin(cutoffFreq*pi*(n-N/2))/(pi*(n-N/2)).
 * and the field h holds h[-n].
 */
float->float filter LowPassFilter(float g, float cutoffFreq, int N) {
    float[N] h;

    /* since the impulse response is symmetric, I don't worry about reversing h[n]. */
    init {
        int OFFSET = N/2;
        for (int i=0; i<N; i++) {
            int idx = i + 1;
            // generate real part
            if (idx == OFFSET) 
                /* take care of div by 0 error (lim x->oo of sin(x)/x actually equals 1)*/
                h[i] = g * cutoffFreq / pi; 
            else 
                h[i] = g * sin(cutoffFreq * (idx-OFFSET)) / (pi*(idx-OFFSET));
        }
    }

    /* Implement the FIR filtering operation as the convolution sum. */
    work peek N pop 1 push 1 {
        float sum = 0;
        for (int i=0; i<N; i++) { 
            sum += h[i]*peek(i);
        }
        push(sum);
        pop();
    }
}
%]
