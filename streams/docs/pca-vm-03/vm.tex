\section{Streaming Virtual Machine API}

Recent trends in both applications and architectures point to the
streaming model of computation as a cornerstone for the next
generation of high-performance computing.  On the applications side,
there is a growing class of programs that fall within the streaming
domain, encompassing areas such as signal processing, graphics,
networks, and scientific computing.  Streaming applications are
distinguished from others by their large degree of task and data
parallelism, as well as regular patterns of communication between
concurrently executing modules.  On the architecture side, there is an
emerging class of communication-exposed machines that contain
distributed memories and replicated processing units, sometimes
including high-performance ``stream co-processors''.  Because
resources in these machines are orchestrated by a software layer, the
compiler can leverage the widespread parallelism and regular
communication patterns of stream programs to obtain high performance.

A number of communication-exposed architectures are being developed as
part of the Polymorphic Computing Architecture (PCA) initiative.  In
addition to providing good support for streams, these machines offer
alternate modes of operation, called ``morphs'', that can be toggled to
meet the demands of the application and the constraints of the
environment.  Not surprisingly, it is a challenging compiler problem
to map a stream program to a PCA architecture.  To obtain good
performance, the compiler needs to detect data, task, and pipeline
parallelism in the source program and calculate a load-balanced
mapping onto the processor resources, all the while taking into
account the large array of morph possibilities.  With a naive
approach, this effort would have to be duplicated every time a given
PCA architecture wants to support a given stream language.

In order to mitigate the implementation complexity for each
architecture team, the PCA program has adopted a two-level compilation
strategy.  Collectively referred to as the Morphware Stable Interface
(MSI), this scheme divides the compilation process into two compilers:
a High Level Compiler (HLC) and a Low Level Compiler (LLC) as shown in
Figure~\ref{fig:msi}.  The HLC inputs a stream language ({\it e.g.,}
Brook or StreamIt) as well as the description of a PCA architecture,
and it outputs a high-performance mapping of logical stream constructs
to the physical resources of the given architecture.  This mapping is
expressed in terms of the Streaming Virtual Machine (SVM), which is
currently an API that is written in C.  The LLC inputs the SVM code
for a given architecture and outputs a binary that is fit for
execution on that architecture.

This two-stage compilation process offers several benefits.  First,
many of the difficult compilation problems are tackled by the HLC and
do not need to be repeated for each architecture (we envision there
being many more LLC's than HLC's.)  The responsibilities of the HLC
include parallelism detection, load balancing, coarse-grained
scheduling of stream computations, and memory management for streaming
data. In turn, the LLC is responsible for generating code for a given
architecture, 
\begin{figure}[h]
\begin{center}
\psfig{figure=msi.eps,width=4in}
\caption{Streaming components of the Morphware Stable Interface.\protect\label{fig:msi}}
\end{center}
\vspace{-6pt}
\end{figure}
which includes software pipelining, detailed routing of data items,
management of instruction memory, and interfacing between stream
processors and control processors.  An additional benefit of this
approach is that the HLC is focused on developing general solutions to
the fundamental issues at hand, rather than over-specializing the
compiler for the idiosyncratic features of any particular
architecture.  It is our expectation that future architectures (and
accompanying LLC's) could effectively reuse an HLC developed in this
program and attain high performance without any modifications to the
HLC.

The Streaming Virtual Machine (SVM) is the interface between the HLC
and the LLC.  Its goal is to provide a means for the HLC to express a
high-performance mapping from the computational elements of the
application to the physical resources of a given architecture.  In
order to enable high performance, it is critical for the SVM to expose
all of the stream-specific information that was available to the HLC,
so that the LLC can take advantage of it while generating code.
Towards this end, the SVM API embodies the following design features:
\begin{itemize}

\item {\bf Separation of control and data-intensive code}, so that
data-intensive code can be explicitly compiled to stream processors.
The SVM provides the {\it kernel} abstraction as a unit of computation
that should execute on a stream processor.  Kernels consist of a
restricted subset of C; they operate on a set of input and output
streams and may contain local state.  Kernels can be asynchronously
executed and monitored from a control thread running on a different
processor.

\item {\bf Explicit communication via streams.}  The SVM exposes the
streaming nature of the computation by embracing a {\it stream}
abstraction.  A stream is a queue of data items through which
different threads and kernels can communicate.  Streams are mapped
either to memories or to hardware FIFOs that provide primitive support
for stream operations.  By mapping streams to FIFOs, the SVM can
express direct processor-to-processor communication (as well as I/O
devices that do not operate via memory.)

\item {\bf Explicit memory management for streaming data.}  The SVM
manages the size and placement of memory-mapped streams, orchestrating
data movement with explicit DMA operations rather than hardware
caching.  This functionality is essential for data-intensive codes
that operate out of local compiler-controlled memories ({\it e.g.,} a
Stream Register File), as stream data must be fetched ahead of time to
keep the execution unit busy.  The SVM also introduces the notion of
streaming DMA, which offers continuous data movement rather than
pre-blocked structures.

\end{itemize} 
In its current instantiation, the SVM implements the above
functionality in terms of an API that is written in C.  Many of the
details of the API reflect the principal design goal of simplifying
the translation process for the LLC.  While the resulting API is
readable and comprehensible by a human, this was only a secondary
consideration; the SVM is not intended as a programming language, but
rather as a detailed compiler-generated mapping.  Also, since the API
is valid C, it can be compiled by a standard C compiler with a simple
runtime library for testing and debugging purposes.

The rest of this document gives a detailed specification for the
Streaming Virtual Machine API.  We review the terminology for the API
in Section~\ref{sec:terminology}.  Then, we describe the API for
streams in Section~\ref{sec:streams}, the API for blocks in
Section~\ref{sec:blocks}, the API for kernels in
Section~\ref{sec:kernel}, and the API for control in
Section~\ref{sec:control}.  We conclude with a discussion of error
handling (Section~\ref{sec:errors}) and a concept of operations
(Section~\ref{sec:conops}).

\subsection{Terminology}
\label{sec:terminology}

The Streaming Virtual Machine (SVM) describes logical and physical
entities recognized by the High Level Compiler (HLC) and Low Level
Compiler (LLC) as a description of a streaming PCA program.  The SVM
describes the valid operations on each entity and the constraints for
mapping logical entities onto physical entities.  During compilation,
the HLC creates a structure or set of structures containing the
logical entities and their mapping onto the physical entities.  The
available physical entities are described in the PCA Machine Model.
The logical structure represents the function and performance of the
HLC source program.  The mapping to physical entities reflects the HLC
resource mapping.  The LLC translates the logical structures to
executable machine code for a specific PCA architecture.

\clearpage
\sss{Logical Entities}

The SVM contains the following logical entities:

\begin{itemize}

\item {\it Streams}.  A stream is a data type that represents a
sequence of data elements of a given type.  Streams are used for
sequential production and consumption of elements.

\item {\it Blocks}. Blocks provide indexed (random) access to a fixed
set of data elements.

\item {\it Kernels (slaves)}. Kernels are a logical entity that
describes a locus of computation consuming 0 or more input streams and
producing 0 or more output streams.  Kernels contain an execution
state (unstarted, waiting, paused, running, or finished).  Kernels are
described with a subset of C code and SVM API commands.

\item {\it Controls (masters)}. Controls are a logical entity that
describes the locus of computation within a processor.  Control code
can initiate, monitor, and terminate the execution of kernels.
Controls are described with a subset of C code and SVM API commands.

\item {\it Data elements}. Data elements of streams and blocks must
have a fixed size.  Variable sized elements can be supported at the
language-level and compiled into the fixed size scheme using a data
encoding.  Supported primitive types are indicated by the set of {\tt
mm.Proc.Supports} fields in the machine model; there is also support
for arrays with a fixed (int literal) length and {\tt struct}'s
containing members of any other type.

\end{itemize}

\sss{Physical Entities}

The following physical entities are contained within the PCA Machine
Model and referenced by the SVM:

\begin{itemize}

\item {\it FIFOs}.  A FIFO is a first-in first-out buffer.  The FIFO
state includes the data elements stored in the FIFO and the number of
items stored in the FIFO which includes the empty state.  The valid
FIFO operations are peek(n), pop, and push.

\item {\it Memories}. A memory is random access storage with an index
pointing to the data element to be read or written.

\item {\it Processors}. A processor is logic with the ability to carry
out an independent execution.  It has internal state.  It can be
independently started or stopped.  A given processor can support a
certain set of pre-defined and black-box kernels, and may support
user-defined kernels and controls.

\item {\it Network links}. Network links pass data elements between
the physical entities.  Links connect processors to memories,
processors to FIFOs, and links to other links.

\end{itemize}

\sss{Processor Properties}

The following processor properties in the PCA Machine Model are
relevant to the operation of the SVM:

\begin{itemize}

\item {\tt mm.Proc.Masters}.  This property is a list of zero or more
processors that control a given processor.  A master is capable of
controlling a kernel's execution on a corresponding slave processor.

%A processor may be its own MASTER (???)

\item {\tt mm.Proc.SupportedKernels}.  This property indicates a list
of pre-defined and black-box kernels that can execute on a given
processor, along with performance metadata for each one.

\item {\tt mm.Proc.SupportsUserCode}.  For a given slave processor,
this property indicates whether user-defined kernels are supported.

\end{itemize}

\sss{Logical to Physical Binding}

The SVM addresses the mapping of logical to physical entities
described above:

\begin{itemize}

\item A Stream is mapped to a FIFO or a memory.

\item A Block is mapped to a memory.

\item A Kernel is mapped to a processor that supports the given kernel.

\item Control code is mapped to processors that have no master.

\end{itemize}

\subsection{Streams}
\label{sec:streams}

A {\it stream} is a data type that represents a sequence of elements
of a given type.  The elements must have a fixed size. Variable sized
elements can be supported at the language-level and compiled into the
fixed size scheme using a data encoding. The stream type is declared
as a black box:
{\small
\begin{verbatim}
  typedef struct {
    // not exposed
  } svm_Stream;
\end{verbatim}}
The API for streams includes built-in support for end-of-stream (EOS)
tags.  In some cases, the number of elements which are pushed onto a
stream is data-dependent and not determined until runtime.  Thus, an
explicit EOS indicator needs to be generated in order for a consumer
kernel to know when its input stream has finished.  It is possible for
a producer kernel to manually encode this information in the data
stream, but doing so often ignores hardware support and incurs
overhead for each element.  For this reason, the API provides the
ability for producers to tag the last stream element with an EOS
indicator ({\tt svm\_streamPushWithEOS}), and for consumers to check
whether or not an element has an EOS tag ({\tt svm\_streamPeekEOS}).
Representing the EOS as a tag rather than a separate element in the
stream serves to simplify several aspects of the stream semantics.
Note that there might be several EOS tags in the lifetime of a stream,
or even at a given point in time; for example, EOS tags could be used
as a boundary between successive lines of video data.  However, the
{\tt svm\_Stream\_One\_EOS} flag allows the High Level Compiler to indicate
cases where only one EOS tag will exist at a given time.

The API also requires the Low Level Compiler to implement memory-based
streams with a very specific data layout: they must be circular
buffers, starting at a given address and wrapping around once the
capacity has been filled (see below for details).  This guarantee is
important for allowing reuse of memory space between streams, as well
as for transferring data between threaded code and streaming code.
However, in cases where the data layout is inconsequential, the High
Level Compiler can use the {\tt svm\_Stream\_Unaliased\_RAM} flag to allow
layout optimizations by the Low Level Compiler.

In order to improve the readability of kernel code, there are several
aliases for the {\tt svm\_Stream} type:
{\small
\begin{verbatim}
  typedef svm_Stream svm_IStream;
  typedef svm_Stream svm_OStream;
  typedef svm_Stream svm_IOStream;
\end{verbatim}}
\noindent The {\tt svm\_IStream}, {\tt svm\_OStream}, and {\tt
svm\_IOStream} aliases indicate the relationship between a given
kernel and a stream.  If a kernel declares a variable of type {\tt
svm\_IStream}, then that stream can only be used for input; likewise,
an {\tt svm\_OStream} can only be used for output, while an {\tt
svm\_IOStream} can be used for both input and output\footnote{Note
that {\tt svm\_IOStream}s are used by a single kernel to both pop and
push elements.  The {\tt svm\_IOStream} is not used between kernels
since such use would be output for one kernel and input for the
other. The primary use of the {\tt svm\_IOStream} is to share a single
physical resource for both an output and input stream.  Data pushed
onto the stream can be popped.  When the resource is full, {\tt push}
stalls.  When the resource is empty, {\tt pop} stalls.  {\tt
svm\_IOStream}s are a first-in and first-out structure.}.  Note that
the use of these aliases is strictly optional; it is simply a way for
the High Level Compiler to improve the readability of its output.

The API for streams consists of a set of initialization and I/O
functions.  An initialization function gives the location and size of
a stream, while the I/O functions allow data to be transferred in and
out of a stream.  A stream must be initialized before being used for
I/O.  These functions are described in detail below:
%%%%%%%%%%%%%%%%%%%%%%%%%
\clearpage
~ \\ \vspace{-24pt} 
\makeline
\name{streamInitRAM}
{\small
\begin{verbatim}
void svm_streamInitRAM(svm_Stream* s, mm_Mem ramLocation, size_t address, 
                       size_t capacity, size_t elementSize, size_t maxBuffering, int flags)
\end{verbatim}}

\spec{Initializes an empty stream that resides in RAM with the
specified location, size, and performance hints.  The {\tt address}
and {\tt capacity} might be dynamically calculated at runtime.\\}

\spec{Unless the {\tt svm\_Stream\_Unaliased\_RAM} flag is set, the
stream must be implemented as a circular buffer that stores items in
consecutive locations starting at {\tt address} and wrapping around
once more than {\tt capacity} elements have been pushed.  That is,
items 0, {\tt capacity}, {\tt 2*capacity}, etc., are all stored at
location {\tt address}.}

\param{
  {\tt s} - stream to initialize.\\
  {\tt ramLocation} - memory resource in which to hold stream (from PCA Machine Model.)\\
  {\tt address} - memory address at which stream begins (in format dictated by PCA Machine Model.)\\
  {\tt capacity} - maximum number of elements in stream.\\
  {\tt elementSize} - number of bytes per element.\\
  {\tt maxBuffering} - performance hint, indicates the maximum number of items that can be buffered at\\ \mbox{~~~~~}the producer for this stream without causing deadlock\footnotemark[2].\\
  {\tt flags} - performance hints to Low Level Compiler (see below).  These can be ignored (assumed to\\ \mbox{~~~~~}be 0) without sacrificing correctness; they are only for the sake of optimization.}

\req{
  {\tt ramLocation}, {\tt elementSize}, {\tt maxBuffering}, and {\tt flags} are literals, and are constant across\\ \mbox{~~~~~}all calls to initialization functions for {\tt s}.\\
  {\tt s} is not used for input or output in a kernel that is currently {\tt Waiting}, {\tt Running}, or {\tt Paused}.}

\footnotetext[2]{In message-passing architectures, this provides a safe
upper bound for message sizes, thereby amortizing the messaging
overhead.}

\callc
%%%%%%%%%%%%%%%%%%%%%%%%%
\makeline
\name{streamInitWithDataRAM}
{\small
\begin{verbatim}
void svm_streamInitWithDataRAM(svm_Stream* s, mm_Mem ramLocation, size_t address, 
                               size_t capacity, size_t elementSize, size_t initLength, 
                               boolean initEOS, size_t maxBuffering, int flags)
\end{verbatim}}

\spec{Initializes a stream in RAM with some elements already in place.
If {\tt initEOS} is true, then the last item is tagged with an EOS
tag.  The stream will have the specified location, size, and
performance hints.  The {\tt address} and {\tt capacity} might be
dynamically calculated at runtime.\\}

\spec{The stream must be implemented as a circular buffer as in {\tt
svm\_streamInitRAM}.}

\param{
  {\tt initLength} - number of items already in place at {\tt address} that should be pushed onto {\tt s} initially.\\
  {\tt initEOS} - whether or not the initial item at position {\tt initLength-1} should be tagged with an EOS \\ \mbox{~~~~~}tag.  That is, indicates the initial value of {\tt streamPeekEOS(s, initLength-1)}.\\
  Other parameters as above.}
\req{
  {\tt initLength} $>$ 0.\\
  {\tt svm\_Stream\_Unaliased\_RAM} flag is NOT set.\\
  {\tt ramLocation}, {\tt elementSize}, {\tt maxBuffering}, and {\tt flags} are literals, and are constant across\\ \mbox{~~~~~}all calls to initialization functions for {\tt s}.\\
  {\tt s} is not used for input or output in a kernel that is currently {\tt Waiting}, {\tt Running}, or {\tt Paused}.}

\callc
%%%%%%%%%%%%%%%%%%%%%%%%%
\clearpage
~ \\ \vspace{-24pt}
\makeline
\name{streamInitFIFO}
{\small
\begin{verbatim}
void svm_streamInitFIFO(svm_Stream* s, mm_Mem fifoLocation, 
                        size_t elementSize, size_t maxBuffering, int flags)
\end{verbatim}}

\spec{Initializes an empty stream mapped to a hardware FIFO with the
given element size and performance hints.  As there is only one stream
allocated to a FIFO at a time, the size of this stream is determined
by the size of the FIFO.}

\param{
  {\tt s} - stream to initialize.\\
  {\tt fifoLocation} - FIFO resource in which to hold stream (from PCA Machine Model).\\
  {\tt elementSize} - number of bytes per element.\\
  {\tt maxBuffering} - performance hint, indicates the maximum number of items that can be buffered at\\ \mbox{~~~~~}the producer for this stream without causing deadlock.\\
  {\tt flags} - performance hints to Low Level Compiler (see below).  These can be ignored (assumed to\\ \mbox{~~~~~}be 0) without sacrificing correctness; they are only for the sake of optimization.}

\req{
  {\tt fifoLocation}, {\tt elementSize}, {\tt maxBuffering}, and {\tt flags} are literals, and are constant across\\ \mbox{~~~~~}all calls to initialization functions for {\tt s}.\\
  {\tt fifoLocation} represents an empty FIFO (no other stream is retaining state in it.)\\
  {\tt s} is not used for input or output in a kernel that is currently {\tt Waiting}, {\tt Running}, or {\tt Paused}.}

\callc
%%%%%%%%%%%%%%%%%%%%%%%%%
\makeline
\name{Stream Flags \medskip}

\noindent These flags are used by the High Level Compiler to pass
performance hints to the Low Level Compiler.  They appear as an
argument to the stream initialization functions above.

\decl{typedef enum \{}

\decl{  svm\_Stream\_Unordered = 0x1,}

\spec{This flag is true if FIFO semantics may be relaxed such that
popping does not return elements in the same order that the elements
were pushed.  This might enable improved routing or instruction
scheduling, {\it e.g., } for applications dealing on unordered network
packets.  Peek cannot be used if this flag is set.\\}

\decl{  svm\_Stream\_Unaliased\_RAM = 0x2,}

\spec{This flag is true if other streams/blocks are guaranteed NOT to
be aliased with this stream.  Thus, the Low Level Compiler can use an
optimized, architecture-specific memory layout for this stream, rather
than guaranteeing that it is implemented as an in-order circular
buffer (as described above).\\}

\decl{  svm\_Stream\_Never\_Wraps = 0x4,}

\spec{This flag is true if the number of elements pushed onto a given
stream will never exceed the stream's capacity.  This allows the Low
Level Compiler to treat the stream as a single-assignment block rather
than a circular buffer.\\}

\decl{  svm\_Stream\_One\_EOS = 0x8}

\spec{This flag is true if at most one element in the stream will ever
have an EOS tag at any given time.  For memory streams, this allows
the Low Level Compiler to implement EOS with a single pointer instead
of tracking a set of EOS tokens.\\}

\decl{\} svm\_Stream\_Flags;}
\makeline
%%%%%%%%%%%%%%%%%%%%%%%%%
\clearpage
\makeline
\name{streamPush}
\decl{void svm\_streamPush(svm\_Stream* s, void* element)}
\spec{Enqueues a value onto the end of a stream.  If the stream cannot hold any more elements, then {\tt svm\_streamPush} stalls until there is space available.}
\param{{\tt s} - stream to push onto.\\
  {\tt element} - value to push.}
\req{{\tt s} has been initialized.}
\callkc
%%%%%%%%%%%%%%%%%%%%%%%%%
\makeline
\name{streamPushMulticast}
\decl{void svm\_streamPushMulticast(int n, void* element, svm\_Stream* s1, svm\_Stream* s2, ...)}

\spec{Pushes an item onto the end of the given streams, in any order
(or simultaneously.)  Will not return until an item has been pushed
onto all streams. \medskip }

\spec{For example, if {\tt s1} is empty and {\tt s2} is full, then the
following are all legal implementations of {\tt svm\_streamMulticastPush(\&x, \&s1,
\&s2)}:
\begin{itemize}
\addtolength{\itemsep}{-4pt} % so this fits on one page
\item Push {\tt x} onto {\tt s1}, wait until {\tt s2} is not full, then push {\tt x} onto {\tt s2}.
\item Wait until {\tt s2} is not full, push {\tt x} onto {\tt s2}, then push {\tt x} onto {\tt s1}.
\item Wait until {\tt s2} is not full, push {\tt x} onto {\tt s1}, then push {\tt x} onto {\tt s2}.
\end{itemize} \smallskip }

\spec{Aside from the stalling behavior, this method is equivalent to a
series of calls to {\tt svm\_streamPush}.  It is included in the API in
order to give the Low Level Compiler an opportunity to optimize
broadcast messages.}

\param{{\tt n} - number of streams.\\
  {\tt s1, s2, ..., sn} - streams to push onto.\\
  {\tt element} - value to push.}

\req{{\tt s1, s2, ..., sn} have been initialized.}

\callkc
%%%%%%%%%%%%%%%%%%%%%%%%%
\makeline
\name{streamPop}
\decl{void svm\_streamPop(svm\_Stream* s, void* element)}

\spec{Dequeues an element from the front of the stream.  If the stream
is empty, then {\tt svm\_streamPop} stalls until an item is present.  Note
that if the popped element has an EOS tag, the tag is also popped, and
the EOS information is no longer accessible.}

\param{{\tt s} - stream to pop from.\\
  {\tt element} - location in which to store the popped value.}

\req{{\tt s} has been initialized.}

\callkc
%%%%%%%%%%%%%%%%%%%%%%%%%
\makeline
\clearpage
\makeline
\name{streamPeek}
\decl{void svm\_streamPeek(svm\_Stream* s, size\_t n, void* element)}

\spec{Retrieves the value at position {\tt n} from the front of the
stream, without changing any state of the stream.  The index {\tt n}
is zero-based, such that {\tt svm\_streamPeek(s, 0, \&x)} and {\tt
svm\_streamPop(s, \&x)} both put the same value in {\tt x}.  If the stream
contains less than {\tt n+1} elements, then {\tt svm\_streamPeek} stalls
until {\tt n+1} elements are available.}

\param{{\tt s} - stream to peek at.\\
  {\tt n} - index at which to peek (zero-based and starting from front of stream).\\
  {\tt element} - location in which to store the peeked value.}

\req{
  if {\tt n} $>$ 0, then {\tt svm\_Stream\_Unordered} flag has NOT been set.\\
  {\tt n} $<$ {\tt capacity} of {\tt s}.\\
  {\tt s} has been initialized.}

\callkc
%%%%%%%%%%%%%%%%%%%%%%%%%
\makeline

\name{streamPushWithEOS}

\decl{void svm\_streamPushWithEOS(svm\_Stream* s, void* element)}

\spec{Pushes an element onto the end of the stream with an EOS tag
attached to the element.  If the stream is full, this stalls until
there is space to push the element.}

\param{{\tt s} - stream to push onto.\\
  {\tt element} - value to push.}

\req{{\tt s} has been initialized.}

\callkc
%%%%%%%%%%%%%%%%%%%%%%%%%
\makeline
\name{streamPushMulticastWithEOS}

\decl{void svm\_streamPushMulticastWithEOS(int n, void* element, svm\_Stream* s1, svm\_Stream* s2, ...)}

\spec{Pushes an item with EOS tag attached onto the given streams, in any
order (or simultaneously.)  Will not return until an item has been
pushed onto all streams.  See {\tt svm\_streamPushMulticast} for an example.}

\param{{\tt n} - number of streams.\\
  {\tt s1, s2, ..., sn} - streams to push onto.\\
  {\tt element} - value to push.}

\req{{\tt s1, s2, ..., sn} have been initialized.}

\callkc
%%%%%%%%%%%%%%%%%%%%%%%%%
\makeline
\clearpage
\makeline
\name{streamPeekEOS}

\decl{boolean svm\_streamPeekEOS(svm\_Stream* s, size\_t n)}

\spec{Returns whether or not there is an EOS tag on element {\tt n}
from the front of the stream; stalls if less than {\tt n+1} elements
are available.}

\param{{\tt s} - stream to peek at EOS.\\
  {\tt n} - index at which to peek (zero-based and starting from front of stream).}
\ret{True if there is an EOS tag at index {\tt n}.}
\req{
  if {\tt n} $>$ 0, then {\tt svm\_Stream\_Unordered} flag has NOT been set.\\
  {\tt n} $<$ {\tt capacity} of {\tt s}.\\
  {\tt s} has been initialized.}
\callkc
%%%%%%%%%%%%%%%%%%%%%%%%%
\makeline

%% \subsection{Memory}
%% \label{sec:memory}

%% The SVM API specifies how data is mapped to memory using stream and
%% block declarations in the control code.  Streams are used for
%% sequential production and consumption of elements, while blocks
%% provide random access to a fixed set of elements.  Streams are mapped
%% to random-access memory or hardware FIFOs. Blocks are only mapped to
%% random-access memory.

\subsection{Blocks}
\label{sec:blocks}

For kernels that require random access to a fixed set of elements, the
API provides the {\it block} abstraction.  A block is simply a region
of memory that can be read and written to.  As with streams, the block
type is a black box with aliases to indicate its relationship to a
given kernel:
{\small
\begin{verbatim}
  typedef struct {
    // not exposed
  } svm_Block;

  typedef svm_Block svm_IBlock;
  typedef svm_Block svm_OBlock;
  typedef svm_Block svm_IOBlock;
\end{verbatim}}
\noindent The API for blocks is as follows:
%%%%%%%%%%%%%%%%%%%%%%%%%
\makeline
\name{blockInit}

\decl{void svm\_blockInit(svm\_Block* b, mm\_Mem ramLocation, size\_t address, size\_t capacity, size\_t elementSize)}

\spec{Initializes a block with the given location and size.  The {\tt
address} and {\tt capacity} might be dynamically calculated at runtime.}

\param{
  {\tt b} - block to initialize.\\
  {\tt ramLocation} - memory resource in which to hold block (from PCA Machine Model.)\\
  {\tt address} - memory address at which block begins (in format dictated by PCA Machine Model.)\\
  {\tt capacity} - number of elements in block.\\
  {\tt elementSize} - number of bytes per element.}

\req{{\tt ramLocation} and {\tt elementSize} are literals, and are constant across all calls to {\tt svm\_blockInit} on {\tt b}.\\
    {\tt b} is not used for input or output in a kernel that is currently {\tt Waiting}, {\tt Running}, or {\tt Paused}.}

\callc
\makeline
%%%%%%%%%%%%%%%%%%%%%%%%%
\clearpage
\makeline
\name{blockWrite}

\decl{void svm\_blockWrite(svm\_Block* b, size\_t index, void* element)}

\spec{Stores an element into a memory block.}

\param{
  {\tt b} - block to write into.\\
  {\tt index} - offset within {\tt b} to write to.\\
  {\tt element} - value to write.}

\req{
  {\tt b} has been initialized.\\
  {\tt index} $<$ {\tt capacity} of {\tt b}.}

\callkc
%%%%%%%%%%%%%%%%%%%%%%%%%
\makeline
\name{blockRead}

\decl{void svm\_blockRead(svm\_Block* b, size\_t index, void* element)}

\spec{Reads an element from a memory block.}

\param{
  {\tt b} - block to read from.\\
  {\tt index} - offset within {\tt b} to read from.\\
  {\tt element} - location to load value into.}

\req{
  {\tt b} has been initialized.\\
  {\tt index} $<$ {\tt capacity} of {\tt b}}

\callkc
%%%%%%%%%%%%%%%%%%%%%%%%%
\makeline

\subsection{Kernels}
\label{sec:kernel}

Kernels can be used to map computation to processors.  The API has
support for three kinds of kernels. Section~\ref{sec:kernelhlc}
describes general user-defined kernels whose behavior is explicitly
described by the High Level Compiler; Section~\ref{sec:kernelsvm}
describes pre-defined kernels that are built into the SVM and must be
supported on every architecture; and Section~\ref{sec:kernelllc}
describes library kernels that appear as a black box to the High Level 
Compiler.

All three types of kernels share a common execution model, as well as
a base data type that specifies the location for kernel execution and
is passed to functions that control kernel execution.

Kernels have the following components:

\begin{enumerate}

\item An initialization function, which receives the following: 
\begin{itemize}
\item The processor resource where the kernel will execute. 
\item (Optional) A block of memory for spilling local variables. 
\item The input and output streams and blocks for the kernel. 
\item Any other kernel-specific initialization data. 
\end{itemize}

\item A {\it work} function that defines the execution of the kernel.

\item (Optional) A data type which represents kernel data that is
accessible to the control thread before, after, and at certain times
during kernel execution.

\item A {\it status} that reflects the current state of the kernel.

\item Control functions used to execute the kernel.

\end{enumerate}

\sss{Kernel Base Data Type}

All kernels share a base data type which is extended to create
specific kernels. The kernel type and enumerations are declared as
follows:

{\small
\begin{verbatim}
  typedef struct {
    // not exposed
  } svm_Kernel;

  typedef void (*svm_ExtKernelWork) (void* extKernelData);
\end{verbatim}}
\spec{This is a function pointer to the work function of a kernel.
The work function represents the entire execution of the kernel; it is
called only once by the control thread, and should contain internal
loops if some function is to be executed repeatedly.\\}
{\small
\begin{verbatim}
  typedef enum {
      svm_Kernel_Unstarted,
      svm_Kernel_Waiting,
      svm_Kernel_Running,
      svm_Kernel_Paused,
      svm_Kernel_Finished
  } svm_Kernel_Status;
\end{verbatim}}
\spec{These are the status codes of a kernel.  In our terminology, the
status codes refer to the state of a kernel that has been initialized
(via a call to {\tt svm\_kernelInit}).  An uninitialized kernel does not
have a well-defined status.\\}

\noindent The API for controlling kernels is as follows:
%%%%%%%%%%%%%%%%%%%%%%%%%
\makeline
\name{kernelInit}
{\small
\begin{verbatim}
void svm_kernelInit(svm_Kernel* k, mm_Proc procLocation, svm_IOBlock* scratch,
                    void* extKernelData, size_t extKernelDataSize, svm_ExtKernelWork extKernelWork)
\end{verbatim}}

\spec{Initializes a kernel for execution on a given processor with a
given scratch space for spilling local variables.  This function
should be called from the initializer for an extension of the base
kernel type; it inputs the state variables and work function of the
extended kernel instance.  After calling {\tt svm\_kernelInit}, the
kernel is in the {\tt Unstarted} state.}

\param{
  {\tt k} - kernel to be initialized.\\
  {\tt procLocation} - processor resource on which to execute the kernel (from PCA Machine Model.)\\
  {\tt scratch} - memory block in which to spill variables (optional; a value of 0 represents no scratch.)\\
  {\tt extKernelData} - structure containing state variables for this kernel.\\
  {\tt extKernelDataSize} - size (in bytes) of this kernel's structure of state variables.\\
  {\tt extKernelWork} - work function to be executed for this kernel.}

\req{
  {\tt procLocation} is a literal, and is constant across all statements that call {\tt svm\_kernelInit} on {\tt k}.\\
  {\tt k} is {\tt Unstarted}, {\tt Finished}, or uninitialized.}

\callc
\makeline
%%%%%%%%%%%%%%%%%%%%%%%%%
\clearpage
\makeline
\name{kernelAddDependence}

\decl{void svm\_kernelAddDependence(svm\_Kernel* k, svm\_Kernel* dependsOnKernel)}

\spec{Non-blocking function indicating that kernel {\tt k} should not
execute its work function until kernel {\tt dependsOnKernel} has
entered the {\tt Finished} state.  If {\tt dependsOnKernel} is {\tt
Unstarted} or {\tt Finished}, then this function has no effect (it
does nothing).  The dependence is satisfied as soon as {\tt
dependsOnKernel} first enters the {\tt Finished} state, even if {\tt
dependsOnKernel} leaves the {\tt Finished} state before {\tt k} can
execute; for instance, {\tt dependsOnKernel} could be re-initialized
and run again while {\tt k} is still waiting for other kernels.}
\param{
  {\tt k} - kernel that is dependent on another.\\
  {\tt dependsOnKernel} - kernel that must finish first.}
\req{
  {\tt k} is {\tt Unstarted}.\\
  {\tt dependsOnKernel} has been initialized.\\
  {\tt k} can read from a memory that {\tt dependsOnKernel} can write to.  That is, the processor executing\\ \mbox{~~~~~}{\tt k} and the processor executing {\tt dependsOnKernel} must both be connected to some common \\ \mbox{~~~~~}memory.  (This allows {\tt dependsOnKernel} to notify {\tt k} when the dependence is satisfied.)}

\callc
%%%%%%%%%%%%%%%%%%%%%%%%%
\makeline
\name{kernelReady}
\decl{void svm\_kernelReady(svm\_Kernel* k)}

\spec{This gives a hint to the Low Level Compiler to ready a kernel for execution.}

\param{
  {\tt k} - kernel to prepare for execution.
}

\req{
  {\tt k} is {\tt Unstarted}.}

\callc
%%%%%%%%%%%%%%%%%%%%%%%%%
\makeline
\name{kernelRun}
\decl{void svm\_kernelRun(svm\_Kernel* k)}

\spec{Non-blocking function that starts or resumes execution of a
kernel.  The kernel continues to run until its status is {\tt Paused}
or {\tt Finished}.}

\param{{\tt k} - kernel to run.}

\req{
  {\tt k} is {\tt Unstarted} or {\tt Paused}.\\
  Of the kernels that are currently {\tt Waiting}, {\tt Running}, or
  {\tt Paused}, none of them write to an output \\ \mbox{~~~~~}stream
  of {\tt k} or read from an input stream of {\tt k}.}

\callc
%%%%%%%%%%%%%%%%%%%%%%%%%
\makeline
\clearpage
\makeline
\name{kernelRunMultiplexed}
\decl{void svm\_kernelRunMultiplexed(svm\_Kernel* k, int N)}

\spec{Executes a pre-defined kernel on a multiplexed processor,
yielding every N iterations or when unable to perform a peek, pop, or
push (for a precise definition of ``iteration'', see
Section~\ref{sec:predef} and Appendix~\ref{sec:referenceimpl}.)  The
kernel continues to run until its status is {\tt Paused} or {\tt
Finished}.}

\param{
  {\tt k} - kernel to run.\\
  {\tt N} - number of iterations to run at a time.}

\req{
  {\tt k} is a pre-defined kernel.\\
  {\tt k} is {\tt Unstarted} or {\tt Paused}.\\
  Of the kernels that are currently {\tt Waiting}, {\tt Running}, or
  {\tt Paused}, none of them write to an output \\ \mbox{~~~~~}stream
  of {\tt k} or read from an input stream of {\tt k}.}

\callc
%%%%%%%%%%%%%%%%%%%%%%%%%
\makeline
\name{kernelPause}
\decl{void svm\_kernelPause(svm\_Kernel* k)}

\spec{Non-blocking function that interrupts execution of a kernel and
(if the kernel was {\tt Running}) sets the kernel's status to {\tt
Paused}.  If called from the control thread, the interruption is done
on a best-effort basis, and the control thread should follow up with a
call to {\tt svm\_kernelWait} to ensure that the kernel is paused.  If
called from within the kernel, the interruption is immediate ({\it
i.e.,} no other statement of the kernel is executed before being
paused.)  However, there is still a delay before the control thread
can recognize the {\tt Paused} state.\\}

\spec{If {\tt k} has a status of {\tt Paused} or {\tt Finished}, then
this call has no effect.}

\param{
  {\tt k} - kernel to pause.}

\req{
  {\tt k} is {\tt Running}, {\tt Paused}, or {\tt Finished}.\\
  If called from within a kernel, then that kernel is {\tt k}.  (Kernels may not pause other kernels.)}

\callkc
%%%%%%%%%%%%%%%%%%%%%%%%%
\makeline
\name{kernelEnd}
\decl{void svm\_kernelEnd(svm\_Kernel* k)}

\spec{Non-blocking function that interrupts execution of a kernel and
(if the kernel was {\tt Running}) sets the kernel's status to {\tt
Finished}.  If called from the control thread, the interruption is
done on a best-effort basis, and the control thread should follow up
with a call to {\tt svm\_kernelWait} to ensure that the kernel is finished.
If called from within the kernel, the interruption is immediate ({\it
i.e.,} no other statement of the kernel is executed before becoming
finished.)  However, there is still a delay before the control thread
can recognize the {\tt Finished} state.\\}

\spec{If the kernel is {\tt Paused}, then it is immediately changed to
{\tt Finished} (without executing another statement from within the
kernel.)  If the kernel is {\tt Finished}, then this call has no
effect.\\}

\spec{This function is implicitly called when a kernel returns from
its work function.}

\param{
  {\tt k} - kernel to end.}

\req{
  {\tt k} is {\tt Running}, {\tt Paused}, or {\tt Finished}.\\
  If called from within a kernel, then that kernel is {\tt k}.  (Kernels may not end other kernels.)}

\callkc
%%%%%%%%%%%%%%%%%%%%%%%%%
\makeline
\name{kernelWait}
\decl{void svm\_kernelWait(svm\_Kernel* k)}

\spec{Blocking function that does not return until the kernel's status
is {\tt Unstarted}, {\tt Paused} or {\tt Finished}.}

\param{
  {\tt k} - kernel to wait for.}

\req{{\tt k} has been initialized.}

\callc
%%%%%%%%%%%%%%%%%%%%%%%%%
\makeline
\name{kernelWaitMultiple}
\decl{void svm\_kernelWaitMultiple(int n, svm\_Kernel* k1, svm\_Kernel* k2, ...)}

\spec{Considers the set $K$ of kernels in {\tt k1}, {\tt k2}, {\tt
...}, {\tt kn} that have a status other than {\tt Unstarted}, and does
not return until one of the following is true: 1) at least one kernel
in $K$ is {\tt Paused}, 2) all kernels in $K$ are {\tt Finished}, or
3) $K$ is empty.\\}

\spec{Note that {\tt svm\_kernelWaitMultiple} cannot be emulated by a
sequence of calls to {\tt svm\_kernelWait}, as the former returns as soon
as any concurrently running kernel has paused.}

\param{
  {\tt n} - number of kernels.\\
  {\tt k1, k2, ..., kn} - kernels to wait for.}

\req{{\tt k1, k2, ..., kn} have been initialized.}

\callc
%%%%%%%%%%%%%%%%%%%%%%%%%
\makeline
\name{kernelGetStatus}
\decl{svm\_Kernel\_Status svm\_kernelGetStatus(svm\_Kernel* k)}

\spec{Returns the status of the kernel as best known by the control thread.}

\param{
  {\tt k} - kernel to inspect.}

\req{{\tt k} has been initialized.}

\callc
\makeline

%%%%%%%%%%%%%%%%%%%%%%%%%
%%%%%%%%%%%%%%%%%%%%%%%%%%%%%%%%%%%%%%%%%%%%%%%%%%%%%%%%%%%%%%%%%%%%%%%%%%%%%%%%%%%%%%%
%%%%  BEGIN FIGURE
%%%%%%%%%%%%%%%%%%%%%%%%%%%%%%%%%%%%%%%%%%%%%%%%%%%%%%%%%%%%%%%%%%%%%%%%%%%%%%%%%%%%%%%
\clearpage
\begin{figure}[t]
%\framebox[6.5in]{
\begin{minipage}{6in}
\vspace{-6pt}
\begin{center}
\psfig{figure=State.eps,width=5in}
\end{center}
\vspace{-16pt}
\caption{Legal transitions of a kernel's {\tt status} \protect\label{fig:kernel-status}}
\end{minipage}
%}
\vspace{28pt}

%\framebox[6.5in]{
\begin{minipage}{6.5in}
\begin{center}
\psfig{figure=timeline.eps,width=6.5in}
\end{center}
\vspace{-16pt}
\caption{Sample timeline for an interaction between the control thread and a kernel. \protect\label{fig:kernel-timeline}}
\end{minipage}
%}
\end{figure}
\clearpage
%%%%%%%%%%%%%%%%%%%%%%%%%%%%%%%%%%%%%%%%%%%%%%%%%%%%%%%%%%%%%%%%%%%%%%%%%%%%%%%%%%%%%%%
%%%%  END FIGURE
%%%%%%%%%%%%%%%%%%%%%%%%%%%%%%%%%%%%%%%%%%%%%%%%%%%%%%%%%%%%%%%%%%%%%%%%%%%%%%%%%%%%%%%

\sss{Kernel Execution Model}

Kernels have a simple execution model. A transition diagram for the
legal states of a kernel appears in Figure~\ref{fig:kernel-status},
while a timeline for a sample interaction with a kernel appears in
Figure~\ref{fig:kernel-timeline}.

A kernel is initially {\tt Unstarted}.  The control thread calls {\tt
kernelRun}, and the kernel becomes {\tt Waiting}.  When all kernels it
depends on (as explicitly indicated by the {\tt svm\_kernelAddDependence}
function) are {\tt Finished}, the kernel becomes {\tt Running}. The
kernel executes its {\tt work} function.  If either the control thread
or the work function calls {\tt svm\_kernelPause}, the kernel pauses in
executing the work function and becomes {\tt Paused}. It remains {\tt
Paused} until the control thread calls {\tt svm\_kernelRun} again. If
either the control thread or the work function calls {\tt svm\_kernelEnd}
at any time when the kernel is not {\tt Finished}, the kernel stops
executing the work function and becomes {\tt Finished}. The kernel
implicitly calls {\tt svm\_kernelEnd} when the work function returns. If
control flow causes a kernel to be initialized again, it becomes {\tt
Unstarted}.

Section~\ref{sec:control} gives a detailed explanation of how kernels
should be controlled.  However, we also give some brief examples here
to clarify the model of computation for kernels.  The most common
control sequence for initializing and running a kernel is as follows:

{\small
\begin{verbatim}
  static MyKernel k;

  svm_kernelInit(&k.kernel, ...);
  svm_kernelRun(&k.kernel);
  svm_kernelWait(&k.kernel);
\end{verbatim}}

In addition, multiple kernels can be run in parallel, with items being
transferred from one to the other using streams.  In this case, no
explicit dependences are needed.  For example:

{\small
\begin{verbatim}
  static svm_Stream s;
  static MyKernel1 k1;   // k1 will write to s
  static MyKernel2 k2;   // k2 will read from s

  svm_streamInitRAM(&s, ...);

  svm_kernelInit(&k1.kernel, ...);  
  svm_kernelRun(&k1.kernel);

  svm_kernelInit(&k2.kernel, ...);
  svm_kernelRun(&k2.kernel);

  svm_kernelWait(&k2.kernel);
\end{verbatim}}

Thus, the {\tt svm\_kernelAddDependence} function does NOT need to be
used to synchronize the transfer of data elements through a stream
object.  Instead, it is intended for cases in which there is a memory
dependence that is not captured by the stream semantics.  This can
occur in two cases: 1) a kernel writes to a block that a subsequent
kernel reads from, and 2) two kernels are reading/writing to streams
or blocks that point to overlapping regions in memory.  It would be
possible to use {\tt svm\_kernelWait} to enforce these dependences
from the control thread, but the dependence tracking mechanism is
added for performance reasons; with stream processing, a round-trip
communication to the control thread is often too expensive.  A simple
use of dependences is as follows:

{\small
\begin{verbatim}
  static svm_Block b;
  static MyKernel1 k1;   // k1 will write to b
  static MyKernel2 k2;   // k2 will read from b

  svm_blockInit(&b, ...);

  svm_kernelInit(&k1.kernel, ...);  
  svm_kernelRun(&k1.kernel);

  svm_kernelInit(&k2.kernel, ...);
  svm_kernelAddDependence(&k2.kernel, &k1.kernel);
  svm_kernelRun(&k2.kernel);

  svm_kernelWait(&k2.kernel);
\end{verbatim}}

Finally, there is a subtle point: the specification supports the use
of dependent kernels in a loop.  In this scenario, each kernel is run
multiple times, and each call to {\tt svm\_kernelAddDependence(k2,
k1)} makes {\tt k2} dependent on the last execution of {\tt k1},
whenever it occurred. If {\tt k1} is {\tt Unstarted}, the {\tt
svm\_kernelAddDependence} call is ignored. The following code
illustrates these concepts:

{\small
\begin{verbatim}
  while (...) {
    static K1 k1;
    static K2 k2;

    svm_kernelWait(&k1);
    svm_kernelInit(&k1.kernel, ...);
    svm_kernelAddDependence(&k1.kernel, &k2.kernel);
    svm_kernelRun(&k1.kernel);

    svm_kernelWait(&k2);
    svm_kernelInit(&k2.kernel, ...);
    svm_kernelAddDependence(&k2.kernel, &k1.kernel);
    svm_kernelRun(&k2.kernel);
  }
\end{verbatim}}

\subsubsection{User-Defined Kernels}
\label{sec:kernelhlc}

The kernel base data type is extended to different kinds of kernels by
declaring a new data type for each kernel which includes the kernel
base data type, an initialization function that calls the kernel
initialization function, and a work function that operates on the new
data type.

For example, a kernel for a simple amplifier could be as follows:
 
{\small
\begin{verbatim}
  typedef struct {
    svm_Kernel kernel;
    svm_IStream* in;
    svm_OStream* out;
    int N;
  } Amplifier;

  void amplifierWork(Amplifier* amp);

  void amplifierInit(Amplifier* amp, mm_Proc procLocation, svm_IOBlock* scratch, 
                     svm_IStream* _in, svm_OStream* _out, int _N) {
      svm_kernelInit(&amp->kernel, procLocation, scratch, 
        amp, sizeof(Amplifier), (svm_ExtKernelWork)&amplifierWork);
      amp->in = _in;
      amp->out = _out;
      amp->N = _N;
  }
 
  void amplifierWork(Amplifier* amp) {
      float x;
      while (!streamPeekEOS(amp->in, 0)) {
        svm_streamPop(amp->in, &x);
        x = x * amp->N;
        svm_streamPush(amp->out, &x);
      }
      svm_streamPop(amp->in, &x);
      x = x * amp->N;
      svm_streamPushWithEOS(amp->out, &x);
  }
\end{verbatim}}

\noindent If the input stream is known to be infinite, then the work
function can be simplified as follows:

{\small
\begin{verbatim}
  void amplifierWork(Amplifier* amp) {
      while(true) {
        float x;
        svm_streamPop(amp->in, &x);
        x = x * amp->N;
        svm_streamPush(amp->out, &x);
      }
  }
\end{verbatim}}

\begin{figure}[t]
\vspace{-12pt}
\framebox[6.5in]{
\begin{minipage}{6in}
\begin{enumerate}

\item No pointers, except for those directly passed to stream, block,
and kernel functions as described in this document.

\item No dynamic memory allocation.

\item No accesses to global variables.

\item No GOTO statements (all control flow is structured).

\item No recursive functions (all function calls have inline semantics).

\item No calls to functions that violate any of these restrictions.

\item Supported opcodes are only the logical, arithmetic, and boolean
operations found in C (no special-purpose DSP operations at this
time\footnote{DSP operations may be added at a future date pending
further discussion by the forum.}).

\item Supported primitive types are indicated by the set of {\tt
mm.Proc.Supports} fields in the machine model (e.g., the {\tt
mm.Proc.SupportsInt} field indicates whether or not integers are
supported); there is also support for arrays with a fixed (int
literal) length and {\tt struct}'s containing members of any other
type.

\end{enumerate}

\caption{Restrictions on C code within kernels.\protect\label{fig:restrict}}
\end{minipage}}
\end{figure}

\sss{User-defined Kernel Restrictions}

The initialization function must statically associate the input and
output streams with a one-to-one mapping from arguments to streams,
and no surrounding control flow or condition statements that can
change the assignment.  This ensures that the Low Level Compiler can
statically calculate which streams are accessed from the work
function.

Also, the work function supports only a subset of C; restrictions are
listed in Figure~\ref{fig:restrict}.

\subsubsection{Pre-Defined Kernels}
\label{sec:kernelsvm}

\label{sec:predef}

Special pre-defined kernels are used to move data between streams and
blocks. These kernels are typically executed by DMA processors, but
may be executed by stream processors on some architectures.  A
reference implementation of the pre-defined kernels can be found in
Appendix~\ref{sec:referenceimpl}.

%%%%%%%%%%%%%%%%%%%%%%%%%
\clearpage
\makeline
\name{Constants}
\decl{const int svm\_Stream\_Length\_All = -1;}

\spec{Indicates that a pre-defined kernel should execute until it
finds an EOS in its input stream.}  
%%%%%%%%%%%%%%%%%%%%%%%%%
\makeline
\name{Move}
{\small
\begin{verbatim}
typedef struct {
  svm_Kernel kernel;
  // not exposed
} svm_Move;

void svm_moveInit(svm_Move* move, mm_Proc location, svm_IStream* srcStr, 
                  svm_OStream* destStr, size_t length, boolean setEOS);
\end{verbatim}}

\spec{The primary use of this kernel is for moving elements between
streams in different memory banks, though it can also be used for
moving between streams in a single memory.  A given number of elements
are popped off of {\tt srcStr} and pushed onto {\tt destStr}, with EOS
tags preserved.  If desired, an EOS tag is also added to the last
element pushed.  Each iteration of this kernel moves a single item.\\}

\spec{Note that {\tt svm\_moveInit} function is only the initialization
routine (as in setting up a DMA processor); it does not do the data
move itself.}

\param{
  {\tt move} - kernel to initialize.\\
  {\tt location} - processor resource on which to execute the kernel (from PCA Machine Model.)\\
  {\tt srcStr} - source stream, from which elements are popped.\\
  {\tt destStr} - destination stream, onto which elements are pushed.\\
  {\tt length} - number of elements to move; {\tt svm\_Stream\_Length\_All}
  executes until an element with an \\ \mbox{~~~~~}EOS tag is popped from {\tt
  srcStr}.\\
   {\tt setEOS} - if true, then the last element pushed onto {\tt
   destStr} is given an EOS tag.  Otherwise \\ \mbox{~~~~~}no element pushed is
   tagged with an EOS.}

\req{
  {\tt move} is {\tt Unstarted}, {\tt Finished}, or uninitialized.\\
  {\tt srcStr} and {\tt destStr} have been initialized.\\
  {\tt srcStr} and {\tt destStr} have the same {\tt elementSize}.\\
  {\tt length} $>$ 0 or {\tt length} = {\tt svm\_Stream\_Length\_All}.}
      
\callc
\makeline
%%%%%%%%%%%%%%%%%%%%%%%%%
\name{StridedScatter}
{\small
\begin{verbatim}
typedef struct {
  svm_Kernel kernel;
  // not exposed
} svm_StridedScatter;

void svm_stridedScatterInit(svm_StridedScatter* scatter, mm_Proc location, 
                            svm_IStream* srcStr, svm_OBlock* destBlock, size_t length,
                            size_t destStride, size_t elementsPerStride)
\end{verbatim}}

\spec{Moves segments of {\tt elementsPerStride} elements from a
continuous input stream to strided locations of a destination block.
The distance between strides in the destination ({\tt destStride})
might be greater than the number of elements moved with each stride
({\tt elementsPerStride}), in which case the elements moved appear at
the beginning of each stride.  Each iteration of this kernel
processes a single stride.}

\param{
  {\tt scatter} - kernel to initialize.\\
  {\tt location} - processor resource on which to execute the kernel (from PCA Machine Model.)\\
  {\tt srcStr} - source stream, from which elements are popped.\\
  {\tt destBlock} - destination block, into which elements are written.\\
  {\tt length} - total number of elements to move; {\tt svm\_Stream\_Length\_All} executes until an \\ \mbox{~~~~~}element with an EOS tag is
  popped from {\tt srcStr}.\\
  {\tt destStride} - number of elements between the start of adjacent strides in {\tt destBlock}.\\
  {\tt elementsPerStride} - number of elements that are moved from {\tt srcStr} for each stride in {\tt destBlock}.}

\req{
  {\tt scatter} is {\tt Unstarted}, {\tt Finished}, or uninitialized.\\
  {\tt srcStr} and {\tt destBlock} have been initialized.\\
  {\tt srcStr} and {\tt destBlock} have the same {\tt elementSize}.\\
  {\tt length} $>$ 0 or {\tt length} = {\tt svm\_Stream\_Length\_All}.\\
  All accesses to {\tt destBlock} are within its capacity.}

\callc
%%%%%%%%%%%%%%%%%%%%%%%%%
\makeline
\name{StridedGather}
{\small
\begin{verbatim}
typedef struct {
  svm_Kernel kernel;
  // not exposed
} svm_StridedGather;

void svm_stridedGatherInit(svm_StridedGather* gather, mm_Proc location, 
                           svm_IBlock* srcBlock, svm_OStream* destStr, size_t length,
                           size_t srcStride, size_t elementsPerStride, boolean setEOS)
\end{verbatim}}

\spec{Copies segments of {\tt elementsPerStride} elements from strided
locations of an input block to a continuous destination stream.  The
distance between strides ({\tt srcStride}) might be greater than the
number of elements copied from each stride ({\tt elementsPerStride}),
in which case the elements are copied from the beginning of the
stride.  Each iteration of this kernel processes a single stride.}

\param{
  {\tt gather} - kernel to initialize.\\
  {\tt location} - processor resource on which to execute the kernel (from PCA Machine Model.)\\
  {\tt srcBlock} - source block, from which elements are copied.\\
  {\tt destStr} - destination stream, onto which elements are pushed.\\
  {\tt length} - total number of elements to copy; {\tt svm\_Stream\_Length\_All} considers all elements of \\\mbox{~~~~~}{\tt srcBlock}.\\
  {\tt srcStride} - number of elements between the start of adjacent strides in {\tt srcBlock}.\\
  {\tt elementsPerStride} - number of elements that are copied from start of each stride.\\
   {\tt setEOS} - if true, then the last element pushed onto {\tt destStr} is given an EOS tag.  Otherwise \\\mbox{~~~~~}no element pushed is tagged with an EOS.}

\req{
  {\tt gather} is {\tt Unstarted}, {\tt Finished}, or uninitialized.\\
  {\tt srcBlock} and {\tt destStr} have been initialized.\\
  {\tt srcBlock} and {\tt destStr} have the same {\tt elementSize}.\\
  {\tt length} $>$ 0 or {\tt length} = {\tt svm\_Stream\_Length\_All}.\\
  All accesses to {\tt srcBlock} are within its capacity.}

\callc
%%%%%%%%%%%%%%%%%%%%%%%%%
\makeline
\clearpage
\makeline
\name{IndexedScatter}
{\small
\begin{verbatim}
typedef struct {
  svm_Kernel kernel;
  // not exposed
} svm_IndexedScatter;

void svm_indexedScatterInit(svm_IndexedScatter* scatter, mm_Proc location,
                            svm_IStream* srcStr, svm_IStream* indexStr, svm_OBlock* destBlock, 
                            size_t length, size_t elementsPerIndex)
\end{verbatim}}

\spec{This kernel allows irregular scattering of elements from an
input stream to an output block, using an index stream to determine
the locations where elements are written.  For each element of the
index stream, {\tt elementsPerIndex} elements of the source stream are
moved to successive locations of the destination block (starting at
the given index).  Elements are popped from both the source and index
streams as they are scattered.  Each iteration of this kernel
processes a single index.}

\param{
  {\tt scatter} - kernel to initialize.\\
  {\tt location} - processor resource on which to execute the kernel (from PCA Machine Model.)\\
  {\tt srcStr} - source stream, from which elements are moved.\\
  {\tt indexStr} - index stream, containing the locations of {\tt destBlock} to which elements should be written.\\
  {\tt destBlock} - destination block, into which elements are written.\\
  {\tt length} - total number of elements to copy; {\tt svm\_Stream\_Length\_All} executes until an element with \\ \mbox{~~~~~}an EOS tag is  popped from {\tt indexStr}.\\
  {\tt elementsPerIndex} - number of elements to move from {\tt srcStr} for each element of {\tt indexStr}.}

\req{
  {\tt scatter} is {\tt Unstarted}, {\tt Finished}, or uninitialized.\\
  {\tt srcStr}, {\tt indexStr}, and {\tt destBlock} have been initialized.\\
  {\tt srcStr} and {\tt destBlock} have the same {\tt elementSize}.\\
  {\tt length} $>$ 0 or {\tt length} = {\tt svm\_Stream\_Length\_All}. \\
  {\tt indexStr} uses an unsigned representation for the indices.\\
  All accesses to {\tt destBlock} are within its capacity.}

\callc
%%%%%%%%%%%%%%%%%%%%%%%%%
\makeline
\name{IndexedGather}
{\small
\begin{verbatim}
typedef struct {
  svm_Kernel kernel;
  // not exposed
} svm_IndexedGather;

void svm_indexedGatherInit(svm_IndexedGather* gather, mm_Proc location, 
                           svm_IBlock* srcBlock, svm_IStream* indexStr, svm_OStream* destStr, 
                           size_t length, size_t elementsPerIndex, boolean setEOS)
\end{verbatim}}

\spec{This kernel allows irregular gathering of elements from an input
block to an output stream, using an index stream to determine the
locations where elements are read.  For each element of the index
stream, {\tt elementsPerIndex} successive elements of the source block
(starting at the given index) are copied into the destination stream.
Elements are popped from the index stream as they are gathered.  Each
iteration of this kernel processes a single index.}

\param{
  {\tt gather} - kernel to initialize.\\
  {\tt location} - processor resource on which to execute the kernel (from PCA Machine Model.)\\
  {\tt srcBlock} - source block, from which elements are copied.\\
  {\tt indexStr} - index stream, containing the locations of {\tt srcBlock} from which elements should be read.\\
  {\tt destStr} - destination stream, onto which elements are pushed.\\
  {\tt length} - total number of elements to copy; {\tt svm\_Stream\_Length\_All} executes until an element with \\ \mbox{~~~~~}an EOS tag is popped from {\tt indexStr}.\\
  {\tt elementsPerIndex} - number of elements to copy from {\tt
  srcBlock} for each element of {\tt indexStr}.\\
  {\tt setEOS} - if true, then the last element pushed onto {\tt destStr} is given an EOS tag.  Otherwise \\ \mbox{~~~~~}no element pushed is tagged with an EOS.}

\req{
  {\tt gather} is {\tt Unstarted}, {\tt Finished}, or uninitialized.\\
  {\tt srcBlock}, {\tt indexStr} and {\tt destStr} have been initialized.\\
  {\tt srcBlock} and {\tt destStr} have the same {\tt elementSize}.\\
  {\tt length} $>$ 0 or {\tt length} = {\tt svm\_Stream\_Length\_All}. \\
  {\tt indexStr} uses an unsigned representation for the indices.\\
  All accesses to {\tt srcBlock} are within its capacity.}

\callc
\makeline
%%%%%%%%%%%%%%%%%%%%%%%%%

\subsubsection{Black-Box Library Kernels}
\label{sec:kernelllc}

The API supports library routines via black-box kernels that are
defined in the PCA Machine Model and then instantiated by name in the
stream control code.  These kernels are identical to user-defined
kernels, except that their {\tt work} functions are not defined;
instead, they are recognized by the Low Level Compiler for the
architecture and are translated into hand-optimized library code.

To ensure that the Low Level Compiler can recognize a black-box
kernel, the High Level Compiler guarantees that it will not merge
black-box kernels with other kernels, or rename a black-box kernel so
that it would become unrecognizable to the Low Level Compiler.

\subsection{Control}
\label{sec:control}

Threaded code is used to declare streams, blocks, and kernels and to
control the execution of kernels.  A given kernel is controlled
exclusively by a single thread; results from a kernel's execution can
then be communicated between threads ({\it e.g.,} by using shared
memory or streams.)  Also, to simplify initialization rules, all
stream, kernel, and block variables must be declared as static
variables.

An example follows:

{\small
\begin{verbatim}
  int length1, length2;
  static svm_Stream s0, s1, s2, s4;
  static svm_Block scratch;
  static svm_Move move01, move24;
  static RLE rle; // source code for the user-defined RLE kernel is not included in this example

  // read file into memory location of s0 (not part of API)
  length1 = readFile("input.dat", GLOBALMEM1, 0x1000, 1024);

  svm_streamInitWithDataRAM(&s0, GLOBALMEM1, 0x1000, 1024, 4, length1, TRUE, length1, svm_Stream_Never_Wraps);
  svm_streamInitRAM(&s1, LOCALMEM1, 0x0, 256, 4, 256, svm_Stream_Unaliased_RAM);
  svm_streamInitRAM(&s2, LOCALMEM1, 0x100, 128, 4, 128, svm_Stream_Never_Wraps);
  svm_blockInit(&scratch, LOCALMEM1, 0x180, 32, 1);

  // move from memory
  svm_moveInit(&move01, DMA1, &s0, &s1, svm_Stream_Length_All, TRUE);
  svm_kernelRun(&move01.kernel);
  
  // run rle (source for rleInit not included in this example)
  rleInit(&rle, PROC1, &scratch, &s1, &s2);
  svm_kernelRun(&rle.kernel);

  // get output length
  svm_kernelWait(&rle.kernel);
  length2 = rle.outputLength;
  
  // if the output is still too large, run additional compression, overwriting s2 in place
  if (length2 > SIZE_THRESHOLD) {
    static svm_Stream s3;
    static svm_Block scratch2;
    static Zip zip; // source code for the user-defined Zip kernel is not included in this example

    svm_streamInitRAM(&s3, LOCALMEM1, 0x100, 128, 4, 128, svm_Stream_Never_Wraps);
    svm_blockInit(&scratch2, LOCALMEM1, 0x180, 32, 1);

    // run zip kernel (source for zipInit not included in this example)
    zipInit(&zip, PROC1, &scratch2, &s2, &s3);
    svm_kernelRun(&zip.kernel);
    
    // get output length
    svm_kernelWait(&zip.kernel);
    length2 = zip.outputLength;

    svm_streamInitWithDataRAM(&s2, LOCALMEM1, 0x100, 128, 4, length2, TRUE, 128, svm_Stream_Never_Wraps);
  }
  svm_streamInitRAM(&s4, GLOBALMEM1, 0x2000, 1024, 4, 1024, svm_Stream_Never_Wraps),
  
  // move to memory
  svm_moveInit(&move24, DMA1, &s2, &s4, svm_Stream_Length_All, TRUE);
  svm_kernelRun(&move24.kernel);
  svm_kernelWait(&move24.kernel);
  // store the result from memory to "output.dat" (not part of API)
  writeFile("output.dat", GLOBALMEM1, 0x2000, length2); 
\end{verbatim}}

The above code fragment illustrates several aspects of the stream
control API.  In the rest of this section, we describe the
capabilities and limitations of the stream control API.

\sss{Transferring Data Between Kernels}

There are two ways to transfer the outputs of one kernel to the inputs
of another.  Perhaps the most natural way is to reuse the same stream,
thereby carrying over the results; in our example, stream {\tt s2} is
used by multiple kernels.

The other way to transfer elements is by allocating a new input stream
that overlaps with the output stream in memory.  In this case, the
control code should indicate that the kernel reading the new stream is
dependent on the kernel that wrote the original stream; this is done
by use of the kernel's {\tt svm\_kernelAddDependence} function, thereby
saving the Low Level Compiler from doing location-based memory
analysis to discover dependences between kernels.

\sss{Transferring Data Between Threaded Code and Kernels}

Before kernel execution, streams can be initialized with (possibly
non-streaming) data from general-purpose threaded code.  Likewise, the
results of a streaming computation can be used in the threaded code
following the kernel's execution.  Both of these transfers are done
through memory, either by directly accessing the memory assigned to a
stream and then using memory management kernels (as in the example),
or by calling stream I/O functions from the control code.  The control
thread can rely on the sequential data layout guaranteed by streams
when managing this communication (see Section~\ref{sec:streams}).
Also note that most device I/O (such as file handling and terminal
interaction) is done by threaded code, and then made available to
streams through memory.  However, for direct interaction with streams,
I/O devices can be described as processors that support a single
black-box kernel and hence I/O data does not need to move through
memories.

It is also possible for the control processor to inspect and modify
the data members of a kernel when it is not in the {\tt Waiting} or
{\tt Running} states. Accesses to kernel fields are useful for passing
parameters to a kernel or retrieving reduction values from a kernel.
For example:

{\small
\begin{verbatim}
  // --- kernel code ---
  typedef struct {
    svm_Kernel kernel;
    svm_IStream* in;
    int sum;
  } SumKernel;

  void sumKernelWork();
  
  sumInit(SumKernel* k, mm_Proc procLocation, svm_IOBlock* scratch, svm_IStream* _in) {
    svm_kernelInit(&k->kernel, procLocation, scratch,
      k, sizeof(SumKernel), (svm_ExtKernelWork)&sumKernelWork); 
    k->in = _in;
    k->sum = 0;
  }

  void sumKernelWork(SumKernel* k) {
      int x;
      while (!svm_streamPeekEOS(k->in, 0)) {
          svm_streamPop(k->in, &x);
          k->sum += x;
      }
      svm_streamPop(k->in, &x);
      k->sum += x;
  }

  // --- control code ---
  static svm_Stream s1;
  static svm_Block scratch1;
  static SumKernel sk;
  int finalSum;

  // assumes prior code initializes data in RAM
  svm_streamInitWithDataRAM(&s1, LOCALMEM1, 0x100, 128, 4, 128, TRUE, 128, 0);

  // run sum kernel
  sumInit(&sk, PROC1, &scratch1, &s1);
  svm_blockInit(&scratch1, LOCALMEM1, 0x180, 16, 1);
  svm_kernelRun(&sk.kernel);

  // get output sum
  svm_kernelWait(&sk.kernel);
  finalSum = sk.sum; 
\end{verbatim}}

\sss{FIFO Cleanup}
\label{sec:fifoclean}

It is the responsibility of the High Level Compiler to ensure that all
elements are drained from FIFOs before a fresh set of streams and
kernels are mapped to the resources.  In the common case, a FIFO is
empty when the neighboring kernels terminate, as the kernels execute
until they push or pop an item with an EOS tag.  However, in
exceptional cases, a kernel might finish before reaching the EOS, and
leftover items need to be cleared from a FIFO before additional
kernels are executed.

There are many ways in which this draining functionality can be
implemented using the primitives described in this document.  An
in-depth example appears in Appendix~\ref{sec:fifocleanexample}.

\sss{Managing Kernel Instruction Memory}

The High Level Compiler might anticipate upcoming kernel
executions. The {\tt svm\_kernelReady} function allows this
information to be transferred to the Low Level Compiler so that the
kernel can be loaded into the processor's instruction memory ahead of
time.

\subsubsection{Restrictions on Stream Control}

There are restrictions on the stream control code to ensure a valid
hardware mapping and allow simple analysis by the Low Level
Compiler. However, there are no restrictions on non-streaming
statements, which can be freely mixed with the stream statements;
these statements can be arbitrarily complex threaded code.

In order to represent a valid hardware mapping, there are several restrictions:

\renewcommand{\labelenumi}{A\theenumi.}

\begin{enumerate}

\item Only one user-defined kernel may be {\tt Running} or {\tt
Paused} on a stream processor at any given time.

\item A number of pre-defined memory management kernels may be {\tt
Running} or {\tt Paused} on a DMA processor at any given time.  The
maximum number is given by the {\tt mm.Proc.Multiplexing} field in the
PCA Machine Model.

\item A kernel may only read from or write to streams that are mapped
to memories or FIFOs connected to the processor executing the kernel
(there must exist a series of links with no intervening processors or
memories that connects the processor running the kernel to the memory
or FIFO that contains the stream.)

\item Only one stream mapped to a hardware FIFO may retain state at
any given time. The SVM code should ensure that the state is empty
before calling an initialization function that maps any stream to the
FIFO.  The state is empty if there are no data elements in the FIFO.

\end{enumerate}

\noindent In order to facilitate static analysis in the Low Level
Compiler, there are a number of restrictions:

\renewcommand{\labelenumi}{B\theenumi.}

\begin{enumerate}

\item If there are re-initializations of a given kernel instance, then
the same stream and block arguments must be passed to each invocation
of the initialization function.

\item It is illegal to take the address of a stream, block, or kernel
variable, except to pass it directly to a function described in this
document. Thus, these variables cannot be aliased other than streams
and blocks within extensions of the kernel base data type.

\item Stream, block, and kernel variables may not be part of
structures except for extensions of the kernel base data type.

\item Any function that contains a stream, block, or kernel variable
must NOT be called recursively or through a function pointer, even
indirectly (e.g. cannot be called from a recursive function, or called
from a function called from a recursive function, etc.).

\item A given kernel may be controlled by only one thread.  This
allows a thread to maintain a consistent view of the status of each
kernel it is controlling.

\end{enumerate}

\subsection{Error Handling}
\label{sec:errors}

If any of the above restrictions are violated, or if there is an
operation that is disallowed by the API ({\it e.g.,} peeking beyond
the capacity of a stream) then the behavior is undefined.  All error
detection, error handling, and error recovery are defined by the
implementation of the Low Level Compiler.  It is recommended that each
Low Level Compiler provides a mode where it checks assertions on the
restrictions above and fails cleanly (with an assertion error) if a
restriction is violated.  However, these assertions could be turned
off to obtain high performance if the Low Level Compiler trusts that
the High Level Compiler generated safe code.

Note that the potential for undefined behavior is introduced for the
sake of performance.  If each architecture were required to verify the
restrictions above, the overhead would be prohibitive.

\subsection{Concept of Operations}
\label{sec:conops}

The following is an incomplete list of SVM/HLC/LLC expectations:

\renewcommand{\labelenumi}{\theenumi.}

\begin{enumerate}

\item One or more Thread processors may be working cooperatively on a
single program.

\item Streaming operations are performed by kernels that accept zero
or more input streams and produce zero or more output streams.  Stream
functions can also be called from control code.

\item Streams are sequences of data items, that may be stored in
memories or FIFOs.

\item Multiple user-defined kernels can be assigned to a streaming
processor, but only one {\tt Running} user-defined kernel can be
assigned; the rest must be {\tt Waiting} for a previous kernel to
complete.  However, multiple pre-defined kernels can be {\tt Running}
on the same processor, up to the limit specified by the {\tt
mm.Proc.Multiplexing} field in the PCA Machine Model.

\item At any given time, each {\tt Stream} has a single producer and a
single consumer.  However, the contents of a {\tt Stream} stored in
memory can be aliased such that several different {\tt Stream}s can
read the same values.

\item The machine model describes the available resource pool that the
HLC can work with.  This may not be the hardware's full capability -
some of the resources may be assigned to other programs outside the
compiler's immediate view.  The machine model may describe a multichip
environment.

\item Multiple programs may be running on the PCA hardware,
concurrently.

\item The PCA elements that control the morphing of the processor both
set the machine model resource environment for a compiled program and
apply directions from the HLC as to how the morphed assets are to be
used ({\it i.e.,} which morph configuration was used for the specific
compiled program).

\item The HLC assesses the alternative morph configurations as defined
in the machine model to choose the preferred set of resources to use
for the compilation.

\item A cost function could be provided for each of the SVM APIs as
for each resource instance - thus allowing the HLC to assess which
logical entities to assign to each resource.  The details of this cost
estimation will be addressed in the future.

\end{enumerate}

\appendix

\clearpage
\section{FIFO Cleanup Example}  % use *-form to suppress numbering
\label{sec:fifocleanexample}
As described in Section~\ref{sec:fifoclean}, the High Level Compiler
is responsible for draining FIFOs before a fresh set of streams and
kernels are mapped to the resources.  In the common case, FIFOs will
be drained by the kernels themselves.  However, if a kernel terminates
before consuming all data elements in the FIFOs, the control thread
will have to take additional steps to ensure that the FIFOs are clear
before additional kernels can be run.

For instance, consider a producer kernel {\tt k1} that communicates
through a FIFO stream {\tt s} to a consumer kernel {\tt k2}.  The
application specifies that {\tt k2} might terminate abnormally (before
the end of stream), in which case {\tt k1} should be automatically
interrupted and {\tt s} should be drained.  The High Level Compiler
can support these semantics by testing the status of {\tt k1} upon the
termination of {\tt k2} and running custom draining kernels if
necessary.  

The following code implements this example.  It works by using two
kernels to replace {\tt k1} and {\tt k2}.  First, a {\tt PopKernel}
replaces {\tt k2} and executes one pop at a time (pausing in between)
until {\tt k1} has finished.  Then, a {\tt PushKernel} is started in
place of {\tt k1}, which pushes a fixed number of dummy items (to
clear leftover junk items from {\tt k1}) before pushing a single item
with an EOS tag.  Finally, {\tt PopKernel} is modified to enter a
``bounded remainder mode'' in which it pops into the dummy buffer of
{\tt PushKernel} before looking for the single item with an EOS tag.
Once it finds this EOS item, both kernels terminate naturally and the
stream is drained.

{\small
\begin{verbatim}
  // ------------- control code -------------

  // assume that:
  //   k1 is running on processor P1
  //   k2 is running on processor P2
  //   s has a capacity of "c"

  // upstream draining kernel
  static PushKernel pushKernel;
  // downstream draining kernel
  static PopKernel popKernel;

  // start k1 and k2 running
  svm_kernelRun(&k1.kernel)
  svm_kernelRun(&k2.kernel)

  // wait for k2 to finish
  svm_kernelWait(&k2.kernel);

  // if k1 did not finish, run the drain procedure
  if (svm_kernelGetStatus(&k1.kernel)==svm_Kernel_Running) {

    svm_kernelEnd(&k1.kernel);

    // start PopKernel running in its "unbounded remainder mode", in which
    // it pauses itself after each item that it pops.
    PopKernelInit(&popKernel, P2, &s1, c, FALSE);
    svm_kernelRun(&popKernel.kernel);

    // pop one item at a time from s until k1 has finished
    do {
      svm_kernelWaitMultiple(&k1.kernel, &popKernel.kernel);
      if (svm_kernelGetStatus(&popKernel.kernel)==svm_Kernel_Paused) {
        svm_kernelRun(&popKernel.kernel);
      }
      if (svm_kernelGetStatus(&k1.kernel)==svm_Kernel_Paused) {
        svm_kernelRun(&k1.kernel);
      }
    } while (svm_kernelGetStatus(&k1.kernel)!=svm_Kernel_Finished);
    
    // start PushKernel running
    PushKernelInit(&pushKernel, P1, &s1, c);
    svm_kernelRun(&pushKernel.kernel);


    // change the PopKernel's state to be in "bounded remainder mode", in which
    // it will pop c items and then keep popping until it finds an EOS tag
    svm_kernelWait(&popKernel.kernel);
    popKernel.boundedRemainder = TRUE;
    svm_kernelRun(&popKernel.kernel);

    // wait for PopKernel to finish
    svm_kernelWait(&popKernel.kernel);

    // at this point, we are guaranteed that both kernels have finished, 
    // and that the stream s has been completely drained
    ASSERT(svm_kernelGetStatus(&popKernel.kernel)==svm_Kernel_Finished);
    ASSERT(svm_kernelGetStatus(&pushKernel.kernel)==svm_Kernel_Finished);
  }

  // ------------- kernel code for PopKernel -------------

  typedef struct {
    svm_Kernel kernel;
    svm_IStream* in;
    int capacity;
    boolean boundedRemainder;
  } svm_PopKernel;

  void popKernelWork();
  
  pushInit(PopKernel* k, mm_Proc procLocation, svm_IStream* _in, int _capacity, boolean _boundedRemainder) {
    svm_kernelInit(&k->kernel, procLocation, 0, k, sizeof(PopKernel), (svm_ExtKernelWork)&popKernelWork); 
    k->in = _in;
    k->capacity = _capacity;
    k->boundedRemainder = _boundedRemainder;
  }

  void popKernelWork(PopKernel* k) {
    int i, x;
    boolean done = FALSE;
    do {
      if (k->boundedRemainder) {
        // if we have a bounded remainder, first pop as many items
        // as will fit in the stream, as these could be junk leftovers 
        // from the original upstream kernel.
        for (i = 0; i < k->capacity; i++) {
          svm_streamPop(&k->in, &x);
        }
        done = FALSE;
        // then, pop until we find the EOS tag that the pushKernel output
        do {
          done = svm_streamPeekEOS(&k->in, 0);
          svm_streamPop(&k->in, &x);
        } while (!done);
      } else {
        // if we do not have a bounded remainder, then pop one item and pause
        svm_streamPop(&k->in, &x);
        svm_kernelPause(&k->kernel);
      }
    } while (!done);
  }

  // ------------- kernel code for PushKernel -------------

  typedef struct {
    svm_Kernel kernel;
    svm_OStream* out;
    int capacity;
  } PushKernel;

  void pushKernelWork();
  
  sumInit(PushKernel* k, mm_Proc procLocation, svm_OStream* _out, int _capacity) {
    svm_kernelInit(&k->kernel, procLocation, 0, k, sizeof(PushKernel), (svm_ExtKernelWork)&pushKernelWork); 
    k->out = _out;
    k->capacity = _capacity;
  }

  void pushKernelWork(PushKernel* k) {
    int i, dummy = 0;
    // push as many dummy items as would fill up the stream, in case
    // the original kernel did not leave any junk leftovers for PopKernel 
    // to consume
    for (i = 0; i < k->capacity; i++) {
      svm_streamPush(&k->kernel, &dummy);
    }
    // push a single item with an EOS tag, indicating the true end of stream
    svm_streamPushEOS(&k->kernel, &dummy);
  }
\end{verbatim}}

The above procedure is an example of a basic draining operation that
can be constructed using the primitives in the SVM.  The High Level
Compiler should employ similar mechanisms in other scenarios to ensure
that it always has a consistent view of FIFO contents.

\clearpage
\section{Pre-Defined Kernels:  Reference Implementation}  % use *-form to suppress numbering
\label{sec:referenceimpl}

This section provides a reference implementation for pre-defined
kernels, with the intent of clarifying any ambiguities in the
specification of Section~\ref{sec:predef}.  Towards this end,
high-level SVM primitives are used to provide a legal
implementation\footnote{This implementation also uses dynamic memory
allocation within the work functions of kernels, which is prohibited
by the SVM.  A Low Level Compiler can avoid dynamic allocation by
inferring the {\tt elementSize} of the streams and blocks in a given
application.} for each pre-defined kernel.  This is only for the sake
of illustration; in most cases, pre-defined kernels should be
implemented natively by the Low Level Compiler to target DMA engines
or other specialized hardware.

Note also that the following implementation gives a precise definition
of the ``iteration'' for each pre-defined kernel, as referenced by the
{\tt svm\_kernelRunMultiplexed} function.  In each kernel's work
function, the {\tt iteration} variable marks the current iteration.

\makeline
\name{Move}
{\small
\begin{verbatim}
  typedef struct {
    svm_Kernel kernel;
    svm_IStream* srcStr;
    svm_OStream* destStr;
    size_t elementSize;
    size_t length;
    boolean setEOS;
  } svm_Move;

  void svm_moveInit(svm_Move* move, mm_Proc location, svm_IStream* _srcStr, 
                    svm_OStream* _destStr, size_t _length, boolean _setEOS) {
    kernelInit(&move->kernel, location, 0, move, sizeof(svm_Move), (ExtKernelWork)&svm_moveWork);
    move->srcStr = _srcStr;
    move->destStr = _destStr;
    move->elementSize = _elementSize;
    move->length = _length;
    move->setEOS = _setEOS;
  }

  void svm_moveWork(svm_Move* move) {
    size_t iteration;
    void* element = malloc(move->elementSize);
    for (iteration=0; iteration<move->length; iteration++) {
      boolean eos = svm_streamPeekEOS(move->srcStr, 0);
      svm_streamPop(move->srcStr, element);
      // push item with EOS tag if source has tag, or if setEOS and at end of stream
      if (eos || (move->setEOS && iteration==move->length-1)) {
        svm_streamPushEOS(move->destStr, element);
        if (move->length == svm_Stream_Length_All) {
          // if we're set to all, quit after pushing an EOS item
          break;
        }
      } else {
        svm_streamPush(move->destStr, element);
      }
    }
    free(element);
  }
\end{verbatim}}

\clearpage
\makeline
\name{StridedScatter}
{\small
\begin{verbatim}
  typedef struct {
    svm_Kernel kernel;
    svm_IStream* srcStr;
    svm_OBlock* destBlock;
    size_t length;
    size_t destStride;
    size_t elementsPerStride;
  } svm_StridedScatter;

  void svm_stridedScatterInit(svm_StridedScatter* scatter, mm_Proc location, 
                              svm_IStream* _srcStr, svm_OBlock* _destBlock, size_t _length,
                              size_t _destStride, size_t _elementsPerStride) {
    kernelInit(&scatter->kernel, location, 0, scatter, sizeof(svm_StridedScatter), 
               (ExtKernelWork)&svm_stridedScatterWork);
    scatter->srcStr = _srcStr;
    scatter->destBlock = _destBlock;
    scatter->length = _length;
    scatter->destStride = _destStride;
    scatter->elementsPerStride = _elementsPerStride;
  }

  void svm_stridedScatterWork(svm_StridedScatter* scatter) {
    size_t iteration, j, num_items = 0;
    boolean done_stream_all = false;
    void* element = malloc(scatter->elementSize);
    for (iteration=0; num_items<scatter->length && !done_stream_all; iteration++) {
      for (j=0; 
           j<scatter->elementsPerStride && num_items<scatter->length && !done_stream_all; 
           j++) {
        done_stream_all = scatter->length == svm_Stream_Length_All && 
                          svm_streamPeekEOS(scatter->srcStr, 0);
        svm_streamPop(scatter->srcStr, element);
        svm_blockWrite(scatter->destBlock, iteration*scatter->destStride + j, element);
        num_items++;
      }
    }
    free(element);
  }
\end{verbatim}}

\clearpage
\makeline
\name{StridedGather}
{\small
\begin{verbatim}
  typedef struct {
    svm_Kernel kernel;
    svm_IBlock* srcBlock;
    svm_OStream* destStr;
    size_t length;
    size_t srcStride;
    size_t elementsPerStride;
    boolean setEOS;
  } svm_StridedGather;

  void svm_stridedGatherInit(svm_StridedGather* gather, mm_Proc location, 
                             svm_IBlock* _srcBlock, svm_OStream* _destStr, size_t _length,
                             size_t _srcStride, size_t _elementsPerStride, boolean setEOS) {
    kernelInit(&gather->kernel, location, 0, gather, sizeof(svm_StridedGather), 
               (ExtKernelWork)&svm_stridedGatherWork);
    gather->srcBlock = _srcBlock;
    gather->destStr = _destStr;
    gather->length = _length;
    gather->srcStride = _srcStride;
    gather->elementsPerStride = _elementsPerStride;
    gather->setEOS = _setEOS;
  }

  void svm_stridedGatherWork(svm_StridedGather* gather) {
    size_t iteration, j, num_items = 0;
    boolean done_stream_all = false;
    void* element = malloc(gather->elementSize);

    for (iteration=0; num_items<gather->length && !done_stream_all; iteration++) {
      for (j=0; 
           j<gather->elementsPerStride && num_items<gather->length && !done_stream_all; 
           j++) {
        done_stream_all = gather->length == svm_Stream_Length_All && 
                          iteration*gather->srcStride + j >= gather->srcBlock->capacity-1;
        svm_blockRead(gather->srcBlock, iteration*gather->srcStride + j, element);
        num_items++;
        if (gather->setEOS && (done_stream_all || num_items==gather->length-1)) {
          svm_streamPushEOS(gather->destStr, element);
        } else {
          svm_streamPush(gather->destStr, element);
        }
      }
    }

    free(element);
  }
\end{verbatim}}

\clearpage
\makeline
\name{IndexedScatter}
{\small
\begin{verbatim}
  typedef struct {
    svm_Kernel kernel;
    svm_IStream* srcStr;
    svm_IStream* indexStr;
    svm_OBlock* destBlock;
    size_t length;
    size_t elementsPerIndex;
  } svm_IndexedScatter;

  void svm_indexedScatterInit(svm_IndexedScatter* scatter, mm_Proc location, 
                              svm_IStream* _srcStr, svm_IStream* _indexStr, 
                              svm_OBlock* _destBlock, size_t _length, size_t _elementsPerIndex) {
    kernelInit(&scatter->kernel, location, 0, scatter, sizeof(svm_IndexedScatter),
               (ExtKernelWork)&svm_indexedScatterWork);
    scatter->srcStr = _srcStr;
    scatter->indexStr = _indexStr;
    scatter->destBlock = _destBlock;
    scatter->length = _length;
    scatter->elementsPerIndex = _elementsPerIndex;
  }

  void svm_indexedScatterWork(svm_IndexedScatter* scatter) {
    size_t iteration, j;
    boolean done_stream_all = false;
    void* element = malloc(scatter->srcStr->elementSize);
    void* index = malloc(scatter->indexStr->elementSize);
    for (iteration=0; iteration<scatter->length && !done_stream_all; iteration++) {
      done_stream_all = scatter->length == svm_Stream_Length_all && 
                        svm_streamPeekEOS(scatter->indexStr, 0);
      svm_streamPop(scatter->indexStr, index);
      size_t full_index = unpack_index((int8_t*)index, scatter->indexStr->elementSize);
      for (j=0; j<scatter->elementsPerIndex; j++) {
        svm_streamPop(scatter->srcStr, element);
        svm_blockWrite(scatter->destBlock, full_index+j, element);
      }
    }
    free(element);
    free(index);
  }

  // unpacks a generic index of <indexSize> bytes for a given architecture
  size_t unpack_index(int8_t* index, int indexSize) {
    size_t result = 0;
    int8_t temp;
    int i;
    for (i=0; i<indexSize; i++) {
      if (mm_BigEndian) {
        result = (result << 8) + index[i];
      } else {
        result = (result << 8) + index[indexSize-i-1];
      }
    }
    return result;
  }
\end{verbatim}}

\clearpage
\makeline
\name{IndexedGather}
{\small
\begin{verbatim}
  typedef struct {
    svm_Kernel kernel;
    svm_IBlock* srcBlock;
    svm_IStream* indexStr;
    svm_OStream* destStr;
    size_t length;
    size_t elementsPerIndex;
    boolean setEOS;
  } svm_IndexedGather;

  void svm_indexedGatherInit(svm_IndexedGather* gather, mm_Proc location, 
                              svm_IBlock* _srcBlock, svm_IStream* _indexStr, svm_OStream* _destStr,
                              size_t _length, size_t _elementsPerIndex, boolean _setEOS) {
    kernelInit(&gather->kernel, location, 0, gather, sizeof(svm_IndexedGather), 
               (ExtKernelWork)&svm_indexedGatherWork);
    gather->srcBlock = _srcBlock;
    gather->indexStr = _indexStr;
    gather->destStr = _destStr;
    gather->length = _length;
    gather->elementsPerIndex = _elementsPerIndex;
    gather->setEOS = _setEOS;
  }

  void svm_indexedGatherWork(svm_IndexedGather* gather) {
    size_t iteration, j;
    boolean done_stream_all = false;
    void* element = malloc(gather->srcBlock->elementSize);
    void* index = malloc(gather->indexStr->elementSize);
    for (iteration=0; iteration<gather->length && !done_stream_all; iteration++) {
      done_stream_all = gather->length == svm_Stream_Length_all && 
                        svm_streamPeekEOS(gather->indexStr, 0);
      svm_streamPop(gather->indexStr, index);
      size_t full_index = unpack_index((int8_t*)index, gather->indexStr->elementSize);
      for (j=0; j<gather->elementsPerIndex-1; j++) {
        svm_blockRead(gather->srcBlock, full_index+j, element);
        // detect the last item pushed, since we might need to include an EOS tag
        if (gather->setEOS && 
            j==gather->elementsPerIndex-1 && (done_stream_all || iteration==gather->length-1)) {
          svm_streamPushEOS(gather->destStr, element);
        } else {
          svm_streamPush(gather->destStr, element);
        }
      }
    }
    free(element);
    free(index);
  }

  // unpacks a generic index of <indexSize> bytes for a given architecture
  size_t unpack_index(int8_t* index, int indexSize) {
    // as in IndexedScatter
  }
\end{verbatim}}
