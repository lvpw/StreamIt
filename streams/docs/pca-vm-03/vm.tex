\section{Streaming Virtual Machine API}

The Streaming Virtual Machine API consists of a set of streaming
computation kernels that are connected in a stream graph.
General-purpose threaded code controls graph construction and
execution.  We describe the API for memory in
Section~\ref{sec:memory}, the API for kernels in
Section~\ref{sec:kernel}, and the API for graphs and stream control in
Section~\ref{sec:graphs}.

\subsection{Memory}
\label{sec:memory}

The streaming API provides two interfaces to storage space: streams
and blocks.  Streams are intended for sequential production and
consumption of data items, while blocks provide random access to a
fixed set of elements.  We describe streams in
Section~\ref{sec:streams} and blocks in Section~\ref{sec:blocks}.

\subsubsection{Streams}
\label{sec:streams}

A {\it stream} is a C++ data type that represents a sequence of items
of a given type.  From the perspective of a kernel, streams can vary
in two respects:
\begin{enumerate}

\item Whether the stream is used for input, output, or both.

\item Whether the elements of the stream are ordered or unordered.
Ordered streams have the semantics of a FIFO queue, whereas unordered
streams are useful when a pool of data needs to be processed in any
order, {\it e.g.}, for network packets.

\end{enumerate}
All six combinations of these attributes represent a valid interface
from a kernel to a stream, and we define a separate stream type for
each (see Figure~\ref{fig:declgrid}).  For the sake of clarifying the
specification and the implementation of the runtime system, this
proposal uses inheritance to describe these types\footnote{However,
nothing in the specification relies on the inheritance relationships
for correctness, so these could be expanded into six stand-alone types
if desired.}(see Figure~\ref{fig:inherit}).  Note that from the
control code, all streams will be instantiated using the {\tt Stream}
class at the bottom of the hierarchy.  The other classes are strictly
for the purpose of allowing kernels to specify their relationship to a
given stream.

%% \begin{figure}[t]
%% \begin{center}
%% \psfig{figure=inherit.eps,width=6in}
%% \end{center}
%% \vspace{-12pt}
%% \caption{Class hierarchy diagram for streams and blocks.\protect\label{fig:inherit}}
%% \end{figure}

\sss{BaseStorage}

The inheritance tree for streams starts with {\tt BaseStorage}, which
represents a contiguous region of storage space.  This space can
either be in memory, or in the registers of a processor.  {\small
\begin{verbatim}
  template<class T>
  class BaseStorage {
  public:
    // make a store that can hold at most <capacity> elements,
    // allocated at <address> of memory node <memLocation>.
    BaseStorage(int capacity, VM_NODE_TYPE_MEM memLocation, int address);

    // same as above, but held in registers of a processor
    BaseStorage(int capacity, VM_NODE_TYPE_PROC procLocation);

    // returns capacity of the storage
    int getCapacity();
  }
\end{verbatim}}

Each {\tt BaseStorage} object is based on a template type {\tt T} to
allow type-safe access to its data elements.  This type must have a
fixed size.  Variable sized records can be supported at the
language-level and compiled into the fixed size scheme using a data
encoding.

\newcommand{\lefttab}[1]{\begin{minipage}{0.6in}\begin{center}#1\end{center}\end{minipage}}
\begin{figure}[t]
\hspace{-13pt}\begin{minipage}{6in}
\begin{tabular}{c|c|c|c}
~ & {\small Input} & {\small Output} & {\small Input and Output}
\\ \hline
\lefttab{{\small Unordered \\ Stream}}
&
\begin{minipage}{1.8in}
  \scriptsize
  \begin{verbatim}

template <class T>
class IStreamUnordered :
      public virtual BaseStream {
public:
  T pop();
  boolean canPop(int n);
}
  \end{verbatim}
\end{minipage}
&
\begin{minipage}{2.1in}
  \scriptsize
  \begin{verbatim}

template <class T>
class OStreamUnordered :
      public virtual BaseStream {
public:
  void push(T& val);
  boolean canPush(int n);
}
  \end{verbatim}
\end{minipage}
&
\begin{minipage}{1.8in}
  \scriptsize
  \begin{verbatim}

template <class T>
class StreamUnordered : 
      public IStreamUnordered<T>, 
      public OStreamUnordered<T> {
}
  \end{verbatim}
\end{minipage}
\\ \hline
\lefttab{{\small Ordered \\ Stream}}
&
\begin{minipage}{1.8in}
  \scriptsize
  \begin{verbatim}

template <class T>
class IStream : 
      public IStreamUnordered<T> {
public:
  T pop();
  T peek(int index);
}
  \end{verbatim}
\end{minipage}
&
\begin{minipage}{2.1in}
  \scriptsize
  \begin{verbatim}

template <class T>
class OStream : 
      public OStreamUnordered<T> {
  void push(T& val);
}
  \end{verbatim}
\end{minipage}
&
\begin{minipage}{1.8in}
  \scriptsize
  \begin{verbatim}

template <class T>
class Stream : 
      public IStream<T>, 
      public OStream<T> {
}
  \end{verbatim}
\end{minipage}
\\ \hline
\lefttab{\small Block}
&
\begin{minipage}{1.8in}
  \scriptsize
  \begin{verbatim}

template <class T>
class IBlock : 
      public virtual BaseBlock {
public:
  T read(int index);
}
  \end{verbatim}
\end{minipage}
&
\begin{minipage}{2.1in}
  \scriptsize
  \begin{verbatim}

template <class T>
class OBlock :
      public virtual BaseBlock {
public:
  void write(int index, T& val);
}
  \end{verbatim}
\end{minipage}
&
\begin{minipage}{1.8in}
  \scriptsize
  \begin{verbatim}

template <class T>
class Block : 
      public IBlock<T>, 
      public OBlock<T> {
}
  \end{verbatim}
\end{minipage}
\end{tabular}
\end{minipage}
\caption{Class declarations for streams and blocks, with constructors ommitted for the sake of brevity.  The constructors for each stream and block have the same signatures as those in {\tt BaseStream} and {\tt BaseBlock}, respectively.\protect\label{fig:declgrid}}
\vspace{0.2in}
\begin{center}
\psfig{figure=inherit.eps,width=6in}
\caption{Class hierarchy diagram for streams and blocks.\protect\label{fig:inherit}}
\end{center}
\end{figure}

\clearpage

\sss{BaseStream}

The base type of all streams is {\tt BaseStream}:
{\small
\begin{verbatim}
  template<class T>
  class BaseStream : public BaseStorage {
  public:
    // make a stream that can hold at most <capacity> elements,
    // allocated at <address> of memory node <memLocation>, with <initLength>
    // items already in place.  The last two arguments are optional hints to
    // the low-level compiler:  <aliased> is true if some other stream 
    // might overlap this in memory, and <wraps> is true if more than <capacity>
    // elements might be pushed to the stream during its lifetime.
    BaseStream(int capacity, VM_NODE_TYPE_MEM memLocation, int address, 
               int initLength=0, boolean aliased=true, boolean wraps=true);

    // same as above, for stream held in registers of a processor
    BaseStream(int capacity, VM_NODE_TYPE_PROC procLocation, int initLength=0);

    // returns number of live items the stream
    int getLength();

    // returns total number of items that were pushed onto this since
    // construction or last reset
    int getTotalLength();

    // clears the elements of the stream, and resets totalLength
    void reset();

    // protected data included only for benefit of reference implementation (see text)
  protected:
    T* data;
    int* capacity;
    int* start;
    int* end;
    int* totalLength;
  }
\end{verbatim}}

Each memory-based stream {\tt s} makes an important guarantee on its
data layout: it is implemented as a circular buffer of length {\tt
capacity} that starts at the specified {\tt address} and wraps around
to the beginning when more than {\tt capacity} elements have been
pushed.  This guarantee is important for allowing reuse of memory
space between streams, as well as for transfering data between
threaded code and streaming code.  However, there is no contract on
data layout at any time if the {\tt aliased} parameter is false in the
{\tt BaseStream}'s constructor, as this represents a case where the
high-level compiler guarantees that no operation will access the
locations allocated to the stream.  This gives the low-level compiler
the freedom to implement streams with an architecture-specific
representation if their contents are never aliased.

The constructor's {\tt initLength} parameter indicates how many items
are already in place at {\tt address} that should be pushed onto the
stream initially. Thus, the stream should increment its own {\tt
length} and {\tt totalLength} by {\tt initLength} during construction.

The {\tt getTotalLength} method returns how many items have been
pushed onto the stream since the last call to {\tt reset}.  In other
words, {\tt getTotalLength} returns how many elements would be in the
stream if no storage was ever reused, and it was implemented as an
acyclic list instead of a wrap-around array.  Note that this measure
is unrelated to the capacity of the stream.  It is also unrelated to
how many items will be written to the stream in the future; if this
quantity is predictable, then it can be calculated as a function of
the {\tt totalLength} of other streams.

The values of {\tt getLength} and {\tt getTotalLength} are undefined
when a stream is being accessed by a kernel that is in the RUNNING
state (see Section~\ref{sec:kernelhlc}).

\newpage
\sss{Interface to Streams}

We now turn our attention to the other stream functions, which are
implemented in derived classes (see Figure~\ref{fig:declgrid}).

\ssss{constructors} Each stream provides two constructors, which have
the same signatures as those in {\tt BaseStream}.

\ssss{push} For output streams, the {\tt push} method enqueues a value
onto the end of the stream.  If the stream cannot hold any more items
({\it i.e.}, if {\tt canPush(1)} is {\tt false}), then its behavior is
undefined.

{\small
\begin{verbatim}
  template <class T>
  void OStream<T> :: push(T& val) {
    if (getLength() == *capacity) {
      waitForDrain();
    }
    if (*end == *capacity) {
      *end = 0;
    }
    data[(*end)++] = val;
    // keep count of items pushed for use by BaseStream.getTotalLength()
    totalLength++;
  }
\end{verbatim}}

\ssss{pop} For input streams, the {\tt pop} method dequeues a value
from the front of the stream.  If the stream has ended ({\it i.e.}, if
{\tt canPop(1)} is {\tt false}), then its behavior is undefined.

{\small
\begin{verbatim}
  template <class T>
  T IStream<T> :: pop() {
    if (getLength() < 1) {
      waitForFill();
    }
    if (*start == *capacity) {
      *start = 0;
    }
    T result = data[(*start)++];
    return result;
  }
\end{verbatim}}

\ssss{canPush} For output streams, the {\tt canPush} method returns
whether or not it will be possible to push $n$ additional items onto
the stream.  Note that for this could return {\tt true} even if there
are currently less than $n$ free spaces in the buffer, since further
execution of the stream graph could consume some items.  The function
must take this into account, and only return {\tt false} if it will
never be possible to push $n$ items at any point in the future.

Note that {\tt canPush} is allowed to cause a change in the number of
free spaces in the stream.  In the reference implementation, we drain
the stream graph as much as possible in the event that fewer than $n$
items are currently available.

{\small
\begin{verbatim}
  template <class T>
  boolean OStreamUnordered<T> :: canPush(int n) {
    if (*capacity - getLength() >= n) {
      return true;
    } else {
      if (canDrain()) {
        waitForDrain();
        return (*capacity - getLength() >= n);
      } else {
        return false;
      }
    }
  }
\end{verbatim}}

\ssss{canPop} For input streams, the {\tt canPop} method returns
whether or not it will be possible to pop $n$ additional items from
the stream.  Just as with {\tt canPush} above, {\tt canPop} returns
false only if it will never be possible to pop an item from the stream
at some point in the future.

{\small
\begin{verbatim}
  template <class T>
  boolean IStreamUnordered<T> :: canPop(int n) {
    if (getLength() >= n) {
      return true;
    } else {
      if (canFill()) {
        waitForFill();
        return (getLength() >= n);
      } else {
        return false;
      }
    }
  }
\end{verbatim}}

\ssss{peek} For ordered input streams, the {\tt peek} method returns
the element at position {\it index}, where {\it index} is zero-indexed
(such that {\tt peek(0)} gives the same value as {\tt pop()}).  If
there are fewer than $\mt{index}+1$ items in the stream ({\it i.e.},
if {\tt canPop(index+1)} is {\tt false}), or if $\mt{index}+1$ exceeds
the capacity of the stream, then the return value is undefined.

{\small
\begin{verbatim}
  template <class T>
  T IStream<T> :: peek(int index) {
    if (getLength() < index+1) {
      waitForFill();
    }
    int i = (*start)+index;
    if (i >= *capacity) {
      i -= *capacity;
    }
    return data[i];
  }
\end{verbatim}}

\subsubsection{Blocks}
\label{sec:blocks}

For kernels that require random access to a fixed set of elements, the
API provides the {\it block} abstraction.  A block is simply a region
of memory that can be read and written to.  Since some kernels use a
given block only for input or only for output, we divide the block
classes into {\tt IBlock} and {\tt OBlock}, with {\tt Block} extending
both (see Figures~\ref{fig:declgrid} and~\ref{fig:inherit}).

\ssss{write} For output blocks, the {\tt write} function stores an
item into memory.  It requires that the {\tt index} written to is less
than the block's {\tt capacity}.  
{\small
\begin{verbatim}
  template<class T>
  void OBlock<T> :: write(int index, T& val) {
    data[index] = val;
  }
\end{verbatim}}

\ssss{read} For input blocks, the {\tt read} function reads a location
in memory.  It requires that the {\tt index} read is less than the
block's {\tt capacity}.
{\small
\begin{verbatim}
  template<class T>
  T IBlock<T> :: read(int index) {
    return data[index];
  }
\end{verbatim}}

\subsection{Kernels}
\label{sec:kernel}

The API has support for three kinds of kernels.
Section~\ref{sec:kernelhlc} describes general user-defined kernels
whose behavior is explicitly described by the high-level compiler;
Section~\ref{sec:kernelsvm} describes pre-defined kernels that are
built into the SVM and must be supported on every architecture; and
Section~\ref{sec:kernelllc} describes library kernels that appear as a
black box to the high-level compiler.

\subsubsection{User-Defined Kernels}
\label{sec:kernelhlc}

\sss{Overview}

Each kernel is represented as a C++ object, with the following
components:

\begin{enumerate}

\item A constructor, which receives the following:

\begin{itemize}
\item The architecture resource where the kernel will execute.
\item A block of memory for spilling local variables.
\item The inputs and outputs for the kernel.
\item Any other kernel-specific initialization data.
\end{itemize}

\item A {\it work} function that represents the steady-state execution
step.

\item A {\it workInfo} function that describes the properties of {\tt
work} to the low-level compiler.

\item (Optional) A {\it prework} function that is called before the
first execution of {\it work}, as well as an associated {\it
preworkInfo} function.

\item (Optional) A {\it postwork} function that is called after the
last execution of {\it work}, as well as an associated {\it
postworkInfo} function.

\item (Optional) Data members, which represent local kernel data that
are preserved between invocations of {\it work}.

\end{enumerate}

\noindent For example, a kernel for a simple amplifier could be as follows:
{\small
\begin{verbatim}
  class AmplifierKernel : public Kernel {
    IStream<float> in;
    OStream<float> out;
    int N;

  public:
    AmplifierKernel(int _N, VM_NODE location, Block<byte> scratch, 
                    IStream<float> _in, OStream<float> _out) : 
                    public Kernel (location, scratch, _in, _out) {
      in = _in;
      out = _out;
      N = _N;
    }

  protected:
    STATUS work() {
      if (canPop(1) && canPush(1)) {
        out.push(in.pop() * N);
        return RUNNING;
      } else {
        return FINISHED;
      }
    }

    void workInfo() {
      KernelInfo.setPop(in, 1);
      KernelInfo.setPush(out, 1);
      KernelInfo.isSIMD();
    }
  }  
\end{verbatim}}

\sss{Kernel Base Class}
\label{sec:kernels}

Each kernel is described as a subclass of the basic kernel class:
{\small
\begin{verbatim}
  class Kernel {
    // status codes
    enum STATUS {
      UNSTARTED,
      RUNNING,
      SUSPENDED,
      FINISHED
    }
  protected:
    // construct a kernel on <location> that uses <scratch> to spill local variables,
    // and is connected to streams or blocks <bs1>, <bs2>, ... .  Also, set private 
    // <status> to UNSTARTED.
    Kernel(VM_NODE location, Block<byte> scratch, BaseStorage bs1, BaseStorage bs2=0, ... );

    // steady-state execution step.  Returns the new status of the kernel.
    virtual STATUS work();

    // (optional) Called before first invocation of work.  Returns status of kernel.
    virtual STATUS prework();

    // (optional) Called after last invocation of work.
    virtual void postwork();

    // annotations for work, prework, and postwork functions
    virtual void workInfo();
    virtual void preworkInfo();
    virtual void postworkInfo();

  private:
    // non-blocking method that starts or resumes execution of this kernel.  The kernel 
    // continues to run until its status is SUSPENDED or FINISHED.  This method is only
    // called from Graph.run.
    void run();
   
  public:
    // non-blocking method that indicates that the kernel should not execute the
    // work, prework, or postwork function again until kernel <k> is FINISHED.  This
    // can only be called when the kernel is in the UNSTARTED or SUSPENDED state.
    void addDependence(Kernel& k);

    // non-blocking method that returns the current status of the kernel
    STATUS getStatus();

    // non-blocking method that interrupts execution of a kernel on a best-effort basis,
    // and then sets the kernel's status to FINISHED.  This should be followed by a call to 
    // wait() if the control thread wants to ensure that the kernel is finished.
    void terminate();

  private:
    // status of kernel
    STATUS status;
    // Graph is a friend in order for it to call run
    friend class Graph;
  }  
\end{verbatim}}
The constructor of {\tt Kernel} receives two special arguments that
are signals to the low-level compiler.  The first is the {\tt
location}, which indicates which processor resource should be used to
execute the kernel.  The second is a {\tt scratch} space, which is a
block of memory in which the low-level compiler can store local
variables if they overflow the registers of the kernel processor.  The
scratch space is always of type {\tt Block<byte>}.

Also, in any subclass of {\tt Kernel}, the input and output streams
are received in the constructor and must be directly stored as fields
(with a one-to-one mapping from fields to streams, and no surrounding
control flow).  This allows {\tt prework}, {\tt work}, and {\tt
postwork} to access the streams.

Within a kernel, standard {\tt public}, {\tt protected}, and {\tt
private} modifiers are used to indicate what is visible outside the
class--in particular, to the stream control code.  As described in
Section~\ref{sec:graphs}, the stream control code can read and write
the public fields of a kernel when it is not in the {\tt RUNNING}
state.  The kernel can also declare public methods that can be invoked
by the control code; however, these methods cannot access a private
field in any way (either directly or via other methods.)  This allows
the private state of the kernel to be completely isolated in the
resource that is executing it.

\sss{Runtime Model}

\begin{figure}[t]
\begin{center}
\psfig{figure=kernel-status.eps,width=3.5in}
\end{center}
\vspace{-12pt}
\caption{Legal transitions of a kernel's {\tt status} before and after
a call to {\tt run}.\protect\label{fig:kernel-status}}
\end{figure}

The {\tt run} function is the interface for executing a kernel.  It
operates as follows: {\small
\begin{verbatim}
  void Kernel::run() {
    // set externally visible status to show kernel is running
    STATUS tempStatus = status;
    status = RUNNING;

    // wait for dependent kernels to finish
    for (all kernels k specified in previous calls to addDependence) {
      k.wait();
    }

    if (tempStatus == UNSTARTED) {
      tempStatus = prework();
    } else if (tempStatus == SUSPENDED) {
      tempStatus = RUNNING;
    }
    while (tempStatus == RUNNING) {
      tempStatus = work();
    }
    if (tempStatus == FINISHED) {
      postWork();
    }

    // set externally visible status to result of run
    status = tempStatus;
  } 
\end{verbatim}}
That is, the {\tt run} function calls the component work functions in
the order specified until one of them returns a status of {\tt
SUSPENDED} or {\tt FINISHED}.  If the status is {\tt SUSPENDED}, then
the {\tt run} function might be called again.  For example, if the
control code is monitoring some aspect of the kernel, then the kernel
should suspend itself after executing for a certain number of
iterations so that the control code can inspect its public fields.
Alternatively, the kernel itself could do the monitoring and suspend
itself only when a particular condition is satisfied.  A transition
diagram for the legal states of a kernel appears in
Figure~\ref{fig:kernel-status}.

Note that no kernel can override the {\tt run} function.  The behavior
of {\tt run} is exactly specified by the above code.

\sss{Annotations}

There are several pieces of information that are available to the
high-level compiler which should be transferred to the low-level
compiler in the form of annotations.  Each annotation takes the form
of a function call to the {\tt KernelInfo} class.  Calls from the {\tt
workInfo} function apply to {\tt work}, calls from {\tt preworkInfo}
apply to {\tt prework}, and calls from {\tt postworkInfo} apply to
{\tt postWork}.
{\small
\begin{verbatim}
  class KernelInfo {
  public:
    // these functions declare the push, pop, and peek rate of a
    // work or prework function with respect to a given stream.
    static void setPush(BaseStream str, int push);
    static void setPop(BaseStream str, int pop);
    static void setPeek(BaseStream str, int peek);

    // these functions indicate a dynamic rate that is unknown at compile time.  However, 
    // they can optionally provide the maximum rate and an estimate of the average rate 
    // to help the low-level compiler.
    static void setDynamicPush(BaseStream str, int max=ALL, float estimate=UNKNOWN);
    static void setDynamicPop(BaseStream str, int max=ALL, float estimate=UNKNOWN);
    static void setDynamicPeek(BaseStream str, int max=ALL, float estimate=UNKNOWN);

    // indicates that a work or prework function is data-parallel and fit for SIMD execution
    static void isSIMD();
  }  
\end{verbatim}}

All arguments to annotations must be compile-time constants.  For
example, in a kernel from a MergeSort, the input rates are marked
dynamic (with an estimate of 0.5 items per stream on average) because
they depend on values from the input streams:
{\small
\begin{verbatim}
  class MergeKernel : public Kernel {
    IStream<int> in1;
    IStream<int> in2;
    OStream<int> out;
  public:
    MergeKernel(VM_NODE location, Block<byte> scratch,
                IStream<int> _in1, IStream<int> _in2, OStream<int> _out) :
                Kernel (location, scratch, _in1, _in2, _out) {
      in1 = _in1;
      in2 = _in2;
      out = _out;
    }

  protected:
    STATUS work() {
      // we're done if both input streams have ended, or if there will never be space for output
      if ((!in1.canPop(1) && !in2.canPop(1)) || !out.canPush(1)) {
        return FINISHED;
      }

      // if in1 is empty, draw from in2
      if (!in1.canPop(1)) {          
        out.push(in2.pop());
        return false;
      }

      // if in2 is empty, draw from in1
      if (!in2.canPop(1)) {
        out.push(in1.pop());
        return false;
      } 

      // otherwise, compare elements from in1 and in2
      if (in1.peek(0) < in2.peek(0)) {
        out.push(in1.pop());
      } else {
        out.push(in2.pop());
      }

      return RUNNING;
    }

    void workInfo() {
      KernelInfo.setDynamicPush(in1, 0.5);
      KernelInfo.setDynamicPop(in2, 0.5);
      KernelInfo.setPush(out, 1);
    }
  }  
\end{verbatim}}

Any instance of a stream that supports a push, pop, or peek operation
must be annotated with its corresponding rates.

\sss{Kernel Restrictions}

Only a subset of C++ is supported from within a kernel; restrictions
are listed in Figure~\ref{fig:restrict}.  

\begin{figure}[t]
\framebox[6.5in]{
\begin{minipage}{6in}

\begin{enumerate}

\item No pointers.

\item No dynamic memory allocation.

\item No accesses to global memory.

\item No GOTO statements (all control flow is structured).

\item No recursive functions (all function calls have inline
semantics).

\item No calls to functions that violate any of these restrictions.

\item No references to templates or objects besides the classes
described in this document.  Further, no creation or casting of
objects within kernels.

\item All kernel functions (including the constructor) must receive
their arguments by value (not by reference.)  Also, the kernel
constructor cannot invoke any member functions of a stream object.

\item Supported opcodes are only the logical, arithmetic, and boolean
operations found in C (no special-purpose DSP operations at this
time\footnote{DSP operations may be added at a future date pending
further discussion by the forum.}).

\item Supported types include 64-bit {\tt double}, 64-bit signed and
unsigned {\tt long}, 32-bit {\tt float}, 32-bit signed and unsigned
{\tt int}, 16-bit signed and unsigned {\tt short}, 8-bit {\tt byte},
{\tt boolean}, arrays with a fixed (int literal) length, and {\tt
struct}'s containing members of any other type.

\end{enumerate}

\caption{Restrictions on C++ code within kernels.\protect\label{fig:restrict}}
\end{minipage}}
\end{figure}

\subsubsection{Pre-Defined Kernels}
\label{sec:kernelsvm}

\label{sec:predef}

All memory management and network support for streams are integrated
into the graph model of Section~\ref{sec:graphs}.  This is done
using pre-defined kernels that can connect streams in different
memories, or route streams to network channels.

\subsubsection*{Memory Management Kernels}

\ssss{Copy} The primary use of this kernel is for copying items
between streams in different memory banks, though it can also be used
for copying between streams in a single memory.
{\small
\begin{verbatim}
  template <class T>
  class Copy : public Kernel {
  public:
    Copy(VM_NODE location, IStream<T> srcStr, OStream<T> destStr, int length = ENTIRE_STREAM);
  }
\end{verbatim}}
The {\tt length} indicates how many items should be transferred; by
default, all elements in the {\tt srcStr} are copied.

\ssss{Scatter/Gather} A set of scatter and gather kernels allow copies
between non-contiguous elements of streams.  The strided kernels are
for copying regularly spaced chunks into or out of a block:
{\small
\begin{verbatim}
  template <class T>
  class StridedScatter : public Kernel {
  public:
    StridedScatter(VM_NODE location, IStream<T> srcStr, OBlock<T> destBlock, int length,
                   int destStride = 1, int itemsPerChunk = 1);
  }

  template <class T>
  class StridedGather : public Kernel {
  public:
    StridedGather(VM_NODE location, IBlock<T> srcBlock, OStream<T> destStr, int length,
                  int srcStride = 1, int itemsPerChunk = 1);
  }  
\end{verbatim}}
The above kernels copy items in segments of {\tt itemsPerChunk} items.
The {\tt srcStride} or {\tt destStride} indicates the number of items
between the start of adjacent chunks, while {\tt length} represents
the number of items that should be copied.

The indexed scatter and gather kernels allow irregular accesses to a
source or destination block:
{\small
\begin{verbatim}
  template <class T>
  class IndexedScatter : public Kernel {
  public:
    IndexedScatter(VM_NODE location, IStream<T> srcStr, IStream<int> indexStr, OBlock<T> destBlock, 
                   int length, int itemsPerChunk = 1);
  }

  template <class T>
  class IndexedGather : public Kernel {
  public:
    IndexedGather(VM_NODE location, IBlock<T> srcBlock, IStream<int> indexStr, OStream<T> destStr, 
                  int length, int itemsPerChunk = 1);
  }  
\end{verbatim}}
In the {\tt IndexedScatter} kernel, the {\tt indexStr} indicates the
positions in the output stream at which the records should be written;
in the {\tt IndexedGather} kernel, the {\tt indexStr} indicates the
positions in the input stream at which the records should be read.
The {\tt length} argument indicates the number of items that should be
copied.

\subsubsection*{Network Kernels}

The network kernels are for synchronized communication between
resources.

\ssss{Send} The {\tt Send} kernel sends a stream from one processor to
another, subject to the connection protocol described below.
{\small
\begin{verbatim}
  template <class T>
  class Send : public Kernel {
  public:
    // send <srcStr> on <channel> of <connection>
    Send(VM_NODE location, IStream<T> srcStr,  VM_EDGE connection, int channel);
  }
\end{verbatim}}
Since the kernel is executing on processor {\tt location}, we require
that {\tt connection} is an edge from {\tt location} to a neighboring
processor.

\ssss{Receive} The {\tt Receive} kernel receives a stream from a
neighboring processor, subject to the connection protocol described
below.  
{\small
\begin{verbatim}
  template <class T>
  class Receive : public Kernel {
  public:
    // receive <destStr> from <channel> of <connection>.
    Receive(VM_NODE location, OStream<T> destStr,  VM_EDGE connection, int channel);
  }  
\end{verbatim}}
Since the kernel is executing on processor {\tt location}, we require
that {\tt connection} is an edge into {\tt location} from a
neighboring processor.

\ssss{Send/Receive Protocol} We refer to channel number $n$ of
connection $c$ as the pair $(c, n)$.  Note that $n$ is a virtual
channel identifier; $n$ does not need to fall within $[0,
\mbox{VM\_PROP\_CHAN\_NUM}]$ for connection $c$.  Rather, the
communication protocol will ensure that there are less than
VM\_PROP\_CHAN\_NUM active channels at a time.

The protocol maintains a queue of {\tt Send} and {\tt Receive} kernels
that are waiting to communicate across each $(c, n)$; let them be
$\mt{SendQ}(c, n)$ and $\mt{ReceiveQ}(c, n)$, respectively.  Kernels
are pushed onto these queues in the same order that they are executed
from the stream processor API.  We disallow the case where multiple
kernels in a given graph are targeting the same queue.  Thus, the
order of the kernels in the queues is well-defined\footnote{Unless
there are multiple threads executing on the control processor, in
which case synchronization should be used to ensure a deterministic
ordering of the send/receive kernels across threads.}.

To open a new session of data transfer across $(c, n)$, the following
conditions must be met:
\begin{enumerate}

\item Channel $n$ is {\it free} on connection $c$.  That is, no other
session is open on $(c, n)$, and $c$ has room for another active
channel.

\item $\mt{SendQ}(c, n)$ and $\mt{ReceiveQ}(c, n)$ are non-empty.

\end{enumerate}
If these conditions are satisfied, then a new session is opened
between the kernels at the front of $\mt{SendQ}(c, n)$ and
$\mt{ReceiveQ}(c, n)$.  Items are transmitted across the channel until
either the {\tt Send} kernel or the {\tt Receive} kernel goes into the
FINISHED state.  At this point, {\tt terminate} is called on both
kernels, and they are removed from the respective queues for $(c, n)$.
The session on $(c, n)$ is finished.

Note that until a session is opened, all pending kernels are blocked.
The graphs that contain these kernels could possibly execute other
nodes, but the {\tt Send} or {\tt Receive} nodes must wait until the
channel is ready.

\ssss{Example} An example utilizing {\tt Send} and {\tt Receive}
kernels is given in Section~\ref{sec:example}.

\subsubsection{Black-Box Library Kernels}
\label{sec:kernelllc}

The API supports library routines via black-box kernels that are
defined in the metadata and then instantiated by name in the stream
control code.  These kernels are identical to user-defined kernels,
except that they have no {\tt work}, {\tt prework}, or {\tt postwork}
functions; instead, they are recognized by the low-level compiler for
the architecture and are translated into hand-optimized library code.
The kernels should include {\tt workInfo}, {\tt preworkInfo}, and {\tt
postworkInfo}, however, so that the high-level compiler can judge how
many items will be consumed in each firing.

To ensure that the low-level compiler can recognize a black-box
kernel, the high-level compiler guarantees that it will not merge
black-box kernels with other kernels, or rename the kernel so that it
would become unrecognizable to the low-level compiler.

\subsection{Stream Control}
\label{sec:graphs}

Stream control is done from general-purpose threaded code.  All static
stream operations are represented by explicit stream graphs, in which
kernel objects are the nodes and streams are the edges.  Dynamic
operations and dynamic control flow are fully supported in the code
that constructs stream graphs and transitions between them.

An example of the stream control API is as follows:
{\small
\begin{verbatim}
  // read file into memory location of s0 (not part of API)
  int length1 = readFile("input.dat", 1024, MEM1, 0x1000);

  // declare streams with their size and the location in which they are held
  Stream<float> s0(length1, MEM1, 0x1000, length1), 
                s1(128, SRF1, 0x200), 
                s2(1024, SRF1, 0x280);
  Block<byte> scratch2(32, SRF1, 0x680);

  // set up a graph to do some audio filtering and compression
  Kernel copy = new Copy<float>(s0, s1, PROC1);
  Kernel rle  = new RunLengthEncode(PROC1, scratch2, s1, s2);
  Graph compress1(copy, rle);

  // run the kernels
  compress1.run();
  compress1.wait();

  // if the output is still too large, run additional compression, 
  // overwriting s2 in place
  int length2 = s2.getTotalLength();
  if (length2 > SIZE_THRESHOLD) {
    Stream<float> s3(1024, SRF1, 0x280);
    Block<byte> scratch3(32, SRF1, 0x680);
    
    Kernel zip = new ZipCompress(PROC2, scratch3, s2, s3)
    Graph compress2(zip);
    compress2.run();
    compress2.wait();
    length2 = s3.getTotalLength();
  }

  // store the result from memory to "output.dat" (not part of API)
  writeFile("output.dat", length2, MEM1, 0x280);
\end{verbatim}}
The above code fragment illustrates several aspects of the stream
control API.  In the rest of this section, we describe how to
construct and execute stream graphs, as well as the restrictions on
stream control.

\subsubsection{Stream Graphs}
\label{sec:streamgraph}

A stream graph represents a set of connected kernels that should be
executed concurrently.  The purpose of a {\tt Graph} is to show the
low-level compiler which pairs of kernels will be simultaneously
reading and writing to a given stream.  That is, the high-level
compiler guarantees that no two kernels will try to access the same
stream object at the same time unless they are in the same graph.
{\small
\begin{verbatim}
  class Graph {
  public:
    // construct graph out of set of kernels.
    Graph(Kernel k1, Kernel kernel2 = 0, ...);

    // calls k.run() on all kernels k in the graph that have k.getStatus()!=RUNNING
    Kernel::STATUS run();

    // blocking method that waits until at least one kernel in the graph 
    // has a status other than RUNNING.  Then proceed as follows:
    //  - if all kernels have status UNSTARTED, then return UNSTARTED
    //  - if all kernels have status FINISHED, then return FINISHED
    //  - otherwise, return SUSPENDED
    Kernel::STATUS wait();
  }
\end{verbatim}}
\noindent The connectivity of the kernels in a graph is implicit in
the streams that are shared between the input of one kernel and the
output of another.  The graph does not need to be structured (as in
StreamIt).  However, no two kernels in a graph may read from (or write
to) the same stream.  Also, a kernel may appear in only one graph
during its entire lifetime (this is to avoid the complexity of having
unfinished graphs with kernel components that have been moved to other
graphs.)

In a given graph, at most one user-defined kernel can be assigned to a
given processor resource.  (However, the pre-defined kernels of
Section~\ref{sec:predef} can be assigned to a resource with other
kernels in the same graph.)  In addition, the high-level compiler
guarantees that no two kernels from different graphs will try to run
on the same resource at the same time.  This restriction applies to
both user-defined and pre-defined kernels.  In order to meet these
restrictions, the high level compiler will either have to merge
neighboring kernels into one, or spread them across multiple graphs.

Each edge of a stream graph can span at most one connection in the
architecture graph.  That is, for each stream that a kernel writes to,
there must be a connection in the architecture graph from the location
of the kernel to the location of the stream.  For each stream that a
kernel reads from, there must be a connection in the architecture
graph from the location of the stream to the location of the kernel.
(Of course, a kernel can also be located at the same node as one of
its input or output streams; we assume that self-loops in the
architecture graph capture this property.)

\sss{Executing Graphs}

The control code executes a stream graph by calling its {\tt run}
method.  The graph then calls {\tt run} on each of the component
kernels.  In this way, the graph is the basic execution unit for
kernel execution.

Before graph execution, streams can be initialized with (possibly
non-streaming) data from general-purpose threaded code.  Likewise, the
results of a streaming computation can be used in the threaded code
following the graph's termination.  Both of these transfers are done
through memory, either by directly accessing the memory assigned to a
stream, or by utilizing a stream's {\tt push}, {\tt pop}, and {\tt
peek} methods from the control code.  The control thread can rely on
the sequential data layout guaranteed by streams when managing this
communication (see Section~\ref{sec:streams}).  Also note that all
device I/O (such as file handling) is done by threaded code, and then
made available to streams through memory.

During the execution of a graph, any interleaving of kernel execution
is valid so long as streams do not overflow or underflow.  In other
words, the low-level compiler can select any schedule respecting the
data dependences of the graphs; no memory analysis is needed.  If
there are location-based dependences due to overlapping streams or
parallel writes, the high-level compiler should insert synchronization
(for inter-graph dependences) or verify that other graph dependences
will ensure a consistent execution order (for intra-graph
dependences).

The high-level compiler might anticipate upcoming kernel executions.
The following method allows this information to be transfered to the
low-level compiler so that the kernel can be loaded into IMEM ahead of
time:
\begin{verbatim}
  // gives a hint to the low-level compiler that it should be loading the instructions 
  // for <kernel> at the time of this call, as it is likely to be used in the near future
  void loadKernel(Kernel k);
\end{verbatim}

\sss{Transitioning Between Graphs}

A {\tt Graph} represents only the static sections of the stream graph.
As in the example at the beginning of Section~\ref{sec:graphs},
dynamic control flow can surround the construction of graphs and can
predicate their execution.

There are two ways to transfer the outputs of one stream graph to the
inputs of another.  Perhaps the most natural way is to reuse the same
stream in both graphs, thereby carrying over the results; in our
example, stream {\tt s2} is used in multiple graphs.  The other way to
transfer items is by allocating a new input stream that overlaps with
the output stream in memory.  This could be desirable if, for example,
a single output stream is being split between multiple input streams,
all of which are in the local memory of some processor.  In this case,
the control code should indicate that the kernel reading the new
stream is dependent on the kernel that wrote the original stream; this
is done by use of the kernel's {\tt addDependence} method, thereby
saving the low-level compiler from doing location-based memory
analysis to discover dependences between kernels.

It is also possible for the control processor to inspect and modify
the public fields of a kernel when it is not in the {\tt RUNNING}
state.  Accesses to kernel fields are useful for passing parameters to
subsequent stream graphs, or for retrieving a reduction value from
inside a kernel.  For example: 
{\small
\begin{verbatim}
  // --- kernel code ---
  class SumKernel : public Kernel {
  public:
    IStreamUnordered<int> in;
    int sum;

    SumKernel(IStreamUnordered<int> _in, VM_NODE location, Block<byte> scratch) : 
              Kernel (_in, location, scratch) {
      in = _in;
      sum = 0;
    }

  protected:
    STATUS work() {
      if (in.canPop(1)) {
        sum += in.pop();
        return RUNNING;
      } else {
        return FINISHED;
      }
    }

    void workInfo() {
      KernelInfo.setPop(in, 1);
    }
  }

  // --- control code ---
  Stream<int> s1(128, SRF, 0x100);
  Block<byte> scratch1(16, SRF, 0x180);

  SumKernel sk(s1, PROC1, scratch1);
  Graph g(sk);
  g.run();
  g.wait();

  int finalSum = sk.sum;
\end{verbatim}}

\subsubsection{Restrictions on Stream Control}

A stream graph can be constructed and executed from general-purpose
threaded code.  In order to facilitate static analysis in the
low-level compiler, there are a number of restrictions on the
statements dealing with streams, blocks, kernels, and graphs.
However, there are no restrictions on non-streaming statements, which
can be finely interleaved with the stream statements; these statements
can be arbitrarily complex threaded code (although, in accordance with
the threaded API, they must adhere to C instead of C++).

The restrictions on stream control code are as follows:

\begin{enumerate}

\item All stream, block, kernel, and graph variables have a static
single assignment in the program.  This exposes exactly which
constructor is invoked for a given variable name.

\item Architecture resources must be specified by literals when they
are passed to stream, block, kernel, or graph functions.  That is,
constants denoting processor nodes, memory nodes, connections, and
channels must be passed to API functions directly, rather than passing
the value in a variable.  Similarly, identifiers of objects can only
be used in three contexts: 1) in their own declaration, 2) as the
direct target of a method call, and 3) as an argument to a function.
This ensures that all references to streams, kernels, and graphs are
resolvable by the low-level compiler.

\item There are no references to templates or objects besides the
classes described in this document.

\item Any function that contains a reference to a stream, kernel, or
graph object must NOT be recursive.

\end{enumerate}

\subsubsection{Example}
\label{sec:example}

We consider one more example in order to illustrate graphs with
pre-defined kernels.  In this example, there are two processors that
each contain their own memory:

\begin{figure}[h]
\begin{center}
\psfig{figure=ex1.eps,width=1.8in}
\end{center}
\vspace{-12pt}
\end{figure}

The application does audio segmentation on a series of 10 input files
and plays a summary of each file on a speaker.  The first processor
does the segmentation itself, while the second processor filters the
extracted segments to provide a smooth transition between them.

{\small
\begin{verbatim}
  // --- code for PROC1 ---

  for (int i=0; i<10; i++) {
    // read file and put in memory (not part of API)
    int fileLength = readFile(filename[i], 10000, MEM1, 0x1000);

    // allocate overlapping stream in memory for raw data
    Stream<float> rawStr(fileLength, MEM1, 0x1000, fileLength), 
    // also allocate block for gather operation
    Block<float> rawBlock(fileLength, MEM1, 0x1000), 

    // allocate other streams
    Stream<float> segIndices(10, PROC1), sumData(10, PROC1);
    Block<byte> scratch1(128, MEM1, 0x4000);

    Kernel k1 = new ExtractSegments(PROC1, scratch1,               // make indices of summary segments
                                    rawStr, segIndices);
    Kernel k2 = new Gather<float>(rawBlock, segIndices, sumData),  // gather summary audio in sumData
    Kernel k3 = new Send<float>(PROC1, sumData, c1, 3);            // send over connection c1, channel 3
    Graph g(k1, k2, k3);
            
    g.run();                                                       // run for whole length of file
    g.wait();
  }

  // --- code for PROC2 ---

  for (int i=0; i<10; i++) {
    stream<float> sumData(128, MEM2, 0x2000), smoothData(100, MEM2, 0x2080);
    stream<byte> scratch2(128, MEM2, 0x2160);

    Kernel k1 = new Receive(PROC2, sumData, c1, 3),                // receive summaries over channel
    Kernel k2 = new FIRFilter(PROC2, scratch,                      // filter summaries
                              sumData, smoothData);
    Kernel k3 = new Speaker(smoothData);                           // send to speaker
    Graph g(k1, k2, k3);

    g.run();
    g.wait();
  }   
\end{verbatim}}
In processor 1, a {\tt Gather} kernel is used to load the audio file
at the indices where the summary segments appear.  The {\tt Gather}
kernel is directly connected to a {\tt Send} kernel which sends the
summary segments across virtual channel 3 of connection {\tt c1}.
Processor 2 uses a {\tt Receive} kernel to receive the summary
segments before filtering them and sending them to a speaker.  Note
that there are 10 sessions of data transfer between the processors,
and the amount of data transferred during each session depends on the
length of the audio file; a session is finished when processor 1
finishes executing its stream graph.
