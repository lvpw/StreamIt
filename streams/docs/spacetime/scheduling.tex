\section{Space-Time Scheduling}
\label{sec:scheduling}
\subsection{Steady-State Schedule}

%\begin{figure}
%\centering
%\psfig{figure=schedule_ex_1.eps,width=3in}
%\caption{Something should go here.
%\protect\label{fig:1d}}
%\end{figure}

what we get at this stage

3d bin packing translation

What the pieces look like 

saman's figures...

We arrive at a solution to the 3d bin packing by using simulated
annealing \cite{simanneal}, a type of iterative improvement.  A
detailed explanation of simulated annealing is beyond the scope of
this paper.  Simulated annealing is a form of stochastic
hill-climbing. Unlike most other methods for cost function
minimization, simulated annealing is suitable for problems where there
are many local minima.  Simulated annealing achieves its success by
allowing the system to go uphill with some probability as it searches
for the global minima.  As the simulation proceeds, the probability of
climbing uphill decreases.


\subsection{Cost Function}
The cost function for the simulated annealing solution to our 3D bin
packing problem attempts to incorporate three different aspects of the
scheduling problem.  First that the critical path of the schedule is
determined by the node(s) with the maximum work.  We do not generate a
true modulo schedule because all nodes must wait for the bottleneck
node to complete to perform the data-reorganization.  Next, we assume
that the work estimation performed by our compiler is inaccurate, so
we bias the estimate to reflect this observation.  Finally, we take
into account the communication cost of the data-reorganization phase.



what it does!

\subsection{Prologue Schedule}
how contructed, what it guarantees.

\subsection{Peek Initialization Schedule}
Before both the prologue schedule is executed and the steady-state
schedule is commenced, we perform a {\it peek initialization
schedule}.  This schedule is required to make sure that we can create
a cyclic steady-state schedule that respects StreamIt's peeking
operation.  In this schedule, slices are executed in data-flow order
and the buffers are not rotated.  Therefore, the prologue schedule
starts with the rotating buffers untouched.  See
\cite{streamitcc} for a more complete discussion of the peek
initialization schedule.


