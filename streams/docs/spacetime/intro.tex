\section{Introduction}

Due to the inexorable advance of VLSI technology, transistors are
essentially free.  Microprocessor design is now dominated by wire
delay concerns.  One popular solution harnesses the plethora of
chip-level resources by replicating processing units.  This limits
wire delay because wires need not be longer than a replicated unit.
The multiple units communicate through shared memory or through an
on-chip communication network.  Recently the architecture community
has witnessed the ascendancy of {\it communication-exposed architectures},
examples include Raw \cite{raw}, Smart Memories
\cite{smartmemories}, Merrimac \cite{merrimac-sc03}, TRIPS
\cite{trips}, WaveScalar \cite{wavescalar}, and RDR \cite{rdr}.  These
machines propose to solve the wire delay problem by replicating
processing units and exposing the interconnect between these units to
a software layer.  The architecture remains simple and scalable, while
complexity is shifted to the compiler. For these architectures to gain
programmer acceptance, there must exist a high-level, portable
programming language that can be compiled efficiently to any of the
candidate targets.

A number of efforts have focused on stream programming as a paradigm
for producing high-level, efficient, and retargetable application code
for wire-exposed architectures \cite{streamit-asplos,imagine-ieee,merrimac-sc03,trips-isca03}.
In a stream program, computation is expressed as a set of filters that
operate over sequences of data.  Because each filter has an
independent address space and program counter, there is abundant task
and data parallelism that can be recognized by the compiler.  Also,
the data streams expose the communication between filters; the
compiler can predict the flow of data and orchestrate data movement.
These properties distinguish stream programs from imperative languages
such as C and FORTRAN.  While imperative languages were a good match
for von-Neumann machines, they are obsolete for architectures
containing multiple instruction streams and distributed memory banks.

\begin{figure*}[th]
  \centering
  \vspace{-18pt}
  \psfig{figure=space-vs-time.eps,width=5.5in}
  \vspace{-12pt}
  \caption{Techniques for scheduling stream programs. \protect\label{fig:spacevstime}}
  \vspace{-6pt}
\end{figure*}

Hitherto, there have been two basic approaches for compiling a stream
program to a communication-exposed architecture.  {\it Time
multiplexing} utilizes the entire chip for each filter, switching
between filters over time (see Figure~\ref{fig:spacevstime}b).  Time
multiplexing's efficacy extends from its freedom from having to
balance the workload between filters.  However, this technique can
lead to long latencies, increased memory traffic, and the utilization
is highly dependent upon how effectively each filter can be
parallelized across the machine.

Conversely, {\it space multiplexing} distributes filters across the
entire chip, running them continuously and in parallel (see
Figure~\ref{fig:spacevstime}a).  Space multiplexing affords $(i)$ no
filter swapping, $(ii)$ reduced memory traffic, $(iii)$ localized
communication, and $(iv)$ tighter latencies.  Because it distributes
computation across decentralized processing units, space multiplexing
supports architectures that scale spatially without any global wires.
Unfortunately, this approach is highly dependent on effective load
balancing techniques: the goal is to merge and split filters (into new
filters) until all computation resources are assigned filters with a
uniform work distribution.  Such load balancing can be very difficult
for applications with an irregular distribution of work.

This paper proposes a hybrid approach, {\it space-time multiplexing},
that exploits the advantages of both space multiplexing and time
multiplexing (see Figure~\ref{fig:spacevstime}c).  This technique uses
space multiplexing to schedule a group of filters for parallel
execution on part of the chip; however, time multiplexing is used to
switch between different groups as execution progresses.  Space-time
multiplexing affords the flexible load balancing of time multiplexing,
while preserving the locality and latency benefits of space
multiplexing.

Scales well with communicate exposed architectures (space multiplexing
within a trace leverages the on-chip interconnect).

In this paper, we describe and evaluate our SpaceTime Compiler, a
fully-automatic implementation of space-time multiplexing.  Our source
language is StreamIt~\cite{streamitcc}, a high-level stream
programming language that aims to be portable across next-generation
communication-exposed architectures.  Our target is the Raw
microprocessor~\cite{raw10,raw_isca}, a tiled architecture with
fine-grained, programmable communication between processors.

Our implementation of space-time multiplexing treats each group of
space-multiplexed filters, which we refer to as a {\it slice}, as an
atomic unit for scheduling.  Due to the implicit outer loop of an
infinite stream graph (or subgraph), slices can be scheduled using
techniques traditionally limited to individual instructions in a given
loop nest.  Our approach extends these loop-level techniques---in
particular, software pipelining---to a coarse level of granularity
that encompasses multiple program modules at a time.  

Inter-slice communication leverages abundant off-chip DRAM resources.
This enables us to aggressively buffer data between slices, amortizing
the cost of slice pipeline startup and providing scheduling
flexibility. The space-time execution model uses rotating buffers to
buffer data between slices to remove dependences between slices and to
allow slices of different iterations of the stream graph to execute
concurrently.  These rotating buffers are akin to a rotating register
file found in some superscalar and VLIW architectures and are
distributed amongst the off-chip dram banks. Furthermore, the
inter-slice off-chip memory accesses are highly regular (unit-stride
bulk transfers) that can utilize batch processing modes of modern
DRAMs. \todo{Anything else here}?

We use the on-chip interconnect of the target architecture for
intra-slice communication, for space multiplexing each slice.  




%Also, since each trace is independent, a specific code generation
%strategy can be applied depending on the properties of the trace.  In
%this paper, we recognize traces that compute a {\it linear} function
%of their input.  Using the coefficients and I/O rates of the linear
%trace (extracted automatically from the code), we generate template
%assembly code based on a hand-optimized systolic algorithm.  By
%decoupling memory accesses from computation and carefully utilizing
%the register-mapped communication network on Raw, this code achieves
%near-optimal performance for linear functions.

Somewhat restricted model of spacetime execution, but still good
result.  Right?

In summary, this paper makes the following contributions:

\begin{itemize}
\item A procedure for eliminating synchronization from a hierarchical
stream graph, yielding a flat graph that directly exposes
communication patterns.
\item An algorithm for extracting load-balanced traces from a flat
stream graph.
\item An optimized code generation strategy for traces that compute a
linear function.
\item A modified software pipelining algorithm for scheduling traces,
respecting layout and occupancy constraints on a tiled architecture.
\item An evaluation of space-time multiplexing in the StreamIt-to-Raw
compiler, demonstrating an average improvement of \todo{??} over a
space multiplexing approach.
\end{itemize}

The remainder of this paper is organized as follows.  Section
\ref{sec:streamit} gives an introduction to the StreamIt programming
language and Section \ref{sec:raw} provides an overview of
Raw. Section~\ref{sec:example} gives an illustrative example for our
technique.  Sections 5 through 10 describe our compilation framework,
including synchronization removal, trace extraction and scheduling,
and code generation.  Section \ref{sec:results} presents our results,
Section \ref{sec:related} reviews related work and Section
\ref{sec:conclusion} concludes.

%% In this paper we present a compiler for the StreamIt programming
%% language \cite{streamitcc}.  StreamIt is a high-level stream
%% programming language that aims to be portable across next-generation
%% communication-exposed architectures.  StreamIt contains basic
%% constructs that expose the parallelism and communication of streaming
%% applications without depending on the topology or granularity of the
%% target architecture \cite{streamit-asplos}. In StreamIt the basic unit
%% of computation is a {\it filter}, a single-input, single-output block.
%% Filters are composed into a communication network using hierarchical,
%% structured constructs (introduced below).  A {\it stream graph},
%% composed of filters and uni-directional FIFO channels connecting the
%% filters, describes the resulting computation. The StreamIt compiler
%% currently targets the Raw microprocessor, a tiled architecture with
%% fine-grained, programmable communication between processors.

%% Neglecting the irrelevant initial passes, the flow of the compiler is
%% as follows.  The compiler reaches the space-time backend with a
%% structured, hierarchical stream graph, where all filters of the
%% application are explicitly represented.  We first convert this stream
%% graph into to a flat, non-hierarchical graph with unnecessary
%% communication channels removed.  Next, we identify the linear
%% sub-components of the stream graph.  Our compiler then extracts the
%% traces from the stream graph, considering the concurrency,
%% communication, layout, and type (linear or non-linear) of each filter.
%% Next, we schedule the traces, producing both a multi-stage initialization
%% schedule and a steady-state schedule.  Lastly, we generate both
%% communication and computation code for Raw.  The contributions of this
%% monograph include:


