\section{Trace Extraction}
\label{sec:extraction}	
Trace extraction refers to the process of assigning each filter of the
stream graph to a trace.  As stated previously, a trace is a
contiguous section of the stream graph that is scheduled for execution
as a group. Each trace occupies a portion of the chip as it executes
and is swapped out when it completes execution.  Each filter in the
stream graph is a member of exactly one trace.

Our initial algorithm for trace extraction is rather straightforward
and a brief description of the algorithm should suffice.  Due to the
restrictions of the current low-level implementation, as we traverse
the stream graph, we introduce trace boundaries {\it before} a filter
with multiple inputs and {\it after} a filter with multiple outputs,
thus restricting the traces to pipelines of filters. We introduce a
new trace when within a pipeline a non-linear filter directly
communicates to a linear filter and vice versa. Finally, as we are
adding filters to the trace, we must introduce a trace boundary if the
size of the trace is equal to the number of tiles in the Raw
configuration.

%%what about linear filters, are they restricted to being pipelines???

Additionally, we try to coax the generation of load-balanced traces.
For each filter, we calculate a static work estimation of the filter
based on an analysis of the {\tt work} function.  Because of the
static I/O (push, pop, and peek) rates in this version of StreamIt,
most loop bounds within {\tt work} can be resolved, allowing a close
approximation of the actual cycle count.  This estimate is multiplied
by the number of times the filter executes in the steady-state.

As we are adding filters to the trace, we compare the work estimation
of the current filter we are examining to the work estimation for the
first filter of the trace.  If the ratio of the work in the current
filter to the work in the first filter is within a predefined
threshold, termed the {\it work threshold}, then we proceed to add the
filter to the trace.  Otherwise, begin a new trace with the current
filter as the first filter in this new trace.  After experimentation,
we found that smaller, well-balanced traces are preferable.  The
results given in Section \ref{sec:results} were generated using a work
threshold of 0.80, meaning that each filter of the trace has a
workload that is within 80-125\% of the workload of the first filter
of the trace.

After the trace extraction stage, we can build a {\it trace graph}
where nodes of the graph represent traces and an edge between $t_1$
and $t_2$ denotes that a channel exists between the last filter of
trace $t_1$ and the first filter of trace $t_2$. This graph describes
the flow of data between traces and is used by subsequent passes.

\section{Trace Scheduling}
Some software pipelining terminology, should use the terminology
throughout. 
Span:  The number of iterations executed in the software pipelined
loop.


\label{sec:scheduling}
In the trace scheduling pass of the compiler, we calculate a relative
execution ordering of the traces and assign each filter of each trace
to a tile on the Raw chip.  The current implementation of the trace
scheduling pass implements a modified list scheduler. The scheduler
attempts to optimize for occupancy and chip utilization.  At each
iteration, the scheduler traverses the list until it finds a trace to
schedule; if it cannot, the current time is incremented.  The
scheduler maintains a block reservation table with entries for each
tile of the Raw chip.  The reservation table stores the next cycle in
which the tile is available for the assignment of a filter.  At the
end of the trace scheduling phase we have an execution ordering of
traces and for each tile, an ordered list of the filters that execute
on it, each filter from a different trace. Note that the linear
ordering of traces calculated does not preclude the parallel execution
of traces, as traces may execute on distinct Raw tiles and communicate
to their neighbors through distinct memory modules.

The scheduler begins by ordering the traces based on a given priority
function, ignoring the data-flow dependencies inherent in the stream
graph.  It then attempts to schedule the traces in order of decreasing
priority.  When attempting to schedule a trace, the scheduler checks
if the trace can be legally scheduled at this time (see Section
\ref{sec:softpipe}) and attempt to find a set of tiles for which to
assign the filters of this trace. If the trace is currently
schedulable, update the reservation table entry for each tile assigned
by adding the work estimation for the filter assigned to the tile to
the current schedule time. Then remove the trace from the list.  If
the current trace cannot be scheduled, attempt to schedule the next
trace in the list and so on.  If no trace can be scheduled, increment
the current time to the earliest time that a busy tile will become
available.

When trying to schedule a trace, the scheduler exhaustively searches
the chip for a possible layout, using the first legal layout
encountered. Of course, at the current scheduling time-step, the
scheduler may not be able to find tile assignments for the filters of
the trace due to the lack of free (non-busy) tiles.  It then moves on
to the next trace in the list. In the current version of the backend,
each trace is composed of a pipeline of filters and each filter must
be directly connected to its downstream consumer filter.  Again, the
endpoints of traces must be assigned to the border tiles of the chip.
There are numerous tile assignment heuristics that are included in the
current version of the compiler, but their discussion would provide
little benefit to the reader.

For the performance results in Section \ref{sec:results}, we use the
{\it bottleneck workload} of each trace as the priority function of
the scheduler. The bottleneck workload for a trace is defined to be
the maximum work estimation of the filters that compose the trace.
This priority function was not chosen arbitrarily; we initially tried
multiple orderings including a height-based priority function and a
communication-based priority function.
% and priority functions that
%concurrently modeled communication, workload, and height.  
The bottleneck workload consistently achieved the greatest
performance. After trace scheduling, we are ready to generate code for
both the compute processor and switch processor of each tile.
