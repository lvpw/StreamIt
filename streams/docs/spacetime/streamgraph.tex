\section{Stream Graph} 

\begin{figure*}
\centering
\psfig{figure=fm_example.eps,width=6.5in}
\caption{FMRadio with a 12-way equalizer after the passes of the
SpaceTime Compiler.
\protect\label{fig:fm-ex}}
\end{figure*}

After the StreamIt frontend executes, it passes our compiler a
complete stream graph representing the computation and communication
of the application.  In StreamIt, the stream graph is a structured,
hierarchical composition of pipelines, splitjoins, and feedbackloops
with filters, splitters, and joiners as the leaf nodes of the graph
(see Section
\ref{sec:streamit}).  

Throughout the following sections, we will use our FMRadio benchmark
to elucidate our discussion.  The FMRadio application is a software
implementation of FM radio with a 12-way equalizer.  StreamIt's stream
graph is given in Figure \ref{fig:fm-ex}a.  In the figure, we
represent splitters, joiners, and filters with circles.  Containers
(splitjoins and pipelines) are represented as rectangles around the
nodes they contain.  In the figure, the filters colored red have the
highest workload per steady-state execution (in this case, the red
nodes account for over 90\% of the computation in the steady-state).

The initial pass of our compiler takes this structured stream graph
and de-structures it.  We remove all hierarchy and structure of the
containers and are left with only filters, splitters and joiners.
Furthermore, we introduce our own notion of a splitter and a joiner,
each slightly more powerful than their analog in StreamIt.  We still
have separate nodes for splitting an output stream and joining an
input stream, but these nodes can handle more complicated
data-reorganization patterns.  For example, our splitter node can
describe a round-robin distribution with duplication.  A more exact
discussion of our splitters and joiners is omitted, but it suffices to
say that we designed each node to closely match the data-organizations
that are possible with a scalar operand network \cite{scalaroperands}.
We next run a synchronization removal pass that converts StreamIt
splitters and joiners into our own notion of data-distribution nodes.
In the process we remove unnecessary synchronization points present in
the stream graph.  We introduce a (more powerful) splitter node after
each filter whose output is read by multiple filters and we introduce
a (more powerful) joiner node before each filter that reads data
produced by multiple filters.

After synchronization removal, we are left with a flat stream graph
composed of filters, and data-reorganization nodes (our more powerful
splitters and joiners), see Figure \ref{fig:fm-ex}b FMRadio's
flattened stream graph; note that some splitters have been
coalesced. For the remainder of this paper the terms splitter and
joiner refer to our more powerful splitting and joining nodes and the
term stream graph refers to the flattened graph composed of these
nodes and filters.  At this point we are ready to extract the slices
from the graph and schedule them for execution.

