\section{Slice Extraction}

\begin{figure*}
\centering
\psfig{figure=fm_example.eps,width=6.5in}
\caption{FMRadio with a 7-way equalizer after the passes of the
SpaceTime Compiler.
\protect\label{fig:fm-ex}}
\end{figure*}

\label{sec:extract}
After the StreamIt frontend executes, it passes our compiler a
complete stream graph representing the computation and communication
of the application.  In StreamIt, the stream graph is a structured,
hierarchical composition of pipelines, splitjoins, and feedbackloops
with filters, splitters, and joiners as the leaf nodes of the graph
(see Section \ref{sec:streamit}).

We will use our FMRadio benchmark to elucidate our discussion.  The
FMRadio application is a software implementation of FM radio with a
7-way equalizer.  StreamIt's stream graph is given in Figure
\ref{fig:fm-ex}a.  In the figure, we represent splitters, joiners, and
filters with circles.  Containers (splitjoins and pipelines) are
represented as rectangles around the nodes they contain.  In the
figure, the filters colored red have the highest workload per
steady-state execution (in this case, the red nodes account for over
90\% of the computation in the steady-state).

The initial pass of our compiler takes this structured stream graph
and de-structures and canonicalizes it.  We remove all hierarchy and
structure of the containers and are left with only filters, splitters
and joiners. We are left with a flat stream graph composed of filters,
and data-reorganization nodes (splitters and joiners), see Figure
\ref{fig:fm-ex}b FMRadio's flattened stream graph; note that some
splitters have been coalesced.

Conceptually, the slice is the atomic unit of scheduling in our
compiler.  Our scheduling algorithm operates at the slice level.  Each
slice is composed of a contiguous region of the stream graph with
restrictions on its composition.  Slice extraction refers to the
process of assigning each joiner, filter, and splitter of the stream
graph to a slice.  As stated previously, a slice is a contiguous
section of the stream graph that is scheduled for execution as a
group. Each slice occupies a portion of the chip as it executes and is
``swapped out'' when it completes execution, not to execute again
until the next steady-state.  Each filter in the stream graph is a
member of exactly one slice.

We traverse the stream graph in depth-first order.  Each time we visit
a node, we must decide if the node should be added to the slice we are
currently constructing.  Due to the restrictions of the current
implementation, as we traverse the stream graph, we end the current
slice {\it before} a joiner node and we end the current slice {\it
after} a splitter node, thus restricting the slices to pipelines of
filters.  Also, as we are adding filters to the slice, we must
introduce a slice boundary if the size of the slice is equal to the
number of tiles in the Raw configuration.

Additionally, we try to coax the generation of load-balanced slices.
For each filter, we calculate a static work estimation of the filter
based on an analysis of the {\tt work} function
\cite{streamit-asplos}.  Because of the static I/O (push, pop, and
peek) rates in this version of StreamIt, most loop bounds within {\tt
work} can be resolved, allowing a close approximation of the actual
cycle count.  This estimate is multiplied by the number of times the
filter executes in the steady-state.

As we are adding filters to the slice, we compare the work estimation
of the current filter we are examining to the work estimation for the
filter of the slice that performs the most work (the current {\it
bottleneck} of the slice).  If the ratio of the work of the current
filter to the work of the bottleneck is within a predefined threshold,
termed the {\it work threshold}, then we proceed to add the filter to
the slice.  Otherwise, begin a new slice with the current filter as
the first filter in this new slice. In the case where a slice will end
in a filter, we will insert a dummy splitter at the end of the
just-completed slice.  Conversely, if needed, we will insert a dummy
joiner at the beginning of a newly created slice if the slice begins
with a filter. Any data-flow arcs that existed from splitters and to
joiners will persist as arcs to nodes of other slices.  The arcs of
the introduced splitters and joiners assume the connections of their
insertion point.

In Figure \ref{fig:fm-ex}c, we give a possible slice graph for the
7-way FMRadio application.  The blue boxes represent individual
slices.

