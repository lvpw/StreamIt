\section{The Slice}
Before we move on to describing the algorithms that compose the
compiler, we will describe our implementation of a
slice. Conceptually, the slice is the atomic unit of scheduling in our
compiler.  Our scheduling algorithm operates at the slice level.
Each slice is composed of a contiguous region of the stream
graph with restrictions on its composition. 

A slice always begins with a joiner node to join the (possibly)
multiple inputs of the slice.  The slice continues with a pipeline of
filters from the stream graph.  The slice ends with a splitter node to
distribute the slice's output to its (possibly) multiple readers (see
Figure \ref{fig:slice}).  More specifically, only the end-points of
slices can communicate with more than one filter (these filters are
contained in other slices); inner filters are restricted to single
input and single output.  Each filter of the original stream graph is
contained in exactly one slice.  Slices are not hierarchical.  The
number of filters in a slice must be less than or equal to the number
of computational nodes of the architecture to which we are targeting.
In Figure \ref{fig:fm-ex}c, we give a possible slice graph for the
12-way FMRadio application.  The blue boxes represent individual
slices.

These restrictions are not fundamental to space-time multiplexing
execution model; they simplify the compiler.  In future work, we plan
to relax some of the restrictions.

In Section \ref{sec:extract} we describe our algorithm for extracting
slices from a stream graph.

