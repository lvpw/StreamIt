As embedded DSP applications become more complex, it is increasingly
important to provide high-level stream abstractions that can be
compiled without sacrificing efficiency.  In this paper, we describe
scheduler support for StreamIt, a high-level language for signal
processing applications.  A StreamIt program consists of a set of
autonomous filters that communicate with each other via FIFO queues.
As in Synchronous Dataflow (SDF), the input and output rates of each
filter are known at compile time.  However, unlike SDF, the stream
graph is represented using hierarchical structures, each of which has
a single input and a single output.

We describe a scheduling algorithm that leverages the structure of
StreamIt to provide a flexible tradeoff between code size and buffer
size.  The algorithm describes the execution of each hierarchical unit
as a set of phases.  A complete cycle through the phases represents a
single steady-state execution.  By varying the granularity of a phase,
our algorithm provides a continuum between single appearance schedules
and minimum latency schedules.  We demonstrate that a minimal latency
schedule is effective in decreasing buffer requirements for some
applications, while the phased representation mitigates the associated
increase in code size.

%% Applications structured around some notion of a ``stream'' are
%% becoming increasingly important and widespread. \cite{Rix98} provides
%% evidence that streaming media applications are already consuming most
%% of the cycles on consumer machines, and their use is continuing to
%% grow. The streaming computation model is pervasive and ranges from
%% small, embedded systems (ex. cell phones) to large, computationally
%% powerful machines (ex. cell base stations). In this paper, we describe
%% a novel technique for scheduling execution of synchronous data flow
%% streaming applications exhibiting hierarchical properties. A vital
%% aspect in compiling such programs is finding an efficient
%% schedule. The technique presented here focuses on producing schedules
%% that are optimized for the amount of space required for buffering and
%% storing the schedule. A wide variety of real-life applications and a
%% few synthetic applications are surveyed. Applications benefit from an
%% average 14.5\% decrease in buffer requirements, with a peak of 93\%
%% savings in buffer size. No application requires more space than the
%% most popular technique used today (Single Appearance Scheduling).
