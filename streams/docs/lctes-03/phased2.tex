\section{Phased Scheduling}
\label{chpt:phased}

\begin{figure}[t]
\psfig{figure=pseudocode5.eps,width=3.5in}
\caption{Phased Scheduling Algorithm\protect\label{fig:phasealg}}
\end{figure}

\begin{figure}[t]
\psfig{figure=pseudocode6.eps,width=3.5in}
\caption{Algorithm to detect granularity of feedback loop.\protect\label{fig:feedalg}}
\end{figure}

A schedule for a given hierarchical node of a {\StreamIt} program is a
list of the node's immediate children, specifying the order in which
they should be executed.  More precisly, since filters (and, as we
will see, hierarchical nodes as well) can have multiple phases, a
schedule is a list of phases of child nodes.  In order for a schedule
to be legal, it must satisfy two conditions: first, for every
execution of a node, a sufficient amount of data must be present on
its {\Input} {\Channel}(s); second, in the case of the steady state,
an infinite repetition of the schedule must require a finite amount of
memory.  The second condition is ensured by using the steady state
multiplicities calculated in the previous section, while the first
condition is one that we must respect when choosing an ordering for
the nodes.

Our phased scheduling algorithm, shown in Figure~\ref{fig:phasealg},
operates in a hierarchical fashion.  That is, it constructs a schedule
for a given pipeline, splitjoin, or feedbackloop as a sequence of the
schedules of its children.  A schedule is represented as a sequence of
phases.  In the base case of a filter, these phases are specified by
the StreamIt program (with one small modification, described below),
while at hierarchical nodes they are computed by our algorithm.  To
schedule an entire StreamIt program, our algorithm should be applied
as a post-order traversal of the stream graph.

Intuitively, our algorithm is based on the observation that a
hierarchical stream displays cyclic behavior as it executes its
components.  At the coarsest level of granularity, these cycles are
evident in the steady state schedule: each iteration of the steady
state is exactly the same.  The aim of our algorithm is to exploit a
finer level of granularity in execution behavior---the basic unit
being a phase of the push schedule for the stream.  Generally
speaking, a phase of the push schedule holds the smallest sequence of
filter executions that will both consume input and produce output for
the stream.  Our algorithm allows a parameterized level of granularity
by collapsing some of these fine-grained phases together and shuffling
the resulting schedule so that the phases of a given child stream are
all adjacent.  As we demonstrate below, a single appearance schedule
and minimum latency schedule are both special cases of a parameterized
phased schedule.  For a more mathematical description of the phased
scheduling technique, see~\cite{karczma-thesis}.

\subsection{Algorithm Details}

We now consider in more detail the pseudocode in
Figure~\ref{fig:phasealg}; refer to Figure~\ref{fig:sjex} for an
example.  The algorithm inputs a Stream {\tt s} and returns a
sequences of phases that represent the schedule for that stream.  It
also inputs two additional parameters: {\tt maxPhases}, which
specifies the maximum number of phases in the resulting schedule, and
{\tt mode}, which indicates whether we are scheduling for the initial
or steady-state epoch.  The algorithm starts by assembling a series of
fine-grained phases, each of which corresponds to a push schedule as
built by the {\tt pushSchedule} routine.

The {\tt pushSchedule} routine simulates a push schedule until the
bottom node is fired at least once (in most cases, this will
correspond to an output being produced).  A push schedule is one in
which downstream nodes are fired as much as possible before upstream
nodes are considered.  The routine starts with the entrance node of
the stream, {\it i.e.,} the first child of a pipeline, the splitter of
a splitjoin, or the joiner of a feedbackloop.  It then pushes live
items as far forward as possible, only executing the entrance node
again if the exit node could not fire.

There are two subtleties in the {\tt pushSchedule} procedure.  First,
note that it always flushes extra items from the stream: the exit node
might fire multiple times, even though all firings were caused by a
single execution of the entrance node.  Second, in the case of a
feedbackloop, it is careful to push items around the feedback path
even after the splitter (the exit node) has fired.  That is, the
ranking of nodes in a feedback loop is $(\mt{joiner}, \mt{body},
\mt{splitter}, \mt{loop})$, and pushing of items through the loop node
is necessary to ensure a correct steady-state schedule.

The {\tt phasedSchedule} routine builds up a maximal list of phases
from the push schedule.  In the steady state, this list is complete
when each node has completed its steady state repetitions, while in
the initialization mode, simulation is finished when each node has
executed its initial phases.  To ensure that the initialization
schedule provides enough data items for the peeking requirements of
the steady state, we add an extra initialization phase to each filter
before running the algorithm.  For filter $f$, this phase has rates
$\mt{peek'}=\mt{peek}_f - \mt{pop}_f, \mt{pop'}=\mt{push'}=0$.  Since
this phase must execute in the initial schedule, it ensures that there
will be $\mt{peek}-\mt{pop}$ items present at the start of the steady
state.  A steady state schedule is then possible to construct, since
the filter can return the buffer to this state by firing once with
$\mt{peek}$ items on the channel.

Once it has gathered the list of maximally fine-grained phases, the
{\tt pushSchedule} algorithm makes two modifications.  First, it
combines some adjacent phases so that only {\tt maxPhases} are
returned.  Combination works simply by concatanating the sequence of
child executions from the given phases.  Second, even if no phases are
combined, the algorithm re-arranges the order of child phases so that
all phases corresponding to a given stream are adjacent.  This is an
attempt to provide a canonical form for a given series of executions,
so that phases with the same form can be compressed in the resulting
schedule~(see Section~\ref{sec:schedrep}).

Note that the pseudocode given in Figure~\ref{fig:phasealg} specifies only
the behavior of the algorithm, rather than the implementation.  In our
implementation, we avoid symbolic execution of the entire steady state
by calculating, from the bottom-up, the number of node firings that
will be required in each phase.  If upstream nodes produce more data
than is necessary, then we drain this data through the stream by
firing downstream nodes again.  In this technique, each child is
visited no more than twice per calculation of a parent phase.

%% \begin{table*}[t] \centering  \scriptsize
%% \begin{tabular}{|c|c|c|c|c|c|c|c|c|c|c|c|}
%% \hline
%% \multicolumn{4}{|c|}{data in {\Channel}} & \multicolumn{4}{c|}{\parbox{1in}{\centering phase executions left}} & \parbox{0.5in}{\centering child considered} & \parbox{0.6in}{\centering phases executed} & \parbox{0.6in}{\centering {\pipeline} consumption} \\
%% \cline{1-8} $in_A$ & $out_A$ & $in_B$ & $out_B$ & split & A & B & join & & & \\
%% \hline 0 (0) & 0 (0) & 0 (0) & 0 (0) & 0 & 0 & 1 & 0 & join & - & $[0\ 0\ 0]$ \\
%% \hline 0 (0) & 0 (0) & 0 (0) & 0 (0) & 0 & 0 & 1 & 0 & A & - & $[0\ 0\ 0]$ \\
%% \hline 0 (0) & 0 (0) & 0 (0) & 0 (0) & 0 & 0 & 1 & 0 & B & $A^i_{B,0}$ & $[0\ 0\ 0]$ \\
%% \hline 0 (0) & 0 (0) & 0 (1) & 0 (0) & 0 & 0 & 0 & 0 & split & split & $[3\ 3\ 0]$ \\
%% \hline 2 (0) & 0 (0) & 1 (0) & 0 (0) & 0 & 0 & 0 & 0 & A & $A^i_{A,0}$ & $[0\ 0\ 0]$ \\
%% \hline 0 (0) & 1 (0) & 1 (0) & 0 (0) & 0 & 0 & 0 & 0 & B & - & $[0\ 0\ 0]$ \\
%% \hline 0 (0) & 1 (0) & 1 (0) & 0 (0) & 0 & 0 & 0 & 0 & join & - & $[0\ 0\ 0]$ \\
%% \hline 0 (0) &  1 (0) &  1 (0) &  0 (0) & \multicolumn{7}{|c|}{init phase 0 done, init done} \\
%% \hline 0 (0) & 1 (0) & 1 (0) & 0 (0) & 2 & 2 & 1 & 2 & join & join & $[0\ 0\ 4]$ \\
%% \hline 0 (0) & 0 (0) & 1 (0) & -3 (3) & 2 & 2 & 1 & 2 & A & - & $[0\ 0\ 0]$ \\
%% \hline 0 (0) & 0 (0) & 1 (0) & -3 (3) & 2 & 2 & 1 & 1 & B & $A_{B,0}$ & $[0\ 0\ 0]$ \\
%% \hline 0 (0) & 0 (0) & -1 (2) & 3 (0) & 2 & 2 & 0 & 2 & split & $\{2split\}$ & $[6\ 6\ 0]$ \\
%% \hline 4 (0) & 0 (0) & 1 (0) & 3 (0) & 0 & 2 & 0 & 2 & A & $\{2A_{A,0}\}$ & $[0\ 0\ 0]$ \\
%% \hline 0 (0) & 2 (0) & 1 (0) & 3 (0) & 0 & 0 & 0 & 0 & B & - & $[0\ 0\ 0]$ \\
%% \hline 0 (0) & 2 (0) & 1 (0) & 3 (0) & 0 & 0 & 0 & 1 & join & join  & $[0\ 0\ 4]$ \\
%% \hline 0 (0) &  1 (0) &  1 (0) &  0 (0) & \multicolumn{7}{|c|}{phase 0 done, steady state schedule done} \\
%% \hline
%% \end{tabular}
%% \caption[Execution of Minimal Latency Scheduling Algorithm on a
%% {\splitjoin}]{Execution of Minimal Latency Scheduling Algorithm on
%% {\splitjoin} from Figure \ref{fig:steady-state}(b). In the ``data
%% in {\Channel}'' columns, the first value represents the actual
%% number of data in the {\Channel}, which can be negative due to
%% data borrowing. The second value is the minimal number of data
%% items borrowed from the {\Channel}.}
%% \label{tbl:min-lat-sj}
%% \end{table*}

%% Table \ref{tbl:min-lat-sj} contains a trace of execution of our
%% algorithm on the sample {\splitjoin} from Figure
%% \ref{fig:steady-state}(b). Below is the phased schedule for the
%% {\splitjoin}. Note that this example produces a phased single
%% appearance schedule.

%% \begin{displaymath} \scriptsize
%% P_{sj} = \left\{
%% \begin{array}{c}
%% T_{sj} = \left\{
%% \begin{array}{c}
%% A_{sj,0} = \left\{ \{\{2 split\}\{2A\}B\{2 join\}\}, \left[\begin{array}{c} 6 \\ 6 \\ 8 \end{array}\right]\right\} \\
%% \end{array}\right\}, \\
%% I_{sj} = \left\{ A^i_{sj,0} = \left\{
%% \{split\ A^i_{A,0}\ A^i_{B,0}\}, \left[\begin{array}{c} 3 \\ 3 \\ 0 \\
%% \end{array}\right]\right\}
%% \right\}, \\
%% c_{sj} = \left[ \begin{array}{c} 6 \\ 6 \\ 8 \end{array} \right],
%% c^i_{sj} = \left[ \begin{array}{c} 3 \\ 3 \\ 0 \end{array}
%% \right],
%% \end{array}
%% \right\}
%% \end{displaymath}

\subsection{Generalizing Other Techniques}

As alluded to above, single appearance scheduling and minimum latency
scheduling are special cases of our parameterized phased scheduling
algorithm.  The initial and steady state schedules for these cases can
be constructed with the following calls:
\[
\small
\begin{array}{rcl}
\mt{singleAppSchedule}[s]  & = & \langle \mt{phasedSchedule}(s, 1, \mbox{INIT}), \\
                         ~ & ~ & ~ \mt{phasedSchedule}(s, 1, \mbox{STEADY}) \rangle \\ ~ \\
\mt{minLatencySchedule}[s] & = & \langle \mt{phasedSchedule}(s, \infty, \mbox{INIT}), \\
                         ~ & ~ & ~ \mt{phasedSchedule}(s, \infty, \mbox{STEADY}) \rangle
\end{array}
\]
That is, a single appearance schedule is the same as a phased schedule
with a single phase, while a minimum latency schedule is a phased
schedule with an unlimited number of phases.  Other values of the {\tt
maxPhases} parameter indicate compromises between these two extremes.
Also, note that different levels of granularity could be applied to
different streams in the same graph, depending on the constraints; the
algorithm does not depend on the granularity of the children when it
is scheduling a parent node.

\begin{figure}[t]
\begin{center}
\psfig{figure=gsmfl.eps,width=1.8in}
\caption{Tightly coupled feedback loop from our GSM benchmark.\protect\label{fig:gsm}}
\end{center}
\end{figure}

\begin{figure*}[t]
\psfig{figure=gsmfl-sched.eps,width=7in}

\hspace{43pt} 
phase 1 \hspace{94pt}
phase 2 \hspace{94pt}
phase 3 \hspace{94pt}
phase 4
\caption{Phased schedule for feedback loop of Figure~\ref{fig:gsm}.
No single appearance schedule exists for this
loop.\protect\label{fig:gsm-phases}}
\end{figure*}

\subsection{Scheduling Feedback Loops}

Some feedback loops require a minimum number of phases in order to
construct a valid schedule.  This is because if the latency of child
streams is too high, then a node could deadlock waiting for its own
(upcoming) output to propagate through the loop.  For example, in our
GSM benchmark, there is a tightly constrainted feedback loop (see
Figure~\ref{fig:gsm}).  While it is impossible to schedule this loop
with a single appearance schedule, a minimum latency schedule results
in a legal ordering (see~Figure~\ref{fig:gsm-phases}).

Figure~\ref{fig:feedalg} provides an algorithm for calculating the
minimum number of phases that are required to schedule a feedback
loop.  The routine's functionality is similar to the phased scheduler,
except for one key difference: the joiner is executed as much as
possible before the items that it pushes are propagated around the
loop.  This ensures that the reshuffling step of the phased scheduling
algorithm will be legal, since no element in the schedule will depend
on items that it produced earlier in the same phase.  Note that the
{\tt phasedSchedule} algorithm gives an undefined result if a given
loop is impossible to schedule with the requested number of phases;
thus, {\tt phasesForFeedback} should always be called first to see how
many phases are needed.

\subsection{Schedule Representation}
\label{sec:schedrep}

In the discussion above, a schedule is represented simply as a
sequence of phases for child nodes.  However, since this
representation can become large for some programs, our implementation
employs compression to keep code size to a minimum.  

We compress the schedule in three simple ways.  First, we collect
repetition of identical phases into a loop.  For example, if {\tt A}
is a phase:
\[
\mbox{{\tt A=BBB}} ~\rightarrow~ \mbox{{\tt A=\{3B\}}}
\]
Second, if a hierarchical phase contains only one execution of a
child, then we substitute all occurrences of the parent with a direct
call to the child:
\[
\mbox{{\tt A=BCD, C=E}} ~\rightarrow~ \mbox{{\tt A=BED}}
\]
Finally, if a phase is used only once, then it can be replaced by its
child phases, even if there are multiple children:
\[
\mbox{{\tt A=BCD, C=EFG, C}}~\mt{used}~\mt{only}~\mt{once}~ ~\rightarrow~ \mbox{{\tt A=BEFGD}}
\]
To improve the compression of the schedule, we repeatedly apply the
above three transformations until no further changes can be made.

%% During testing it was found that in some applications some operators
%% had many phases that were identical to other phases of the
%% operator. Instead of including these phases in the final schedule
%% multiple times, they were listed only once, and references to the
%% duplicate phases have been replaced with references to the one copy.

%% This optimization lead to improvements in schedule size for two
%% reasons. First, operators now have fewer phases, so their schedules
%% take up less space. Second, applications using the phased schedules
%% can now execute the same phase multiple times in a row, which was
%% optimized out using run length encoding.

In the future, an additional optimization could be explored regarding
schedule compression.  Instead of representing different phases for a
given stream by distinct entries in the schedule, we could record only
the name of the stream in the schedule and postpone the resolution of
the current phase number until runtime.  This would allow more
opportunities for schedule compression, as two different phases would
be considered equal if they call the same child streams, rather than
requiring them to call the same phases on those children.  However,
proper evaulation of this technique would need to take into account
the overhead of this indirection at runtime, so we do not evaluate it
here.