\section{Results}
\label{chpt:results}

This section presents results of creating schedules using the
technique described in Section \ref{chpt:phased}.  Section
\ref{sec:results:apps} presents the applications used for evaluation.
Section \ref{sec:results:results} presents the results and analysis.

\subsection{Benchmarks}
\label{sec:results:apps}

Our benchmark suite contains 17 applications. Out of those
applications, 15 represent useful practical computation taken from
real-life applications, while two were chosen to highlight the
effectiveness of phased scheduling.

SJPeek1024 and SJPeek31 are synthetic benchmarks, designed to
highlight the strengths of phased schedules. SJPeek1024 requires an
initialization schedule which benefits from the finer granularity of
minimum latency scheduling. SJPeek31 contains a push/pop mismatch
which causes a combinatorial blow-up using single appearance
scheduling.

Nine test applications (BitonicSort, FFT, FilterBank, FIR, Radio, GSM,
3GPP, Radar and Vocoder) are applications used in
\cite{Gordo02}. BitonicSort performs a 32 element bitonic sort; FFT
performs a 64-element FFT; FilterBank is an 8 channel filter bank; FIR
is a 64-tap fine-grained FIR filter; Radio is an FM radio decoder with
an equalizer; 3GPP is a 3GPP Radio Access Protocol application; Radar
is a radar array front-end application with beamforming; Vocoder is a
28 channel Vocoder.

Two test applications (QMF and CD-DAT) are applications used in
another publication on scheduling streaming applications
(\cite{murthy99buffer}). QMF is a filter bank application. CD-DAT is a
sample rate conversion application. The code inside of the {\filters}
has not been implemented. QMF is a filter bank application which uses
a 1/2-1/2 split for the spectrum up to a depth of 3 (qmf12\_3d
in~\cite{murthy99buffer}).  It was slightly modified to use
{\StreamIt}'s pre-defined {\splitter} and {\joiner} constructs.  The
high-pass and low-pass filtering in multiple-output blocks has been
converted to a splitter followed by filters on each of the output
channels. The low and highb pass filters have also been given a peek
amount of 16 so they can perform their function in the way intended in
{\StreamIt}.  CD-DAT is exactly the same application as that described
in \cite{murthy99buffer}.

The remaining 4 applications were chosen from our sample applications
used for testing the StreamIt compiler. HDTV performs a HDTV signal
decoding/encoding. CFAR implements PCA Constant False Alarm Rate
detection. Block Matrix Mult performs a blocked matrix multiplication
- it multiplies a 12x12 matrix by a 9x12 matrix in blocks of 3x3
submatrices. Trellis performs trellis encoding/decoding.

\begin{comment}

\subsection{Methodology}
\label{sec:results:methodology}

The following data has been collected: number of nodes, number of
node executions per steady state, schedule size and buffer size
for pseudo single appearance and minimal latency schedules.

\subsubsection{Schedule Compression}

\end{comment}

\subsection{Results}
\label{sec:results:results}

\begin{figure}[t]
\psfig{figure=buffer-graph.eps,width=3.35in}
\caption{Buffer sizes.\protect\label{fig:buffergraph}}
\end{figure}

\begin{figure}[t]
\psfig{figure=code-graph.eps,width=3.35in}
\caption{Code sizes.\protect\label{fig:codegraph}}
\end{figure}

\begin{figure}[t]
\psfig{figure=total-size-graph.eps,width=3.35in}
\caption{Sum of code size and buffer size.\protect\label{fig:sumgraph}}
\end{figure}

\begin{table*} \centering \small
\begin{tabular}{|c|c|c|c|c|c|c|}
\hline benchmark & \parbox{0.5in}{\centering number of nodes} & \parbox{0.5in}{\centering number of node executions} & \multicolumn{2}{c|}{pseudo single appearance} & \multicolumn{2}{c|}{minimal latency} \\
\cline{4-7} & & & \parbox{0.5in}{\centering schedule size} & \parbox{0.5in}{\centering buffer size} & \parbox{0.5in}{\centering schedule size} & \parbox{0.5in}{\centering buffer size} \\
\hline SJPeek31 & 6 & 12063 & 8 & 19964 & 24 & 874 \\
\hline HDTV & 170 & 390038 & 230 & 550692 & 1190 & 28300 \\
\hline CD-DAT & 6 & 612 & 6 & 1021 & 64 & 72 \\
\hline CFAR & 4 & 193 & 7 & 193 & 9 & 129 \\
\hline SJPeek1024 & 6 & 3081 & 8 & 7168 & 13 & 4864 \\
\hline Block Matrix Mult & 43 & 1956 & 48 & 4212 & 56 & 3132 \\
\hline Vocoder & 117 & 415 & 156 & 1285 & 205 & 1094 \\
\hline Radar & 68 & 161 & 68 & 332 & 68 & 332 \\
\hline BitonicSort & 370 & 468 & 370 & 2112 & 370 & 2112 \\
\hline 3GPP & 94 & 356 & 104 & 986 & 108 & 970 \\
\hline Trellis & 14 & 301 & 14 & 538 & 17 & 499 \\
\hline FIRfine & 132 & 152 & 132 & 1560 & 132 & 1560 \\
\hline FilterBank & 53 & 312 & 95 & 2063 & 116 & 1991 \\
\hline QMF & 65 & 184 & 85 & 1225 & 85 & 1225 \\
\hline Radio & 30 & 43 & 35 & 1351 & 35 & 1351 \\
\hline FFT & 26 & 448 & 26 & 3584 & 26 & 3584 \\
\hline GSM & 47 & 3356 & - & - & 64 & 3900 \\
\hline
\end{tabular}
\caption{Results of running pseudo single appearance and minimal
latency scheduling algorithms on various applications.}
\label{tbl:results}
\end{table*}

\begin{comment}
\begin{figure}
\centering \psfig{figure=kz-1.eps,width=6in} \caption[Buffer
storage space savings of Phased Minimal Latency schedule vs.
Hierarchical schedule.]{Buffer storage space savings of Phased
Minimal Latency schedule vs. Hierarchical schedule. All data in
all {\Channels} is assume to consume same amount of space.}
\end{figure}

\begin{figure}
\centering \psfig{figure=kz-2.eps,width=6in} \caption[Storage
usage comparison]{Storage usage for compressed Minimal Latency
Phased schedule vs. Hierarchical schedule. Left bars are for
Hierarchical schedules. Numbers are normalized to total storage
required by Hierarchical schedule. Each entry in every schedule
and data items in all {\Channels} are assumed to consume same
amount of space.}
\end{figure}
\end{comment}

Table \ref{tbl:results} presents the buffer and schedule sizes
required by our hierarchical single appearance and minimum latency
scheduling algorithms for our benchmark suite.  Note that the GSM
application cannot be scheduled using a single appearance schedule,
because it has a tightly constrained feedback loop (see
Figure~\ref{fig:gsm}).

Several applications show a very large improvement in buffer size
necessary for execution (see Figure~\ref{fig:buffergraph}).  These
improvements are usually coupled with an increase in code size
(Figure~\ref{fig:codegraph}).  However, as shown in
Figure~\ref{fig:sumgraph}, minimum latency scheduling never increases
the sum of code size and data size for any application.

The CD-DAT benchmark exhibits a buffer size decreases from 1021 to 72,
a 93\% improvement. \cite{murthy99buffer} reports a buffer size of 226
after applying buffer merging techniques. Our improvement is due to
reducing the combinatorial growth of the buffers using phased
scheduling.  

For our synthetic benchmarks SJPeek31 and SJPeek1024, buffer sizes
decrease by 95\% and 32\%, respectively. In the case of SJPeek1024,
the improvement is due to creating fine grained phases which allow the
initialization schedule to transfer smaller amount of data and allow
the children of a {\splitjoin} to drain their data before the
{\splitter} provides them with more. This improvement is only evident
in the presence of peeking. In the case of SJPeek31, the improvement
reflects reduced combinatorial growth in addition to the fine-grained
benefit with peeking.

It is important to note that the schedules we consider in our
evaluation have the elements of a hierarchical phase sorted as
described in Section~\ref{chpt:phased}: all of the phases of a given
child stream are executed before advancing to the next child.  For
both single-appearance and minimum latency scheduling, this represents
only one possible execution order for child phases; in particular, a
more fine-grained interleaving of children could reduce buffer
requirements.  While we do not explore the range of possible
interleavings within a hierarchical node, note that the hierarchy of
the stream graph provides a set granularity at which the leaf nodes of
the graph can be interleaved---for example, in a hierarchical
single-appearance schedule, two consecutive executions of a pipeline
construct would execute all of its nodes once before executing all of
the nodes again.  We are currently exploring other interleaving
strategies for the nodes within a given phase.
