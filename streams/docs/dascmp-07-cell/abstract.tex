% $Log: not supported by cvs2svn $
% Revision 1.10  2007/11/22 06:34:21  qjli
% minor edits
%
% Revision 1.9  2007/11/22 06:05:52  dxzhang
% final edits
%
% Revision 1.8  2007/11/21 22:50:37  rabbah
% updated abstract and intro.
%
% Revision 1.7  2007/11/20 05:05:12  qjli
% More revisions
%
% Revision 1.6  2007/11/15 06:25:07  qjli
% Mostly typo corrections
%
% Revision 1.5  2007/09/18 01:17:50  qjli
% Revisions suggested by Mike
%
% Revision 1.4  2007/09/17 22:36:53  qjli
% Completed abstract
%
% Revision 1.3  2007/09/17 22:03:56  qjli
% Revised section on schedulers. Beginning of abstract.
%
% Revision 1.2  2007/09/17 17:24:06  rabbah
% need new abstract and new contributions for intro, along with paper roadmap.
%
% Revision 1.1  2007/09/15 05:59:03  rabbah
% reformatted david's thesis for conference format. 15 pages, need to cut a few more and edit figures for size.
%
% Revision 1.1  2007/09/07 20:47:38  dxzhang
% Added thesis
%
% Revision 1.1  93/05/14  14:56:25  starflt
% Initial revision
% 
% Revision 1.1  90/05/04  10:41:01  lwvanels
% Initial revision
% 
%
%% The text of your abstract and nothing else (other than comments) goes here.
%% It will be single-spaced and the rest of the text that is supposed to go on
%% the abstract page will be generated by the abstractpage environment.  This
%% file should be \input (not \include 'd) from cover.tex.

As multicore architectures gain widespread use, it becomes
increasingly important to be able to harness their additional processing
power to achieve higher performance. However, exploiting parallel
cores to improve single-program performance is difficult from a
programmer's perspective because most existing programming languages
dictate a sequential method of execution.

Stream programming, which organizes programs by independent filters
communicating over explicit data channels,  exposes useful types of
parallelism that can be exploited. However, there is still the burden
of mapping high-level stream programs to specific multicore
architectures. The complexities of each architecture's underlying
details makes it difficult to schedule the execution of a stream
program with high performance.

In this paper, we present the specifications for an intermediate layer
between the stream program and the target architecture. This multicore
streaming layer (MSL) provides a common level of abstraction that
facilitates efficient execution of stream programs by making it easier
for compilers to manage computation, and by providing automatic
orchestration and optimization of communication when appropriate. We
implemented a framework for one such instance of the MSL targeted to
the Cell processor and the StreamIt language and achieved greater than
88\% utilization on all benchmarks with relatively small amounts of
code. The framework can also be applied to other architectures and
stream programming languages to enhance generality and portability.

%%The Cell architecture is a heterogeneous, distributed-memory multicore architecture that features a novel on-chip communication network. Stream programs are particularly well-suited for execution on Cell.

%%This thesis implements a runtime library on Cell specifically designed to support streaming applications and streaming language compilers. The runtime library abstracts the details of Cell's communication network and provides facilities that simplify the task of scheduling stream actors. The library is designed in the context of the StreamIt programming language.

%%This library is used to implement a dynamic scheduling framework. The programmability of high-level schedulers with and without the library is analyzed. We observe that the library does not add significant overhead, provides a number of useful features for schedulers, and greatly simplifies the code required to implement schedulers.
