\section{Performance}\label{ch:perf}

We evaluate the performance of the MSL library and StreamIt compiler backend
on a set of four StreamIt applications using different scheduling
methodologies. The applications are described in figure~\ref{fig:perf:apps}.
For statically-scheduled benchmarks, the scheduler executes a sufficiently
coarsened steady state to reduce library overhead and imposes an explicit
synchronization barrier between steady state iterations. The dynamic scheduler
automaticaly coarsens work functions as necessary.

\begin{figure}[!htb]
\begin{center}
\begin{tabular}{|l|p{2.25in}|}
\hline
BitonicSort & 8-element bitonic sort \\
\hline
DCT & 16x16 IEEE reference DCT \\
\hline
FFT & 256-element FFT \\
\hline
MPEG & MPEG-2 block and motion vector decoding (subset of full MPEG-2 decoder) \\
\hline
\end{tabular}
\end{center}
\caption{Benchmark applications.}
\label{fig:perf:apps}
\end{figure}

The benchmarks were executed on PlayStation 3 hardware, which only provides
six usable SPEs. Benchmarks were run for a large number of steady state
iterations to smooth out any one-time execution startup overhead.
Performance numbers are given in Figure~\ref{fig:perf:stats}.

\begin{figure}[!htb]
\begin{center}
\begin{tabular}{|l|r|r|r|r|}
\hline
& Cores & Run~\% & Work~\% & GOPS \\
\hline
\textsf{BitonicSort} & 6 & 98.2 & 97.8 & 0.5 \\
\hline
\textsf{DCT} & 6 & 98.5 & 97.5 & 3.2 \\
\hline
\textsf{FFT} & 6 & 99.0 & 98.4 & 1.9 \\
\hline
\textsf{FFT\_dynamic} & 6 & 99.3 & 89.8 & 2.2 \\
\hline
\textsf{FFT\_pipeline~*} & 6 & 95.9 & 92.2 & 1.9 \\
\hline
\textsf{MPEG\_static~*} & 5 & 98.2 & 97.7 & N/A \\
\hline
\textsf{MPEG\_dynamic} & 5 & 98.8 & 96.9 & N/A \\
\hline
\end{tabular}
\end{center}
\caption{Benchmark performance.}
\label{fig:perf:stats}
\end{figure}

The \emph{Cores} column gives the number of SPEs each benchmark is
scheduled for and run on. Utilization statistics are maintained by the
MSL library. The \emph{Run~\%} column gives the percentage of total time the SPE had an active \textsf{filter\_run} command. The remainder is overhead due to either \emph{i}) durations when SPEs do not have useful work to do (such as when waiting for filters to be loaded) or \emph{ii}) inadequate double-buffering of input or output data. In both cases, the scheduler (or the nature of the program) creates insufficient communication--computation concurrency. The \emph{Work~\%} column gives the percentage of total time actually spent in the work function. The remainder represents the total overhead added by the library or caused by the scheduling algorithm.
The Run~\% and Work~\% numbers given for each benchmark are for the SPE
that reported the greatest work percentage.
For \textsf{DCT} and the FFT benchmarks, the \emph{GOPS} column gives GFLOPS.
For \textsf{BitonicSort}, the GOPS column gives billions of
compare operations per second.

The \textsf{FFT\_pipeline} and \textsf{MPEG\_static} benchmarks (indicated
with a \textsf{*} in the figure) exhibit behavior that is not adequately
described by the statistics alone, which we discuss below.

The \textsf{BitonicSort}, \textsf{DCT}, and \textsf{FFT} benchmarks are statically-scheduled and generated by the StreamIt compiler. These application are fully data-parallel and the compiler fuses the stream graph into a single filter. This is then fissed to the number of cores. 
As the number of SPEs is varied from one to six, utilization numbers remain nearly the same. Utilization and runtimes are always nearly identical across all SPEs. This demonstrates nearly perfectly linear speedup.
From Work~\%, the overhead added by the library is less than 2.5\%. Run~\% is high, although not exactly 100\% due to the overhead of starting each steady state and the synchronization barrier after each steady state.

For a single fused data-parallel filter, the dynamic scheduler provides identical performance as static scheduling. Results for the dynamic scheduler on these applications are not separately given.

The \textsf{FFT\_dynamic} and \textsf{FFT\_pipeline} benchmarks are different manual implementations of the FFT application. The FFT stream graph consists of a single pipeline with 15 filters. \textsf{FFT\_dynamic} schedules this pipeline using the dynamic scheduler. In \textsf{FFT\_pipeline}, the stream graph is first manually fused into a pipeline of six filters, each of which is statically placed on a different SPE. All communication except for input and output is done directly between SPE local stores and remains entirely on-chip. GFLOPS numbers for \textsf{FFT\_dynamic} and \textsf{FFT\_pipeline} can be compared with each other but not directly to \textsf{FFT}; the former two have manual work function implementations that are slightly more efficient than the compiler-generated code.

For \textsf{FFT\_dynamic}, utilization and runtimes are nearly identical across all SPEs and Run~\% is very close to 100\%. This indicates that the dynamic scheduler has no difficulties keeping all SPEs supplied with work. Moreover, it validates the critical assumption made in the design of the dynamic scheduler: the Cell architecture's memory bandwidth is indeed sufficient to perform all buffering to memory, avoiding SPE--SPE communication entirely. Utilization numbers remain nearly identical as the number of SPEs is varied from one to six, indicating almost perfectly linear speedup. 

However, Work~\% shows that this benchmark has approximately 10\% overhead. From the high Run~\%, almost all of this is due to the library. This overhead results from two factors: \emph{i}) individual filters in the pipeline have much lower communication--computation ratios than the single fused filter in the \textsf{FFT} benchmark; and \emph{ii}) the dynamic scheduler continually issues additional commands to switch filters between SPEs which are not needed in a static schedule.

For \textsf{FFT\_pipeline}, a single filter/SPE in the middle of the pipeline is the bottleneck. Although GFLOPS is 12\% lower than \textsf{FFT\_dynamic}, no SPE except for the bottleneck was fully utilized: two other SPEs had Work~\% around 60\%. In general, this illustrates the difficulty of performing direct static SPE--SPE pipelining. Although SPE--SPE pipelining keeps communication on-chip, where extremely high bandwidth is available and there is no danger of exhausting comparatively limited memory bandwidth, work imbalances between filters make it difficult to fully utilize all SPEs.

MPEG has a small amount of state and cannot be fused into a single filter. The compiler fuses the stream graph down to a single stateful filter and a single stateless filter as the branches of a two-way splitjoin. For mapping to the MSL library, both statically- and dynamically-scheduled benchmarks explicitly treat the splitter and joiner as separate stateless filters, resulting in a total of four filters to schedule.

The \textsf{MPEG\_static} benchmark statically software-pipelines these four filters. We generated a static schedule for five SPEs by manually profiling, fissing, and partitioning filters and manually generating control code to issue the resulting schedule. The mapping of filters to SPEs in the partition that we obtained is given in figure~\ref{fig:perf:mpegs}.

\begin{figure}[!htb]
\begin{center}
\begin{tabular}{|r|l|}
\hline
SPE & Filters \\
\hline
0 & $n$ splitter, $n$ stateful \\
\hline
1 & $6n$ joiner \\
\hline
2 & $2n$ stateless \\
\hline
3 & $2n$ stateless \\
\hline
4 & $2n$ stateless \\
\hline
\end{tabular}
\end{center}
\caption{Partition for \textsf{MPEG\_static} benchmark.}
\label{fig:perf:mpegs}
\end{figure}

This partition has a slight work imbalance. The Run~\% and Work~\% numbers given in figure~\ref{fig:perf:stats} are for SPEs~2--4, which have the highest utilization. SPE~0 has the lowest utilization (88.3\%). However, aside from the work imbalance, which can be minimized with a better partition, this benchmark shows that static software pipelining using the MSL library can be done with very low overhead.

The \textsf{MPEG\_dynamic} benchmark uses the dynamic scheduler to schedule the four filters. 
As with \textsf{FFT\_dynamic}, utilization and runtimes are nearly identical across all SPEs and there is nearly perfectly linear speedup as the number of SPEs is varied. Compared to the static schedule, the dynamic scheduler is not limited by the concept of a steady state and does not impose any synchronization barriers; however, in order to dynamically switch filters on cores, it must issue more commands. This translates into slightly higher Run~\% but slightly lower Work~\%. However, the dynamic scheduler is able to more efficiently distribute work across all available cores; overall, this results in a 2.5\% performance improvement over \textsf{MPEG\_static}.
