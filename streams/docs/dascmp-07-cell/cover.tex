% -*-latex-*-
% $Log: not supported by cvs2svn $
% Revision 1.1  2007/09/15 05:59:04  rabbah
% reformatted david's thesis for conference format. 15 pages, need to cut a few more and edit figures for size.
%
% Revision 1.1  2007/09/07 20:47:38  dxzhang
% Added thesis
%
% Revision 1.7  2001/02/08 18:53:16  boojum
% changed some \newpages to \cleardoublepages
%
% Revision 1.6  1999/10/21 14:49:31  boojum
% changed comment referring to documentstyle
%
% Revision 1.5  1999/10/21 14:39:04  boojum
% *** empty log message ***
%
% Revision 1.4  1997/04/18  17:54:10  othomas
% added page numbers on abstract and cover, and made 1 abstract
% page the default rather than 2.  (anne hunter tells me this
% is the new institute standard.)
%
% Revision 1.4  1997/04/18  17:54:10  othomas
% added page numbers on abstract and cover, and made 1 abstract
% page the default rather than 2.  (anne hunter tells me this
% is the new institute standard.)
%
% Revision 1.3  93/05/17  17:06:29  starflt
% Added acknowledgements section (suggested by tompalka)
% 
% Revision 1.2  92/04/22  13:13:13  epeisach
% Fixes for 1991 course 6 requirements
% Phrase "and to grant others the right to do so" has been added to 
% permission clause
% Second copy of abstract is not counted as separate pages so numbering works
% out
% 
% Revision 1.1  92/04/22  13:08:20  epeisach
\title{A Versatile Streaming Layer for Multicore Execution}

\authorinfo{Xin David Zhang, Qiuyuan J. Li, Rodric Rabbah, Saman Amarasinghe}
% Make the titlepage based on the above information.  If you need
% something special and can't use the standard form, you can specify
% the exact text of the titlepage yourself.  Put it in a titlepage
% environment and leave blank lines where you want vertical space.
% The spaces will be adjusted to fill the entire page.  The dotted
% lines for the signatures are made with the \signature command.
\maketitle

% The abstractpage environment sets up everything on the page except
% the text itself.  The title and other header material are put at the
% top of the page, and the supervisors are listed at the bottom.  A
% new page is begun both before and after.  Of course, an abstract may
% be more than one page itself.  If you need more control over the
% format of the page, you can use the abstract environment, which puts
% the word "Abstract" at the beginning and single spaces its text.

%% You can either \input (*not* \include) your abstract file, or you can put
%% the text of the abstract directly between the \begin{abstractpage} and
%% \end{abstractpage} commands.

% First copy: start a new page, and save the page number.

% Uncomment the next line if you do NOT want a page number on your
% abstract and acknowledgments pages.
% \pagestyle{empty}

\begin{abstract}
As DSP programming is becoming more complex, there is an increasing
need for high-level abstractions that can be efficiently compiled.
Toward this end, we present a set of aggressive optimizations that
target linear sections of a stream program.  Our input language is
StreamIt, which represents programs as a hierarchical graph of
autonomous filters.  A filter is linear if each of its outputs can be
represented as an affine combination of its inputs.  Linear filters
are common in DSP applications; examples include FIR filters,
expanders, compressors, FFTs and DCTs.

We present a linear extraction analysis that automatically detects
linear filters based on the C-like code in their {\tt work} function.
Once linear filters are identified, we show how neighboring nodes can
be collapsed into a single linear representation, thereby eliminating
many redundant computations.  Also, we describe a method for
automatically translating linear nodes into the frequency domain,
thereby yielding algorithmic savings for convolutional filters.

We have completed a fully-automatic implementation of the above
techniques as part of the StreamIt compiler, and we demonstrate
performance improvements that average 400\% over our benchmark
applications.




\end{abstract}

% Additional copy: start a new page, and reset the page number.  This way,
% the second copy of the abstract is not counted as separate pages.
% Uncomment the next 6 lines if you need two copies of the abstract
% page.
% \setcounter{page}{\thesavepage}
% \begin{abstractpage}
% As DSP programming is becoming more complex, there is an increasing
need for high-level abstractions that can be efficiently compiled.
Toward this end, we present a set of aggressive optimizations that
target linear sections of a stream program.  Our input language is
StreamIt, which represents programs as a hierarchical graph of
autonomous filters.  A filter is linear if each of its outputs can be
represented as an affine combination of its inputs.  Linear filters
are common in DSP applications; examples include FIR filters,
expanders, compressors, FFTs and DCTs.

We present a linear extraction analysis that automatically detects
linear filters based on the C-like code in their {\tt work} function.
Once linear filters are identified, we show how neighboring nodes can
be collapsed into a single linear representation, thereby eliminating
many redundant computations.  Also, we describe a method for
automatically translating linear nodes into the frequency domain,
thereby yielding algorithmic savings for convolutional filters.

We have completed a fully-automatic implementation of the above
techniques as part of the StreamIt compiler, and we demonstrate
performance improvements that average 400\% over our benchmark
applications.




% \end{abstractpage}

