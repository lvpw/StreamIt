\section{Introduction}

Multicore architectures have become the rule rather than the exception in the changing computing landscape. With single-core performance limited by power consumption, memory latency, and circuit complexity, almost all new architectures (certain mobile and embedded applications excepted) are branching into more cores rather than better cores. Exploiting parallelism has already become absolutely critical if applications wish to make full use of current and future architectures.

Compilers for traditional imperative languages are faced with a daunting task when attempting to aid the programmer in this regard: it is very difficult to automatically extract parallelism from a sequential program written in a von Neumann language such as C. Much of the time, the task of parallelizing a program remains in the hands of the programmer, who must manually convert a single-threaded sequential program into a multithreaded parallel one. While doing so, he or she must contend with issues specific to the architecture he or she is targeting, limiting portability, and worry about introducing race conditions or other bugs that typically plague multithreaded programs. Programmers do have access to a number of frameworks such as MPI and OpenMP; however, parallelization of sequential programs remains a difficult process.

Streaming languages provide a way to alleviate the burden of manually parallelizing applications. In a streaming language, the programmer defines actors that operate on streams of data; the programmer then composes actors and streams into a program. The structure that is explicitly expressed by a streaming language exposes rather than hides the parallelism present in a program, making it much easier for the compiler to automatically extract parallelism. For the programmer, many applications fit within the streaming model, and can be naturally expressed in StreamIt or other streaming languages.

Ideally, a compiler for a streaming language would like to be able to focus on high-level scheduling issues: finding parallelism and scheduling actors to obtain the best possible utilization of available computation resources. However, when presented with a heterogeneous, distributed-memory architecture, the compiler must also contend with numerous low-level issues.

The Cell architecture is a novel multicore processor that is an example of such an architecture. In this regard, it differs significantly from traditional SMP architectures: it is both a heterogeneous architecture and one that does \emph{not} use shared memory. Compared to traditional SMP architectures, the design of Cell is a trade-off for computing power, ease of manufacturing, and low power consumption at the cost of programming complexity. To a programmer writing an application that runs on Cell, this only leaves him with additional complications.

The goal of this paper is to create a general runtime framework for streaming applications to alleviate these complexities and that can be targeted to any multicore architecture. This runtime framework abstracts away the architecture-specific details that would otherwise complicate the scheduling of filters in a stream program, whether this is done by a programmer or a compiler. However, the framework is geared towards compilers rather than programmers. Just as a programmer manually parallelizing a sequential program has access to frameworks like MPI that abstract certain low-level operations, the goal of the streaming framework is to provide similar functionality to streaming language compilers and schedulers that target multicore architectures. In particular, this paper demonstrates an implementation of the framework for the Cell processor.

To this end, this paper contributes the following:
\begin{enumerate}
\item A specification of a general runtime framework for streaming applications.
\item An implementation of such a runtime framework for the Cell processor
\item A dynamic scheduler implemented on top of the runtime library that dynamically schedules a stream graph.
\item A static scheduler implemented on top of the runtime library that statically schedules a stream graph.
\end{enumerate}

With our implementation for Cell, we achieved at least 97\% utilization on data parallel applications, giving a reasonable 3\% of library overhead, and at least 90\% utilization on pipelined applications, giving an acceptable 10\% of library overhead. The amount of code needed is also significantly reduced compared to programming directly at a lower level, thereby simplifying the implementation of a scheduler.

%% new roadmap here
%%Chapters~2 presents basic background information and related work on streaming languages and parallelization frameworks, and describes the StreamIt language and the Cell architecture in detail. Chapter~3 presents the runtime library along with a discussion of design decisions and implementation issues. Chapter~4 describes how common scheduling patterns a compiler or programmer might generate can be mapped to the library, and compares code complexity and flexibility to programming for Cell directly. Chapter~5 discusses the dynamic scheduler. Chapter~6 gives a brief analysis of the performance of the framework on a test benchmark. Chapter~7 concludes with a discussion of future work and extensions.
