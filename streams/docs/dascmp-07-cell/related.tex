\section{Related Work}\label{ch:bg}

Streaming languages provide an attractive alternative to sequential
languages for many applications. In a streaming language, the
programmer defines actors that operate on data streams and composes
programs by connecting actors. Many common high-performance
applications, especially those involving signal, audio, image, and
video processing, have explicit block diagram structure and are most
easily conceptualized in that manner, making streaming languages a
natural choice for these applications.
Because programs written in streaming languages are explicitly
structured as communicating actors, they typically expose a high
degree of parallelism. This allows streaming language compilers to
easily analyze stream programs and efficiently parallelize
them. Compared to sequential languages, streaming languages free the
programmer from the burden of having to explicitly write parallelized
code for specific architectures, allowing him to focus on expressing
the high-level semantics of the algorithm in a natural way.
Recent developments in streaming languages include Brook~\cite{brook},
Cg~\cite{cg}, StreamC/KernelC~\cite{streamc}, and StreamIt~\cite{streamitweb}.
The MSL is not tied to any particular language choice and simply
relies on the expression of computation as a dataflow graph.
 
MPI and OpenMP are probably the two most well-known and widely used
parallelization frameworks for computing clusters and traditional SMP
architectures. These two programming APIs represent opposite ends of a
spectrum: MPI defines a language-independent, low-level network
communication API, while OpenMP defines a set of 
high-level annotations that programs can use to cause compatible
compilers to generate multi-threaded code.  The multicore streaming
layer and library are intended to provide the same kind of low-level
functionality as MPI, with two major differences: \emph{i}) it is more
suited for multicores instead of networks, and \emph{ii}) it is
specific to streaming applications and provides additional control
functionality.

The work presented here also complements parallelization frameworks
that have been designed specifically for multicores or Cell; see
\cite{cell:pf} for a review. RapidMind~\cite{rapidmind} and MPI
Microtask~\cite{mpimicrotask} follow a
``task-based'' approach by providing runtime systems that help
schedule program-defined tasks, or kernels. Mercury's MultiCore
Framework~\cite{mcf} adopts a similar approach. However, it is
primarily designed for exploiting data parallelism in large matrix computations instead of streaming. Stream Virtual Machine~\cite{svm} is a generic model for mapping stream computation; it is much broader in scope than what the MSL attempts to accomplish.
CellSs~\cite{cellss} takes
this a degree further by automatically generating tasks from
annotations to linear code. Sequoia~\cite{sequoia} is a language that exposes an abstract hierarchical memory model and allows the programmer to define communication and local computation; its compiler and runtime system can efficiently execute Sequoia programs on Cell. The MSL allows for the exploration of
various scheduling methodologies for executing programs on multicores.

%%XXX TODO: add SVM and MERCURY 
