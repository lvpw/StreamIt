% Name your report <group>-<unique-name>.tex, (e.g., sls-jupiter.tex)
% to avoid name collisions.  For \label{} entries within a file,
% please use \<group>-<unique-name>-<your label} (e.g.,
% \label{sls-jupiter-figure}), again to avoid name collisions.  There
% is no need to use unique names with \cite{} since each report has
% its own citation namespace.

% You can also use \formattitle{Title}{Authors} to gain formatting
% control (e.g., footnotes on author names or controlling line breaks
% with \\) and \formatcontents{Title}{Authors}.  Since we did not need
% such control, we simply used \formattitlecontents{Title}{Authors}
% here.  Please do not put any special formatting in
% \formattitlecontents or \formatcontents as this will disturb the
% table of contents.

\formattitlecontents 
{Scheduling of Execution of StreamIt Programs}
{Michal Karczmarek, Saman Amarasinghe}


% Please use the following \formatsection entries unless they are
% inappropriate: Introduction, Approach, Progress, Future, Research
% Support.
%
% Do not put a blank line after a \formatsection, as this affects the
% formatting.
\formatsection{Introduction}
Applications structured around some notion of a ``stream'' are
becoming increasingly important and widespread. \cite{Rix98}
provides evidence that streaming media applications are already
consuming most of the cycles on consumer machines, and their use
is continuing to grow. The streaming computation model is
pervasive and ranges from small, embedded systems (ex. cell
phones) to large, computationally powerful machines (ex. cell base
stations).

StreamIt \cite{streamitcc} is a language and a compiler designed to deal
with the new problems created by the shift in computation from general
purpose computing to stream processing.  One of the problems
faced by a compiler is scheduling of streaming computation in a way
that minimizes usage of available resources. In this document, we
describe a set of techniques developed to solve the problem of
scheduling. For a more detailed description of algorithms introduced
here, see \cite{karczma-thesis}.

\formatsection{Approach} StreamIt introduces a novel concept of
structure to streaming programming. The structures used by StreamIt
programs map the flow of streaming data into simple to understand
patterns. These patterns introduce an inherent hierarchy to the flow
of computation.

Our scheduling algorithms take advantage of this hierarchy in order
to simplify the task of scheduling: each structure only needs to
learn of its children's properties in order to produce a valid 
schedule. No global view of flow of data is required.

\noindent {\it Phased Scheduling}

The dominant technique used for scheduling streaming programs is
called Single Appearance Scheduling. In this technique, every
node in the stream program appears only once in every schedule.
This assures that the schedules produced are small. The trade-off,
however, is that these schedules often require a large amount of
storage memory to execute.

Phased Scheduling is a technique developed to deal with the issue
of large storage requirements, without overly growing the size of the
schedules. The technique relies on producing several small schedules
for each hierarchical element in the StreamIt program, each of
which works on a small dataset. These schedules are then grouped and 
interleaved in order to produce schedules that are small, but do not
require lots of data buffering.

\noindent {\it Scheduling Strategies}
\begin{table} \centering \small
\begin{tabular}{|c|c|c|c|c|c|c|c|}
\hline benchmark & \parbox{0.5in}{\centering number of nodes} & \parbox{0.5in}{\centering number of node executions} & \multicolumn{2}{c|}{pseudo single appearance} & \multicolumn{3}{c|}{minimal latency} \\
\cline{4-8} & & & \parbox{0.5in}{\centering schedule size} & \parbox{0.5in}{\centering buffer size} & \parbox{0.5in}{\centering schedule size} & \parbox{0.8in}{\centering compressed schedule size} & \parbox{0.5in}{\centering buffer size} \\
\hline bitonic sort & 370 & 468 & 439 & 2112 & 448 & 448 & 2112 \\
\hline CD-DAT & 6 & 612 & 7 & 1021 & 170 & 72 & 72 \\
\hline FFT & 26 & 488 & 31 & 3584 & 31 & 31 & 3584 \\
\hline filter bank & 53 & 312 & 166 & 2063 & 160 & 145 & 1991 \\
\hline FIR & 132 & 152 & 133 & 1560 & 133 & 133& 1560 \\
\hline radio & 30 & 43 & 58 & 1351 & 50 & 50 & 1351 \\
\hline GSM & 47& 3356 & - & - & 724 & 78 & 3900 \\
\hline 3GPP & 94 & 356 & 147 & 986 & 149 & 137 & 970 \\
\hline QMF & 65 & 184 & 143 & 1225 & 132 & 122 & 1225 \\
\hline radar & 68 & 161 & 100 & 332 & 100 & 100 & 332 \\
\hline SJ\_PEEK\_1024 & 6 & 3081 & 11 & 7168 & 40 & 16 & 4864 \\
\hline SJ\_PEEK\_31 & 6 & 12063 & 11 & 19964 & 250 & 24 & 12063 \\
\hline vocoder & 117 & 415 & 172 & 1285 & 293 & 206 & 1094 \\
\hline
\end{tabular}
\caption{Results of running pseudo single appearance and minimal
latency scheduling algorithms on various applications.}
\label{tbl:results}
\end{table}

The flexible approach to scheduling used by Phased Scheduling allows
for a wide variety of methods for producing schedules for each
hierarchical component in a StreamIt program as well as a wide variety
of techniques for grouping and interleaving these schedules
together. The combination of this flexibility allows us to select
whether emphasis should be put on minimizing resources necessary for
storing streaming data, minimizing resources necessary for storing
execution schedules, or finding an acceptable solution in between
these two conflicting goals.

\formatsection{Progress}
We have implemented Phased Scheduling as part of the StreamIt system.
Our current results from using a certain scheduling strategy (depicted
in Table~\ref{tbl:results}) show we are able to consistently reduce the
combined amount of resources necessary for storage of the schedule and
buffer space.  We achieve this by reducing buffer requirements by an
average of 14.5\% on a variety of benchmarks, with a peak savings of
93\%.

\formatsection{Future Work}
The current results of our scheduling techniques are promising, but
require much more work. We are planning on designing techniques which
will be able to conform to hard constraints, such as very limited amount
of memory and real-time latency constraints.

We also plan on designing scheduling algorithms that will allow us to
take advantage of low-bandwidth, asynchronous communication constructs
used by StreamIt to send non-streaming data between computational elements.

\formatsection{Research Support}
This work is supported by DARPA (PCA F29601-04-2-0166),
the NSF (CISE EIA-0071841), and fellowships from the
Singapore-MIT Alliance and the MIT-Oxygen Project.

\bibtex{cag-commit-streamit-scheduling}
