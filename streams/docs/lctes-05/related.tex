\section{Related Work}
\label{sec:related-work}

There is a large body of literature on scheduling synchronous dataflow
(SDF) graphs to optimize various
metrics~\cite{bhattacharyya99synthesis,leesdf}.  The work most closely
related to ours is a recent study by Kohli~\cite{kohli04} on
cache-aware scheduling of SDF graphs, implemented as part of the
Ptolemy framework for simulating heterogeneous embedded
systems~\cite{ptolemy03overview}.  Kohli develops a Cache-Aware
Scheduling (CAS) heuristic for an embedded target with a
software-managed scratchpad instruction cache.  His algorithm greedily
decides how many times to execute a given actor based on estimates of
the data cache and instruction cache penalties associated with
switching to the next actor.  In contrast, our algorithm considers the
buffering requirements of all filters in a given container and
increases the multiplicity so long as 90\% of buffers are contained
within the data cache.  Kohli does not consider buffer management
strategies, and the evaluation is limited to one 6-filter pipeline and
an assortment of random SDF graphs.  An empirical comparison of our
heuristics on a common architectural target would be an interesting
direction for future work.

It is recognized that there is a tradeoff between code size and buffer
size when determining an SDF schedule.  Most techniques to date have
focussed on ``single appearance schedules'' in which each filter
appears at only one position in the loop nest denoting the schedule.
Such schedules guarantee minimal code size and facilitate the inlining
of filters.  There are a number of approaches to minimizing the buffer
requirements for single-appearance schedules (see
\cite{bhattacharyya99synthesis} for a review).  While it has been
shown that obtaining the minimal memory requirements for general
graphs is NP-complete~\cite{Bhatta97}, there are two complimentary
heuristics, APGAN (Pairwise Grouping of Adjacent Nodes) and RPMC
(Recursive Partitioning by Minimum Cuts), that have shown to be
effective when applied together~\cite{Bhatta97}.  Buffer
merging\cite{murt1999x3,murt2000x2} represents another technique for
decreasing buffer sizes , which could be integrated with our approach
in the future.

Govindarajan et al. develop a linear programming framework for
determining the ``rate-optimal schedule'' with the minimal memory
requirement~\cite{GGD94}.  A rate-optimal schedule is one that takes
advantage of parallel resources to execute the graph with the maximal
throughput.  However, the technique is specific to rate-optimal
schedules and can result in a code size explosion, as the same node
could be executed in many different contexts.

The work described above is related to ours in that minimizing buffer
requirements can also improve caching behavior.  However, our goal is
different in that we aim to improve spatial and temporal locality
instead of simply decreasing the size of the live data set.  In fact,
our scaling transformation actually {\it increases} the size of the
data buffers, leading to higher performance across our benchmark
suite.  Our transformations also take into account the size of the
instruction and data caches to select an appropriate scaling and
partitioning for the stream graph.

Proebsting and Watterson \cite{pro96} give a fusion algorithm that
interleaves the control flow graphs of adjacent filters.  However,
their algorithm only supports synchronous {\tt get} and {\tt put}
operations; StreamIt's {\tt peek} operation necessitates buffer
management between filters.

There are a large number of stream programming languages; see
\cite{survey97} for a review.  The Brook language~\cite{Buck04}
extends C to include data-parallel kernels and multi-dimensional
streams that can be manipulated via predefined operators.  Synchronous
languages such as Esterel~\cite{esterel92} and LUSTRE~\cite{lustre}
also target the embedded domain, but they are more control-oriented
than StreamIt and are less amenable to compile-time optimizations.
Benveniste et al.~\cite{benveniste93dataflow} also provides an
overview of dataflow synchronous languages.  Sisal (Stream and
Iteration in a Single Assignment Language) is a high-performance,
implicitly parallel functional language~\cite{sisal}.  We are not
aware of any cache-aware optimizations in these stream languages.

%(cache models?)