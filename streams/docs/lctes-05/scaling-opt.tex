We have already alluded to multiplicity scaling in previous
sections. As the instruction cache model shows, increasing the number
of consecutive firings of the same actor leads to lower miss
rates. However, scaling increases the data buffers that are maintained
between actors. Thus it is prudent that we account for the data
working set requirements as we scale a steady state.

\begin{figure}[t]
\begin{center}
\framebox{\parbox{3.25in}{
\mbox{} \textsc{Algorithm 1.} {\it (Calculate Multiplicity Factor)} \vspace{6pt}\\
\mbox{} ~~~~// {\it Returns a multiplicity factor for steady state $S$} \\
\mbox{} ~~~~// - $C_D$ is the data cache size\\
\mbox{} ~~~~// - $\alpha$ is the fraction of $C_D$ dedicated for I/O $(0 < \alpha)$\\
\mbox{} ~~~~// - $p$ is the desired percentile of all actors that are\\
\mbox{} ~~~~// satisfied by chosen multiplicity $(0 < p \le 1$)\\
\mbox{} ~~~~{\bf calculateMultiplicity}($S$, $C_D$, $\alpha$, $p$) \{\\
\mbox{} ~~~~~~~{\bf create} $\mt{array}$ of size $|S|$ and {\bf initialize} all entries to 0\\
\mbox{} ~~~~~~~{\bf for} $i$ = 1 to $|S|$ \{ \\
\mbox{} ~~~~~~~~~~$X = S[i]$\\
\mbox{} ~~~~~~~~~~// calculate effective cache size\\
\mbox{} ~~~~~~~~~~$C = \alpha \times (C_D - \mt{State}(X))$\\
%\mbox{} ~~~~~~~~~~// estimate I/O requirements\\
%\mbox{} ~~~~~~~~~~$d = \mt{IO}(X)$\\
\mbox{} ~~~~~~~~~~// calculate largest possible scaling for $X$\\
\mbox{} ~~~~~~~~~~$M_X = \lfloor C / \mt{IO}(X) \rfloor$\\
\mbox{} ~~~~~~~~~~// array indexed from 0 \\
\mbox{} ~~~~~~~~~~$\mt{array}[i - 1] = M_X$\\
\mbox{} ~~~~~~~\} \\
\mbox{} ~~~~~~~{\bf sort} $\mt{array}$ into ascending numerical order\\
\mbox{} ~~~~~~~$i = \mt{{\bf round}}((1-p) \times (|S| - 1))$ \\
\mbox{} ~~~~~~~return $\mt{array}[i]$\\
\mbox{} ~~~~\}
}}
\end{center}
\vspace{-12pt}
\end{figure}

Our approach is to scale the entire steady state by a single
multiplicity factor, with the constraint that only a small percentage
of the actors overflow the data cache. Our algorithm is described in
Algorithm~1. First, it calculates the largest possible multiplicity
for every actor in the steady state. Second, it choses the largest
factor that allows a fraction ($p$) of the steady state actors to be
scaled safely (i.e., the cache is adequate for their I/O
requirements).  For example, the algorithm might calculate
$M_\texttt{A} = 10$,
$M_\texttt{B} = 20$, 
$M_\texttt{C} = 30$, and 
$M_\texttt{D} = 40$, for four actors in some steady state.
In this case, the largest $M$ that allows 90\% of the actors to be
scaled without violating the cache constraints is 10.
Similarly, to allow for the safe scaling of 75\% of the actors, the
largest factor we can choose is 20.

In our implementation, we use the 90-10 heuristic. In other words, we
set $p=90\%$. We empirically determined this value via a series of
experiments using our benchmark suite. Specifically, we varied the
multiplicity factor across a wide range of values,
and measured the resulting performance. We found that
increasing the threshold beyond 90\% leads to marginal
performance gains. Due to the limited space we cannot include all of
the data that was collected. However, the interested reader can find
the data in a thesis by one of the authors~\cite{janis-thesis}.

Note that our algorithm for adjusting the multiplicity of a steady
state adjusts the effective cache size that is reserved for an actor's
dynamic working set (i.e., data accessed via {\tt pop} and {\tt
push}). This adjustment allows us to control the fraction of the cache
that is used for reading and writing data---and affords some
flexibility in targeting various cache organizations.  For example,
architectures with highly associative and multilevel caches may benefit
from scaling up the effective cache size (i.e., $\alpha > 1$), whereas
a direct mapped cache that is more prone to conflicts may benefit from
scaling down the cache (i.e., $alpha < 1$). In our implementation, we
found $\alpha=2/3$ to work well. However, we note that tuning the
effective cache size is not only dependent on the underlying cache
organization, but also dependent on the instruction  working
set size, and the  I/O properties of the actors in the steady
state; this is an interesting issue that warrants further investigation.