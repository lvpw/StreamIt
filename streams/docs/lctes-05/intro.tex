\section{Introduction}


Stream computing represents an increasingly important class of
applications. In streaming codes, there is an abundance of parallelism that
is easier to extract compared to traditional desktop workloads (e.g.,
pointer-based computing). As a result, the extraction of parallelism
in streaming codes does not require heroic efforts, and thus,
processors can deliver higher performance with significantly lower
power costs. This is especially important since
leading microprocessor companies have realized that modern general
purpose architectures are near their  performance limits for  the
amount of power they consume. Thus, the future will place a greater
emphasis on exploiting the properties of streaming workloads in
conventional von~Neumann architectures.

Streaming is a model of computation that uses sequences of data
and computation kernels to expose concurrency and locality for
efficiency~\cite{wss}. In general purpose processors, improving locality 
translates to an effective management of the memory hierarchy at all
levels, including the register file. In this paper, we present a
methodology for compiling streaming codes to general purpose,
cache-based architectures. We first introduce a simple model for
reasoning effectively about the caching behavior of streaming
workloads. This models serves as a foundation for several {\it cache-aware
optimizations} that are geared toward the concomitant increase of instruction
and data {\it temporal locality}. These
optimization lead to significantly better utilization of the memory
system, and as such, they deliver performance gains ranging from XXX
to XXX\% for our streaming benchmark suite.

The context for our work is StreamIt, an architecture-independent
language that is engineered for streaming
applications~\cite{streamit}. It adopts the 
Cyclo-Static Dataflow~\cite{BELP96} model of computation which is a
generalization of Synchronous Dataflow~\cite{LM87-i} (SDF).  
SDF is a popular  modelthat  is well suited for
streaming codes. In SDF, computation is represented as a graph
consisting of {\it  actors} connected by communication channels; the
actors consume  and produce a constant number  of items from their
input and output  channels every time they execute. SDF is appealing
because it is amenable to static scheduling and optimization. 

From a general purpose architecture's point of view, actors represent
computation kernels, and the communication between actors represents
data buffers that must be streamed to and from the processor. Thus
the size of an actor and the
order of actor executions are critical properties that
impact the performance of the instruction cache. For example, the
compiler must make sure the actor's code size is not
greater than the instruction cache. Furthermore, we must {\it scale}
the execution of the actor so that it runs several times before we move
on to some other actor in the stream 
graph. This serves to amortize the cost of fetching the actor's
instructions into the cache from memory (an expensive operation), and
improves the instruction temporal locality, leading to better overall
performance. However, as our cache model will show, we 
cannot arbitrarily scale the execution frequency of an actor. This
is because actors produce data that must be buffered, and therefore,
we must also consider the amount of data an actor produces and
consumes if we are to adequately manage the data cache. This paper is unique
in that it is the first to present a unified optimization methodology
that simultaneously considers instruction and data locality for
mapping streaming computation to cache-based architectures.

In terms of improving the data cache behavior, the compiler schedules
actor firing such that the producer-consumer locality is
preserved. Furthermore,  the compiler may {\it fuse}
together two or more actors to form a coarser grained kernel.
The fusion allows for better register allocation as we can
destroy the arrays used to buffer data between the actors and replace
the corresponding array references with scalars.  It also allows for
various competing implementations for managing the buffers between the
fused actors.  This paper evaluates several implementation
alternatives (for buffer management) and introduces a formal model for
estimating their impact on performance.

The methodology for fusing actors leverages a distinguishing StreamIt
characteristic, namely, the hierarchical organization of
the stream graph. Furthermore, the algorithm for fusing actors applies
for the various topologies allowed by StreamIt.
It also considers another distinguishing characteristics of StreamIt,
namely the {\tt peek} operation whereby an actor may inspect data
items in its input buffer without consuming them until some future
execution. While peeking is a powerful language feature, it does pose
some challenges to the compiler and the cache optimizations. Peeking
also impacts the choice for the best buffer management strategy, as our
model will show.

%% the comment about p3 and itanium not being embedded architectures
%% is out of the blue! need a better transition.
Fusion alone delivers significant performance gains, although our
evaluation shows that fusion with scaling leads to the best
performance on a general purpose, cache-based architecture. For our
experiments, we use two different processors: a superscalar out-of-order
processor, and an in-order VLIW processor. The former is a Pentium~3
whereas the latter is an Itanium~2. While these architectures are not
particularly suited for an embedded system, they do exhibit some
properties that are worthy of investigation. Furthermore, that we can
demonstrate measurable performance gains on real systems is far more
convincing than using a simulation-based environment. We chose the
Pentium~3 processor because it has very few registers in its ISA
(instruction set architecture). The Itanium by contrast has a much
larger and richer repertoire of registers. The two architectures serve
to validate our fusion optimizations, in that we expect an
architecture with more register to benefit more from optimization such
as scalar replacement. On average, fusion leads to a XXX\% improvement
on the Pentium~3, and XXX\% on the Itanium~2.

The two architectures also differ in terms of their memory system
organization. The Itanium is an in-order VLIW processor and does not
tolerate a memory stall as well as its out-of-order
counterpart. Therefore we expect different gains from the scaling
optimization which amortize the long access latencies for instruction
and data caches. On average, scaling leads to a XXX\% improvement on
the Itanium~2, and XXX\% on the Pentium~3.

While both scaling and fusion lead to modest performance gains, we
must combine the two to deliver the best possible performance. When we
do so, we can further improve the performance of our benchmarks by
XXX\% on average for the Pentium~3, and XXX\% for the Itanium~2.

\subsection{Summary of Contributions}

This paper makes the following contributions:
\begin{itemize}
\item XXX cache-model for stream computing
\item XXX cache-aware scheduling
\item XXX cache-aware graph partitioning
\item XXX fusion model
\item XXX fusion optimizations
\item XXX full implementation 
\item XXX analysis and results
\end{itemize}

\subsection{Paper Roadmap}

The remainder of the paper is organized as follows. Section~\ref{sec:streamit}
describes StreamIt and introduces our motivating example.
Section~\ref{sec:cache-model} introduces our cache model for 
reasoning about the performance of a streaming
computation. Section~\ref{sec:cache-opt} describes our cache-aware
optimizations, and Section~\ref{sec:buffer} describes the 
optimization enabled by fusion. Section~\ref{sec:evaluation} describes
our evaluation methodology and present our experimental
analysis. Sections~\ref{sec:related-work}~and~\ref{sec:conclusion}
discuss related work and conclude the paper.
