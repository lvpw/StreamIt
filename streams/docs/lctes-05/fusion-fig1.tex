\begin{figure*}[t]
\begin{minipage}{1.7in}
\centering
\psfig{figure=fusion-pipeline.eps,width=0.7in}

\caption{Stream graph for a synthetic buffer test.\protect\label{fig:code-graph}}
\end{minipage}
\hspace{0.3in}
\begin{minipage}{2.2in}
\centering
{\scriptsize
\begin{verbatim}
void->void pipeline BufferTest {
  add Source();
  add FIR();
}

void->float filter Source {
  work push 1 {
    push(random());
  }
}

float->void filter FIR {
  int PEEK = 4;
  work pop 1 peek PEEK {
    float result = 0;
    for (int i=1; i<PEEK; i++) {
      result += i*peek(i);
    }
    pop();
    print(result);
  }
}
\end{verbatim}}

\caption{Original StreamIt code for the buffer test.\protect\label{fig:code-orig}}
\end{minipage}
\hspace{0.3in}
%
\begin{minipage}{2.2in}
\centering
%% modulation
{\scriptsize
\begin{verbatim}
void->void pipeline BufferTest {
  int PEEK = 4;
  float[4] BUFFER;
  int push_index = 0;
  int pop_index = 0;

  prework {
    for (int i=0; i<PEEK-1; i++) {
      BUFFER[push_index++] = random();
    }
  }

  work {
    // run Source
    BUFFER[push_index] = random();
    push_index = (push_index + 1) & 3;
    
    // run FIR
    float result = 0;
    for (int i=1; i<PEEK; i++) {
      result += i*BUFFER[(pop_index + i) & 3];
    }
    pop_index = (pop_index + 1) & 3;
    print(result);
  }
}
\end{verbatim}}

\caption{Fused buffer test using modulation buffer management strategy.\protect\label{fig:code-modulation}}
\end{minipage}
\vspace{6pt}
\hrule
\vspace{6pt}
\end{figure*}
