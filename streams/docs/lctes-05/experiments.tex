\section{Experimental Evaluation}
\label{sec:evaluation}

\begin{table}[t]
\center
\label{tab:benchmarks}
\vspace{-12pt}
{\tiny
\begin{tabular}{|c|c|c|} \hline
{\bf benchmark}&{\bf description}&{\bf \# of actors}\\ \hline \hline
\texttt{bitonic	} &bitonic sort of 64 integers	&	972 \\ \hline
\texttt{fir	      } &finite impulse response program	&	132 \\ \hline
\texttt{fft-fine	} &fine grained FFT implementation	&	267 \\ \hline
\texttt{fft-coarse} &coarse grained FFT implementation	&	26 \\ \hline
\texttt{3gpp	} &3GPP Radio Access Protocol application	&	105 \\ \hline
\texttt{beamformer} &beamformer with 64 sources and 1 detector& 197 \\ \hline
\texttt{matmult	} &matrix multiplication	&	48 \\ \hline
\texttt{fmradio	} &FM Radio with 10 way equalizer	&	49 \\ \hline
\texttt{filterbank} &simple filterbank program	&	53 \\ \hline
\texttt{filterbank2}&alternate implementation of a filterbank &	37 \\ \hline
\texttt{ofdm	 }& Orthogonal Frequency Division Multiplexor~\cite{spectrumware}	&	16 \\ \hline
\end{tabular}
}
\vspace{-12pt}
\caption{Evaluation benchmark suite.}
\end{table}

\begin{figure}
\begin{minipage}{3.35in}
\psfig{figure=arm1.eps,width=3.4in}
\caption{Full fusion versus \texttt{CAF+scaling+buffer} on a StrongARM.\protect\label{fig:arm-perf}}

\psfig{figure=p3-1.eps,width=3.4in}
\caption{Full fusion versus \texttt{CAF+scaling+buffer} on a Pentium~3\protect\label{fig:p3-perf}}

\psfig{figure=i2-1.eps,width=3.4in}
\caption{Full fusion versus \texttt{CAF+scaling+buffer} on an Itanium~2\protect\label{fig:i2-perf}}
\end{minipage}
\end{figure}

\begin{figure}
\begin{minipage}{3.35in}
\psfig{figure=arm2.eps,width=3.4in}
\caption{Performance of various compilation strategies on a StrongARM.\protect\label{fig:arm-perf2}}

\psfig{figure=p3-2.eps,width=3.4in}
\caption{Performance of various compilation strategies on a Pentium~3.\protect\label{fig:p3-perf2}}

\psfig{figure=i2-2.eps,width=3.4in}
\caption{Performance of various compilation strategies on an Itanium~2.\protect\label{fig:i2-perf2}}
\end{minipage}
\end{figure}

In this section we evaluate the merits of the proposed cache-aware
optimizations and buffer management strategies.  We use three
different architectures: a 137MHz StrongARM-1110, a 600~MHz Pentium~3
and a 1.3~GHz Itanium~2. The StrongARM results reflect performance for
an embedded target; it has a 16Kb L1 instruction cache, an 8Kb L1 data
cache, and no L2 cache.  The Pentium~3 and Itanium~2 reflect desktop
performance; they have a 16Kb L1 instruction cache, 16Kb L1 data
cache, and 256Kb shared L2 cache.

Our benchmark suite (see Table~\ref{tab:benchmarks}) consists of
eleven StreamIt applications. They are compiled with the StreamIt
compiler which applies the optimizations described in this paper, as
well as aggressive loop unrolling (by a factor of 128 for all
benchmarks) to facilitate scalar replacement
(Section~\ref{sec:buffer}).  The StreamIt compiler outputs a
functionally equivalent C program that is compiled with \texttt{gcc}
(v3.4, -O3) for the StrongARM and for the Pentium~3 and with
\texttt{ecc} (v7.0, -O3) for the Itanium~2.  Each benchmark is then
run five times, and the median user time is recorded.

\paragraph*{Overall Speedup} 
The overall speedups offered by our techniques are illustrated in
Figure~\ref{fig:arm-perf} (StrongARM), Figure~\ref{fig:p3-perf}
(Pentium~3), and \ref{fig:p3-perf} (Itanium~2).  These graphs have two
bars: one for ``full fusion'', in which all filters are fused into a
single function (with scalar replacement), and one labelled
``CAF+scaling+buffer'', representing all of our optimizations (cache
aware fusion, execution scaling, and buffer optimizations) applied
together.  We include the comparison to full fusion because it
represents the most simple approach for eliminating function call
overhead and optimizing across filter boundaries; however, it is
oblivious to instruction and data locality, and lacks aggressive
buffer optimizations.  Performance is normalized to unoptimized
StreamIt, in which no filters are fused (but there is still unrolling
by 128).  Average execution times are shown on the right of each
graph.

On StrongARM (Figure~\ref{fig:arm-perf}), our cache optimizations
offer a 249\% average speedup over the baseline and a 162\%
average speedup over full fusion.  Cache optimizations always perform
better than the baseline, and they perform better than full fusion in
all cases except for \texttt{3gpp}, where they yield a 45\% slowdown.
This slowdown is due to conservative code size estimation: the
compiler predicts that the fused version of \texttt{3gpp} will not fit
into the instruction cache, thereby preventing fusion.  However, due
to optimizations by {\tt gcc}, the final code size is smaller than
expected and does fit within the cache.  While such inaccuracies could
be improved by adding feedback between the output of {\tt gcc} and our
code estimation, each fusion possibility would need to be evaluated
separately as the fusion boundary affects the impact of low-level
optimizations (and thus the final code size).

The speedups offered by cache optimizations over a full fusion
strategy are more modest for the desktop processors: 34\% average
speedup on Pentium~3 (Figure~\ref{fig:p3-perf}) and 6\% speedup on
Itanium~2 (Figure~\ref{fig:i2-perf}).  Out of the 11 benchmarks, cache
optimizations perform as well or better than full fusion for 7
benchmarks on the Pentium~3 and 5 benchmarks on the
Itanium~2.  Performance on any architecture is a tradoeff between two
factors: 1) the benefit of data and instruction locality, and 2) the
benefit of fusion, which reduces memory accesses due to improved
register allocation across filter boundaries.  Compared to the
StrongARM, the Pentium~3 and Itanium~2 offer an L2 cache (as well as a
larger L1 data cache), thereby lessening the impact of
locality-enhancing cache optimizations.  However, the fusion benefit
remains a significant factor; for example, using Intel VTune on the
Pentium~3, we measured that full fusion offers a 50\% reduction in
memory accesses over the cache-optimized version.  This effect may be
pronounced in Itanium~2 due to the larger number of registers on that
architecture (128 general, 128 floating point).  While fusion benefits
are also present on StrongARM, cache optimizations are more important
on that processor due to the large penalty for cache misses.

The cache-aware fusion algorithm can be adjusted to account for the
caching hierarchy on desktop machines.  Specifically, if cache-aware
fusion is modified to allow fused filters to consume up to one half of
the L2 cache (rather than L1 cache), then the performance of cache
optimizations is closer or equal to full fusion, for cases it was
previously trailing on Pentium~3 or Itanium~2 (data not shown).  The
only benchmark that is negatively impacted by this change is {\tt
ofdm}, where two large filters are fused despite a very low
communication to computation ratio, thereby lessening the impact of
eliminated memory accesses while nonetheless worsening the instruction
locality.

\paragraph*{Impact of Each Optimization}
To better understand the overall speedups, we assess the individual
performance impact of execution scaling, cache aware fusion, and
buffer management optimizations.  These results are illustrated in
Figures~\ref{fig:arm-perf2}, \ref{fig:p3-perf2} and~\ref{fig:i2-perf2}
for the StrongARM, Pentium~3, and Itanium~2, respectively.  There are
four bars per benchmark; as in the previous analysis, all performance
is normalized to unoptimized StreamIt (with 128-way unrolling).  The
first bar, labeled {\tt scaling}, applies only execution scaling.  The
second bar, labeled {\tt CAF}, applies only cache-aware fusion.  The
third bar, labeled {\tt CAF+scaling}, first applies cache-aware fusion
and then applies execution-scaling to the granularity-adjusted
filters.  The fourth bar, labeled {\tt CAF+scaling+buffer},
additionally applies buffer management optimizations (detailed later);
this bar is equivalent to the ``best'' cache-optimized performance
illustrated in Figures~\ref{fig:arm-perf}, \ref{fig:p3-perf}, and
\ref{fig:i2-perf}.

Execution scaling improves performance over unoptimized StreamIt, with
average speedups of 145\% for StrongARM, 58\% for Pentium~3, and 57\%
for Itanium~2.  {\bf [[refs to scaling charts?]]}.  The 90-10
heuristic works quite well and there is only one instance where
scaling results in a performance degradation. The granularity-adjusted
\texttt{3gpp} on StrongARM has a 17\% slowdown due to scaling (compare
{\tt CAF} to {\tt CAF+scaling} in Figure~\ref{fig:arm-perf}).  This is
possibly due to items in flight between the granularity-adjusted
actors overwriting the state of an executing actor in the data cache.
Since StrongARM has no L2 cache then such eviction can be quite
expensive.

Independently, cache-aware fusion also improves performance by 
84\% on StrongARM, 101\% on Pentium~3 and a 103\% on the Itanium~2. 
Cache-aware fusion degrades performance only for
\texttt{filter}-\texttt{bank2} (by 6\% on StrongARM). When we combine
cache-aware fusion with execution scaling, the performance
consistently improves.  The speedup of \texttt{CAF+scaling} over
baseline is 241\% on StrongARM, 146\% on Pentium~3 and 144\% on
Itanium~2.

However, after coarsening the actors with cache-aware fusion, scaling
results in less additional speedup than it did relative to the
baseline.  The speedup of \texttt{CAF+scaling} over \texttt{CAF} is
86\% for StrongARM, 22\% for Pentium~3 and only 20\% for Itanium~2.
This is because some actors are implicity scaled by fusion to match
input/output rates of succesive actors within a fused block.

Note that the \texttt{ofmd} benchmark does not benefit from fusion or
scaling. This is because \texttt{ofmd} has few filters, some of which
consume and produce a total of 16-66Kb data; consequently, execution
scaling does not apply.  Also there is limited oportunity to fuse
filters within \texttt{ofmd}, as there are filters that have an
instruction size of 9Kb and fusing them with other filters would
exceed the instruction cache.

The last bar, {\tt CAF+scaling+buffer}, illustrates the benefit of
buffer management optimizations for filters that peek.  As detailed in
section~\ref{sec:fusion-opt}, such filters demand specialized buffer
management, as they reuse items on the input tape across successive
iterations.  In our benchmark suite, peeking occurs only in the last
four applications (i.e., \texttt{fmradio}, \texttt{filterbank},
\texttt{filterbank2} and \texttt{ofdm}).  Thus, the {\tt CAF+scaling}
bar is equivalent to {\tt CAF+scaling+buffer} for all benchmarks from
{\tt bitonic} through {\tt matmult}.

The buffer optimization applied in Figures~\ref{fig:arm-perf},
\ref{fig:p3-perf}, and \ref{fig:i2-perf} is termed {\tt cutpeek}.  In
this optimization, the cache-aware fusion algorithm is modified so
that two adjacent filters are never fused if the downstream filter
performs any peeking (i.e., it has $\mbox{\it peek} > \mbox{\it pop}$.
After we apply execution scaling to the granularity-adjusted actors,
we apply buffer expansion.  The live items are copied to the start of
the buffer once after executing the granularity-adjusted actor for the
number of times that is equal to the global multiplicity scaling. This
reduces the overhead of copying live items.  Optimized buffer
management offers the largest gains for {\tt filterbank2}: 45\%
on StrongARM, 38\% on Pentium~3, and 36\% on Itanium~2.
This is due to a large peek rate (100 items) in addition a large
global multiplicity of 300 that amortizes the cost of copying
items.

While we found that {\tt cutpeek} is, on average, the best buffer
management strategy across the three platforms, we also evaluated
several others.  For {\tt fmradio}, up to a 5\% improvement is offered
by a strategy called {\tt peekscale}: filters with $\mbox{\it
peek}>\mbox{\it pop}$ are replaced with an equivalent filter that
executes the original multiple times to ensure that {\it pop} is at
least $4*\mbox{\it peek}$.  Like {\tt cutpeek}, this serves to
amortize the overhead of copy-shift buffer management, but it can also
hamper the execution scaling optimization.  Because certain other
filters in the graph must fire more frequently to compensate for the
scaled peeking filter, there is less room for global execution
scaling.  On the StrongARM, {\tt peekscale} also benefits {\tt ofdm}
by up to {\bf ???}.  Finally, on the {\tt filterbank} benchmark,
highest performance results from unoptimized copy-shift buffer
management.  We also evaluated buffer management using modulation for
indexing, but this did not yield the best performance for any
benchmark or platform.

%% \texttt{CAF+scaling} is our baseline for evaluating optimized buffer 
%% management strategies. In \texttt{CAF+scaling} the live items are 
%% copied to the beginning of the buffer after every execution of a filter 
%% that has $peek>pop$.

%% \texttt{CAF+scaling+peekscale} strategy replaces filters that have
%% $peek>pop$ with an equivalent filter that executes the original filter 
%% multiple times to make sure that $pop$ is at least $4*(peek-pop)$.
%% This reduces the overhead of copying live items, since we only copy live 
%% items once after each execution of the scaled actor.

%% \texttt{CAF+scaling+mod} strategy uses modular buffers for channels where 
%% downstream filter has $peek>pop$.

%% \texttt{CAF+scaling+cutpeek} modifies the CAF algorithm so that two 
%% adjacent filters are never fused if the downstream filter has 
%% $peek>pop$. After we apply multiplicity scaling to the 
%% granularity adjusted actors we apply buffer expansion. The 
%% live items are copied to the start of the buffer once after executing 
%% the granularity adjusted actor for the number of times that is equal 
%% to the global multiplicity scaling. This reduces the overhead of 
%% copying live items.

%% \texttt{CAF+scaling+cutpeek+mod} also modifies the CAF algorithm so 
%% that two adjacent filters are never fused if the downstream filter has 
%% $peek>pop$ and modular buffers are used for channels 
%% where downstream filter has $peek>pop$.

%% We found that on average \texttt{CAF+scaling+cutpeek} works best across 
%% the 3 platforms and this is the buffer optimizatio included in our 
%% performance numbers. However, \texttt{CAF+scaling+peekscale} is the 
%% best strategy for fmradio benchmark and our baseline \texttt{CAF+scaling} 
%% is best for the filterbank benchmark.

%% The Itanium~2 has less speedup possibly
%% because of data and instruction prefetch instructions generated by the 
%% \texttt{ecc} compiler.

%% By comparsion, the scaling results on the 
%% Itanium are more uniform, and in fact, scaling does not degrade the
%% performance of any of the benchmarks in suite. 

%% The discrepancy between
%% the Pentium and Itanium results is likely due to the in-order
%% nature of the latter. Unlike the Pentium which can tolerate a cache
%% miss by issuing instructions out-of-order, the Itanium pipeline stalls
%% when an outstanding memory request is not serviced in time for
%% its consuming instruction to execute. The Itanium VLIW architecture
%% therefore places a greater emphasis on the memory hierarchy, and our
%% locality enhancing optimizations help to bridge the gap between
%% processor and memory speeds.

%% whereas it improved the performance of the applications on the Itanium.

%% a few benchmarks on both platforms.
%% A performance degradation as a result of fusion is generally caused by
%% an increase in the buffer management overhead. 


%% Note that our benchmark suite
%% includes two implementations of \texttt{FFT}. The first is a
%% fine-grained implementation (\texttt{fft-fine}) and the other is a
%% coarse-grained implementation (\texttt{fft-coarse}). The former
%% benefits more from our cache-aware fusion because there is greater
%% versatility in fusing filters to achieve balanced actors. On the other hand
%% a coarse grained implementation restricts the amount of fusion that
%% can be performed and increases the burden for more efficient buffer
%% management. 


%% In the case of \texttt{fft-coarse}, the scalar-replacement
%% strategy regains nearly 30-70\% of the performance when applied in
%% conjunction with fusion.

%% In this case, the granularity-adjusted actors
%% are scaled to maximize cache locality. In addition, the scalar replacement
%% and copy-shift in conjunction with scaling helps to reduce the overhead of 
%% moving data between actors. As a result, the performance gains can be quite
%% dramatic, ranging up to a 5800\% speedup for \texttt{bitonic} on Pentium~3
%% and up to a 6400\% speedup for \texttt{bitonic} on Itanium~2.
