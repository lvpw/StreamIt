\SubSection{   }

% Note: Some of this is repeat information from the mpeg.tex file. The
% first sentence is verbatim from mpeg.tex. Do we want to include
% less detail in the MPEG section or less detail here? We probably
% don't want to be totally redundant.

Macroblocks specify colors using a {\it luminance} channel to
represent saturation (color intensity), and two {\it chrominance}
channels to represent hue. The human eye is more sensitive to changes
in saturation than changes in hue, so the chrominance channels
are frequently compressed by downsampling the chrominance data within
a macroblock. The type of chrominance downsampling an MPEG-2 file
uses is its {\it chrominance format}. The most common chrominance
format is 4:2:0, which uses a single block for each of the chrominance
channels, downsampling each of the two channels from 16x16 to 8x8. 
The other chrominance format with downsampling is 4:2:2, which uses two
blocks for each chrominance channel, downsampling each of the channels
from 16x16 to 8x16, The possible chrominance formats are shown in
Figure~\ref{chroma_format}.

ADD SOME TEXT HERE

Adding support for the 4:2:2 chrominance format required modifying 31 lines
and adding 27 additional lines. Of the 31 modified lines, 23 were trivial
modifications to pass a variable representing the chrominance format
as a stream parameter or replace hardcoded block count values with variables.
Of the 27 added lines, 7 were fixing a flawed assumption in the MPEG stream
parsing which 4:2:0 had not uncovered and had nothing to do with the 4:2:2
data. A typical set of changes is shown in Figures~\ref{chroma_original} and~\ref{chroma_modified}, which
illustrates the code wrapping the parallel processing of the different
color channels. The changes needed to support the additional 4:2:2 format
are base

% In what follows some of the variables are renamed from what they
% are in the actual MPEGdecoder.str.pre file since the names aren't
% as meaningful in the context of the paper.
% The changes are listed here:
% UpdatePortal_picture_type -> pic_messages
% the_chroma_format -> chroma_format
% LuminanceChannelProcessing -> LuminanceUpsampleAndMotionPredict
% ChrominanceChannelProcessing -> ChrominanceUpsampleAndMotionPredict
\begin{figure*}[t]
  \begin{minipage}[t]{4.0in}
    {
  \begin{scriptsize}
    \begin{verbatim}
add int->int splitjoin {


  split roundrobin(4*75, 1*75, 1*75);
  

  add LuminanceUpsampleAndMotionPredict
    (width, height, pic_messages);
  

  add ChrominanceUpsampleAndMotionPredict
    (width, height, pic_messages);
  add ChrominanceUpsampleAndMotionPredict
    (width, height, pic_messages);
  

  join roundrobin(1,1,1);  
}
 	  \end{verbatim}
	\end{scriptsize}
    }
    % \vspace{-3pt}
    \caption{High level distribution of data (4:2:0 chrominance format).}
    \label{chroma_original}
    % \label{fig:zigzag-filter}
  \end{minipage}
  ~~\vrule~~
  \begin{minipage}[t]{3.0in}
    {  
	\begin{scriptsize}
	  \begin{verbatim}
add int->int splitjoin {
  if (the_chroma_format == 1) {
    split roundrobin(4*75, 2*75);
  } else {
    split roundrobin(4*75, 4*75);
  }
  add LuminanceUpsampleAndMotionPredict
    (width, height, pic_messages, chroma_format);
  add int->int splitjoin {
    split roundrobin(75);
    add ChrominanceUpsampleAndMotionPredict
      (width, height, pic_messages, chroma_format);  
    add ChrominanceUpsampleAndMotionPredict
      (width, height, pic_messages, chroma_format);  
    join roundrobin(1, 1);
  }
  join roundrobin(1, 2);
}
	  \end{verbatim}
	\end{scriptsize}
    }
    % \vspace{-3pt}
    \caption{High level distribution of data (4:2:0 or 4:2:2 chrominance format).}
    \label{chroma_modified}
  \end{minipage}
\end{figure*}









