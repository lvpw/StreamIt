\Section{Related Work}

% TODO:
% - fill in first few references
% - improve citations of multicore architectures
% - sanitize references (fields ok, consistent abbrev, capitalization)

Video codecs such as MPEG-2 have been a longtime focus of the embedded
and high-performance computing communities.  We consider related work
in modeling environments, stream languages and parallel computing.

There have been many efforts to develop expressive and efficient
models of computation for use in rapid prototyping environments such
as Ptolemy~\cite{ptolemy03overview}, GRAPE-II~\cite{grape-ii}, and
COSSAP~\cite{cossap}.  The Synchronous Dataflow model (SDF) represents
computation as an independent set of actors that communicate at fixed
rates~\cite{lee87static}.  StreamIt leverages the SDF model of
computation, though also supports dynamic communication rates and
out-of-band control messages.  There are other extensions to SDF that
provide similar dynamic constructs.  Synchronous Piggybacked Dataflow
(SPDF) supports control messages in the form of a global state table
with well-timed reads and writes~\cite{park99spdf2,park02spdf3}.  SPDF
is evaluated using MP3 decoding, and would also be effective for
MPEG-2 decoding.  However, control messages in StreamIt are more
expressive than SPDF, as they allow messages to travel upstream
(opposite the direction of dataflow), with adjustable latency, and
with more fine-grained delivery (i.e., allowing multiple execution
phases per actor and multiple messages per phase).  Moreover, our
focus is on providing a high-level programming abstraction rather than
an underlying model of computation.

Ko and Bhattacharyya also extend SDF with the dynamism needed for
MPEG-2 encoding; they use ``blocked dataflow'' to reconfigure
sub-graphs based on parameters embedded in the data
stream~\cite{bhatta05block} and a ``dynamic graph topology'' to extend
compile-time scheduling optimizations to each runtime
possibility~\cite{ko05dgt}. Neuendorffer and Lee also extend SDF to
support hierarchical parameter reconfiguration, subject to semantic
constraints~\cite{neuendorffer04hierarchical}.  Unlike our description
of control messages, these models allow reconfiguration of filter I/O
rates and thus require alternate or parameterized schedules.  MPEG-2
encoding has also been expressed in formalisms such as Petri
nets~\cite{valero02petri} and process algebras~\cite{pelayo01rosa}.

There are a number of stream-oriented languages besides StreamIt,
drawing from functional, dataflow, CSP and synchronous programming
styles~\cite{survey97}.  Synchronous languages which target embedded
applications include Esterel~\cite{Esterel}, Lustre~\cite{Lustre},
Signal~\cite{Signal}, Lucid~\cite{Lucid77}, and Lucid
Synchrone~\cite{Lucid-Synchrone}.  Additional languages of recent
interest are Cg~\cite{cg03}, Brook~\cite{brook04},
Spidle~\cite{spidle03}, StreamC/KernelC \cite{imagine03ieee},
Occam\cite{Occam}, Parallel Haskell~\cite{ph} and Sisal \cite{sisal}.
The primary differences between StreamIt and these languages are $(i)$
StreamIt supports (but is no longer limited to) the Synchronous
Dataflow~\cite{lee87static} model of computation, $(ii)$ StreamIt
offers a ``peek'' construct that inspects an item without consuming it
from the channel, $(iii)$ the single-input, single-output hierarchical
structure that StreamIt imposes on the stream graph, and $(iv)$ the
teleport messaging feature for out-of-band communication.

Many researchers have developed both hardware and software schemes for
parallel video compression; see Ahmad et al.~\cite{ahmad01compression}
and Shen et al.~\cite{shen94overview} for reviews.  We focus on
programming models used to implement MPEG on general-purpose hardware.
Assay et al. present a syntax of parallel tasks, forall loops, and
dependence annotations for exposing fine-grained parallelism in an
MPEG-4 encoder~\cite{assayad05mpeg4b}.  A series of loop
transformations (currently done by hand) lowers the representation to
an MPI program for an SMP target.  The system allows parallel
components to communicate some values through shared memory, with
execution constraints specified by the programmer.  In comparison,
StreamIt adopts a pure dataflow model with a focus on making the
programming model as simple as possible.  Another programming model is
the Y-Chart Applications Programmers Interface
(YAPI)~\cite{kock00yapi}, which is a C++ runtime library extending
Kahn process networks with flexible channel selection.  Researchers
have used YAPI to leverage programmer-extracted parallelism in
JPEG~\cite{kock02jpeg} and MPEG-2~\cite{dwivedi01exploring}.  Other
high-performance programming models for MPEG-2 include manual
conversion of C/C++ to SystemC~\cite{pazos04soc}, manual conversion to
POSIX threads~\cite{li05alpbench}, and custom mappings to
multiprocessors~\cite{iwata98coarse, ahmad01multiproc}.  Our focus
again lies on the programmability: StreamIt provides an
architecture-independent representation that is natural for the
programmer while exposing pipeline and data parallelism to the
compiler.

% Not bothering to cite, though relevant:
%
% - \cite{li05thesis}
%
% - \cite{assayad05mpeg4a}
%   4b is better
%
% - \cite{schneider99spec} 
%   more of an architectural synthsis paper
% 
% - Extracting fine- and coarse-grained from C
%   MPEG just a benchmark \cite{karkowski98parallelism}
%
% - older publications from Ahmad:
%   http://ranger.uta.edu/~iahmad/pub4.html
%
% - Investigation of Thread-Level Parallelism in
%   the Architectural Complexity Reduction of
%   MPEG2, XviD and H.264 Video Encoders
%   http://www.lboro.ac.uk/departments/el/research/conferences/esc2005/Tom_Jacobs.pdf
%
% - more on SPDF:
%  - implemented in RTL:
%    http://iris.snu.ac.kr/publications/data/64/ISSS2000park_c.pdf
%  - for 3-D graphics pipeline:
%    http://iris.snu.ac.kr/publications/data/68/aspdac_01.pdf
% 
% @Article{,
%   author =   {},
%   title =   "{}",
%   journal = "{}",
%   volume =   {},
%   number =   {},
%   year =     {},
%   mynotes = {
% -------------------------------------------------------------------------------------------
% TODO
% -------------------------------------------------------------------------------------------
% }
% }
%
% @InProceedings{,
%  author =     {},
%   title =     "{}",
%   booktitle = "{}",
%   year =       {},
%   mynotes = {
% -------------------------------------------------------------------------------------------
% TODO
% -------------------------------------------------------------------------------------------
% }
% }                                                                                           
