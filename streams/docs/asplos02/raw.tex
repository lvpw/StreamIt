\begin{figure}
\centering
\psfig{figure=raw-diagram.eps,width=3.5in}
\caption{A block diagram of the Raw architecture.
\protect\label{fig:raw-diagram}}
\end{figure}

\section{The Raw Architecture}
\label{sec:raw}

The Raw Microprocessor \cite{raw10,raw} addresses the wire delay
problem \cite{raw13} by providing direct instruction set architecture
(ISA) analogs to three underlying physical resources of the processor:
gates, wires and pins. Because ISA primitives exist for these
resources, a compiler such as StreamIt has direct control over both
the computation and the communication of values between the functional
units of the microprocessor, as well as across the pins of the
processor.

The architecture exposes the gate resources as a scalable 2-D array of
identical, programmable tiles, that are connected to their immediate
neighbors by four on-chip networks.  Each network is 32-bit,
full-duplex, flow-controlled and point-to-point. On the edges of the
array, these networks are connected via logical channels \cite{raw11}
to the pins.  Thus, values routed through the networks off of the side
of the array appear on the pins, and values placed on the pins by
external devices (for example, wide-word A/Ds, DRAMS, video streams
and PCI-X buses) will appear on the networks.

Each of the tiles contains a compute processor, some memory and two
types of routers -- one static, one dynamic -- that control the flow
of data over the networks as well as into the compute processor (see
Figure \ref{fig:raw-diagram}). The compute processor interfaces to the
network through a bypassed, register-mapped interface \cite{raw10}
that allows instructions to use the networks and the register files
interchangeably. In other words, a single instruction can read up to
two values from the networks, compute on them, and send the result out
onto the networks, with no penalty.  Reads and writes in this fashion
are blocking and flow-controlled, which allows for the computation to
remain unperturbed by unpredictable timing variations such as cache
misses and interrupts.

Each tile's static router has a virtualized instruction memory to
control the crossbars of the two static networks. Collectively, the
static routers can reconfigure the communication pattern across these
networks every cycle.  The instruction set of the static router is
encoded as a 64-bit VLIW word that includes basic instructions
(conditional branch with/without decrement, move, and nop) that
operate on values from the network or from the local 4-element
register file. Each instruction also has 13 fields that specify the
connections between each output of the two crossbars and the network
input FIFOs, which store values that have arrived from neighboring
tiles or the local compute processor.  The input and output
possibilities for each crossbar are: North, East, South, West,
Processor, to the other crossbar, and into the static router. The
FIFOs are typically four or eight elements large.

To route a word from one tile to another, the compiler inserts a route
instruction on every intermediate static router \cite{lee98spacetime}.
Because the routers are pipelined and compile-time scheduled, they can
deliver a value from the ALU of one tile to the ALU of a neighboring
tile in 3 cycles, or more generally, 2+N hops for a inter-tile
distance of N hops.

The results of this paper were generated using btl, a cycle-accurate
simulator that models arrays of Raw tiles identical to those in the
.15 micron 16-tile Raw prototype chip.  With a target clock rate of
250 MHz, the tile employs as compute processor an 8-stage, single
issue, in-order MIPS-style pipeline that has a 32 KB data cache, 32 KB
of instruction memory, and 64 KB of static router memory. All
functional units except the floating point and integer dividers are
fully pipelined. The mispredict penalty of the static branch predictor
is three cycles, as is the load latency. The compute processor's
pipelined single-precision FPU operations have a latency of 4 cycles,
and the integer multiplier has a latency of 2 cycles.

%The Raw Processor is a general-purpose microprocessor being developed
%in the Computer Architecture Group at The Massachusetts Institute of
%Technology.

%The general organization of the Raw Processor is as a chip
%multiprocessor with multiple fine-grain, first-class, register mapped
%communication networks \cite{raw}.  The processor contains a 2-D mesh
%of identical tiles, see Figure \ref{fig:raw-diagram}.  A tile consists
%of a tile processor, memory, two dynamic network routers, two static
%switch crossbars and a static switch processor.  Tiles are connected
%to each of their four nearest neighbors by the two sets of static
%network interconnect and two sets of dynamic network interconnect.
%The tile processor uses a 32-bit MIPS-like instruction set with 32K
%SRAM data memory and 32K SRAM instruction memory.

%The StreamIt Compiler maps the infinite, FIFO channels of the language
%to Raw's static networks.  Each static network routes single-word
%quantities of data (with no header) between the switch processor of
%nearest neighbors.  The tile processor communicates with the switch
%processor using buffered, blocking sends and receives.  The switch
%processors communicate using the same mechanism. Each tile must know
%in advance to whom it is sending data and from whom it is receiving
%data.  It is the task of the compiler to generate the appropriate
%route instructions at compiler time.  The static network allows
%3-cycle nearest neighbor ALU to ALU communication latency.

%The switch processor controls the static networks of the chip.  The
%switch processor uses a stripped down MIPS-like instruction set
%containing only moves and branches.  The switch processor has 4
%registers, an 8096-instruction instruction memory, and no data
%memory. Each switch instruction has a ROUTE component, executed in
%parallel, that specifies the transfer of values on the static network
%between the switch and its neighboring switches.  Each switch
%instruction can execute multiple moves in parallel using a VLIW-like
%instruction encoding for the ROUTE component.  The Raw instruction set
%architecture (ISA) works together with this parallel architecture by
%exposing both the computational and communication resources up to the
%software.

%Currently, the StreamIt Compiler generates code that executes on Raw's
%cycle accurate simulator. The simulator can model Raw configurations
%of up to 8 tiles per side.  During the summer of 2002, a prototype 4x4
%tile Raw chip will be available.  With a target clock rate of 225MHz,
%the chip will support 16 OPS/FLOPS per cycle, 3.6 GLOPS per second,
%and a bisection bandwidth of 230 Gb/sec.

