\chapter{Related Work}

    This thesis builds directly on the work done to analyze and
optimize linear components in StreamIt graphs \cite{Lamb}. We have
extended the theoretical framework for linear analysis to
state-space analysis in order to apply our optimizations to a
wider class of applications. We have also changed some parts of
the underlying representations. Previously, constants were handled
separately and peeked items beyond the pop rate were considered
inputs. For our current work we have placed both types of items in
states.

    Many other groups are researching methods for automated DSP
application optimizations. SPIRAL \cite{Spiral} is a system
developed to generate libraries of DSP transforms. These libraries
are designed for specific architectures, and can be re-optimized
when hardware is upgraded or replaced. Other such libraries that
have been designed include a package for linear algebra
manipulations by the ATLAS project \cite{Atlas} and a set of
optimized FFTs (Fast Fourier Transforms) \cite{fftw}.

    Aside from StreamIt, other programming languages have been
designed for streaming data. Synchronous languages which target
embedded applications include LUSTRE \cite{Lustre}, Esterel
\cite{Esterel}, and Signal \cite{Signal}. Other stream-based
languages are Occam \cite{Occam}, SISAL \cite{sisal}, and StreamC
\cite{streamc}. These are designed to exploit vector and parallel
processing. However, none of these languages have compilers that
run state-space or linear analysis.
