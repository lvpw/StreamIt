\chapter{Results}

    We have implemented the extraction, combination, and
optimization (except multiple execution stages) procedures within
the StreamIt compiler, which uses the KOPI java compiler
infrastructure \cite{kopi}. We measure performance by counting the
number of floating point operations (additions and
multiplications) executed in a given benchmark. The program
DynamoRIO \cite{dynamo} is used to count operations.

    We have collected the data for the base case (no replacement) and
with state-space and linear replacement \cite{Lamb}. For each
benchmark, values are expressed as a ratio of floating point
operations with state-space or linear replacement\footnote{We do
not consider frequency replacement in our comparison} over
operations with no replacement.
\begin{center}
\begin{table}[tbp]
\begin{tabular} {c|c|c}
Application & Statespace & Linear \\ \hline
FM Radio & 0.1740 & 0.1692 \\
FIR Program &  0.9961 & 0.9980 \\
Channel Vocoder & 0.2601 & 0.2620 \\
FilterBank & 1.000 & 1.001 \\
FFT (16 pt) & 2.938 & 3.000 \\
Linear Difference Equation & 1.005 & 1.000 \\
IIR & 1.005 & 1.000 \\
IIR + 1/2 Decimator & 0.6441 & 1.000 \\
IIR + 1/16 Decimator & 0.3393 & 1.000 \\
IIR + FIR & 0.9413 & 1.000 \\
FIR + IIR + IIR & 0.9214 & 1.000
\end{tabular}
\caption{Floating point operations with state-space and linear
replacement normalized for no replacement}
\end{table}
\end{center}

    For the first 5 applications, which mainly have linear
components without state, linear replacement and state-space
replacement are equally effective. There is a huge performance
downgrade for an FFT (Fast Fourier Transform) for both types of
replacement. This is not surprising, since an FFT performs its
computations sparsely across multiple filters. Combining these
filters creates one filter densely packed with computations. This
is exactly a conversion form an FFT to a DFT (Discrete Fourier
Transform). We would need staged execution with minimal
parameterization to convert the DFT back to an FFT.

    The remaining applications have filters with state, thus they
cannot be analyzed linearly. Therefore we use the value 1.000 as
the ratio in the linear column. For a simple standalone Linear
Difference Equation or IIR (Infinite Impulse Response) filter,
state-space replacement shows a very slight degradation in
performance. The reason for this is that there are a few extra
operations performed by the prework function. In terms of
steady-state behavior, state-space replacement is equivalent to no
replacement. There is no gain in performance because the filters
are written optimally.

    When we combine state filters with other filters, we notice a
performance improvement using state-space replacement. For
example, combining an IIR filter with a decimator that leaves 1
out of every 16 values has a 66\% improvement. Combining an IIR
filter with an FIR filter has a 6\% improvement. In the case of an
IIR filter with a decimator, there are extraneous computations
performed by the IIR filter that are thrown away by the decimator.
Combining their respective matrices removes these computations. In
the case of an IIR filter with an FIR filter, the computations in
both filters can be merged to a single set of computations. This
indicates that state-space replacement is more useful when applied
to combined filters than when applied to individual filters.
