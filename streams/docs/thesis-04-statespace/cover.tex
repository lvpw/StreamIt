% -*-latex-*-

\title{Linear State-Space Analysis and Optimization of StreamIt Programs}

\author{Sitij Agrawal}
\department{Department of Electrical Engineering and Computer Science}
% If the thesis is for two degrees simultaneously, list them both
% separated by \and like this:
% \degree{Doctor of Philosophy \and Master of Science}
\degree{Master of Engineering in Computer Science and Engineering}
\degreemonth{August}
\degreeyear{2004}
\thesisdate{August 26, 2004}

%% By default, the thesis will be copyrighted to MIT.  If you need to copyright
%% the thesis to yourself, just specify the `vi' documentclass option.  If for
%% some reason you want to exactly specify the copyright notice text, you can
%% use the \copyrightnoticetext command.
%\copyrightnoticetext{\copyright IBM, 1990.  Do not open till Xmas.}

% If there is more than one supervisor, use the \supervisor command
% once for each.
\supervisor{Saman Amarasinghe}{Associate Professor}

% This is the department committee chairman, not the thesis committee
% chairman.  You should replace this with your Department's Committee
% Chairman.
\chairman{Arthur C. Smith}{Chairman, Department Committee on Graduate Students}

% Make the titlepage based on the above information.  If you need
% something special and can't use the standard form, you can specify
% the exact text of the titlepage yourself.  Put it in a titlepage
% environment and leave blank lines where you want vertical space.
% The spaces will be adjusted to fill the entire page.  The dotted
% lines for the signatures are made with the \signature command.
\maketitle

% The abstractpage environment sets up everything on the page except
% the text itself.  The title and other header material are put at the
% top of the page, and the supervisors are listed at the bottom.  A
% new page is begun both before and after.  Of course, an abstract may
% be more than one page itself.  If you need more control over the
% format of the page, you can use the abstract environment, which puts
% the word "Abstract" at the beginning and single spaces its text.

%% You can either \input (*not* \include) your abstract file, or you can put
%% the text of the abstract directly between the \begin{abstractpage} and
%% \end{abstractpage} commands.

% First copy: start a new page, and save the page number.
\cleardoublepage
% Uncomment the next line if you do NOT want a page number on your
% abstract and acknowledgments pages.
% \pagestyle{empty}
\setcounter{savepage}{\thepage}
\begin{abstractpage}
As DSP programming is becoming more complex, there is an increasing
need for high-level abstractions that can be efficiently compiled.
Toward this end, we present a set of aggressive optimizations that
target linear sections of a stream program.  Our input language is
StreamIt, which represents programs as a hierarchical graph of
autonomous filters.  A filter is linear if each of its outputs can be
represented as an affine combination of its inputs.  Linear filters
are common in DSP applications; examples include FIR filters,
expanders, compressors, FFTs and DCTs.

We present a linear extraction analysis that automatically detects
linear filters based on the C-like code in their {\tt work} function.
Once linear filters are identified, we show how neighboring nodes can
be collapsed into a single linear representation, thereby eliminating
many redundant computations.  Also, we describe a method for
automatically translating linear nodes into the frequency domain,
thereby yielding algorithmic savings for convolutional filters.

We have completed a fully-automatic implementation of the above
techniques as part of the StreamIt compiler, and we demonstrate
performance improvements that average 400\% over our benchmark
applications.




\end{abstractpage}

% Additional copy: start a new page, and reset the page number.  This way,
% the second copy of the abstract is not counted as separate pages.
% Uncomment the next 6 lines if you need two copies of the abstract
% page.
% \setcounter{page}{\thesavepage}
% \begin{abstractpage}
% As DSP programming is becoming more complex, there is an increasing
need for high-level abstractions that can be efficiently compiled.
Toward this end, we present a set of aggressive optimizations that
target linear sections of a stream program.  Our input language is
StreamIt, which represents programs as a hierarchical graph of
autonomous filters.  A filter is linear if each of its outputs can be
represented as an affine combination of its inputs.  Linear filters
are common in DSP applications; examples include FIR filters,
expanders, compressors, FFTs and DCTs.

We present a linear extraction analysis that automatically detects
linear filters based on the C-like code in their {\tt work} function.
Once linear filters are identified, we show how neighboring nodes can
be collapsed into a single linear representation, thereby eliminating
many redundant computations.  Also, we describe a method for
automatically translating linear nodes into the frequency domain,
thereby yielding algorithmic savings for convolutional filters.

We have completed a fully-automatic implementation of the above
techniques as part of the StreamIt compiler, and we demonstrate
performance improvements that average 400\% over our benchmark
applications.




% \end{abstractpage}

\cleardoublepage

\section*{Acknowledgments}

    First, I would like to thank my family for all their support throughout my college years.
I would like to thank the members of the StreamIt group - in particular Jasper Lin and David Maze - for patiently
answering all my questions and for helping me understand the StreamIt language and compiler. 
I would like to thank Andrew Lamb, whose work on linear analysis of StreamIt programs provided the foundation for 
my own work on state-space analysis. His thesis and well-constructed code were invaluable to me. Rodric Rabbah, another
member of our group, gave me excellent comments about the writing in this thesis.
I would like to thank my advisor, Saman Amarasinghe, for giving me the opportunity to work on the StreamIt project and for funding
my research. Finally, I would like to thank Bill Thies for guiding me through every step of my project. That
I was able to complete this thesis is a testament to his mentoring ability. I could not have done it without him.


%%%%%%%%%%%%%%%%%%%%%%%%%%%%%%%%%%%%%%%%%%%%%%%%%%%%%%%%%%%%%%%%%%%%%%
% -*-latex-*-
