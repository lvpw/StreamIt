\subsection{Types}

StreaMIT language uses a type system similar to those defined by Java and
C.  Following types are defined: bool, byte, word, int, long, float 
and double.  These types can be combined into {\tt struct}s, similar 
to those defined by C.  The language allows taking a (Java style) 
reference of a {\tt struct}.  The reference can be saved inside of 
Filters, etc, just like any other type.

An array can be defined using syntax similar to that of C:

element-type array-name [30];

Strings are represented as references null-terminated arrays of chars.

Finally, StreaMIT uses a tagged union type.  A tagged union type is similar
to a C union, except it checks which field of the union is currently active
(using an automatically maintained tag).  It is a run-time exception to
read a field which is not the most-recently assigned one.  In addition
to run-time checking of accessed fields, StreaMIT's tagged unions can also
test if a given field is the current field:

\begin{verbatim}
union u { int num; float frac }
u data;
data.num = 3;
if (data.isActive (num)) { /* do stuff if num is currently active */ }
\end{verbatim}


