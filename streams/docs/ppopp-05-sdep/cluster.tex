\section{Cluster Backend}

The cluster backend compiles the StreamIt code to a set of threads
that can be executed on a single computer or a networked cluster. In
order to exchange data threads establish TCP/IP connections. A set of
connections is created that represents the flow of data between
filters, splitters and joiners. A separate set of connections is
established between message sender and message receivers in each
portal.

The connections between message sender and receivers are used for both
transmitting the messages and also for sending credits. Credits
represent the number of iterations that a filter is allowed to execute
and this enforces that messages can be executed at given latency.

If a message is sent following data is written to a socket:

{\scriptsize
\begin{verbatim}
int size; // size of message in bytes, including size field
int index; // index of message handler
int iteration_to_execute_at; // target iteration number or
                             // -1 for best effort messages 
int and float parameters;
\end{verbatim}}

Example:

{\scriptsize
\begin{verbatim}
(16, 0, -1, 1000)
\end{verbatim}}

A best effort message for handler Nr. 0 with one integer parameter set
to 1000.

{\scriptsize
\begin{verbatim}
(16, 0, 8, 1000)
\end{verbatim}}

A message for handler Nr. 0 that has to be executed before its work
function is invoked 8th time. The Message has one integer parameter equal
to 1000.

If a credit is being sent following data is written to a socket:

{\scriptsize
\begin{verbatim}
int -1; 
int credit; 
\end{verbatim}}

Example:

(-1, 8) Node can execute only up to first 8 iterations before receiving additional credit.

To calculate the iteration at which to execute the message the SDEP
structure is embedded in the code for the thread sending the
message. Any sdep and reverse sdep value can be calculated from few
parameters and information about the first steady state cycle
dependency.

Currently each message is scheduled exactly at the latest possible
latency and the rest of information is discarded. If one does take the
latency interval into account one could work out a way of sending less
credit messages and still maintaining feasibility. However this is not
implemented for the first version of cluster backend. This could be
implemented in future versions.

If a message is being sent downstream and maximum latency is
non-negative then no credits need to be sent. If a message is being
sent downstream at negative maximum latency or is being sent upstream
then one needs to coordinate the execution of filters by sending
credits. Nodes that are receiving credits can only execute up 
to the maximum credit they have received so far.

In case of a message being sent downstream with negative latency,
during compile time we use the sdep information to calculate following
information:

\begin{enumerate}
\item Number of iterations upstream node can execute before sending any credits,
\item The schedule of sending credits during a steady state cycle after initial number of iterations.
\end{enumerate}

We use the information about credits to send during a steady cycle to
figure out number of credits to send during subsequent steady
cycles. This data is embedded into the message sender thread.

In case of a message being sent upstream, during compile time we use
the sdep information to calculate following information:

\begin{enumerate}
\item The initial credit that downstream node sends upstream before executing any cycle,
\item The schedule of sending credits during a steady state cycle.
\end{enumerate}

We use the information about credits to send during a steady cycle to
figure out number of credits to send during subsequent steady
cycles. This data is embedded into the message sender thread.
