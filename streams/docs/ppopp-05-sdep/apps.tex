\section{Precise Event Handling}
In this section we describe how $\sdep$ information can be
incorporated into the semantics of a language feature that provides
precise delivery of control messages in stream programs.  Our goal is
to improve both programmer productivity and performance.

The Synchronous Dataflow (SDF) domain is well-suited for applications
that have regular, high-bandwidth communication patterns.  However, in
realistic streaming applications there are also irregular,
low-bandwidth control messages that are used to adjust parameters in
various parts of the stream.  For example, a downstream actor might
detect a high signal-to-noise ratio and send a message to the
communications frontend to increase the amplification.  Or, an actor
at the top of the stream graph might detect an invalid checksum for a
packet, and send a message downstream to invalidate the effects of
what has been processed.  Other examples of control messages include:
periodic channel characterization; adaptive beamforming; initiating a
handoff ({\it e.g.,} to a new network protocol); marking the end of a
large data segment; and responding to user inputs, environmental
conditions, or exceptional states.

Generally speaking, control messages are sent at infrequent and
irregular intervals; however, once they are generated there might be
tight constraints on their delivery.  For example, the message
invalidating the effects of a given packet has to be delivered exactly
in sync with the front of the packet itself; otherwise, valid data
will be lost or invalid data will be allowed to pass.

In the rest of this section, we describe language support for control
messages that makes use of $\sdep$ to achieve precise delivery timing.
We first describe the semantics of the language feature, and then we
compare it against other means of implementing messages for a
frequency hopping radio application.

\subsection{Messaging with SDEP}

The messaging system we describe is included as part of the StreamIt
language~\cite{streamitcc}.  In StreamIt, there are two distinct kinds
of communication between filters during steady state execution: 1)
high-bandwidth dataflow over the FIFO channels in the graph, and 2)
low-bandwidth messaging between pairs of filters\footnote{Messaging
is possible whenever there is a downstream path from either filter to
the other.  Filters running in parallel cannot send messages}.  In
order for filter $A$ to send a message to filter $B$, the following
steps need to be taken:
\begin{itemize}

\item $B$ declares a message handler that will be invoked when the
message arrives, for example:
{\small
\begin{verbatim}
handler increaseGain(float amount) {
  this.gain += amount;
}
\end{verbatim}
}
Message handlers are just like normal functions, except that they
cannot access the input/output channels and they have no return value.

\item A parent stream which contains $A$ and $B$ declares a variable
of type {\tt portal<} $T_B$ {\tt >} which can forward messages to
anything of type $T_B$.  The parent adds $B$ to the portal and passes
the portal to $A$ during initialization.

\item To send a message, $A$ invokes the handler method on the portal
from within its steady state work function.  It includes a range of
latencies at which the message should be delivered.  For example:
{\small
\begin{verbatim}
work pop 1 {
  float val = pop();
  if (val < THRESHOLD) {
    portalToB.increaseGain(0.1) [2:3];
  }
}
\end{verbatim}}

\end{itemize}
The most interesting aspect of the messaging system is the semantics
for message latency.  Because there are many legal orderings of actor
executions, there is no notion of ``global time'' in a stream graph.
The only common frame of reference between concurrently executing
actors is the series of data items that is passed between them.  The
$\sdep$ function captures the data dependences in the graph and
provides a natural means of defining a rendezvous point between two
actors.

Intuitively, the message semantics can be thought of in terms of
attaching tags to data items.  If $A$ sends a message to downstream
filter $B$ with a latency $k$, then this could be implemented by
tagging the items that $A$ outputs $k$ iterations later.  These tags
propagate through the stream graph; whenever an actor inputs an item
that is tagged, all of its subsequent outputs are tagged.  Then, the
message handler of $B$ is invoked immediately after the first
invocation of $B$ that inputs a tagged item.  In this sense, the
message has the semantics of traveling ``with the data'' through the
stream graph, even though it does not have to be implemented this way.

The intuition for upstream messages is similar.  Consider that $B$ is
sending a message with latency $k$ to upstream actor $A$ in the stream
graph.  This means that $A$ will receive the message immediately
following the last invocation of its work function which produces an
item affecting the output of $B$'s $k$th firing, counting the current
firing as 0.  As before, we can also think of this in terms of $A$
tagging items and $B$ observing the tags.  In this case, the latency
constraint says that $B$ must input a tagged item before it finishes
$k$ additional executions.  The message is delivered immediately
following the latest firing in $A$ during which tagging could start
without violating this constraint.

The following definition leverages the $\sdep$ formalism to give a
precise meaning to message timing.

\begin{definition}(Message delivery)
Consider that $A$ sends a message to $B$ with latency range
$[k_1:k_2]$ and that the message is sent during the $n$th invocation
of $A$'s work function.  Then the message handler can be invoked in
$B$ immediately after its work function has fired ${\cal M}(A, B, k_1,
k_2, n)$ times, where ${\cal M}$ is constrained as follows.

There are two cases\footnote{In a feedback path, both cases might apply.  In this event, we assume the message is being sent upstream.}:
\begin{enumerate}

\item There is a path in the stream graph from $A$ to $B$.  Then
${\cal M}$ obeys the following constraints:
\[
\begin{array}{l}
\sdepf{A}{B}({\cal M}(A, B, k_1, k_2, n)) \ge n+k_1\\
\sdepf{A}{B}({\cal M}(A, B, k_1, k_2, n)) \le n+k_2
\end{array}
\]

\item There is a path in the stream graph from $B$ to $A$.  Then
${\cal M}$ obeys the following constraints:
\[
\begin{array}{l}
{\cal M}(A, B, k_1, k_2, n) \ge \sdepf{A}{B}(n + k_1)\\
{\cal M}(A, B, k_1, k_2, n) \le \sdepf{A}{B}(n + k_2)
\end{array}
\]
\end{enumerate}
\end{definition}

\begin{table*}[t]
{\small
\begin{tabular}{|r|c|c|} \hline
~ & {\bf Negative latency} & {\bf Positive latency} \\ \hline
{\bf Message travels downstream} & latency in schedule must not be too small & no constraint \\ \hline
{\bf Message travels upstream} & impossible & latency in schedule must not be too big \\ \hline
\end{tabular}}
\caption{\small Effect of message direction and latency on stream graph execution.}
\label{tab:messcons}
\end{table*}

It is instructive to note that there are distinct categories of
message latencies, each of which poses a different constraint on the
execution of the stream graph (see Figure~\ref{tab:messcons}).  A
negative-latency downstream message has the effect of synchronizing
the arrival of the message with some data that was previously output
by the sender ({\it e.g.,} for the checksum example used above).  The
latency requires the downstream actor not to execute too far ahead, or
else it might process the data before the message arrives.  This
translates to a constraint on the minimum latency between the sender
and receiver actors in the schedule of the program.

Similarly, a positive latency upstream message places a constraint on
the maximum latency between the sender and receiver.  Again the
receiver must be throttled so that it doesn't get too far ahead before
the message arrives; however, because the receiver is upstream of the
sender, this constraint represents the need for a tight coupling
between the two filters' executions.

An upstream message with negative latency is impossible to deliver,
because the data dependences imply that the target iteration has
already passed when the message was sent.  Also, a downstream message
with positive latency imposes no constraint, as it is not possible for
the receiver to have executed yet.

\begin{figure}[t]
\begin{center}
\psfig{figure=constrained-example.eps,height=1.5in}
\caption{{\small Example of construction of a constrained schedule. The $\sdepf{R}{S}$ function for filters $R$ and $S$ is given in Table \ref{tab:sdepconst}. The blob between filters $R$ and $S$ illustrates other possible stream elements. $R$ sends a message to $S$ with latency $[1,2]$. Executions of the blob are ommitted, it is assumed that at the point $S$ executes, the blob has drained data provided by $R$.}}
\end{center}
\vspace{-12pt}
\label{fig:sdepconst}
\end{figure}

\begin{table*}[t]
{\small
\begin{tabular}{|c|c|} \hline
{\bf $sdepf{R}{S}$} & {\bf Execs of S} \\ \hline
$9n+2$ & $8n+1$ \\ \hline
$9n+3$ & $8n+2$ \\ \hline
$9n+5$ & $8n+3$ \\ \hline
$9n+5$ & $8n+4$ \\ \hline
$9n+6$ & $8n+5$ \\ \hline
$9n+8$ & $8n+6$ \\ \hline
$9n+9$ & $8n+7$ \\ \hline
$9n+9$ & $8n+8$ \\ \hline
\end{tabular}}
\caption{\small $sdepf{R}{S}$ function for example in Figure \ref{fig:sdepconst}. This particular $\sdep$ function was obtained by setting $push_R=2$, $pop_S=3$ and making the blob between $R$ and $S$ into a filter that pops 3 and pushes 4 every iteration of its work function. No initialization due to peeking is necessary in this example.}
\label{tab:sdepconst}
\end{table*}


Explanation of working of the example:

The example in Figure \ref{fig:sdepconst} illustrates scheduling of a single constraint. The constraint is a message sent upstream with latency $[1,2]$. The resulting schedule consists of two parts, an initialization schedule and a steady state schedule. The initialization schedule is necessary to initialize the constraint, to ensure that the steady schedule can be executed repeatedly forever. Notation used here is: $lastReceived$ denotes the execution of S which sent the last message to be received by R; $n_S$ indicates number of executions of $S$ and $n_R$ indicates number of executions of $R$.

Initialization schedule is computed by executing R $SDEP(minLatency)-1$ number of times, and executing S as many times as possible, given data provided by R. This assures that when the steady schedule starts, all executions of filters will contribute to new message sending and receiving, thus allowing the steady state schedule to execute forever. The initialization schedule does not need to send or receive messages here.

The steady state schedule is computed by executing the receiver R as far as possible without going beyond the boundary of beeing able to receive message sent by S on execution $lastReceived+1$ (indicated by "Oldest msg to receive" in the figure). Now the sender S is executed as many times as possible, given data provided by R. Now S can receive messages sent by S in its executions $[lastReceived+1 ... \min(n_S, $"Newest msg to receive"$)]$.

"Oldest msg to receive" is equal to $iSDEP(n_R)-maxLatency$. "Newest msg to receive" is equal to $m-minLatency$ with $m$ being the greatest integer such that $SDEP(m) \le n_R$.


\subsection{Case Study}

To illustrate the pros and cons of the messaging system, we
implemented a spread-spectrum frequency hopping radio
frontend~\cite{harada02} as appears in Figure~\ref{fig:fhr-streamit}.
A frequency hopping radio is one in which the receiver switches
between a set of known frequencies whenever it hears certain tones
from the transmitter.  The frequency hop is a good match for control
messages because the hopping interval is dynamic (based on on the data
in the stream); it spans a large section of the stream graph (there is
an FFT between the demodulator and the hop detector); and it requires
precise delivery of messages.  The delivery must be precise both to
meet real-time requirements (as the tranceiver will leave the current
frequency soon), and to ensure that the message falls at a logical
packet boundary; if the frequency change is out of sync with the FFT
stage, then the FFT will muddle the spectrum of the old and new
frequency bands.

A StreamIt version of the radio frontend with language support for
messaging appears in Figure~\ref{fig:freq1}.  The Freq\_Hopping\_Radio
pipeline creates a portal and adds the RFtoIF actor as a receiver.
The portal is passed to the Check\_Freq\_Hop stage, where four
parallel detectors send messages into the portal if they detect a hop
to the frequency they are monitoring.  The messages are sent with a
small latency (4-6) to ensure a timely transition.  To make sense of
the latency, note that $\sdepf{RFtoIF}{D}(n) = 64*n$ for each of the
detector actors $D$.  This comes about because the FFT stage consumes
and produces 64 items; each detector fires once per set of outputs
from the FFT, but RFtoIF fires 64 times to fill the FFT input.
Because of this $\sdep$ relationship, messages sent from the detectors
to RFtoIF are guaranteed to arrive only at iterations that are a
multiple of 64.  This satisfies the design criterion that a given FFT
stage will not operate on data that was demodulated at two separate
frequencies.

Another version of the frequency hopping radio appears in
Figures~\ref{fig:fhr-streamit} and~\ref{fig:freq2}.  This version is
functionality equivalent to the first, except that the control
messages are implemented manually by embedding them in the data stream
and introducing a feedback loop.  Because the number of items
transfered around the loop must be constant from one iteration to the
next, a data item is sent whether or not there is a message as part of
the algorithm; a special value of 0 represents that there is no
message on the given iteration (in some other programs, no special
value is available, in which case a structure can be passed through
the stream with a boolean flag indicating whether or not a message is
present).  Then, the RFtoIF filter checks the values from the loop on
every iteration and processes them as a message if they are non-zero.
The I/O rate of the RFtoIF filter has been scaled up to ensure that
the messaging information is received at intervals of 64 iterations
(as in the version with portals).  To achieve the desired messaging
latency, a number of items are enqueued on the feedback path prior to
execution.

Yet another way to approximate the behavior of messaging is with a
direct function call from the detector to the RFtoIF stage.  (Though
such a call is disallowed in StreamIt, it could be an option in a
different programming model.)  While this approach is simple, it does
not have any timing guarantees.  There is no way for the sender to
know when in the course of the target's execution the message will be
received.  This could cause problems both for algorithm development
and for reliability / predictability of software.

\subsection{Discussion}

We believe that the StreamIt messaging system offers several benefits
compared to a manual implementation of equivalent functionality.
While embedding messages in the data stream is equally precise, this
involves several tedious and error-prone changes, not only to the
stream graph but also to the steady state execution code within the
actors.  In particular, the manual derivation of the loop delay,
adjustment of the actor I/O rates, and implicit interleaving of data
items with control messages has a negative impact on the readability
and maintainability of the code.  The messaging construct in StreamIt
provides the same level of precision, but with the simplicity of a
method call.

The messaging construct also has advantages from a compiler
standpoint.  By separating the data-intensive code from the
control-oriented code, the common case of the steady state actor
execution is not sacrificed for the uncommon case of message
processing.  There are no ``dummy items'' serving as placeholders in
on the static-rate channels.  In addition, by exposing the message
latency as part of the language, the compiler can infer the true
dependences between filter firings and reorder the execution so long
as the message constraints are respected.  The actual message delivery
can be implemented in the most efficient way for the given
architecture.

A final benefit of the messaging system is the clean interface
provided by the portals.  Since a portal can have multiple receivers,
it is straightforward to send a message that is delivered
synchronously to two actors in parallel streams.  For example,
consider a vocoder that is separately manipulating the magnitude and
phase components of a signal.  If something triggers an adjusment to
the speech transformation ({\it e.g.,} the speaker requests a change
of pitch) then the mask needs to be updated at the same data-relative
time in both parallel streams.  A portal that contains both components
seamlessly provides this functionality.  Finally, portals are useful
as an external programming interface; an application can export a
portal based on an interface type without exposing the underlying
actor implementation.

\clearpage
\begin{figure}[t]
\psfig{figure=fhr-streamit.eps,width=3.5in}
\caption{\small Stream graph of frequency-hopping radio with language
support for messaging.  A messaging portal delivers point-to-point
latency-constrained messsages from the detectors to the RFtoIF stage.
\protect\label{fig:fhr-streamit}}
\end{figure}

\begin{figure}[t]
\scriptsize
\begin{verbatim}
float->float filter RFtoIF(int N, float START_FREQ) {
  float[N] weights;
  int size, count;
  
  init { set_frequency(START_FREQ); }
  
  work pop 1 push 1 {
    push(pop() * weights[count++]);
    count = count % size;
  }
  
  handler set_frequency(float freq) {
    count = 0;
    size  = (int) (N * START_FREQ / freq);
    for (int i = 0; i < size; i++)
      weights[i] = sin(i * pi / size);
  }
}

float->float splitjoin Check_Freq_Hop(int N, 
                                      float START_FREQ, 
                                      portal<RFtoIF> port) {
  split roundrobin(N/4-2, 1, 1, N/2, 1, 1, N/4-2);
  for (int i=1; i<=7; i++) {
    if (i==1 || i==4 || i==7) {
      add Identity<float>;
    } else {
      add float->float filter {
        work pop 1 push 1 {
          float val = pop();
          push(val);
          if (val > hop_threshold)
            port.set_frequency(START_FREQ + 
                               i/7*Constants.BANDWIDTH)
        }
      }
    }
  }
  join roundrobin(N/4-2, 1, 1, N/2, 1, 1, N/4-2);
}

void->void pipeline Freq_Hopping_Radio {
  int   N          = 32;
  float START_FREQ = 2402000000;
  portal <RFtoIF> port;

  add Read_From_AtoD(N);
  add RFtoIF(N, START_FREQ) to port;
  add FFT(N);
  add Magnitude();
  add Check_Freq_Hop(N, START_FREQ, port);
  add Output()
}
\end{verbatim}
\vspace{-12pt}
\caption{\small Frequency hopping radio with language support for event handling. \protect\label{fig:freq1}}
\end{figure}

\clearpage
\begin{figure}[t]
\psfig{figure=fhr-feedback.eps,width=3.5in}
\caption{\small Stream graph of frequency-hopping radio with control
messages implemented manually.  A feedback loop connects the detectors
with the RFtoIF stage, and an item is sent on every invocation to
indicate whether or not a message is present.  The latency and
periodicity of message delivery are governed by the data rates and the
number of items on the feedback
path. \protect\label{fig:fhr-manual}}
\end{figure}

\begin{figure}[t]
\scriptsize
\begin{verbatim}
 float->float filter RFtoIF(int N, float START_FREQ) {
   float[N] weights;
   int size, count;
   
   init { set_frequency(START_FREQ); }
   
*  work pop 3*N push 2*N {
*    // manual loop to 2*N.  Factor of N because messages 
*    // for given time slice come in groups of N; factor 
*    // of 2 for data-rate conversion of Magnitude filter
*    for (int i=0; i<2*N; i++) {
*      push(pop() * weights[count++]);
*      count = count % size;
*    }
*    // manually check for messages; 
*    // special value of 0 encodes no message
*    for (int i=0; i<N; i++) {
*      float freqHop = pop();
*      if (freqHop!=0)
*        set_frequency(freqHop);
*    }
*  }
   
   handler set_frequency(float freq) {
     count  = 0;
     size   = (int) (N * START_FREQ / freq);
     for (int i = 0; i < size; i++)
       weights[i] = sin(i * pi / size);
   }
 }

 float->float splitjoin Check_Freq_Hop(int N, 
                                       float START_FREQ) {
   split roundrobin(N/4-2, 1, 1, N/2, 1, 1, N/4-2);
   for (int i=1; i<=7; i++) {
     if (i==1 || i==4 || i==7) {
       add float->float filter {
*        work pop 1 push 2 {
           push(pop());
*          push(0);
         }
       }
     } else {
       add float->float filter {
*        work pop 1 push 2 {
           float val = pop();
           push(val);
*          if (val > hop_threshold) {
*            push(val);
*          } else {
*            push(0);
*          }
         }
       }
     }
   }
*  join roundrobin(2*(N/4-2), 2, 2, 2*(N/2), 2, 2, 2*(N/4-2));
 }

 void->void pipeline Freq_Hopping_Radio {
   int   N             = 32;
   float START_FREQ    = 2402000000;
   
   add Read_From_AtoD(N);
*  add float->float feedbackloop {
*    // adjust joiner rates to match data rates in loop
*    join roundrobin(2*N,N);
*    body pipeline {
*      add RFtoIF(N, START_FREQ);
*      add FFT(N);
*      add Magnitude();
*      add Check_Freq_Hop(N, START_FREQ);
*    }
*    split roundrobin();
*    // number of items on loop path = latency * N
*    for (int i=0; i<6*N; i++)
*      enqueue(0);
*  }
   add Output()
 }
\end{verbatim}
\vspace{-12pt}
\caption{\small Frequency hopping radio with manual feedback loop for
event handling.  Lines that differ from Figure~\ref{fig:freq1} are
marked with an asterisk. \protect\label{fig:freq2}}
\end{figure}
\clearpage

\subsection{Experimental Evaluation of Messaging}

In order  to evaluate  our messaging methodology,  we compile  the two
implemenations      of      the      frequency      hopping      radio
(Figures~\ref{}~and~\ref{}) into  a set of  threads. The message-based
implementation  is compiled  into  28 threads,  whereas the  alternate
version---relying on a feedback-loop for notificaition---results in 32
threads.  Each  thread implements a  specific component of  the stream
graph  (e.g., filter)  and is  allocated to  a specific  machine  in a
networked computing  cluster.  The cluster consists  of sixteen 750Mhz
Pentium~III workstations, each with  a 256~Kb cache.  The machines are
interconnected  using a  fully switched  high speed  network,  and the
threads     are    allowed     to     communicate    via     dedicated
channels. Specifically, every channel  in the stream graph is assigned
a  unique TCP/IP  connections for  the exchange  of data  and messages
between actors.

In  order  to  perform  the  thread-to-machine  allocation,  our  {\it
clustering}       backend,       part       of      the       StreamIt
compiler~\cite{streamit-compiler}     infrastructure,     applies    a
partitioning   algorithm~\cite{thies-msp}   to   reduce  the   overall
application bottleneck  while maximizing the throughput  of the output
filter in the stream dependence  graph. For the purpose of this paper,
throughput is  defined as the number  of outputs produced  per unit of
time.

\begin{figure}[t]
\psfig{figure=throughput-graph.eps,width=3.5in}
\caption{\small Throughput as a function of the number of clusters for
the two implementations of the frequency hopping radio.
\protect\label{fig:fhr-throughput}}
\end{figure}

In Figure~\ref{fig:fhr-throughput},  we report the  throughput for the
two applications along the $y$-axis.  Each data point corresponds to a
specific allocation  of the  threads to the  number of  clusters shown
along the $x$-axis.   Note that due to the  limited parallelism in the
two implementations of the  frequency hopper, increasing the number of
clusters beyond five workstations leads to negligle gains with respect
to throughput.   Nonetheless, the use of messaging  achieves a maximal
throughput   that   is    49.8\%   better   than   its   feedback-loop
counterpart. Furthermore, a detailed analysis of the results indicates
a 35\%  reduction in  the total communication  when messaging  is used
instead of a continuous feedback-loop.

\section{Other SDEP Applications}

We believe that there are many interesting applications for $\sdep$ in
programming language design and implementation.  We summarize four
potential applications below.

\subsection{Specifying Latency Constraints}

The latency range included in the messaging construct can be directly
employed as a declarative way to specify latency constraints between
actors.  For instance, our infrastructure can interpret a directive
such as $\mt{maxLatency}(A, B, k)$ to indicate a maximum latency of
$k$ between actors $A$ and $B$.  To implement this directive, the
schedule is constrained as if there were an upstream message with
latency range $[0,k]$.  This kind of data-centric latency constraint
could be important for reactive applications that need to produce an
output before consuming too many items from the input.

\subsection{Debugging}

An immediate application of $\sdep$ is as part of a graphical
debugging environment for stream programs.  For example, if a user is
stepping through the execution of the work function of actor $B$, it
might become apparent that the errant behavior of $B$ is due to
aspects of the input items that originated in actor $A$.  By utilizing
$\sdep$ information, the debugger can provide the iteration of $A$ in
which the items originated, and the user can continue debugging at
that location.

\subsection{Software-Based Speculation}

Software-based speculation is one approach to improving the
performance of irregular scientific applications~\cite{frank-thesis}.
While the graph-level control flow in Synchronous Dataflow is known at
compile time, there could be unpredictable control flow within the
work function of each actor, and a compiler could attempt to improve
performance by speculatively executing a given path.  However, if the
prediction failed, the results of the speculation might have passed
outside the boundaries of the actor and on to other actors in the
graph.  In this case, $\sdep$ provides an exact count of how many
iterations the downstream actors should roll back in order to arrive
at the original state.

\subsection{Program Transformation}

In the realm of scientific computing, a precise notion of dependences
proved essential for developing a robust suite of program analyses and
optimizations.  Representations such as dependence levels~\cite{AK82},
direction vectors~\cite{wolfe82}, and dependence
polyhedra~\cite{Irig88} were important abstractions because they
provided an efficient way to test the validity of program
transformations.  We believe that a similar abstraction of dependences
is needed for the emerging realm of streaming applications, and
$\sdep$ represents our first step towards this goal.
