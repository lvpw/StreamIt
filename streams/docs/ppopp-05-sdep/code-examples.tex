\clearpage
\begin{figure}[t]
\psfig{figure=fhr-streamit.eps,width=3.5in}
\caption{\small Stream graph of frequency-hopping radio with language
support for messaging.  A messaging portal delivers point-to-point
latency-constrained messages from the detectors to the RFtoIF stage.
\protect\label{fig:fhr-streamit}}
\end{figure}

\begin{figure}[t]
\scriptsize
\begin{verbatim}
float->float filter RFtoIF(int N, float START_FREQ) {
  float[N] weights;
  int size, count;
  
  init { set_frequency(START_FREQ); }
  
  work pop 1 push 1 {
    push(pop() * weights[count++]);
    count = count % size;
  }
  
  handler set_frequency(float freq) {
    count = 0;
    size  = (int) (N * START_FREQ / freq);
    for (int i = 0; i < size; i++)
      weights[i] = sin(i * pi / size);
  }
}

float->float splitjoin Check_Freq_Hop(int N, 
                                      float START_FREQ, 
                                      portal<RFtoIF> port) {
  split roundrobin(N/4-2, 1, 1, N/2, 1, 1, N/4-2);
  for (int i=1; i<=7; i++) {
    if (i==1 || i==4 || i==7) {
      add Identity<float>;
    } else {
      add float->float filter {
        work pop 1 push 1 {
          float val = pop();
          push(val);
          if (val > hop_threshold)
            port.set_frequency(START_FREQ + 
                               i/7*Constants.BANDWIDTH)
        }
      }
    }
  }
  join roundrobin(N/4-2, 1, 1, N/2, 1, 1, N/4-2);
}

void->void pipeline Freq_Hopping_Radio {
  int   N          = 32;
  float START_FREQ = 2402000000;
  portal <RFtoIF> port;

  add Read_From_AtoD(N);
  add RFtoIF(N, START_FREQ) to port;
  add FFT(N);
  add Magnitude();
  add Check_Freq_Hop(N, START_FREQ, port);
  add Output()
}
\end{verbatim}
\vspace{-12pt}
\caption{\small Frequency hopping radio with language support for event handling. \protect\label{fig:freq1}}
\end{figure}

\clearpage
\begin{figure}[t]
\psfig{figure=fhr-feedback.eps,width=3.5in}
\caption{\small Stream graph of frequency-hopping radio with control
messages implemented manually.  A feedback loop connects the detectors
with the RFtoIF stage, and an item is sent on every invocation to
indicate whether or not a message is present.  The latency and
periodicity of message delivery are governed by the data rates and the
number of items on the feedback
path. \protect\label{fig:fhr-manual}}
\end{figure}

\begin{figure}[t]
\scriptsize
\begin{verbatim}
 float->float filter RFtoIF(int N, float START_FREQ) {
   float[N] weights;
   int size, count;
   
   init { set_frequency(START_FREQ); }
   
*  work pop 3*N push 2*N {
*    // manual loop to 2*N.  Factor of N because messages 
*    // for given time slice come in groups of N; factor 
*    // of 2 for data-rate conversion of Magnitude filter
*    for (int i=0; i<2*N; i++) {
*      push(pop() * weights[count++]);
*      count = count % size;
*    }
*    // manually check for messages; 
*    // special value of 0 encodes no message
*    for (int i=0; i<N; i++) {
*      float freqHop = pop();
*      if (freqHop!=0)
*        set_frequency(freqHop);
*    }
*  }
   
   handler set_frequency(float freq) {
     count  = 0;
     size   = (int) (N * START_FREQ / freq);
     for (int i = 0; i < size; i++)
       weights[i] = sin(i * pi / size);
   }
 }

 float->float splitjoin Check_Freq_Hop(int N, 
                                       float START_FREQ) {
   split roundrobin(N/4-2, 1, 1, N/2, 1, 1, N/4-2);
   for (int i=1; i<=7; i++) {
     if (i==1 || i==4 || i==7) {
       add float->float filter {
*        work pop 1 push 2 {
           push(pop());
*          push(0);
         }
       }
     } else {
       add float->float filter {
*        work pop 1 push 2 {
           float val = pop();
           push(val);
*          if (val > hop_threshold) {
*            push(val);
*          } else {
*            push(0);
*          }
         }
       }
     }
   }
*  join roundrobin(2*(N/4-2), 2, 2, 2*(N/2), 2, 2, 2*(N/4-2));
 }

 void->void pipeline Freq_Hopping_Radio {
   int   N             = 32;
   float START_FREQ    = 2402000000;
   
   add Read_From_AtoD(N);
*  add float->float feedbackloop {
*    // adjust joiner rates to match data rates in loop
*    join roundrobin(2*N,N);
*    body pipeline {
*      add RFtoIF(N, START_FREQ);
*      add FFT(N);
*      add Magnitude();
*      add Check_Freq_Hop(N, START_FREQ);
*    }
*    split roundrobin();
*    // number of items on loop path = latency * N
*    for (int i=0; i<6*N; i++)
*      enqueue(0);
*  }
   add Output()
 }
\end{verbatim}
\vspace{-12pt}
\caption{\small Frequency hopping radio with manual feedback loop for
event handling.  Lines that differ from Figure~\ref{fig:freq1} are
marked with an asterisk. \protect\label{fig:freq2}}
\end{figure}
\clearpage
