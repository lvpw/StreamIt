\smallskip \subsection{A Minimal Program}

\begin{textpic}{\includegraphics{cookbook.1}}
\begin{lstlisting}{}
void->void pipeline Minimal {
  add IntSource;
  add IntPrinter;
}
void->int filter IntSource {
  int x;
  init { x = 0; }
  work push 1 { push(x++); }
}
int->void filter IntPrinter {
  work pop 1 { print(pop()); }
}
\end{lstlisting}
\end{textpic}

This is the minimal interesting StreamIt program.  \lstinline|Minimal|
is a StreamIt \emph{pipeline}: the output of its first child is
connected to the input of its second child, and so on.  It has two
children, a source and a sink.  Each of these are implemented as
StreamIt \emph{filter} objects.

A filter has two special functions, an \emph{init} function and a
\emph{work} function.  Both of these are present in
\lstinline|IntSource|.  The init function runs once at the start of
the program; the work function runs repeatedly forever.  If the init
function is omitted, as it is in \lstinline|IntPrinter|, it is assumed
to be empty.  Work functions have \emph{static data rates}.  The
source here declares that each iteration of the work function pushes a
single item on to its output; the sink declares that it pops a single
item from its input.

Every StreamIt structure has a single input and a single output.  The
filter and pipeline declarations here show the types of these inputs
and outputs.  C-like \lstinline|int| and \lstinline|float| types are
available, along with \lstinline|bit| for one-bit data and
\lstinline|complex| for complex floating-point data.  \lstinline|void|
is used as a special type to indicate the boundary of the program:
``the program'' in StreamIt is defined as a stream structure with both
\lstinline|void| input and output types.  A filter that takes no input
at all should also be declared to take \lstinline|void| as its input
type, and similarly a \lstinline|void| output can be used if a filter
produces no output.

\subsection*{How to Compile and Run}

The StreamIt compiler script {\bf strc} can be used to compile and
execute StreamIt programs.  For example, if the above program is
stored in {\tt Minimal.str}, then the following command will compile
it for the uniprocessor backend:
\begin{verbatim}
strc Minimal.str -o minimal
\end{verbatim}
The resulting binary is stored in {\tt minimal}, and it can be
executed for 5 iterations as follows:
\begin{verbatim}
minimal -i 5
\end{verbatim}
Doing so will print the integers from 0 to 4, in increasing order.

During the course of compilation, a number of stream graphs are ouput
to {\tt dot} files in the current directory.  The {\tt dot} format can
be displayed and converted to other formats using the Graphviz
software, which is available online\footnote{\tt
http://www.research.att.com/sw/tools/graphviz/}.  Running the
following command will draw the stream graph for the program, as
pictured to the right of the source code above:
\begin{verbatim}
dotty first-sir-tree.dot
\end{verbatim}
There are many other {\tt dot} files that are output by the compiler;
see Section~\ref{sec:compiler} of this document for more details.

{\bf The Java Library.}  In addition to using the StreamIt compiler,
it is possible to convert StreamIt programs into equivalent Java
programs that can be executed using any Java VM.  This is particularly
convenient for testing and debugging, as well as for cases when the
compiler might encounter a bug.

To run the {\tt Minimal} program for 5 iterations in the Java library,
do as follows:
\begin{verbatim}
strc --library Minimal.str -i 5
\end{verbatim}
This command will output a {\tt Minimal.java} file, compile it with a
Java compiler, and run it using {\tt java}.  The output should always
be identical to that obtained using the compiler.  In addition, the
library will output a {\tt Minimal.dot} file that can be visualized
using Graphviz.

For more details on the StreamIt compiler and execution environment,
including instructions for compiling to the {\bf Raw architecture},
please consult Section~\ref{sec:compiler}.
