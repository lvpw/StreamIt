\section{Related Work}
\label{sec:related}

Several other researchers have pursued the idea of operating directly
on compressed data formats.  The novelty of our work is two-fold:
first, in its ability to map an arbitrary stream program, rather than
a single predefined operation, into the compressed domain; and second,
in its focus on lossless compression formats.

Most of the previous work on mapping algorithms into the compressed
domain has focused on formats such as JPEG that utilize a Discrete
Cosine Transform (DCT) to achieve spatial
compression~\cite{smith98,dorai00,dugad01,feng03,mukherjee02,nang00,shen96,shen96b,shen98,smith96b,vasudev98}.
This task requires a different analysis, with particular attention
given to details such as the blocked decomposition of the image,
quantization of DCT coefficients, zig-zag ordering, and so-on.
Because there is also a run-length encoding stage in JPEG, our current
technique might find some application there; however, it appears that
techniques designed for JPEG have limited application to formats such
as LZ77.  Also, we are unaware of any previous methodology for
translating a generic program to operate on compressed data; previous
efforts have mapped each algorithm in a manual and ad-hoc way.

There has been some interest in performing compressed processing on
lossless encodings of black-and-white images.  Shoji presents the pxy
format for performing transpose and other affine
operations~\cite{shoji95}; the memory behavior of the technique was
later improved by Misra et al.~\cite{misra99}.  
%As described in Section~\ref{sec:formats}, 
The pxy format lists the $(x,y)$ coordinate pairs at which a
black-and-white image changes color during a horizontal scan.  If the
image is represented using LZ77 compression, our transformation could
also preserve some compression during a transpose if we include
support for filters with state (see Section~\ref{sec:extensions}.

From the perspective of programming languages, Swartz and Smith
present RIVL, a Resolution Independent Video
Language~\cite{swartz95}.  The language is used to describe
a sequence of image transformations; this allows the compiler to
analyze the sequence and, via lazy evaluation, to eliminate any
operations that do not effect the final output.  Such a technique is
complementary to ours and could also be implemented using StreamIt as
the source language.
