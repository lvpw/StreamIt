Due to the high data rates involved in audio, video, and signal
processing applications, it is imperative to compress the data to
decrease the amount of storage used.  However, compression incurs
extra computational overhead, as any program operating on the data
must be wrapped by a decompression and re-compression stage.

In this paper, we present a program transformation that eliminates
much of the overhead involved in processing compressible data.  Given
a program that operates on uncompressed data, we output an equivalent
program that operates directly on the compressed format.  We currently
support lossless compression formats based on LZ77, a popular
compression algorithm utilized by gzip, PNG, and Apple Animation.  Our
transformations rely on the streaming model of computation, which
exposes the flow of data in the applications.

To evaluate the impact of our transformations, we implemented plugins
for two digital video editing tools: MEncoder and Blender.  For common
operations such as color adjustment and video compositing, computing
directly on compressed data offers a speedup roughly proportional to
the overall compression ratio.  For our benchmark suite of 12 videos
in Apple Animation format, speedups range from 1.0x to 235x, with a
median of 16x.
