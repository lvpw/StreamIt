\documentclass[11pt]{article}

\usepackage{palatino}
\usepackage{streamit}

\title{StreamIt Semantic Checks}

\begin{document}

\maketitle

This document contains a list of semantic checks that should be
performed on StreamIt programs after they have been parsed.  These
assume the program is syntactically correct; it must pass through the
parser correctly.

\section{Statement Placement}

\subsection{Init, Work, and Phase Functions}

\begin{enumerate}
\item \lstinline|add| statements can only appear in init functions in
  split-joins and pipelines;
\item \lstinline|split| and \lstinline|join| statements can only
  appear in init functions in split-joins and feedback loops;
\item \lstinline|loop|, \lstinline|body|, and \lstinline|enqueue|
  statements can only appear in init functions in feedback loops;
\item \lstinline|push| statements can only appear in work and phase
  functions in filters.
\item \lstinline|pop| and \lstinline|peek| expressions can only appear
  in work and phase functions in filters.
\item If a work function does not declare I/O rates, it may not use
  \lstinline|push|, \lstinline|pop|, or \lstinline|peek|.
\end{enumerate}

\subsection{Control Flow and Initialization Code}

Several statements may only be executed once per function; other
statements are sensitive to the order in which they are executed.

\begin{enumerate}
\item \lstinline|split|, \lstinline|join|, \lstinline|body|, and
  \lstinline|loop| statements must be executed exactly once if they
  are allowed at all.  Every path of control through the
  initialization code must result in exactly one execution.
\item For purposes of type-checking for pipelines, a data-flow
  analysis must be able to determine the exact type of the pipeline at
  each point in the initialization code.
\item A data-flow analysis must be able to determine the number and
  sequence of phases, along with any intermediate code, for a phased
  filter.
\item Nothing other than a work function which does not declare I/O
  rates may call a phase function.
\end{enumerate}

\section{Typing}

\subsection{Streams}

``Match'' within this section means an \emph{exact} match in type;
there is no implicit type promotion across tapes.

\begin{enumerate}
\item If the input type of a filter is \lstinline|void|, all of its
  work functions must declare peek and pop rates of 0; if its output
  type is \lstinline|void|, all of its work functions must have a push
  rate of 0.
\item For a pipeline, the input type must match the input type of the
  first child and the output type of the last child must match the
  output type of the pipeline.
\item Within a pipeline, the output type of one child must match the
  input type of the next child.
\item The input type of a split-join must match the input type of all
  of its children.  If the input type is \lstinline|void|, the
  splitter must be \lstinline|roundrobin(0)|.  Identical constraints
  hold for the output.
\item The output type of a feedback loop's loop stream must match the
  input type of its body stream.  If this type is \lstinline|void|,
  the input type of the feedback loop must also be \lstinline|void|
  and the joiner must be \lstinline|roundrobin(0)|.  Otherwise, the
  input type of the feedback loop must be either the same type as the
  body input or \lstinline|void|.  If the feedback loop's input type
  is \lstinline|void| but the input type of the body stream is not,
  then the joiner must be a round-robin joiner that gives a zero
  weight to the external input.  Identical constraints hold for the
  output side of the feedback loop.
\item Exactly one stream in the program has the stream type
  \lstinline|void->void|.  This stream must be named.
\end{enumerate}

\subsection{Statements and Expressions}

\begin{enumerate}
\item Types can be promoted: a \lstinline|bit| can be implicitly
  converted to an \lstinline|int|, an \lstinline|int| can be promoted
  to a \lstinline|float|, and a \lstinline|float| can be promoted to
  \lstinline|complex|.
\item Every referenced variable must have a declared type.
\item Every expression has a type:
\begin{enumerate}
\item The type of a variable reference is the type of the variable;
\item The type of a literal is the lowest type that can contain the
  literal (the type of \lstinline|1| is \lstinline|bit|, the type of
  \lstinline|2| is \lstinline|int|, the type of \lstinline|2.0| is
  \lstinline|float|, and the type of \lstinline|2i| is
  \lstinline|complex|);
\item The type of a unary expression is the type of the child
  expression.  The type of a binary expression is the join of the
  types of the child expression, possibly promoted to a type the
  operator can accept.  The type of the first part of the ternary
  expression must be promotable to \lstinline|int|, and the second and
  third parts must be promotable to a common type; the resulting type
  is the common type.
\item For a field reference \lstinline|a.b|, \lstinline|a| must be of
  some structure type \lstinline|A|, which must have a field named
  \lstinline|b|.  The type of the expression is the type of
  \lstinline|b| in the structure declaration.
\item For an array access \lstinline|a[i]|, \lstinline|a| must be of
  some array type \lstinline|T[n]|, and the type of the expression is
  \lstinline|T|.
\item The type of a \lstinline|peek()| or \lstinline|pop()| expression
  is the input type of the filter.
\item The parameter of a \lstinline|peek()| expression and the index
  of an array must both be of type \lstinline|int|.
\item The type of a call to a helper function is the return type of
  that function.
\end{enumerate}
\item The type of the parameter of a \lstinline|push()| statement must
  be the output type of the filter.
\item The type of the parameter of an \lstinline|enqueue()| statement
  must be the output type of the feedback loop's loop stream.
\item The type of the right-hand side of an assignment statement must
  be promotable to the type of the left-hand side of the statement.
\end{enumerate}

\section{Naming}

\subsection{Object Names}

\begin{enumerate}
\item No two streams or structures may have the same name.
\item Within a single structure, no two member fields may have the
  same name.  Field names must also be distinct from all stream and
  structure names.
\item Stream parameters within a single stream must have distinct
  names from each other and from all stream and structure names.
\item Functions declared within a filter must have unique names; these
  include phase functions, message handlers, and helper functions.
  The name of a function must be distinct from stream and structure
  names.
\item A variable may not be declared with the same name as a stream
  parameter.  (Other variable hiding is legal.)  The name of a
  variable must be distinct from stream, structure, and function
  names within the current filter.
\end{enumerate}

\subsection{Using Names}

\begin{enumerate}
\item A stream constructor (the parameter to \lstinline|add|,
  \lstinline|body|, and \lstinline|loop| statements) must be an
  anonymous stream declaration, or must have a name matching the name
  of a stream object with a parameter list matching the parameter list
  of  the the stream.
\item A variable must be declared before it is used.
\item A stream parameter may not be assigned to or otherwise modified.
\end{enumerate}

\section{Warnings}

Things in this section are technically legal, but indicate
questionable code:

\begin{enumerate}
\item Some things in StreamIt are counted; these include split-join
  children (with fixed-length weighted round-robin splitters or
  joiners), and work function pushes, pops, and peeks.  It should be a
  warning if the compiler cannot statically confirm that these numbers
  are not met exactly (or, for peeks, that the peek rate is not
  exceeded).   For feedback loops, the compiler can also issue a
  warning if it cannot confirm that enough items are enqueued to cause
  the joiner to be fired at the start of the program.
\item The compiler may need to guess at the I/O types of anonymous
  streams.  If the type isn't clear from context a warning should be
  issued.
\item A warning should be issued if a phase function is declared which
  is not used.
\end{enumerate}

\section{Temporary Checks}

These checks are necessary due to current constraints in the StreamIt
compiler.  They should be able to be removed when the compiler is
improved.

\begin{enumerate}
\item A statement cannot contain more than one \lstinline|pop()|
  expression; if it contains any \lstinline|pop()| expressions, it can
  contain no \lstinline|peek()| expressions.  This can be removed once
  the ordering of these expressions is established in the language
  specification and enforced appropriately in the compiler.
\item All child-constructing statements in initialization code must
  have consistent types.  While the front-end can perform minimal
  data-flow analysis to check that the types of pipelines, control
  flow like the following is presently illegal:

\begin{lstlisting}{}
int->int filter IntIntBody { ... }
int->int filter IntIntLoop { ... }
float->int filter FloatIntBody { ... }
int->float filter IntFloatLoop { ... }

void->int feedbackloop SketchyLoop(int p) {
  join roundrobin(0, 1);
  if (p) {
    body IntIntBody();
    loop IntIntLoop();
  } else {
    body FloatIntBody();
    loop IntFloatLoop();
  }
  split duplicate;
  enqueue(1);
}
\end{lstlisting}

  The types of the loop and all of its components are consistent
  regardless of the value of \lstinline|p|, but if \lstinline|p| is
  zero, the \lstinline|enqueue| statement takes a \lstinline|float|
  parameter.  Presently the front-end should reject this construction
  since it is unclear if the types around the joiner are consistent or
  not.  This check can be removed if the compiler does stream
  type-checking after constant propagation.
\item None of the names listed in the ``Implementation Limits''
  section of the language specification should be used.  These include
  the capitalized names \lstinline|SplitJoin|,
  \lstinline|FeedbackLoop|, \lstinline|Filter|, \lstinline|Pipeline|,
  \lstinline|StreamIt|, and \lstinline|Complex|.
\end{enumerate}

\end{document}