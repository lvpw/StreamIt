\section{Design Rationale}
\label{sec:designrationale}

Our approach to removing induction variable state is to introduce a new language construct.  The construct maintains a value indicating how often the corresponding filter has been invoked.  This approach was chosen over implementing a system to automatically recognize induction variable usage in filter construction.

The approach of automatic analysis would require detecting induction variables defined in the filter.  The simplest form of this analysis would idiomatically detect a variable modified by a statement similar to \texttt{var=var+1}.  Very few iteration filters (outside of source filters) use induction state in this limited capacity.  Consider Figure~\ref{fig:weight-calc}, where the induction variable is incremented at each iteration step, until a certain threshold value at which it will reset.  This pattern is very common in programs that use induction filters.  MPD and FIRBank use this technique to iterate across a provided array one element per iteration step.  To ensure consistency between iteration steps, the automatic analysis is required to detect these types of updates as well.  

%\begin{figure}[t]
%{\eightpoint
%\begin{verbatim}
%float->float filter WeightCalc(int n)
%{
%  float[n] window;
%  int windowPos;
%
%  ...
%
%  // the input stream is multiplied with the weights
%  work push 2 pop 2
%  {
%
%    push(pop() * window[windowPos]);
%    push(pop() * window[windowPos]);
%
%    windowPos++;
%    if(windowPos >= n)
%    {
%      windowPos = 0;
%    }
%  }
%}
%\end{verbatim}
%\caption{MPD filter that multiplies stream values with weights.\protect\label{fig:weight-calc}}}
%\end{figure}

A filter may have multiple induction variables that are dependent on one another in defining their values, as in the stateful filter in Figure~\ref{fig:transform-after-twonested}.  The automatic analysis must also be able to detect incrementing statements that may not necessarily be updated on every work call.  The process of simply detecting and identifying induction variables can potentially branch into many cases that need to be specially implemented.

There may potentially be other special cases that must be defined into the automatic analysis.  Filters may define induction variables that start with and reset to a particular value.  Induction variables may increment by a different value other than one at each execution step.  Co-induction variables may be constructed to reset the value of other induction variables after reaching a certain value.  The different uses of induction variables may be difficult to assess.  Though the methods of using induction variables as illustrated in Figures~\ref{fig:wc-example}a. and~\ref{fig:transform-after-twonested} encompass many of the common use cases in the benchmark suite, slight variations in the implementation to the pattern may prevent the induction variable from being detected.

Automatic recognition is fairly inflexible in detecting induction variable state.  The process would restrict data parallelism opportunities to only the filters that fit the implemented templates.  We instead elect to provide the user with the flexibility of defining their own derived induction values.  

It may also be difficult to predict how the induction value will be updated, as some of the use cases described above may demonstrate.  The keyword solution has the added benefit of maintaining a value that is predictable in its updates.  The value that the keyword returns is simply the number of times the filter has been invoked.  This value is always incremented by one at the end of every work call.  

Furthermore, the approach of automatic analysis does little to encourage programming with parallelism in mind.  On inspection, user written code will still maintain state, which actively inhibits data parallelism opportunities.  In introducing a language construct, user written code eliminates explicitly kept state.  This approach encourages users to write code with the intention of exploiting parallelism.


