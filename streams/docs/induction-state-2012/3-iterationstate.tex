\section{Iteration State}

\begin{figure}[t]
{\small
\begin{verbatim}
   float->float filter MovingAverage(int N) {
     float[N] weights;  // a filter field

     // init function initializes the weights
     init {
       for (int i=0; i<N; i++)
         weights[i] = 1/N;
     }

     // work function declares push, pop, peek rates
     work push 1 pop 1 peek N {
       float result = 0;  // a local variable
       for (int i=0; i<N; i++) {
         result += weights[i] * peek(i);
       }
       push(result);
       pop();
     }
   } 
\end{verbatim}
\caption{Example of a StreamIt filter.\protect\label{fig:filter-example}}}
\end{figure}


Traditional induction variables encapsulate all variables that are
increased or decreased with iterations of a loop.  Induction variable
state as applied to stream programming is a class of state that
requires keeping count of how often a filter has been invoked.  Common
usage of induction variable state include performing some special
action after a certain number of iterations and keeping track of array
index positions.  Many filters maintain multiple induction variables
as well, which may be either dependent or independent of each other.

As is most commonly used now, induction variable state is maintained
by creating a mutable state field in the corresponding filter.  This
state can be set to the desired starting value.  The induction
variable is consistently updated at some point during the work call.
For many use cases, this induction variable may need to be reset if it
reaches a certain threshold. [can put example code here]

As presently constructed, induction variable state forces the
corresponding filter to be run in sequential order.  In providing a
mutable state whose value is dependent on the previous execution step,
it is necessary to run a filter execution step and establish the
induction variable value before moving on to the next execution step.
The sequential order of execution for this filter inhibits data
parallelism, as it is impossible to partition the filter's execution
steps into independently running cores.


