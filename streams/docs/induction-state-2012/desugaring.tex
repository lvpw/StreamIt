\section{Desugaring}

\begin{figure}[t]
{\eightpoint
\begin{verbatim}
  int->int filter IterationFilter() {

      work push 1 pop 1{

          int counter = iter();
          ...

      }
  }
\end{verbatim}
\caption{Example of a StreamIt filter using the iteration keyword.\protect\label{fig:iter-filter-example}}}
\end{figure}


\begin{figure}[t]
{\eightpoint
\begin{verbatim}
  int->int filter IterationFilter() {

      int __iterationCount = 0;      

      work push 1 pop 1{

          int counter = __iterationCount;
          ...

          __iterationCount++;

      }
  }
\end{verbatim}
\caption{Example of a StreamIt filter with the iteration keyword desugared.\protect\label{fig:desugar-filter-example}}}
\end{figure}


Under the StreamIt framework, filters are classified as iteration filters when there is usage of the iteration keyword in its work function.  A simple lexer and parser can identify the use of this keyword, and it will be represented accordingly in the intermediate representation.  To have it actually return the correct iteration value, we first convert the keyword to a compiler-usable construct.

When the compiler comes across filter classified as an iteration filter, the iteration keyword is replaced with an access to a field holding the value of the iteration count.  The filter is given a definition for this field only if an iteration keyword use exists in its work or prework definition.  The work and prework function are appended with incrementing statements that update the iteration value.

It is important to note, these filters are not classified as stateful to the user, even though the filter is actually stateful on the iteration count after the desugaring process.  In classifying filters as stateful, the user is made aware of where data parallelism may be inhibited.  Iteration filters will not inhibit data parallelism because it is identifiable to the compiler during the fission process.   
