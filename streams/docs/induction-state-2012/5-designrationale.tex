{\large DESIGN RATIONALE}

Our approach in removing induction variable state is to introduce a new language construct.  The language construct maintains a value indicating how often the corresponding filter has been invoked.  This approach was chosen over implementing a system to automatically recognize induction variable usage in filter construction.

There are several downsides to the approach of automatic analysis.  Automatic recognition is a fairly inflexible process in removing induction variable state.  As illustrated in the previous section, there may be many different ways of defining induction variables.  Nested counters are often used to track row and column indexes in two-dimensional arrays.  Co-induction variables may be constructed to reset the value of other variables after reaching a certain value.  The many different uses of induction variables may be difficult to assess.  Automatic recognition would restrict data parallelism opportunities to only the filters that fit the implemented templates.  We instead elect to provide the user with the flexibility of defining derived induction values while providing parallelism opportunities.  

The approach of automatic analysis also does little to encourage programming with parallelism in mind.  User written code will still maintain state, which inhibits data parallelism opportunities.  In introducing a language construct, user written code eliminates explicitly kept state.  This approach encourages users to write code with the intention of exploiting parallelism.


