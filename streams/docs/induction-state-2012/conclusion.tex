\vspace{-4pt}
\section{Conclusion}
\label{sec:conclusion}

Too often there exists tension between the expressiveness of a
language and the robustness of automatic parallelization techniques
for programs written in the language. The StreamIt language chose to
be expressive in the general case (e.g., allowing stateful computation
and dynamic rates) but to enable parallelization by offering the
programmer helpful idioms to capture common program behavior (e.g.,
static rate definitions and peeking). In this work we continue to add
to the array of idioms in StreamIt that ease implementation burden
while exposing parallelism by capturing and parallelizing a common
form of state: induction variable state.  Our techniques remove a key
sensitivity of the StreamIt compiler for the goal of robust and
scalable automatically-managed parallelization.

This paper demonstrates that induction variable state is prevalent in
real-world applications, and that this type of state, if not
parallelized, represents a significant bottleneck to scalable
parallelization. We introduce a keyword expression that can be used as
the basis of all derived induction variable state. The keyword, when
executed, returns the current execution iteration of a filter. We
present a desugaring of the keyword that enables minimal changes to
the existing StreamIt compiler codebase. We present modifications to
the fundamental filter fission parallelization transformation that
extracts data parallelism from filters that included the new keyword.
Our techniques demonstrate a substantial performance improvement on
the MPEG2 motion estimation subset when explicit iteration state is
replaced by our keyword solution.


%In this paper, we describe a technique to eliminate induction variable state in stream programs.  Eliminating induction variable state allows for the data parallelizing of several programs in the StreamIt framework.  The introduction of a expression keyword can help simplify written code and provides automatic parallelization of induction variable state.  
%
%We also perform an analysis of the potential speedups on stateful programs.  Statefulness is a key inhibitor of data parallelism, which provides load-balanced and limitless parallelism.  It is important to expose data parallelism as processing power scales more with the number of available cores.  We found that even with proportionally small amounts of work performed in stateful filters, it is possible to obtain drastic runtime improvements if the corresponding state were removed.  As more processing cores are made available, filter state serves as a significant throughput bottleneck to scalable parallelization.

