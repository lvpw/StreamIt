\section{The StreamIt Compiler}
\label{sec:compiler}
The phases of the StreamIt compiler are described in
Table~\ref{tab:phases}.  The details of each stage can be found in
~\cite{streamit-asplos}.  The following subsections provide an
overview on the phases of the StreamIt compiler and the changes made
to the corresponding phases in implementing the iteration keyword.

\subsection{StreamIt Compiler Overview}
\label{sec:compiler-overview}

The front end is built on top of KOPI, an open-source compiler 
infrastructure for Java~\cite{kopi}.  We translate the KOPI syntax 
tree into the StreamIt IR (SIR) that encapsulates the hierarchical 
stream graph.  The graph is then expanded for structures that
are parametrized.  Constants are propagated through the program
and the graph is expanded into a static structure.

We can calculate an execution schedule for the nodes of the stream 
graph.  The schedule indicates a multiplicity for each filter in the
stream graph, which determines how often the work function should be
invoked.  The schedule is periodic, meaning its
execution must preserve the number of live items on each channel in
the graph.  Accordingly, there is a {\it steady-state} schedule that is 
periodic which maintains this invariant.  Peeking filters, however, 
introduce another layer of complication.
A filter with {\it peek $>$ pop} leaves {\it peek - pop} items on the
input channel after every firing.  If the periodic schedule also
contains this schedule, every firing would leave extra leftover items
in the channel per invocation.  Accordingly, peeking filters require a separate,
non-periodic {\it intialization} schedule.  




\begin{table}[t]
\begin{center}
\scriptsize
\begin{tabular}{|l|l|} \hline
{\bf Phase} & {\bf Function} \\
\hline \hline
KOPI Front-end & Parses syntax into a Java-like abstract syntax tree. \\ &\\
\hline
SIR Conversion & Converts the AST to the StreamIt IR (SIR). \\ &\\
\hline
Graph Expansion & Expands all parameterized structures in the \\
                & stream graph. \\
\hline
Scheduling & Calculates initialization and steady-state \\
           & execution orderings for filter firings. \\
\hline
Partitioning & Performs fission and fusion transformations \\
             & for load balancing. \\
\hline
Layout & Determines minimum-cost placement of filters \\
       & on grid of Raw tiles. \\
\hline
Communication Scheduling & Orchestrates fine-grained communication \\
                         & between tiles via simulation of the stream graph. \\
\hline
Code generation & Generates code for the tile and switch processors. \\
&\\
\hline
\end{tabular}
\caption{\protect\small Phases of the StreamIt compiler. ~\cite{streamit-asplos}
\label{tab:phases}}
\end{center}
\end{table}

{\it Partitioning} is the process of dividing the stream program into
a set of balanced computation units.  Given a number $N$, representing
the maximum number of computation units that can be supported, the 
partitioning step transforms a stream graph into a set of at most $N$
load-balanced filters.  Each filter can be run on a separate processor.
The partitioning step uses a set of fusion, fission, and reordering 
transformations to achieve the desired granularity and load-balancing~\cite{streamit-asplos}.

The {\it layout} phase assigns nodes in the stream graph to the
computation nodes in the target architecture while minimizing the
communication and synchronization in the final layout.  The {\it
communication scheduling} phase maps the communication explicit
in the stream graph to the interconnect of the target.  The FIFO
abstraction of the stream channels is mapped to the limited resources
of the target.  Finally, {\it code generation} is performed using the
results of all previous phases.  More details about these phases can be 
found in~\cite{streamit-asplos}