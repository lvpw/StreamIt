\section{Related Work}

Printz's ``signal flow graphs'' includes nodes that performed a
sliding window~\cite{printz91thesis}.  Printz presents an
implementation strategy translating the sharing requirement of data
parallelization of peeking filters into communication.  However, his
approach does not alter the steady-state of the graph to reduce
sharing to neighboring fission products, and it does not further alter
the steady-state to reduce sharing to under a given threshold.
Instead, Printz's technique parallelizes the communication required by
sharing via a pipelined sequence of transfers between neighboring
cores on the iWarp machine.  Finally, Printz did not implement his
strategy; instead his evaluation relied on an model of a parallel
architecture.

% In the Warp project, the AL language~\cite{tseng89thesis,tseng90} had
% a window operation for use with arrays.  The AL compiler targeted the
% Warp machine and translated loops into systolic computation.  The AL
% compiler did alter the blocking of distributed loop iterations to
% reduce the sharing requirement for the sharing of the sliding window
% among iterations of the loop.  However, this was in the context of an
% imperative language with parallel arrays and considered a different
% application class: matrix computation. 

The Brook language requires a programmer to explicitly represent the
communication of sliding windows by specifying that a filter reads
overlapping portions of the input. Liao et al. map Brook to multicore
processors by leveraging the affine partitioning
model~\cite{liao06brook}. However, from our understanding, they do not
block across steady-states of the application, meaning the application
needs to be described at a granularity coarse enough to block
iterations of the filter to be parallelized across cores. In other
words, the sizes of the streams between producer and consumer need to
be large enough to enable parallelization and blocking to reduce
sharing caused by sliding windows.  Our framework frees the programmer
from having to consider buffers and granularity, enabling portable,
reusable, and malleable code.

Other languages have included the notion of a sliding window.  The ECOS
graphs language allows actors to specify how many items are read but
not consumed~\cite{huang_ecos_1992}; the Signal language allows access
to the window of values that a variable assumed in the
past~\cite{le_guernic_signal--data_1986}; and the SA-C language
contains a two-dimensional windowing
operation~\cite{draper_compiling_2001}.  However, to the best of our
knowledge, translation systems for these languages do not utilize
sliding windows to improve parallelism and reduce inter-core
communication.

Traditional parallelizing compilers could achieve similar mappings
(i.e., data parallelizing a loop nest and reducing the sharing between
iterations mapped to separate cores), but the application would have
to be expressed at a granularity coarse enough that the compiler could
block the iterations of the inner loops as necessary for the parallel
mapping.  This would require the programmer to consider and explicitly
specify loop bounds on the inner loops representing producers and
consumers.  Given access patterns of the sliding window filters that
we are considering (similar to a stencil pattern in scientific
computing) and the complexity of the fine-grained communication
described by scatter and gather operations, this is a very difficult
task for the applications we consider.

Our key contribution over previous work in both streaming computation
and traditional scientific frameworks is that we automatically adjust
the granularity of the computation in the presence of non-affine
dependencies to decrease inter-core communication, and that we give
expressions for calculating the destination(s) of data items.