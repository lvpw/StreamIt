\section{Streaming Execution Model}

In order to rigorously describe our techniques it is necessary to
define an execution model for streaming computation. The model we use
for this paper is a general model that is agnostic of input language.

Consider a directed acyclic graph $G = (V, E)$ corresponding to a
streaming application. $F \in V$ is a filter in the application and
$(f, g) \in E$ is an edge in the graph that denotes communication from
$f$ to $g$ using a FIFO channel.  A filter may have multiple incoming
edges and multiple outgoing edges.  Filter inputs are organized into a
single FIFO buffer for the filter to read according to an {\it input
distribution pattern}, and filter outputs are distributed from an
output FIFO buffer according to a {\it output distribution pattern}.  Both
are described below.

One execution of a filter is termed a filter {\it firing}. For each
filter $F \in V$ a {\it prework} function, $W_p(F)$, and a {\it work},
$W(F)$ function is defined.  The prework function is executed on the
first firing of the filter and the work function is executed on all
subsequent firings of the filter.  These work functions are the atomic
unit of computation for a filter and denote a function with the
variable $\mathcal{F} \in \{W_p, W\}$.  In the work and prework
functions, a filter accesses its input tape via the expressions {\tt
pop()}, which dequeues and item, and {\tt peek(i)}, which inspects but
does not dequeue the item at index {\tt i} from the beginning of the
buffer (beginning at 0).  {\tt push(val)} enqueues an item onto the
output buffer.

For each filter $F \in V$ we define the following:
\begin{itemize}

\item $o(\mathcal{F}, F)$, the number of items dequeued from $F$'s
input buffer per firing of $\mathcal{F}$,

\item $e(\mathcal{F}, F)$, the number of items read (but not dequeued)
from $F$'s input buffer per firing of $\mathcal{F}$,

\item $u(\mathcal{F}, F)$, the number of items enqueued to $F$'s
output buffer per firing of $\mathcal{F}$,

\end{itemize}

For all $F \in V$, we require that the above quantities be statically
determinable.  Also, we require that the input and output distribution
patterns be statically determinable.  This special case of streaming
computation is termed synchronous dataflow (SDF) and it is natural and
expressive for many domains including DSP, network, image, voice, and
multimedia~\cite{leeSDF}.  Furthermore, applications with dynamic
I/O rates can often be cut into static rate region for which the
techniques in this paper are applicable~\cite{chen:graphics-hardware:2005}.

A schedule of a graph $G$ gives a multiplicity for each filter $F \in
V$ that denotes how many times to fire filter $F$. In the SDF domain,
a {\it steady-state} schedule, $S$, of a graph can be calculated such
that the quantity of items on each of the filter's input buffer and
output buffer remains unchanged by the complete execution of the
schedule~\cite{lee87}.  Thus this schedule can be repeated
indefinitely because buffers to not grow or shrink in size.
Furthermore, because of the {\tt peek} construct, a separate {\it
initialization} schedule, $I$, is required. After the initialization
schedule executes, each filter is guaranteed to have at least $e(W,
F)$ items in its input buffer. The initialization schedule is required
to calculate a steady-state schedule for a graph with an $F$ such that
$e(W, F) > 0$~\cite{karczmarek:lctes:2003}.  During application
execution, The initialization schedule is executed once followed by an
infinite repetition of the steady-state schedule. 

The number of items remaining on a $F$'s input buffer after execution
of the initialization schedule is given by $C(F)$. A schedule of
execution is denoted by the variable $\Sigma$, where $\Sigma \in
\{I, S\}$.  The multiplicity of filter $F$ in schedule $\Sigma$ is
denoted by $M(\Sigma, F)$.

The input distribution pattern for filter $F$ for schedule $\Sigma$ is
a weighted round-robin described by two sequences:

\[ \mt{IW}(\Sigma, F) \in (\mathbb{N}^{*})^n \]

\[ \mt{IE}(\Sigma, F) \in E^n \]
 
Conceptually, to gather the items on the multiple input edges for $F$
into $F$'s input buffer we cycle through the edges $e_i \in
\mt{IE}(\Sigma, F)$ and copy $w_i \in \mt{IW}(\Sigma, F)$ items to the
current position in the input buffer.

The output distribution pattern describes both duplication and
distribution in the same structures. For filter $F$ and for schedule
$\Sigma$, the distribution is given by a sequence of weights and
sequence of set of edges:

\[ \mt{OW}(\Sigma, F)  \in (\mathbb{N}^{*})^m \]

\[ \mt{OE}(\Sigma, F) \in (D_1, D_2, ..., D_m)  \mbox{ where }  D_i \subset
E \]



Each set $D_i$ of the outgoing edges sequence denotes a
duplication sec


