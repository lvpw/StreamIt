\begin{figure}
\centering
\psfig{figure=beam-graph.eps,width=3.2in}
\caption{Stream graph of the Radar Application consisting of 12
channels and 4 beams. The non-linear filters are colored black. 
\label{fig:beam-graph}}
\end{figure}

\section{Motivating Example}
To elucidate the phases of compilation, consider the stream graph
given in Figure \ref{fig:beam-graph}.  The stream graph represents our
Radar application which implements a radar array front-end \cite{pca}.
The stream graph shows how the application was decomposed by the
programmer into filters and how these filters are connected.  The
application is composed of two splitjoins.  The top splitjoin gathers
input from a set of parallel channels, with FIR filters to delay each
channel by a different amount.  Each pipeline of the top splitjoin is
linear.  The bottom slipjoin steers the channels into a set of beams
with a detector to sense if the signal exceeds a given threshold.  
The Magnitude filter at the bottom stage is non-linear.  

\begin{itemize}
\item Flattening: all outputs of top stage communicate to all inputs
of bottom stage.
\item Show Traces that are extracted.
\item Show where / when each trace executed
\end{itemize}
