\begin{figure}
\center
\epsfxsize=2.5in
\epsfbox{images/precoder.eps}
\epsfxsize=2.5in
\epsfbox{images/de-precoder.eps}
\caption{8VSB precoder(left) and de-precoder(right)}
\label{fig:precoder-deprecoder}
\end{figure}

\begin{table}
\center
\begin{tabular}{c|c|c}
input(x2) & output(y2) & state  \\
\hline
0 & 0 & 0 \\
1 & 1 & 0 \\
1 & 0 & 1 \\
0 & 0 & 0 \\
1 & 1 & 0 \\
0 & 1 & 1 \\
1 & 0 & 1 \\
1 & 1 & 0 \\
1 & 0 & 1 \\
0 & 0 & 0 \\
\end{tabular}

\begin{tabular}{c|c|c}
input(y2) & output(x2) & state  \\
\hline
0 & 0 & 0 \\
1 & 1 & 0 \\
0 & 1 & 1 \\
0 & 0 & 0 \\
1 & 1 & 0 \\
1 & 0 & 1 \\
0 & 1 & 1 \\
1 & 1 & 0 \\
0 & 1 & 1 \\
0 & 0 & 0 \\
\end{tabular}
\caption{Example of encoding(above) and de-precoding(below) when the input is 0110101110.}
\label{tbl:precoding_example}
\end{table}

A precoder is a simple circuit which XORs the current input with the previous
output to get the current output. Table~\ref{tbl:precoding_example} 
contains a worked out example of the input, output, and state of the precoder
for coding the data \texttt{01101011110}.

\begin{table}
\center
\end{table}

To decode the precoded data, we simply reverse the process. It might be the case that 
the Viterbi algorithm should be applied to recover the input sequence, and we will
look into it. Table~\ref{tbl:precoding_example} contains a worked out example of
decoding the data \texttt{0100110100}.
